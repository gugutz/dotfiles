% Created 2018-09-09 dom 18:15
% Intended LaTeX compiler: pdflatex
\documentclass[11pt]{article}
\usepackage[utf8]{inputenc}
\usepackage[T1]{fontenc}
\usepackage{graphicx}
\usepackage{grffile}
\usepackage{longtable}
\usepackage{wrapfig}
\usepackage{rotating}
\usepackage[normalem]{ulem}
\usepackage{amsmath}
\usepackage{textcomp}
\usepackage{amssymb}
\usepackage{capt-of}
\usepackage{hyperref}
\author{gugutz}
\date{\today}
\title{emacs configuration file\\\medskip
\large ORGfied configuration for Emacs}
\hypersetup{
 pdfauthor={gugutz},
 pdftitle={emacs configuration file},
 pdfkeywords={},
 pdfsubject={This file is compiled to init.el automatically on every save},
 pdfcreator={Emacs 26.1 (Org mode 9.1.14)}, 
 pdflang={English}}
\begin{document}

\maketitle
\setcounter{tocdepth}{0}
\tableofcontents


\section*{Recompile init.el everytime emacs.org is changed and saved}
\label{sec:org4cc40c2}

   Moved this to beggining of the file to avoid it not being parsed when theres an error in the middle of the file
It was being recompiled without this function so i had to manually re-copy first-init.el to make it compile first time again and again


\begin{verbatim}
(defun /util/tangle-init ()
  (interactive)
  "If the current buffer is init.org' the code-blocks are
tangled, and the tangled file is compiled."
  (when (equal (buffer-file-name)
               (expand-file-name (concat user-emacs-directory "emacs.org")))
    ;; Avoid running hooks when tangling.
    (let ((prog-mode-hook nil))
      (org-babel-tangle)
      (byte-compile-file (concat user-emacs-directory "init.el")))))
\end{verbatim}

\begin{verbatim}
(add-hook 'after-save-hook #'/util/tangle-init)
\end{verbatim}


\section*{Package Repositories}
\label{sec:org5fc04d2}

\begin{verbatim}
(require 'package)
;; add melpa stable emacs package repository
(add-to-list 'package-archives '("melpa" . "https://melpa.org/packages/"))
(add-to-list 'package-archives '("gnu" . "https://elpa.gnu.org/packages/"))
(add-to-list 'package-archives '("org" . "http://orgmode.org/elpa/") t) ; Org-mode's repository
\end{verbatim}

\subsection*{initialize packages}
\label{sec:orgfd4f04e}
\begin{verbatim}
(package-initialize)
\end{verbatim}

moved this part to beggining of the file because if the
custom-safe-themes variable is not set before smart-mode-line (sml) activates
emacs asks 2 annoying confirmations on every startup before actually starting

\begin{verbatim}
(custom-set-variables
 ;; custom-set-variables was added by Custom.
 ;; If you edit it by hand, you could mess it up, so be careful.
 ;; Your init file should contain only one such instance.
 ;; If there is more than one, they won't work right.
 '(custom-safe-themes
   (quote
    ("84d2f9eeb3f82d619ca4bfffe5f157282f4779732f48a5ac1484d94d5ff5b279" "57f95012730e3a03ebddb7f2925861ade87f53d5bbb255398357731a7b1ac0e0" "3c83b3676d796422704082049fc38b6966bcad960f896669dfc21a7a37a748fa" default)))
 '(fci-rule-color "#3E4451")
 '(package-selected-packages
   (quote
    (pdf-tools ox-pandoc ox-reveal org-preview-html latex-preview-pane smart-mode-line-powerline-theme base16-theme gruvbox-theme darktooth-theme rainbow-mode smartscan restclient editorconfig prettier-js pandoc rjsx-mode js2-refactor web-mode evil-org multiple-cursors flycheck smart-mode-line ## evil-leader evil-commentary evil-surround htmlize magit neotree evil json-mode web-serverx org))))
(custom-set-faces
 ;; custom-set-faces was added by Custom.
 ;; If you edit it by hand, you could mess it up, so be careful.
 ;; Your init file should contain only one such instance.
 ;; If there is more than one, they won't work right.
 )
\end{verbatim}


\section*{Bindings}
\label{sec:orgf725535}


These macros are to help me remap keys.

\begin{verbatim}
(defmacro /bindings/define-prefix-keys (keymap prefix &rest body)
  (declare (indent defun))
  `(progn
     ,@(cl-loop for binding in body
                collect
                `(let ((seq ,(car binding))
                       (func ,(cadr binding))
                       (desc ,(caddr binding)))
                   (define-key ,keymap (kbd seq) func)
                   (when desc
                     (which-key-add-key-based-replacements
                       (if ,prefix
                           (concat ,prefix " " seq)
                         seq)
                       desc))))))

(defmacro /bindings/define-keys (keymap &rest body)
  (declare (indent defun))
  `(/bindings/define-prefix-keys ,keymap nil ,@body))

(defmacro /bindings/define-key (keymap sequence binding &optional description)
  (declare (indent defun))
  `(/bindings/define-prefix-keys ,keymap nil
     (,sequence ,binding ,description)))
\end{verbatim}


\section*{Environment setup}
\label{sec:org6d3c1bf}

\subsection*{Allow access from emacsclient}
\label{sec:org439b562}

\begin{verbatim}
(require 'server)
(unless (or (daemonp) (server-running-p))
  (server-start))
\end{verbatim}


\subsection*{Prevent emacs to create lockfiles (.\#files\#).}
\label{sec:orgd6d452c}

PS: this also stops preventing editing colisions, so watch out

\begin{verbatim}
(setq create-lockfiles nil)
\end{verbatim}


\subsection*{Always follow symbolic links to edit the 'actual' file it points to}
\label{sec:org741dc10}

\begin{verbatim}
(setq vc-follow-symlinks t)
\end{verbatim}


\subsection*{Enable mouse support in terminal mode}
\label{sec:org91bbf74}

\begin{verbatim}
(when (eq window-system nil)
  (xterm-mouse-mode 1))
\end{verbatim}


\subsection*{Save all tempfiles in \$TMPDIR/emacs\$UID/}
\label{sec:org2f85889}

\begin{verbatim}
(defconst emacs-tmp-dir (expand-file-name (format "emacs%d" (user-uid)) temporary-file-directory))
(setq backup-directory-alist
    `((".*" . ,emacs-tmp-dir)))
(setq auto-save-file-name-transforms
    `((".*" ,emacs-tmp-dir t)))
(setq auto-save-list-file-prefix
    emacs-tmp-dir)
\end{verbatim}


\subsection*{Disable the annoying Emacs bell ring (beep)}
\label{sec:orgefabb13}

\begin{verbatim}
(setq ring-bell-function 'ignore)
\end{verbatim}


\section*{After}
\label{sec:org25c659e}

with-eval-after-load is a function that lets you defer execution of code until after a feature has been loaded.
It is very useful to only load some packages when they’re, and because of that it is extensively used in this setup. 
So of course there is a macro to make it simpler. It can also run code if a package has been installed by using “pkgname-autoloads” or only if multiple packages have been loaded.
This also avoids loading config for packages that haven’t been loaded yet, resulting in void variables of function definitions. 
This was take from milkypostman (along with some other things).

\begin{verbatim}
;; examples
;; after [evil magit] (
  ;; execute after evil and magit have been loaded
;  )

;; macro definiton
(defmacro after (feature &rest body)
  "Executes BODY after FEATURE has been loaded.

FEATURE may be any one of:
    'evil            => (with-eval-after-load 'evil BODY)
    \"evil-autoloads\" => (with-eval-after-load \"evil-autolaods\" BODY)
    [evil cider]     => (with-eval-after-load 'evil
                          (with-eval-after-load 'cider
                            BODY))
"
  (declare (indent 1))
  (cond
   ((vectorp feature)
    (let ((prog (macroexp-progn body)))
      (cl-loop for f across feature
               do
               (progn
                 (setq prog (append `(',f) `(,prog)))
                 (setq prog (append '(with-eval-after-load) prog))))
      prog))
   (t
    `(with-eval-after-load ,feature ,@body))))
\end{verbatim}


\section*{Packages}
\label{sec:orgacf37ed}

\subsection*{Add the folder 'config' to emacs load-path}
\label{sec:org6b04607}
so i can require stuff from there

\begin{verbatim}
(add-to-list 'load-path (expand-file-name "config" user-emacs-directory))
\end{verbatim}



\subsection*{Require needed packages}
\label{sec:org6f09de3}
\begin{verbatim}
;(after 'evil
;  (require 'evil.tau))

  (require 'evil.tau)

;(after 'org
;  (require 'org.tau))

  (require 'org.tau)

\end{verbatim}


\section*{Evil}
\label{sec:org50ee4b6}

\begin{verbatim}
All Evil settings are meant to be isolates in a separate file evil.tau.
\end{verbatim}

\subsection*{Require Evil related packages}
\label{sec:org5375185}

\begin{verbatim}
(require 'evil)
(evil-mode 1)
\end{verbatim}



\subsection*{Don't wait for any other keys after escape is pressed.}
\label{sec:org12b97f4}
\begin{verbatim}
; (setq evil-esc-delay 0)
\end{verbatim}


\subsection*{Make Evil look a bit more like (n) vim  (??)}
\label{sec:org4d08759}
\begin{verbatim}
not sure what all these options do yet
\end{verbatim}

\begin{verbatim}
(setq evil-search-module 'isearch-regexp)
(setq evil-magic 'very-magic)
(setq evil-shift-width (symbol-value 'tab-width))
(setq evil-regexp-search t)
(setq evil-search-wrap t)
;; (setq evil-want-C-i-jump t)
(setq evil-want-C-u-scroll t)
(setq evil-want-fine-undo nil)
(setq evil-want-integration nil)
;; (setq evil-want-abbrev-on-insert-exit nil)
(setq evil-want-abbrev-expand-on-insert-exit nil)
;; move evil tag to beginning of modeline
(setq evil-mode-line-format '(before . mode-line-front-space))
\end{verbatim}




\subsection*{Simulate Vim behaviour and some bindings}
\label{sec:org49d9156}


\subsubsection*{make esc quit or cancel everything in Emacs}
\label{sec:orgefba423}
\begin{verbatim}
(define-key evil-normal-state-map [escape] 'keyboard-quit)
(define-key evil-visual-state-map [escape] 'keyboard-quit)
(define-key minibuffer-local-map [escape] 'minibuffer-keyboard-quit)
(define-key minibuffer-local-ns-map [escape] 'minibuffer-keyboard-quit)
(define-key minibuffer-local-completion-map [escape] 'minibuffer-keyboard-quit)
(define-key minibuffer-local-must-match-map [escape] 'minibuffer-keyboard-quit)
(define-key minibuffer-local-isearch-map [escape] 'minibuffer-keyboard-quit)
\end{verbatim}




\subsection*{Cursor is alway black because of evil.}
\label{sec:org328acd5}

\begin{verbatim}
Here is the workaround
(@see https://bitbucket.org/lyro/evil/issue/342/evil-default-cursor-setting-should-default)
\end{verbatim}
\begin{verbatim}
(setq evil-default-cursor t)
\end{verbatim}



\subsubsection*{recover native emacs commands that are overriden by evil}
\label{sec:orgc9c5024}
\begin{verbatim}
this gives priority to native emacs behaviour rathen than Vim's
\end{verbatim}

\begin{verbatim}
(define-key evil-normal-state-map (kbd "SPC") 'ace-jump-mode)
(define-key evil-insert-state-map (kbd "C-e") 'move-end-of-line)
(define-key evil-insert-state-map (kbd "C-k") 'kill-line)
(define-key evil-normal-state-map (kbd "C-k") 'kill-line)
(define-key evil-insert-state-map (kbd "C-w") 'kill-region)
(define-key evil-normal-state-map (kbd "C-w") 'kill-region)
(define-key evil-visual-state-map (kbd "C-w") 'kill-region)
(define-key evil-visual-state-map (kbd "C-e") 'move-end-of-line)
(define-key evil-normal-state-map (kbd "C-e") 'move-end-of-line)
(define-key evil-normal-state-map (kbd "C-y") 'yank)
(define-key evil-insert-state-map (kbd "C-y") 'yank)
(define-key evil-visual-state-map (kbd "SPC") 'ace-jump-mode)
(define-key evil-normal-state-map "\C-e" 'evil-end-of-line)
(define-key evil-insert-state-map "\C-e" 'end-of-line)
(define-key evil-visual-state-map "\C-e" 'evil-end-of-line)
(define-key evil-motion-state-map "\C-e" 'evil-end-of-line)
(define-key evil-normal-state-map "\C-f" 'evil-forward-char)
(define-key evil-insert-state-map "\C-f" 'evil-forward-char)
(define-key evil-insert-state-map "\C-f" 'evil-forward-char)
(define-key evil-normal-state-map "\C-b" 'evil-backward-char)
(define-key evil-insert-state-map "\C-b" 'evil-backward-char)
(define-key evil-visual-state-map "\C-b" 'evil-backward-char)
(define-key evil-normal-state-map "\C-d" 'evil-delete-char)
(define-key evil-insert-state-map "\C-d" 'evil-delete-char)
(define-key evil-visual-state-map "\C-d" 'evil-delete-char)
(define-key evil-normal-state-map "\C-n" 'evil-next-line)
(define-key evil-insert-state-map "\C-n" 'evil-next-line)
(define-key evil-visual-state-map "\C-n" 'evil-next-line)
(define-key evil-normal-state-map "\C-p" 'evil-previous-line)
(define-key evil-insert-state-map "\C-p" 'evil-previous-line)
(define-key evil-visual-state-map "\C-p" 'evil-previous-line)
(define-key evil-normal-state-map "\C-w" 'evil-delete)
(define-key evil-insert-state-map "\C-w" 'evil-delete)
(define-key evil-visual-state-map "\C-w" 'evil-delete)
(define-key evil-normal-state-map "\C-y" 'yank)
(define-key evil-insert-state-map "\C-y" 'yank)
(define-key evil-visual-state-map "\C-y" 'yank)
(define-key evil-normal-state-map "\C-k" 'kill-line)
(define-key evil-insert-state-map "\C-k" 'kill-line)
(define-key evil-visual-state-map "\C-k" 'kill-line)
(define-key evil-normal-state-map "Q" 'call-last-kbd-macro)
(define-key evil-visual-state-map "Q" 'call-last-kbd-macro)
(define-key evil-insert-state-map "\C-e" 'end-of-line)
(define-key evil-insert-state-map "\C-r" 'search-backward)
\end{verbatim}


\begin{verbatim}
;; (define-key evil-window-map "\C-h" 'evil-window-left)
;; (define-key evil-window-map "\C-j" 'evil-window-down)
;; (define-key evil-window-map "\C-k" 'evil-window-up)
;; (define-key evil-window-map "\C-l" 'evil-window-right)
\end{verbatim}



\subsubsection*{change cursor color according to mode}
\label{sec:org50eed3a}

\begin{verbatim}
(setq evil-emacs-state-cursor '("#ff0000" box))
(setq evil-motion-state-cursor '("#FFFFFF" box))
(setq evil-normal-state-cursor '("#00ff00" box))
(setq evil-visual-state-cursor '("#abcdef" box))
(setq evil-insert-state-cursor '("#e2f00f" bar))
(setq evil-replace-state-cursor '("red" hbar))
(setq evil-operator-state-cursor '("red" hollow))
\end{verbatim}

\subsubsection*{multiple cursors}
\label{sec:orge056684}

\begin{verbatim}
;; step 1, select thing in visual-mode (OPTIONAL)
;; step 2, `mc/mark-all-like-dwim' or `mc/mark-all-like-this-in-defun'
;; step 3, `ace-mc-add-multiple-cursors' to remove cursor, press RET to confirm
;; step 4, press s or S to start replace
;; step 5, press C-g to quit multiple-cursors
(define-key evil-visual-state-map (kbd "mn") 'mc/mark-next-like-this)
(define-key evil-visual-state-map (kbd "ma") 'mc/mark-all-like-this-dwim)
(define-key evil-visual-state-map (kbd "md") 'mc/mark-all-like-this-in-defun)
(define-key evil-visual-state-map (kbd "mm") 'ace-mc-add-multiple-cursors)
(define-key evil-visual-state-map (kbd "ms") 'ace-mc-add-single-cursor)
\end{verbatim}

\subsubsection*{imitate vim multiple selection behavior with multiple-cursors package}
\label{sec:org887cbbe}
\begin{verbatim}
;; (define-key evil-normal-state-map (kbd "C-n") 'mc/mark-next-like-this)
;; (define-key evil-normal-state-map (kbd "M-N") 'mc/mark-previous-like-this)
\end{verbatim}


\subsubsection*{evil-leader}
\label{sec:orgcd98b28}

\begin{verbatim}
(require 'evil-leader)
\end{verbatim}

\begin{verbatim}
(global-evil-leader-mode)
(evil-leader/set-leader ",")
(evil-leader/set-key
  "e" 'find-file
  "q" 'evil-quit
  "w" 'save-buffer
  "k" 'kill-buffer
  "b" 'switch-to-buffer
  "-" 'split-windowellow
  "|" 'split-window-right)
\end{verbatim}

\subsubsection*{Evil Surround}
\label{sec:orgbe1a706}
\begin{verbatim}
@see https://github.com/timcharper/evil-surround for tutorial
\end{verbatim}

\begin{verbatim}
(require 'evil-surround)
(global-evil-surround-mode 1)
\end{verbatim}

\begin{verbatim}
(defun evil-surround-prog-mode-hook-setup ()
  "Documentation string, idk, put something here later."
  (push '(47 . ("/" . "/")) evil-surround-pairs-alist)
  (push '(40 . ("(" . ")")) evil-surround-pairs-alist)
  (push '(41 . ("(" . ")")) evil-surround-pairs-alist)
  (push '(91 . ("[" . "]")) evil-surround-pairs-alist)
  (push '(93 . ("[" . "]")) evil-surround-pairs-alist))
(add-hook 'prog-mode-hook 'evil-surround-prog-mode-hook-setup)
\end{verbatim}

\begin{verbatim}
(defun evil-surround-js-mode-hook-setup ()
  "ES6." ;  <-- this is a documentation string, a feature in Lisp
  ;; I believe this is for auto closing pairs
  (push '(?1 . ("{`" . "`}")) evil-surround-pairs-alist)
  (push '(?2 . ("${" . "}")) evil-surround-pairs-alist)
  (push '(?4 . ("(e) => " . "(e)")) evil-surround-pairs-alist)
  ;; ReactJS
  (push '(?3 . ("classNames(" . ")")) evil-surround-pairs-alist))
(add-hook 'js2-mode-hook 'evil-surround-js-mode-hook-setup)
\end{verbatim}

\begin{verbatim}
(defun evil-surround-emacs-lisp-mode-hook-setup ()
  (push '(?` . ("`" . "'")) evil-surround-pairs-alist))
(add-hook 'emacs-lisp-mode-hook 'evil-surround-emacs-lisp-mode-hook-setup)
(defun evil-surround-org-mode-hook-setup ()
  (push '(91 . ("[" . "]")) evil-surround-pairs-alist)
  (push '(93 . ("[" . "]")) evil-surround-pairs-alist)
  (push '(?= . ("=" . "=")) evil-surround-pairs-alist))
(add-hook 'org-mode-hook 'evil-surround-org-mode-hook-setup)
\end{verbatim}




\subsection*{Neotree}
\label{sec:orgafd6452}

\begin{verbatim}
(require 'neotree)
\end{verbatim}


\subsubsection*{set NeoTree default window width}
\label{sec:org7c6f4df}
\begin{verbatim}
(setq neo-window-width 30)
\end{verbatim}

\subsubsection*{toggle neotree with F8}
\label{sec:org827fbdb}
\begin{verbatim}
(global-set-key [f8] 'neotree-toggle)
\end{verbatim}


\subsubsection*{make nerdtree open on emacs startup}
\label{sec:org8a18b1d}
\begin{verbatim}
(add-hook 'after-init-hook #'neotree-toggle)
\end{verbatim}


\subsubsection*{neotree 'icons' theme, which supports filetype icons}
\label{sec:org81e8f34}
\begin{verbatim}
(unless (display-graphic-p)
  (setq neo-theme 'icons))
(setq neo-theme (if (display-graphic-p) 'icons 'arrow))
\end{verbatim}


\subsubsection*{make neotree window open and go the file currently opened}
\label{sec:org7149eb4}
\begin{verbatim}
(setq neo-smart-open t)
\end{verbatim}


\subsubsection*{solve keybinding conflicts between neotree with evil mode}
\label{sec:org89b8858}
\begin{verbatim}
(add-hook 'neotree-mode-hook
          (lambda ()
            ; default Neotree bindings
            (define-key evil-normal-state-local-map (kbd "TAB") 'neotree-enter)
            (define-key evil-normal-state-local-map (kbd "SPC") 'neotree-quick-look)
            (define-key evil-normal-state-local-map (kbd "q") 'neotree-hide)
            (define-key evil-normal-state-local-map (kbd "RET") 'neotree-enter)
            (define-key evil-normal-state-local-map (kbd "g") 'neotree-refresh)
            (define-key evil-normal-state-local-map (kbd "n") 'neotree-next-line)
            (define-key evil-normal-state-local-map (kbd "p") 'neotree-previous-line)
            (define-key evil-normal-state-local-map (kbd "A") 'neotree-stretch-toggle)
            (define-key evil-normal-state-local-map (kbd "H") 'neotree-hidden-file-toggle)
            (define-key evil-normal-state-local-map (kbd "|") 'neotree-enter-vertical-split)
            (define-key evil-normal-state-local-map (kbd "-") 'neotree-enter-horizontal-split)
            ; simulating NERDTree bindings in Neotree
            (define-key evil-normal-state-local-map (kbd "R") 'neotree-refresh)
            (define-key evil-normal-state-local-map (kbd "u") 'neotree-refresh)
            (define-key evil-normal-state-local-map (kbd "C") 'neotree-change-root)
            (define-key evil-normal-state-local-map (kbd "c") 'neotree-create-node)))
\end{verbatim}



\subsection*{Vim plugins definitions}
\label{sec:orgedcd99d}

\subsubsection*{To enable evil-commentary permanently, add}
\label{sec:org1236532}
\begin{verbatim}

(evil-commentary-mode)
\end{verbatim}

\subsubsection*{Vim Commentary}
\label{sec:org0695c9b}
\begin{verbatim}
(require 'evil-commentary)
(evil-commentary-mode)
\end{verbatim}

\subsubsection*{Evil-Matchit}
\label{sec:org63943d3}
\begin{verbatim}
(require 'evil-matchit)
(global-evil-matchit-mode 1)
\end{verbatim}




\section*{ORG mode}
\label{sec:org79b968d}

The ORG part of the config compiles to a separate file, inside the config folder, called `org.el`

\subsection*{Require ORG}
\label{sec:org7b9e053}

\begin{verbatim}
(require 'org)
\end{verbatim}

\subsection*{Resolve issue with Tab not working with ORG only in Normal VI Mode in terminal}
\label{sec:org4343aed}

(something with TAB on terminals being related to C-i\ldots{})

\begin{verbatim}
(add-hook 'org-mode-hook                                                                      
          (lambda ()                                                                          
        (define-key evil-normal-state-map (kbd "TAB") 'org-cycle))) 

;; (setq evil-want-C-i-jump nil)
\end{verbatim}




\subsection*{Function to activate export-on-save in org mode}
\label{sec:orge44aa36}

\begin{verbatim}
(defun toggle-org-html-export-on-save ()
  "Make Emacs auto-export to HTML when org file is saved.
Enable calling this function from the file with <M-x>."
  (interactive)
  (if (memq 'org-html-export-to-html after-save-hook)
      (progn
        (remove-hook 'after-save-hook 'org-html-export-to-html t)
        (message "Disabled org html export on save for current buffer..."))
    (add-hook 'after-save-hook 'org-html-export-to-html nil t)
    (message "Enabled org html export on save for current buffer...")))
\end{verbatim}


\subsection*{Add hook to auto-export automatically on saveing ORG files}
\label{sec:orgbcf3fcb}

\begin{verbatim}
(defun org-mode-export-hook ()
  "This exports to diffenent outputs everytime the file is saved.
This will be added to org-mode-hook, so it only activates on ORG files.
Generates outputs in these formats:
- PDF
- HTML
- RevealJS."
   (add-hook 'after-save-hook 'org-beamer-export-to-pdf t t)
   (add-hook 'after-save-hook 'org-reveal-export-to-html t t))

; Finally adds the above hook in org-mode-hook.
;; (add-hook 'org-mode-hook #'org-mode-export-hook)
\end{verbatim}


\subsection*{Evil-ORG}
\label{sec:org68b5d16}


\begin{verbatim}
(after 'org
  (require 'evil-org)
  (require 'evil-org-agenda)
  (add-hook 'org-mode-hook #'evil-org-mode)
  (add-hook 'evil-org-mode-hook
            (lambda ()
              (evil-org-set-key-theme))))
\end{verbatim}

\begin{verbatim}
;; (add-hook 'org-mode-hook 'evil-org-mode)
;; (evil-org-set-key-theme '(navigation insert textobjects additional calendar))
;; (evil-org-agenda-set-keys)
\end{verbatim}



\subsubsection*{Simulate <leader> key with Spacebar}
\label{sec:org03de016}

\begin{verbatim}
 (defvar my-leader-map (make-sparse-keymap)
   "Keymap for \"leader key\" shortcuts.")

 ;; binding "SPC" to the keymap
(define-key evil-normal-state-map (kbd "M-SPC") my-leader-map)

 ;; binding using SPC leader
 (define-key my-leader-map "b" 'list-buffers)
 (define-key my-leader-map "w" 'evil-save)
 (define-key my-leader-map "SPC" ":noh")
\end{verbatim}



\subsection*{ox-pandoc}
\label{sec:org74fff91}

As pandoc supports many number of formats, initial org-export-dispatch
shortcut menu does not show full of its supported formats. You can customize
org-pandoc-menu-entry variable (and probably restart Emacs) to change its
default menu entries.
If you want delayed loading of `ox-pandoc’ when org-pandoc-menu-entry
is customized, please consider the following settings in your init file``

\begin{verbatim}
(with-eval-after-load 'ox
  (require 'ox-pandoc))
\end{verbatim}

\begin{verbatim}
(require 'ox-pandoc)
\end{verbatim}

\begin{verbatim}
;; default options for all output formats
(setq org-pandoc-options '((standalone . t)))
;; cancel above settings only for 'docx' format
(setq org-pandoc-options-for-docx '((standalone . nil)))
;; special settings for beamer-pdf and latex-pdf exporters
(setq org-pandoc-options-for-beamer-pdf '((pdf-engine . "xelatex")))
(setq org-pandoc-options-for-latex-pdf '((pdf-engine . "luatex")))
;; special extensions for markdown_github output
(setq org-pandoc-format-extensions '(markdown_github+pipe_tables+raw_html))
\end{verbatim}


\subsection*{ReveaJS org-reveal:}
\label{sec:orgc09a28e}
\begin{verbatim}
This delay makes the options to export to RevealJS appear on the exporter menu (C-c C-e)
\end{verbatim}

\begin{verbatim}
(with-eval-after-load 'ox
  (require 'ox-reveal))
\end{verbatim}

\begin{verbatim}
(require 'ox-reveal)
\end{verbatim}



\section*{Eshell}
\label{sec:org6ff1514}

\subsection*{Make eshell open in a split-window buffer at the bottom of the screen}
\label{sec:org39da531}

\begin{verbatim}
(defun /eshell/new-window ()
    "Opens up a new shell in the directory associated with the current buffer's file.  The eshell is renamed to match that directory to make multiple eshell windows easier."
    (interactive)
    (let* ((parent (if (buffer-file-name)
                       (file-name-directory (buffer-file-name))
                     default-directory))
           (height (/ (window-total-height) 3))
           (name   (car (last (split-string parent "/" t)))))
      (split-window-vertically (- height))
      (other-window 1)
      (eshell "new")
      (rename-buffer (concat "*eshell: " name "*"))

      (insert (concat "ls"))
      (eshell-send-input)))

; Pull eshell in a new bottom window
(define-key evil-normal-state-map (kbd "!") #'/eshell/new-window)
(define-key evil-visual-state-map (kbd "!") #'/eshell/new-window)
(define-key evil-motion-state-map (kbd "!") #'/eshell/new-window)
\end{verbatim}


\section*{Helm}
\label{sec:org8e35df6}

\begin{verbatim}
(require 'helm)

(setq helm-bookmark-show-location t)
(setq helm-buffer-max-length 40)
(setq helm-split-window-inside-p t)
(setq helm-mode-fuzzy-match t)
(setq helm-ff-file-name-history-use-recentf t)
(setq helm-ff-skip-boring-files t)
(setq helm-follow-mode-persistent t)

(after 'helm-source
  (defun /helm/make-source (f &rest args)
    (let ((source-type (cadr args))
          (props (cddr args)))
      (unless (child-of-class-p source-type 'helm-source-async)
        (plist-put props :fuzzy-match t))
      (apply f args)))
  (advice-add 'helm-make-source :around '/helm/make-source))
\end{verbatim}


\subsection*{Other helm settings}
\label{sec:orgabb377f}

\begin{verbatim}
(after 'helm
  ;; take between 10-30% of screen space
  (setq helm-autoresize-min-height 10)
  (setq helm-autoresize-max-height 30)
  (helm-autoresize-mode t))
\end{verbatim}

Make helm replace the default Find-File and M-x

\begin{verbatim}
(progn
(global-set-key [remap execute-extended-command] #'helm-M-x)
(global-set-key [remap find-file] #'helm-find-files)
(helm-mode t))
\end{verbatim}

\subsection*{Helm related bindings}
\label{sec:orgd442af2}

\begin{verbatim}
(after 'helm
  (require 'helm-config)
  (global-set-key (kbd "C-c h") #'helm-command-prefix)
  (global-unset-key (kbd "C-x c"))
  (global-set-key (kbd "C-h a") #'helm-apropos)
  (global-set-key (kbd "C-x b") #'helm-buffers-list)
  (global-set-key (kbd "C-x C-b") #'helm-mini)
  (global-set-key (kbd "C-x C-f") #'helm-find-files)
  (global-set-key (kbd "C-x r b") #'helm-bookmarks)
  (global-set-key (kbd "M-x") #'helm-M-x)
  (global-set-key (kbd "M-y") #'helm-show-kill-ring)
  (global-set-key (kbd "M-:") #'helm-eval-expression-with-eldoc)
  (define-key helm-map (kbd "<tab>") #'helm-execute-persistent-action)
  (define-key helm-map (kbd "C-z") #'helm-select-action)
)
\end{verbatim}


\section*{Dired}
\label{sec:org5ba160c}

\begin{verbatim}
(after 'dired
  (require 'dired-k)
  (setq dired-k-style 'git)
  (setq dired-k-human-readable t)
  (add-hook 'dired-initial-position-hook #'dired-k))
\end{verbatim}


\section*{Magit}
\label{sec:org04a2d6c}

\subsection*{Load evil-magit with magit buffer}
\label{sec:org7e2f276}

\begin{verbatim}
(after 'magit
  (require 'evil-magit)
  (evil-magit-init))
\end{verbatim}


\subsection*{define global keybing to magit-status}
\label{sec:orge20856e}

\begin{verbatim}
(global-set-key (kbd "C-x g") 'magit-status)
\end{verbatim}


\section*{which-key}
\label{sec:org6a1a572}

\begin{verbatim}
(require 'which-key)
(setq which-key-idle-delay 0.2)
(setq which-key-min-display-lines 3)
(setq which-key-max-description-length 20)
(setq which-key-max-display-columns 6)
(which-key-mode)
\end{verbatim}


\section*{General editor configuration}
\label{sec:org549a0da}


\subsection*{Show line numbers}
\label{sec:org7b7843d}
\begin{verbatim}
(when (version<= "26.0.50" emacs-version )
  (global-display-line-numbers-mode))
\end{verbatim}



\subsection*{Line Number : Pretty format}
\label{sec:org9edaea9}
\begin{verbatim}
(setq linum-format " %d ")
\end{verbatim}



\subsection*{Use the system clipboard}
\label{sec:org33e2ee2}
\begin{verbatim}
(setq x-select-enable-clipboard t)
\end{verbatim}




\subsection*{Window navigation with vim-like bindings}
\label{sec:org2a8db27}

\begin{verbatim}
;; for some readon the bellow lines should be the default native way for navigation on emacs
;; but they dont work
;; using the above package instead til i find a solution
;
;; (windmove-default-keybindings 'control)
;; (global-set-key (kbd "C-h") 'windmove-left)
;; (global-set-key (kbd "C-l") 'windmove-right)
;; (global-set-key (kbd "C-k") 'windmove-up)
;; (global-set-key (kbd "C-j") 'windmove-down)
\end{verbatim}


Bellow i use the `define-keys` function to map window navigation to default Vim bindings <C-hjkl>

First require the file with the function

\begin{verbatim}
;; (require 'evil-tmux-navigator)
\end{verbatim}


Then create the keybindings 
\begin{verbatim}
(define-prefix-command 'evil-window-map)
(define-key evil-window-map "h" 'evil-window-left)
(define-key evil-window-map "j" 'evil-window-down)
(define-key evil-window-map "k" 'evil-window-up)
(define-key evil-window-map "l" 'evil-window-right)
(define-key evil-window-map "b" 'evil-window-bottom-right)
(define-key evil-window-map "c" 'evil-window-delete)
(define-key evil-motion-state-map "\M-w" 'evil-window-map)
\end{verbatim}


\begin{verbatim}
;; (/bindings/define-keys evil-normal-state-map
;;   ("C-w h" #'evil-window-left)
;;   ("C-w j" #'evil-window-down)
;;   ("C-w k" #'evil-window-up)
;;   ("C-w l" #'evil-window-right))
\end{verbatim}


\begin{verbatim}
;; (/bindings/define-keys evil-normal-state-map
;;   ("C-w h" #'evil-window-left)
;;   ("C-w j" #'evil-window-down)
;;   ("C-w k" #'evil-window-up)
;;   ("C-w l" #'evil-window-right))
\end{verbatim}


\subsection*{Increase, decrease and adjust font size}
\label{sec:org9d14cae}

\begin{verbatim}
(global-set-key (kbd "C-+") #'text-scale-increase)
(global-set-key (kbd "C-_") #'text-scale-decrease)
;; (global-set-key (kbd "C-)") #'text-scale-adjust)
\end{verbatim}


\section*{General text editing settings}
\label{sec:org8bb745d}

\subsection*{Turn on auto-revert mode (auto updates files changed on disk)}
\label{sec:org797c0e9}
\begin{verbatim}
(global-auto-revert-mode 1)
(setq auto-revert-interval 0.5)
\end{verbatim}


\subsection*{Spellchecking}
\label{sec:org439356f}
\begin{verbatim}
(defconst *spell-check-support-enabled* t) ;; Enable with t if you prefer
\end{verbatim}

I recommend adding this to your .emacs, as it makes C-n insert newlines if the point is at the end of the buffer. Useful, as it means you won’t have to reach for the return key to add newlines!
\begin{verbatim}
(setq next-line-add-newlines t)
\end{verbatim}


\subsection*{Smartscan mode}
\label{sec:org9b16b1c}
\begin{verbatim}
Usage:
M-n and M-p move between symbols
M-' to replace all symbols in the buffer matching the one under point
C-u M-' to replace symbols in your current defun only (as used by narrow-to-defun.)
\end{verbatim}

\begin{verbatim}
(smartscan-mode 1)
\end{verbatim}


\subsection*{PDF Tools}
\label{sec:org25e3c6f}

\subsubsection*{Install pdf-tools if its not already installed}
\label{sec:org187d80c}
\begin{verbatim}
(pdf-tools-install)
\end{verbatim}


\subsubsection*{Make buffer refresh every 1 second to PDF-tools updates the changed pdf}
\label{sec:org9ede917}
\begin{verbatim}
(add-hook 'TeX-after-compilation-finished-functions #'TeX-revert-document-buffer)
;; (add-hook 'pdf-view-mode-hook 'auto-revert-mode) 
;; (add-hook 'doc-view-mode-hook 'auto-revert-mode) 
\end{verbatim}


\subsubsection*{PDF tools evil keybindings}
\label{sec:orgcc980f2}
\begin{verbatim}
(evil-define-key 'normal pdf-view-mode-map
  "h" 'pdf-view-previous-page-command
  "j" (lambda () (interactive) (pdf-view-next-line-or-next-page 5))
  "k" (lambda () (interactive) (pdf-view-previous-line-or-previous-page 5))
  "l" 'pdf-view-next-page-command)
\end{verbatim}



\section*{Development environment customizations}
\label{sec:orga3f1164}



\subsection*{show mathing parenthesis}
\label{sec:orga451ce6}
\begin{verbatim}
; parentheses
(show-paren-mode t)
\end{verbatim}



\subsection*{indentation}
\label{sec:org61b589e}
\begin{verbatim}
(setq-default indent-tabs-mode nil)
(setq-default c-basic-offset 2)
\end{verbatim}



\subsection*{enable rainbow-mode on relevant filetypes}
\label{sec:orgcbf0a14}

Colorize hex, rgb and named color codes

\begin{verbatim}
(add-hook 'org-mode-hook 'rainbow-mode)
(add-hook 'css-mode-hook 'rainbow-mode)
(add-hook 'php-mode-hook 'rainbow-mode)
(add-hook 'html-mode-hook 'rainbow-mode)
(add-hook 'web-mode-hook 'rainbow-mode)
(add-hook 'js2-mode-hook 'rainbow-mode)
\end{verbatim}



\section*{Appearance}
\label{sec:org74a3297}


\subsection*{Applying my theme}
\label{sec:orgd6940d3}

\begin{verbatim}

(add-to-list 'custom-theme-load-path "~/dotfiles/emacs.d/themes/")
; theme options:
; atom-one-dark (doenst work well with emacsclient, ugly blue bg)
; dracula
; darktooth
; gruvbox-dark-hard
; gruvbox-dark-light
; gruvbox-dark-medium
; base16-default-dark <-- this one is good

(setq my-theme 'dracula)

\end{verbatim}

Load the theme

\begin{verbatim}
(load-theme my-theme t)
\end{verbatim}


\begin{verbatim}

;; (defun load-my-theme (frame)
;;   "Function to load the theme in current FRAME.
;;   sed in conjunction
;;   with bellow snippet to load theme after the frame is loaded
;;   to avoid terminal breaking theme."
;;   (select-frame frame)
;;   (load-theme my-theme t))

;; ; make emacs load the theme after loading the frame
;; ; resolves issue with the theme not loading properly in terminal mode on emacsclient

;; ;; this if was breaking my emacs!!!!!
;;  (add-hook 'after-make-frame-functions #'load-my-theme)
\end{verbatim}



\subsection*{Customizing the mode line}
\label{sec:orga0af57a}

\begin{verbatim}
(require 'smart-mode-line)
(if (require 'smart-mode-line nil 'noerror)
    (progn
      ;( sml/name-width 20)
      ;( sml/mode-width 'full)
      ;( sml/shorten-directory t)
      ;( sml/shorten-modes t)
      (require 'smart-mode-line-powerline-theme)
      ; this must be BEFORE (sml/setup)
      (sml/apply-theme 'powerline)
      ;; Alternatives:
      ;; (sml/apply-theme 'powerline)
      ;; (sml/apply-theme 'dark)
      ;; (sml/apply-theme 'light)
      ;; (sml/apply-theme 'respectful)
      ;; (sml/apply-theme 'automatic)


      (if after-init-time
          (sml/setup)
        (add-hook 'after-init-hook 'sml/setup))


      (display-time-mode 1)

      (add-to-list 'sml/replacer-regexp-list '("^~/Dropbox/" ":DB:"))
      (add-to-list 'sml/replacer-regexp-list
                   '("^~/.*/lib/ruby/gems" ":GEMS" ))
      (add-to-list 'sml/replacer-regexp-list
                   '("^~/Projects/" ":CODE:"))))
\end{verbatim}


\section*{Minor modes}
\label{sec:org015766b}


\subsection*{js2-refactor}
\label{sec:org3468425}

\begin{verbatim}
(add-hook 'js2-mode-hook #'js2-refactor-mode)
\end{verbatim}


\subsubsection*{choose js2-refactor keybinding scheme (this can be changed easily)}
\label{sec:org8321b3f}

\begin{verbatim}
(js2r-add-keybindings-with-prefix "C-c C-m")
\end{verbatim}


\subsubsection*{add prettier to js2, web and rjsx minor modes}
\label{sec:org912a2b1}
\begin{verbatim}
(require 'prettier-js)
(add-hook 'js2-mode-hook 'prettier-js-mode)
(add-hook 'web-mode-hook 'prettier-js-mode)
;; (add-hook 'rjsx-mode-hook 'prettier-js-mode)
\end{verbatim}




\subsection*{web-mode}
\label{sec:orgb04a6ab}

\begin{verbatim}
(require 'web-mode)
\end{verbatim}


\section*{File associations}
\label{sec:orgb0a5684}

\begin{verbatim}
(add-to-list 'auto-mode-alist '("\\.org$" . org-mode))
(add-to-list 'auto-mode-alist '("\\.html?\\'" . web-mode))
(add-to-list 'auto-mode-alist '("\\.js$" . js2-mode))
\end{verbatim}


\subsection*{template engines filetypes}
\label{sec:org5d45a18}

\begin{verbatim}
(add-to-list 'auto-mode-alist '("\\.phtml\\'" . web-mode))
(add-to-list 'auto-mode-alist '("\\.tpl\\.php\\'" . web-mode))
(add-to-list 'auto-mode-alist '("\\.[agj]sp\\'" . web-mode))
(add-to-list 'auto-mode-alist '("\\.as[cp]x\\'" . web-mode))
(add-to-list 'auto-mode-alist '("\\.erb\\'" . web-mode))
(add-to-list 'auto-mode-alist '("\\.mustache\\'" . web-mode))
(add-to-list 'auto-mode-alist '("\\.djhtml\\'" . web-mode))
\end{verbatim}



\subsection*{use rjsx mode for js files in React folder structure}
\label{sec:org8ef7ffe}

better support for JSX and React and GatsbyJs

\begin{verbatim}
(add-to-list 'auto-mode-alist '("components\\/.*\\.js\\'" . rjsx-mode))
(add-to-list 'auto-mode-alist '("pages\\/.*\\.js\\'" . rjsx-mode))
\end{verbatim}




\subsection*{FlyCheck linter}
\label{sec:org14ef9fe}

\begin{verbatim}
(add-hook 'after-init-hook #'global-flycheck-mode)
\end{verbatim}



\section*{Autocompletion and Snippets}
\label{sec:orgded9850}

\subsection*{enable autocompletion engine}
\label{sec:org3f4510e}

\begin{verbatim}
(require 'auto-complete)
(global-auto-complete-mode t)
\end{verbatim}


\subsection*{Company mode (Complete Anything)}
\label{sec:orge7eb444}


\subsubsection*{Basic settings for company-mode}
\label{sec:org8a834ea}
\begin{verbatim}
(require 'company)
(global-company-mode t)
(setq company-tooltip-limit 20)                      ; bigger popup window
(setq company-minimum-prefix-length 1)               ; start completing after 1st char typed
(setq company-idle-delay .1)                         ; decrease delay before autocompletion popup shows
(setq company-echo-delay 0)                          ; remove annoying blinking
(setq company-begin-commands '(self-insert-command)) ; start autocompletion only after typing
(setq company-dabbrev-downcase nil)                  ; Do not convert to lowercase
(setq company-dabbrev-ignore-case t)
(setq company-dabbrev-code-everywhere t)
(setq company-selection-wrap-around t)               ; continue from top when reaching bottom
(setq company-auto-complete 'company-explicit-action-p)
\end{verbatim}


\subsubsection*{Enable company-mode in all buffers}
\label{sec:orgdeb3ec7}
\begin{verbatim}
(add-hook 'after-init-hook 'global-company-mode)
\end{verbatim}
(add

\subsubsection*{Bind <TAB> to company-indent-or-complete}
\label{sec:orgca01115}
\begin{verbatim}
(add-hook 'after-init-hook 'global-company-mode)

; (after "company-autoloads"
;    (define-key evil-insert-state-map (kbd "TAB")
;      #'company-indent-or-complete-common))
\end{verbatim}


\subsection*{Yasnippets}
\label{sec:org6be2ddd}

\begin{verbatim}
;; (add-to-list 'load-path
;;               "~/.emacs.d/plugins/yasnippet")
(require 'yasnippet)
(yas-global-mode 1)
\end{verbatim}

\begin{verbatim}
(setq yas-snippet-dirs
      '("~/.emacs.d/snippets"                 ;; personal snippets
        ))
\end{verbatim}


\section*{Languages specific settings}
\label{sec:orge6631cb}

\subsection*{HTML}
\label{sec:orga73352f}

\subsubsection*{Emmet}
\label{sec:org62db3f0}


Add hook to any markup file to load emmet-mode
\begin{verbatim}
(add-hook 'sgml-mode-hook 'emmet-mode) ;; Auto-start on any markup modes
(add-hook 'css-mode-hook  'emmet-mode) ;; enable Emmet's css abbreviation. 
\end{verbatim}

Use emmet with JSX markup

\begin{verbatim}
(setq emmet-expand-jsx-className? t) ;; default nil
\end{verbatim}

\subsection*{Ruby mode}
\label{sec:org6b223cc}
\begin{verbatim}
;(require 'ruby.tau)
;(add-to-list 'auto-mode-alist '("\\(?:\\.rb\\|ru\\|rake\\|thor\\|jbuilder\\|gemspec\\|podspec\\|/\\(?:Gem\\|Rake\\|Cap\\|Thor\\|Vagrant\\|Guard\\|Pod\\)file\\)\\'" . enh-ruby-mode))
\end{verbatim}


\subsection*{PHP mode}
\label{sec:org472639a}
\begin{verbatim}
(autoload 'php-mode "php-mode" "Major mode for editing PHP code." t)
(add-to-list 'auto-mode-alist '("\\.php$" . php-mode))
(add-to-list 'auto-mode-alist '("\\.inc$" . php-mode))
\end{verbatim}


\subsection*{Go mode}
\label{sec:org476d16e}
\begin{verbatim}
; (autoload 'go-mode "go-mode" "Major mode for editing Go code." t)
; (add-to-list 'auto-mode-alist '("\\.go\\'" . go-mode))
\end{verbatim}


\subsection*{markdown-mode}
\label{sec:org3cc5bf1}
\begin{verbatim}
; (autoload 'mardown-mode "markdown-mode")
; (add-to-list 'auto-mode-alist '("\\.md\\'" . markdown-mode))
\end{verbatim}


\subsection*{haskell-mode}
\label{sec:orgb41d897}
\begin{verbatim}
; ; (require 'haskell-interactive-mode)
; (add-hook 'haskell-mode-hook 'turn-on-haskell-indent)
; (eval-after-load 'flycheck
;                  '(add-hook 'flycheck-mode-hook #'flycheck-haskell-setup))
; (add-hook 'haskell-mode-hook (lambda ()
;                                (electric-indent-mode -1)))
; (add-hook 'haskell-mode-hook 'interactive-haskell-mode)
; (add-hook 'haskell-mode-hook (lambda () (global-set-key (kbd "<f5>") 'haskell-process-cabal-build)))
\end{verbatim}




\section*{\LaTeX{}}
\label{sec:org27321ec}

\begin{verbatim}
(require 'ox-latex)
\end{verbatim}

\subsection*{AucTex settings}
\label{sec:org677565c}

\begin{verbatim}
(require 'tex)
\end{verbatim}

Three steps are required (as according to ORG official docs) to setup AucTex with Emacs:

\subsubsection*{1) Tell emacs where the \LaTeX{} related bins are located in the system}
\label{sec:orgd127ecf}

\begin{verbatim}
(setq exec-path (append exec-path '("/usr/bin/tex")))
\end{verbatim}

\subsubsection*{2) Load AucTex}
\label{sec:orgee19d69}

\begin{verbatim}
;; (load "auctex.el" nil t t)
;; (load "preview-latex.el" nil t t)
\end{verbatim}

\subsubsection*{3) Add Latex to list of org-babel loaded languages}
\label{sec:org1eb3355}

\#+END\(_{\text{SRC}}\)
\begin{verbatim}
(org-babel-do-load-languages
 'org-babel-load-languages
 '((latex . t)))
\end{verbatim}

\begin{verbatim}
(setq TeX-auto-save t)
(setq TeX-parse-self t)
(setq-default TeX-master nil)
\end{verbatim}


\begin{verbatim}
(add-hook 'LaTeX-mode-hook 'visual-line-mode)
(add-hook 'LaTeX-mode-hook 'flyspell-mode)
(add-hook 'LaTeX-mode-hook 'LaTeX-math-mode)   
\end{verbatim}




\subsection*{Latex Classes}
\label{sec:org201d2dc}
\subsubsection*{Add the beamer presentation class template to org}
\label{sec:org745931e}
\begin{verbatim}
(add-to-list 'org-latex-classes
             '("beamer"
               "\\documentclass\[presentation\]\{beamer\}"
               ("\\section\{%s\}" . "\\section*\{%s\}")
               ("\\subsection\{%s\}" . "\\subsection*\{%s\}")
               ("\\subsubsection\{%s\}" . "\\subsubsection*\{%s\}")))
\end{verbatim}


\subsubsection*{Add the memoir class template to org}
\label{sec:org162c80f}
\begin{center}
\begin{tabular}{lrl}
Division & <c>Level & <c>org-equivalent\\
\book & -2 & \\
\part & -1 & \\
\chapter & 0 & *\\
\section & 1 & **\\
\subsection & 2 & \textbf{*}\\
\subsubsection & 3 & \textbf{**}\\
\paragraph & 4 & \textbf{\textbf{*}}\\
\subparagraph & 5 & \textbf{\textbf{**}}\\
\end{tabular}
\end{center}


\begin{verbatim}
(add-to-list 'org-latex-classes
             '("memoir"
               "\\documentclass\[a4paper\]\{memoir\}"
               ("\\book\{%s\}" . "\\book*\{%s\}")
               ("\\part\{%s\}" . "\\part*\{%s\}")
               ("\\chapter\{%s\}" . "\\chapter*\{%s\}")
               ("\\section\{%s\}" . "\\section*\{%s\}")
               ("\\subsection\{%s\}" . "\\subsection*\{%s\}")
               ("\\subsubsection\{%s\}" . "\\subsubsection*\{%s\}")))
\end{verbatim}

\subsubsection*{Add abntex2 class to org list of latex classes}
\label{sec:org92bec6a}
This class is based on the Memoir class
\begin{verbatim}
(add-to-list 'org-latex-classes
          '("abntex2"
            "\\documentclass{abntex2}"
            ("\\part{%s}" . "\\part*{%s}")
            ("\\chapter{%s}" . "\\chapter*{%s}")
            ("\\section{%s}" . "\\section*{%s}")
            ("\\subsection{%s}" . "\\subsection*{%s}")
            ("\\subsubsection{%s}" . "\\subsubsection*{%s}")
            ("\\subsubsubsection{%s}" . "\\subsubsubsection*{%s}")
            ("\\paragraph{%s}" . "\\paragraph*{%s}"))
          ) 
\end{verbatim}


\subsection*{Enable latex-preview-pane}
\label{sec:orge99aa1f}
\begin{verbatim}
(latex-preview-pane-enable)
\end{verbatim}

\subsection*{To compile documents to PDF by default add the following to your \textasciitilde{}/.emacs.}
\label{sec:orgea98f2d}

\begin{verbatim}
(setq TeX-PDF-mode t)
\end{verbatim}

\subsubsection*{If it doesn’t work, try this :}
\label{sec:org7d91a64}

\begin{verbatim}
(TeX-global-PDF-mode t)
\end{verbatim}


\subsection*{To highlight (or font-lock) the “\section{title}” lines:}
\label{sec:org259dbe5}

\begin{verbatim}
(font-lock-add-keywords
   'latex-mode
   `((,(concat "^\\s-*\\\\\\("
               "\\(documentclass\\|\\(sub\\)?section[*]?\\)"
               "\\(\\[[^]% \t\n]*\\]\\)?{[-[:alnum:]_ ]+"
               "\\|"
               "\\(begin\\|end\\){document"
               "\\)}.*\n?")
      (0 'your-face append))))
\end{verbatim}


\subsection*{Convert quotes to \LaTeX{} Smartquotes}
\label{sec:orgef11d49}

\begin{verbatim}
(setq org-export-with-smart-quotes t)
\end{verbatim}

\subsection*{Keep latex logfiles}
\label{sec:orga37c1a2}
\begin{verbatim}
(setq org-latex-remove-logfiles nil)
\end{verbatim}

\section*{Helper functions}
\label{sec:orgf94048d}


\subsection*{Copy/Paste To/From System's Clipboard =D}
\label{sec:org0fdaad8}


\subsubsection*{Copy}
\label{sec:orgc7c54dc}

\begin{verbatim}
(defun copy-to-clipboard ()
  "Make F8 and F9 Copy and Paste to/from OS Clipboard.  Super usefull."
  (interactive)
  (if (display-graphic-p)
      (progn
        (message "Yanked region to x-clipboard!")
        (call-interactively 'clipboard-kill-ring-save)
        )
    (if (region-active-p)
        (progn
          (shell-command-on-region (region-beginning) (region-end) "xsel -i -b")
          (message "Yanked region to clipboard!")
          (deactivate-mark))
      (message "No region active; can't yank to clipboard!")))
  )
\end{verbatim}


\subsubsection*{Paste}
\label{sec:org38ff27b}

\begin{verbatim}
(evil-define-command paste-from-clipboard()
  (if (display-graphic-p)
      (progn
        (clipboard-yank)
        (message "graphics active")
        )
    (insert (shell-command-to-string "xsel -o -b")) ) )
\end{verbatim}

\begin{verbatim}
(global-set-key [f9] 'copy-to-clipboard)
(global-set-key [f10] 'paste-from-clipboard)
\end{verbatim}


\subsection*{Auto save function}
\label{sec:org7500c23}

\begin{verbatim}
(defun my-save ()
  "Save file when leaving insert mode in Evil."
  (if (buffer-file-name)
      (evil-save)))
\end{verbatim}

\begin{verbatim}
(add-hook 'evil-insert-state-exit-hook 'my-save)
\end{verbatim}



\begin{verbatim}
;; Local Variables:
;; coding: utf-8
;; no-byte-compile: t
;; End:


(provide 'init)
;;; .emacs ends here

\end{verbatim}





\section*{Provide packages in separate files}
\label{sec:orgabf3fae}

\subsection*{Provide the evil.tau.el file}
\label{sec:orga6db98c}
\begin{verbatim}
(provide 'evil.tau)
;;; evil.tau.el ends here...
\end{verbatim}

\subsection*{Provide the org.tau.el file}
\label{sec:org0673965}

\begin{verbatim}
(provide 'org.tau)
;;; org.tau.el ends here...
\end{verbatim}
\end{document}