% Created 2019-11-01 sex 10:22
% Intended LaTeX compiler: pdflatex
\documentclass[11pt]{article}
\usepackage[utf8]{inputenc}
\usepackage[T1]{fontenc}
\usepackage{graphicx}
\usepackage{grffile}
\usepackage{longtable}
\usepackage{wrapfig}
\usepackage{rotating}
\usepackage[normalem]{ulem}
\usepackage{amsmath}
\usepackage{textcomp}
\usepackage{amssymb}
\usepackage{capt-of}
\usepackage{hyperref}
\author{Gustavo P Borges}
\date{\today}
\title{gugutz emacs config\\\medskip
\large ORGfied configuration for Emacs}
\hypersetup{
 pdfauthor={Gustavo P Borges},
 pdftitle={gugutz emacs config},
 pdfkeywords={},
 pdfsubject={This file is compiled to init.el automatically on every save},
 pdfcreator={Emacs 26.3 (Org mode 9.2.6)},
 pdflang={English}}
\begin{document}

\maketitle
\setcounter{tocdepth}{0}
\tableofcontents


\section*{Personal information}
\label{sec:org76ecea5}

\begin{verbatim}
(setq user-full-name "Gustavo P Borges")
(setq user-mail-address "gugutz@gmail.com")
(setq work-mail-address "gugutz@stairs.studio")
(setq gmail-address "gugutz@gmail.com")
(setq nickname "gugutz")
\end{verbatim}

\section*{Observations about this config}
\label{sec:orgcccf9cd}
\begin{verbatim}
Keybindings use the <kbd> macro, as recommended in Mastering Emacs:
\end{verbatim}

\url{https://www.masteringemacs.org/article/mastering-key-bindings-emacs}

\begin{verbatim}
For native packages that come with emacs, `:ensure nil` must be set or use-package will try to download those packages from melpa and break
\end{verbatim}

\section*{Recompile init.el everytime emacs.org is changed and saved}
\label{sec:orgfc574e5}

\begin{verbatim}
Moved this to beggining of the file to avoid it not being parsed when theres an error in the middle of the file
It was being recompiled without this function so i had to manually re-copy first-init.el to make it compile first time again and again
\end{verbatim}



\begin{verbatim}
(defun /util/tangle-init ()
  (interactive)
  "If the current buffer is init.org' the code-blocks are
tangled, and the tangled file is compiled."
  (when (equal (buffer-file-name)
               (expand-file-name (concat user-emacs-directory "emacs.org")))
    ;; Avoid running hooks when tangling.
    (let ((prog-mode-hook nil))
      (org-babel-tangle)
      (byte-compile-file (concat user-emacs-directory "init.el")))))
\end{verbatim}

\begin{verbatim}
(add-hook 'after-save-hook #'/util/tangle-init)
\end{verbatim}

\section*{Packages}
\label{sec:org9ed8a4f}

\subsection*{package repositories}
\label{sec:org7c8cec6}

\begin{verbatim}
(require 'package)
;; add melpa stable emacs package repository
(add-to-list 'package-archives '("melpa" . "https://melpa.org/packages/"))
(add-to-list 'package-archives '("gnu" . "https://elpa.gnu.org/packages/"))
(add-to-list 'package-archives '("org" . "http://orgmode.org/elpa/") t) ; Org-mode's repository
\end{verbatim}

\subsection*{initialize packages}
\label{sec:org3fec0ec}
\begin{verbatim}
(package-initialize)
\end{verbatim}

moved this part to beggining of the file because if the
custom-safe-themes variable is not set before smart-mode-line (sml) activates
emacs asks 2 annoying confirmations on every startup before actually starting

\begin{verbatim}
(custom-set-variables
;; custom-set-variables was added by Custom.
;; If you edit it by hand, you could mess it up, so be careful.
;; Your init file should contain only one such instance.
;; If there is more than one, they won't work right.
'(custom-safe-themes
   (quote
   ("84d2f9eeb3f82d619ca4bfffe5f157282f4779732f48a5ac1484d94d5ff5b279" "57f95012730e3a03ebddb7f2925861ade87f53d5bbb255398357731a7b1ac0e0" "3c83b3676d796422704082049fc38b6966bcad960f896669dfc21a7a37a748fa" default)))
   '(fci-rule-color "#3E4451")
   '(package-selected-packages
     (quote
     (pdf-tools ox-pandoc ox-reveal org-preview-html latex-preview-pane smart-mode-line-powerline-theme base16-theme gruvbox-theme darktooth-theme rainbow-mode smartscan restclient editorconfig prettier-js pandoc rjsx-mode js2-refactor web-mode evil-org multiple-cursors flycheck smart-mode-line ## evil-leader evil-commentary evil-surround htmlize magit neotree evil json-mode web-serverx org))))
   (custom-set-faces
   ;; custom-set-faces was added by Custom.
   ;; If you edit it by hand, you could mess it up, so be careful.
   ;; Your init file should contain only one such instance.
   ;; If there is more than one, they won't work right.
   )
\end{verbatim}

\subsection*{Add the folder 'config' to emacs load-path so i can require stuff from there}
\label{sec:org1b3acdc}

\begin{verbatim}
(add-to-list 'load-path (expand-file-name "config" user-emacs-directory))
;; (add-to-list 'load-path "~/dotfiles/emacs.d/config")
\end{verbatim}

\subsection*{preparing environment to load stuff}
\label{sec:org47beffb}

\begin{verbatim}
;; Init time start
(defvar my-init-el-start-time (current-time) "Time when init.el was started")

;; Faster startup
;; (setq gc-cons-threshold 100000000)
;; Defer garbage collection further back in the startup process
(setq gc-cons-threshold (if (display-graphic-p) 400000000 100000000))
\end{verbatim}
\subsection*{require use-package}
\label{sec:orga292658}

\subsubsection*{Install use-package if not already installed}
\label{sec:orgbdb834b}
\begin{verbatim}
(unless (package-installed-p 'use-package)
  (package-refresh-contents)
  (package-install 'use-package)
)
\end{verbatim}

\subsubsection*{load use-package}
\label{sec:org5dcd1eb}
\begin{verbatim}
(eval-when-compile
  (require 'use-package))
\end{verbatim}

\subsubsection*{Enable use-package extension `ensure-system-package`}
\label{sec:org4cd4118}
\begin{verbatim}
(use-package use-package-ensure-system-package
  :ensure t
  :init
  ;; use sudo when needed
  (setq system-packages-use-sudo t)
)
\end{verbatim}

\subsubsection*{Set `:ensure t` globally for all packages using use-package}
\label{sec:org520f195}

\begin{verbatim}
this is disabled for now as i preffer to specify for each package
\end{verbatim}

\begin{verbatim}
(require 'use-package-ensure)
;; (setq use-package-always-ensure t)
\end{verbatim}

\subsubsection*{Auto update packages}
\label{sec:org4bb256c}
\begin{verbatim}
(use-package auto-package-update
  :config
  (setq auto-package-update-interval 7) ;; in days
  (setq auto-package-update-prompt-before-update t)
  (setq auto-package-update-delete-old-versions t)
  (setq auto-package-update-hide-results t)
  (auto-package-update-maybe)
)
\end{verbatim}


\section*{General editor settings}
\label{sec:org3a7c794}

\subsection*{Emacs Server}
\label{sec:org2ff0514}
Allow access from emacsclient
\begin{verbatim}
;; (use-package server
;;   :ensure nil
;;   :init
;;   (unless (or (daemonp) (server-running-p))
;;     (server-start))
;;   :hook (after-init . server-mode))
\end{verbatim}

\begin{verbatim}
(require 'server)
(unless (or (daemonp) (server-running-p))
  (server-start))
\end{verbatim}

\subsection*{set default font}
\label{sec:org8868ef5}

Find the first font in the list and use it

\begin{verbatim}
(require 'cl)
(defun font-candidate (&rest fonts)
  "Return existing font which first match."
  (find-if (lambda (f) (find-font (font-spec :name f))) fonts))

;; define list of fonts to be used in the above function
;; the first one found will be used
(set-face-attribute 'default nil :font (font-candidate '"Hack-10:weight=normal"
                                                        "Consolas-10:weight=normal"
                                                        "Droid Sans Mono-10:weight=normal"
                                                        "DejaVu Sans Mono-10:weight=normal"
                                                        "Ubuntu Mono-12:weight=normal"))
\end{verbatim}

\subsection*{visual-line-mode (word wrap)}
\label{sec:org004a685}
\begin{verbatim}
(use-package visual-line-mode
  :ensure nil
  :hook
  (prog-mode . visual-line-mode)
  (text-mode . visual-line-mode)
)
\end{verbatim}

\subsection*{Prevent emacs to create lockfiles (.\#files\#).}
\label{sec:org90af75b}

PS: this also stops preventing editing colisions, so watch out
\begin{verbatim}
(setq create-lockfiles nil)
\end{verbatim}

\subsection*{Use the system clipboard}
\label{sec:org75ffd60}

Enable copy/past-ing from clipboard

\begin{verbatim}
(setq x-select-enable-clipboard t)
\end{verbatim}

\subsection*{Always follow symbolic links to edit the 'actual' file it points to}
\label{sec:orgdd11057}

\begin{verbatim}
(setq vc-follow-symlinks t)
\end{verbatim}

\subsection*{Save all tempfiles in \$TMPDIR/emacs\$UID/}
\label{sec:org074dfaf}

\begin{verbatim}
(defconst emacs-tmp-dir (expand-file-name (format "emacs%d" (user-uid)) temporary-file-directory))
(setq backup-directory-alist
    `((".*" . ,emacs-tmp-dir)))
(setq auto-save-file-name-transforms
    `((".*" ,emacs-tmp-dir t)))
(setq auto-save-list-file-prefix
    emacs-tmp-dir)
\end{verbatim}

\subsection*{dont make backup files}
\label{sec:org892e052}

\begin{verbatim}
(use-package files
  :ensure nil
  :config
  (setq make-backup-files nil)
  ;; dont ask confirmation to kill processes
  ;;(setq confirm-kill-processes nil)
)
\end{verbatim}

\subsection*{dont ask confirmation to kill processes}
\label{sec:org8c5a04a}

\begin{verbatim}
(setq confirm-kill-processes nil)
\end{verbatim}

\subsection*{Disable the annoying Emacs bell ring (beep)}
\label{sec:orgbdb3d09}

\begin{verbatim}
(setq ring-bell-function 'ignore)
\end{verbatim}

\subsection*{Create alias to yes-or-no anwsers (y-or-n-p}
\label{sec:org038ec93}

\begin{verbatim}
(defalias 'yes-or-no-p 'y-or-n-p)
(fset 'yes-or-no-p 'y-or-n-p)
\end{verbatim}

\subsection*{dont ask for confirmation for opening large files}
\label{sec:org6a6d059}

\begin{verbatim}
(setq large-file-warning-threshold nil) ;; Don’t warn me about opening large files
\end{verbatim}

\subsection*{display-line-numbers}
\label{sec:org5662c1e}

Released with Emacs 26 (released in 2018-05)
\begin{verbatim}

(use-package display-line-numbers
  :if (version<= "26.0.50" emacs-version)
  :ensure nil
  :init
  (setq display-line-numbers-grow-only t)
  (setq display-line-numbers-width-start t)
  ;; old linum-mode variables, check if they work with new display-line-numbers-mode
  ;; (setq linum-format 'dynamic)
  ;; (setq linum-format " %d ") ;; one space separation between the linenumber display and the buffer contents:
  ;; (setq linum-format "%4d “) ;; 4 character and a space for line numbers
  (setq linum-format "%4d \u2502 ") ; 4 chars and a space with solid line separator
  :config
  ;;(global-display-line-numbers-mode)
  ;; for some reason the hooks for diplay line numbers wont work if i put them in use-package `:hook'. it has to be after `:config'
  (add-hook 'prog-mode-hook #'display-line-numbers-mode)
  (add-hook 'text-mode-hook #'display-line-numbers-mode)

  ;; Select lines by click-dragging on the margin. Tested with GNU Emacs 23.3
  (defvar *linum-mdown-line* nil)
  (defun line-at-click ()
    (save-excursion
    (let ((click-y (cdr (cdr (mouse-position))))
        (line-move-visual-store line-move-visual))
      (setq line-move-visual t)
      (goto-char (window-start))
      (next-line (1- click-y))
      (setq line-move-visual line-move-visual-store)
      ;; If you are using tabbar substitute the next line with
      (line-number-at-pos))))

  (defun md-select-linum ()
    (interactive)
    (goto-line (line-at-click))
    (set-mark (point))
    (setq *linum-mdown-line*
      (line-number-at-pos)))

  (defun mu-select-linum ()
    (interactive)
    (when *linum-mdown-line*
    (let (mu-line)
      ;; (goto-line (line-at-click))
      (setq mu-line (line-at-click))
      (goto-line (max *linum-mdown-line* mu-line))
      (set-mark (line-end-position))
      (goto-line (min *linum-mdown-line* mu-line))
      (setq *linum-mdown*
        nil))))

  (global-set-key (kbd "<left-margin> <down-mouse-1>") 'md-select-linum)
  (global-set-key (kbd "<left-margin> <mouse-1>") 'mu-select-linum)
  (global-set-key (kbd "<left-margin> <drag-mouse-1>") 'mu-select-linum)
)

\end{verbatim}

\subsection*{minibuffer history}
\label{sec:org42b3f35}

\begin{verbatim}
(savehist-mode 1)
\end{verbatim}

\subsection*{Turn on auto-revert mode (auto updates files changed on disk)}
\label{sec:orga5af9d1}

\begin{verbatim}
(use-package autorevert
  :ensure nil
  :hook
  (after-init . global-auto-revert-mode)
  :config
  (setq auto-revert-interval 0.5)
  (setq auto-revert-interval 2)
  (setq auto-revert-check-vc-info t)
  (setq auto-revert-verbose nil)
)
\end{verbatim}

\subsection*{C-n insert newlines if the point is at the end of the buffer.}
\label{sec:org008816a}

\begin{verbatim}
Useful, as it means you won’t have to reach for the return key to add newlines!
\end{verbatim}

\begin{verbatim}
(setq next-line-add-newlines t)
\end{verbatim}

\subsection*{Remove the \^{}M characters from files that contains Unix and DOS line endings}
\label{sec:org7c61946}

\begin{verbatim}
(defun remove-dos-eol ()
  "Do not show ^M in files containing mixed UNIX and DOS line endings."
  (interactive)
  (setq buffer-display-table (make-display-table))
  (aset buffer-display-table ?\^M [])
)
\end{verbatim}

\subsubsection*{Hook it to text-mode and prog-mode}
\label{sec:orga9239ff}
\begin{verbatim}
(add-hook 'text-mode-hook 'remove-dos-eol)
(add-hook 'prog-mode-hook 'remove-dos-eol)
\end{verbatim}

\subsection*{Increase, decrease and adjust font size}
\label{sec:org5c825a9}

\begin{verbatim}
(global-set-key (kbd "C-S-+") #'text-scale-increase)
(global-set-key (kbd "C-S-_") #'text-scale-decrease)
(global-set-key (kbd "C-S-)") #'text-scale-adjust)
\end{verbatim}

\subsection*{expand-region}
\label{sec:orga39de64}
\begin{verbatim}
;; (require 'expand-region)
(global-set-key (kbd "C-S-<tab>") 'er/expand-region)
\end{verbatim}

\subsection*{refresh buffer with F5}
\label{sec:orgdb294f6}
\begin{verbatim}
(global-set-key [f5] '(lambda () (interactive) (revert-buffer nil t nil)))
\end{verbatim}
\subsection*{C-k kills current buffer without having to select which buffer}
\label{sec:orgf0b9f87}

By default C-x k prompts to select which buffer should be selected.
I almost always want to kill the current buffer, so this snippet helps in that.
\begin{verbatim}
;; Kill current buffer; prompt only if
;; there are unsaved changes.
(global-set-key (kbd "C-x k")
  '(lambda () (interactive) (kill-buffer (current-buffer)))
)
\end{verbatim}

\subsection*{smooth scrolling}
\label{sec:org028a235}

Disabled in favor of sublimity-scroll, which is better
\begin{verbatim}
(use-package smooth-scrolling
  :disabled
  :ensure t
  :config
  (smooth-scrolling-mode 1)
)
\end{verbatim}

\begin{verbatim}
;; ;; Vertical Scroll
;; (setq scroll-step 1)
;; (setq scroll-margin 1)
;; (setq scroll-conservatively 101)
;; (setq scroll-up-aggressively 0.01)
;; (setq scroll-down-aggressively 0.01)
;; (setq auto-window-vscroll nil)
;; (setq fast-but-imprecise-scrolling nil)
;; (setq mouse-wheel-scroll-amount '(1 ((shift) . 1)))
;; (setq mouse-wheel-progressive-speed nil)
;; ; Horizontal Scroll
;; (setq hscroll-step 1)
;; (setq hscroll-margin 1)
\end{verbatim}

\subsection*{add final newline}
\label{sec:org08f5844}

\begin{verbatim}
(setq require-final-newline t)
\end{verbatim}

\subsection*{fill column}
\label{sec:org381ddd8}

Sets a 80 character line width

\begin{verbatim}
(setq-default fill-column 80)
\end{verbatim}

\subsection*{preffer UTF-8 coding system}
\label{sec:org71335e3}
\begin{verbatim}
(prefer-coding-system 'utf-8) ;; Prefer UTF-8 encoding
\end{verbatim}

\subsection*{auto balance windows on opening and closing frames}
\label{sec:org3ebd34c}

\begin{verbatim}
(setq window-combination-resize t)
\end{verbatim}

\subsection*{set default line spacing}
\label{sec:org0709127}

\begin{verbatim}
;; (setq-default line-spacing 1) ;; A nice line height
(setq-default line-spacing 3)
\end{verbatim}

\subsection*{fix wierd color escape system}
\label{sec:org2f718cb}

\begin{verbatim}
(setq system-uses-terminfo nil) ;; Fix weird color escape sequences
\end{verbatim}

\subsection*{confirm before closing emacs}
\label{sec:orgb9138fa}

\begin{verbatim}
;; (setq confirm-kill-emacs 'yes-or-no-p) ;; Ask for confirmation before closing emacs
\end{verbatim}


\section*{Code editing settings}
\label{sec:org4faa648}
\subsection*{subword-mode}
\label{sec:org53bd712}

\begin{verbatim}
Alt+x subword-mode. It change all cursor movement/edit commands to stop in-between the “camelCase” words.
subword-mode and superword-mode are mutally exclusive. Turning one on turns off the other.
\end{verbatim}


\begin{verbatim}
(use-package subword
  :ensure nil
  :hook
  (clojure-mode . subword-mode)
  (ruby-mode . subword-mode)
  (enh-ruby-mode . subword-mode)
  (elixir-mode . subword-mode)
)
\end{verbatim}

\subsection*{superword-mode}
\label{sec:org6799a96}

\begin{verbatim}
Alt+x superword-mode (emacs 24.4) is similar. It treats text like “x_y” as one word. Useful for “snake_case”.
subword-mode and superword-mode are mutally exclusive. Turning one on turns off the other.
\end{verbatim}


\begin{verbatim}
(use-package superword
  :ensure nil
  :hook
  (js2-mode . superword-mode)
)
\end{verbatim}

\subsection*{default indentation}
\label{sec:orgc82118e}
\begin{verbatim}
(setq-default indent-tabs-mode nil)
;; C e C-like langs default indent size
(setq-default tab-width 2)
;; Perl default indent size
(setq-default cperl-basic-offset 2)
(setq-default c-basic-offset 2)
\end{verbatim}

\subsection*{Use unix-conf-mode for .*rc files}
\label{sec:org6a49976}
\begin{verbatim}
(use-package conf-mode
  :mode
  (;; systemd
    ("\\.service\\'"     . conf-unix-mode)
    ("\\.timer\\'"      . conf-unix-mode)
    ("\\.target\\'"     . conf-unix-mode)
    ("\\.mount\\'"      . conf-unix-mode)
    ("\\.automount\\'"  . conf-unix-mode)
    ("\\.slice\\'"      . conf-unix-mode)
    ("\\.socket\\'"     . conf-unix-mode)
    ("\\.path\\'"       . conf-unix-mode)

    ;; general
    ("conf\\(ig\\)?$"   . conf-mode)
    ("rc$"              . conf-mode))
)
;; (add-to-list 'auto-mode-alist '("\\.*rc$" . conf-unix-mode))
\end{verbatim}

\subsection*{iedit}
\label{sec:orgbb76244}
\begin{verbatim}
(use-package iedit
  :config
  (set-face-background 'iedit-occurrence "Magenta")
  :bind
  ("C-;" . iedit-mode)
)
\end{verbatim}

\subsection*{eldoc}
\label{sec:org4e08aa1}

Enable documentation for programming languages

\begin{verbatim}
(use-package eldoc
  :ensure nil
  :hook
  (prog-mode . eldoc-mode)
  ;;(prog-mode       . turn-on-eldoc-mode)
  ;; (cider-repl-mode . turn-on-eldoc-mode)
  :config
  ;; (global-eldoc-mode -1)
  ;; (add-hook 'prog-mode-hook 'eldoc-mode)
  (setq eldoc-idle-delay 0.4)
)
\end{verbatim}


\subsection*{aggressive-indent-mode}
\label{sec:org9b5cfc5}

\begin{verbatim}
(use-package aggressive-indent
  :ensure t
  :hook
  (emacs-lisp-mode . aggressive-indent-mode)
  (css-mode . aggressive-indent-mode)
  :config
)
\end{verbatim}

\subsection*{interactive-align}
\label{sec:org111e78c}

Keymap used in the minibuffer when ialign command is executed.
\begin{center}
\begin{tabular}{ll}
\hline
Key & Command\\
\hline
C-c C-r & ialign-toggle-repeat\\
C-c C-t & ialign-toggle-tabs\\
C-c M-c & ialign-toggle-case-fold\\
C-c + & ialign-increment-spacing\\
C-c - & ialign-decrement-spacing\\
C-c [ & ialign-decrement-group\\
C-c ] & ialign-increment-group\\
C-c C-f & ialign-set-group\\
C-c C-s & ialign-set-spacing\\
C-c RET & ialign-commit\\
C-c C-c & ialign-update\\
C-c ? & ialign-show-help\\
\hline
\end{tabular}
\end{center}

\begin{verbatim}
(use-package ialign
  :ensure t
  :bind
  ("C-x l" . ialign)
  :config
  ;;(setq ialign-default-spacing 32)
  (setq ialign-align-with-tabs nil) ;; default nil
  (setq ialign-auto-update t) ;; default t
)
\end{verbatim}

\subsection*{align.el}
\label{sec:orgdf7ae41}

align text to a specific column, by regexp

This mode allows you to align regions in a context-sensitive fashion.
The classic use is to align assignments:

int a = 1;
short foo = 2;
double blah = 4;

becomes

int    a    = 1;
short  foo  = 2;
double blah = 4;

\begin{verbatim}
(defun align-values (start end)
  "Vertically aligns region based on lengths of the first value of each line.
Example output:

  foo        bar
  foofoo     bar
  foofoofoo  bar"
  (interactive "r")
  (align-regexp start end
        "\\S-+\\(\\s-+\\)"
        1 1 nil))


\end{verbatim}

\subsection*{dumb-jump}
\label{sec:org370b3f5}
Emacs jump to definition tool

\begin{verbatim}
(use-package dumb-jump
  :ensure t
  :after helm
  :bind (("M-g o" . dumb-jump-go-other-window)
         ("M-g j" . dumb-jump-go)
         ("M-g b" . dumb-jump-back)
         ("M-g i" . dumb-jump-go-prompt)
         ("M-g x" . dumb-jump-go-prefer-external)
         ("M-g z" . dumb-jump-go-prefer-external-other-window))
  :config
  (setq dumb-jump-selector 'helm)
  ;; (setq dumb-jump-selector 'ivy)
)
\end{verbatim}

\section*{Text editing settings}
\label{sec:org34d0ab0}

\subsection*{Helper functions for casing words}
\label{sec:org9773b37}

\begin{verbatim}
(defun upcase-backward-word (arg)
  (interactive "p")
  (upcase-word (- arg))
)
\end{verbatim}

\begin{verbatim}
(defun downcase-backward-word (arg)
  (interactive "p")
  (downcase-word (- arg))
)
\end{verbatim}

\begin{verbatim}
(defun capitalize-backward-word (arg)
  (interactive "p")
  (capitalize-word (- arg))
)
\end{verbatim}

\begin{verbatim}
(global-set-key (kbd "C-M-u")	 'upcase-backward-word)
(global-set-key (kbd "C-M-l")	 'downcase-backward-WORD)
;; this replaces native capitlize word!
(global-set-key (kbd "M-c")	 'capitalize-backward-word)
\end{verbatim}

\subsection*{Spellchecking}
\label{sec:org1ce2eac}

\begin{verbatim}
(defconst *spell-check-support-enabled* t) ;; Enable with t if you prefer
\end{verbatim}


\subsection*{move-text}
\label{sec:org806abc5}

\begin{verbatim}
(use-package move-text
  :ensure t
  :after evil
  :bind
  ([(meta shift up)] . move-text-up)
  ([(meta shift down)] . move-text-down)
  ([(meta k)] . move-text-up)
  ([(meta j)] . move-text-down)
  ([(meta shift k)] . move-text-line-up)
  ([(meta shift j)] . move-text-line-down)
  :init
  ;; free the bindings used by this plugin from windmove and other areas that use the same keys
  (global-unset-key (kbd "M-j"))
  (global-unset-key (kbd "M-k"))
  (global-unset-key (kbd "C-S-j"))
  (global-unset-key (kbd "C-S-k"))
  :config
  (move-text-default-bindings)
  ;; tried setting these in :bind but use package executes :bind along with init, and i needed to free the keys before
  (define-key evil-normal-state-map (kbd "M-j") 'move-text-down)
  (define-key evil-normal-state-map (kbd "M-k") 'move-text-up)
  (define-key evil-visual-state-map (kbd "M-j") 'move-text-region-up)
  (define-key evil-visual-state-map (kbd "M-k") 'move-text-region-down)
)
\end{verbatim}

\subsection*{Flyspell}
\label{sec:org11e4c08}

Change dictionaries with F12

\begin{verbatim}
;(defun fd-switch-dictionary()
;(interactive)
;(let* ((dic ispell-current-dictionary)
;    (change (if (string= dic "deutsch8") "english" "deutsch8")))
;  (ispell-change-dictionary change)
;  (message "Dictionary switched from %s to %s" dic change)
;  ))

;(global-set-key (kbd "<f12>")   'fd-switch-dictionary)
\end{verbatim}

\begin{verbatim}
;; Change dictionaries with F12 (teste pt-br)
(let ((langs '("american" "brasileiro")))
  (setq lang-ring (make-ring (length langs)))
  (dolist (elem langs) (ring-insert lang-ring elem))
)

(defun cycle-ispell-languages ()
   (interactive)
   (let ((lang (ring-ref lang-ring -1)))
     (ring-insert lang-ring lang)
     (ispell-change-dictionary lang))
)

(global-set-key (kbd "<f12>")   'cycle-ispell-languages)
\end{verbatim}

\begin{verbatim}
(use-package flyspell
  :defer 1
  :hook
  (text-mode . flyspell-mode)
  :config
  ;; ignore org source blocks from spellchecking
  (add-to-list 'ispell-skip-region-alist '(":\\(PROPERTIES\\|LOGBOOK\\):" . ":END:"))
  (add-to-list 'ispell-skip-region-alist '("^#+BEGIN_SRC" . "^#+END_SRC"))

  ;; global ispell settings (disabled in favor of conditional hunspell setup bellow)
  ;; (setenv "LANG" "en_US.UTF-8")
  ;; (setq ispell-program-name "aspell")
  ;; (setq ispell-program-name "hunspell")
  ;; (setq ispell-dictionary "en_US")
  ;; (setq ispell-local-dictionary "pt_BR")
  ;; (setq ispell-local-dictionary "en_US")

  ;; Hunspell settings
  ;; find aspell and hunspell automatically
;;  (cond
;;    ;; try aspell first in case both aspell and hunspell are installed, it will
;;    ;; set `ispell-program-name' to use hunspell
;;    ((executable-find "aspell")
;;      (setq ispell-program-name "aspell")
;;      ;; Please note `ispell-extra-args' contains ACTUAL parameters passed to aspell
;;      (setq ispell-extra-args '("--sug-mode=ultra" "--lang=en_US"))
;;      ;;(setq ispell-local-dictionary "pt_BR")
;;    )
;;   ;; if hunspell is available, use it instead of aspell for multilang support
;;    ((executable-find "hunspell")
;;      (setq ispell-program-name "hunspell")
;;      ;; i could set `ispell-dictionary' instead but `ispell-local-dictionary' has higher priority
;;      (setq ispell-local-dictionary "en_US")
;;      ;; setup both en_US and pt_BR dictionaries in hunspell
;;      (ispell-hunspell-add-multi-dic "en_US,pt_BR")
;;
;;      (setq ispell-local-dictionary-alist
;;         ;; Please note the list `("-d" "en_US")` contains ACTUAL parameters passed to hunspell
;;         ;; You could use `("-d" "en_US,en_US-med")` to check with multiple dictionaries
;;         '(("en_US" "[[:alpha:]]" "[^[:alpha:]]" "[']" nil ("-d" "en_US,pt_BR") nil utf-8))
;;      )
;;    )
;;  )

)
\end{verbatim}

\subsection*{guess-language}
\label{sec:orgdb19ddb}

Automatic guess the language of the paragraph im writing in
Works with mutilang documents

\begin{verbatim}
(use-package guess-language         ; Automatically detect language for Flyspell
  :ensure t
  :defer t
  :hook
  (text-mode . guess-language-mode)
  ;; :init (add-hook 'text-mode-hook #'guess-language-mode)
  :config
  (setq guess-language-langcodes '((en . ("en_US" "English"))
                                   (br . ("pt_BR" "Portuguese Brazilian"))
                                  )
  guess-language-languages '(en br)
  guess-language-min-paragraph-length 45)
)
\end{verbatim}


\section*{GPG Encryption}
\label{sec:orgb2246d1}

\begin{verbatim}
(use-package epa-file
  :config
  (epa-file-enable)
  (setq epa-file-encrypt-to '("gugutz@gmail.com"))

  ;; Control whether or not to pop up the key selection dialog.
  (setq epa-file-select-keys 0)
  ;; Cache passphrase for symmetric encryption.
  (setq epa-file-cache-passphrase-for-symmetric-encryption t)
)
(require 'epa-file)
(epa-file-enable)
\end{verbatim}

\section*{exec-path-from-shell}
\label{sec:org478b65f}

Make emacs use \$PATH defined in the systems shell

\begin{verbatim}
snippet taken from oficial use package github page
\end{verbatim}

\begin{verbatim}
(use-package exec-path-from-shell
  :if (memq window-system '(mac ns x))
  :ensure t
  :init
  ;;(setenv "SHELL" "/bin/zsh")
  ;;(setq explicit-shell-file-name "/bin/zsh")
  ;;(setq shell-file-name "zsh")
  :config
  ;; This sets $MANPATH, $PATH and exec-path from your shell, but only on OS X and Linux.
  (exec-path-from-shell-initialize)
  ;; Its possible to copy values from other SHELL variables using one of the two methods bellow
  ;; either using the `exec-path-from-shell-copy-env' functon or setting the variable `exec-path-from-shell-variables'
  ;; (exec-path-from-shell-copy-env "PYTHONPATH")
  ;; (setq exec-path-from-shell-variables '("PYTHONPATH" "GOPATH"))
)
\end{verbatim}

\section*{Mouse configuration}
\label{sec:org9c03fda}
\subsection*{Enable mouse support in terminal mode}
\label{sec:org029f738}

\begin{verbatim}
(when (eq window-system nil)
  (xterm-mouse-mode 1))
\end{verbatim}

\begin{verbatim}
;; (use-package mouse3
;;     :config
;; (global-set-key (kbd "<mouse-3>") 'mouse3-popup-menu))
\end{verbatim}

\subsection*{right-click-context-menu}
\label{sec:orgb154e82}

\begin{verbatim}
(use-package right-click-context
  :ensure t
  :config
  (global-set-key (kbd "<menu>") 'right-click-context-menu)
  (global-set-key (kbd "<mouse-3>") 'right-click-context-menu)
  (bind-key "C-c <mouse-3>" 'right-click-context-menu)

  ;; (setq right-click-context-mode-lighter "🐭")

  ;; customize the right-click-context-menu
  (let ((right-click-context-local-menu-tree
       (append right-click-context-global-menu-tree
             '(("Insert"
                ("Go to definition" :call (lsp-goto-type-definition)
                ("FooBar" :call (insert "FooBar"))
                )))))
  (right-click-context-menu)))
)
\end{verbatim}

\subsection*{zoom buffers with Mouse+Scroll<Up/Down> like in the browser}
\label{sec:org8ff1f5c}

\begin{verbatim}
;; zoom in/out like we do everywhere else.
(global-set-key (kbd "C-=") 'text-scale-increase)
(global-set-key (kbd "C--") 'text-scale-decrease)
(global-set-key (kbd "<C-wheel-down>") 'text-scale-decrease)
(global-set-key (kbd "<C-wheel-up>") 'text-scale-increase)
\end{verbatim}

\section*{hippie-expand (native emacs expand function)}
\label{sec:org7f90654}

\begin{verbatim}
(use-package hippie-exp
  ;;:ensure nil
  :defer t
  :bind
  ("<tab>" . hippie-expand)
  ("<C-return>" . hippie-expand)
  ("C-M-SPC" . hippie-expand)
  (:map evil-insert-state-map
  ("<tab>" . hippie-expand)
  )
  :config
  (setq-default hippie-expand-try-functions-list
        '(yas-hippie-try-expand
          indent-according-to-mode
          emmet-expand-line
          company-indent-or-complete-common
          )
  )
)
\end{verbatim}


\section*{Evil}
\label{sec:org9f6af6e}

\begin{verbatim}
(use-package evil
    :ensure t
    :init
    (setq evil-ex-complete-emacs-commands nil)
    (setq evil-vsplit-window-right t)
    (setq evil-split-window-below t)
    (setq evil-shift-round nil)
    (setq evil-esc-delay 0)  ;; Don't wait for any other keys after escape is pressed.
    ;; Make Evil look a bit more like (n) vim  (??)
    (setq evil-search-module 'isearch-regexp)
    ;; (setq evil-search-module 'evil-search)
    (setq evil-magic 'very-magic)
    (setq evil-shift-width (symbol-value 'tab-width))
    (setq evil-regexp-search t)
    (setq evil-search-wrap t)
    ;; (setq evil-want-C-i-jump t)
    (setq evil-want-C-u-scroll t)
    (setq evil-want-fine-undo nil)
    (setq evil-want-integration nil)
    ;; (setq evil-want-abbrev-on-insert-exit nil)
    (setq evil-want-abbrev-expand-on-insert-exit nil)
    (setq evil-mode-line-format '(before . mode-line-front-space)) ;; move evil tag to beginning of modeline
    ;; Cursor is alway black because of evil.
    ;; Here is the workaround
    ;; (@see https://bitbucket.org/lyro/evil/issue/342/evil-default-cursor-setting-should-default)
    (setq evil-default-cursor t)
    ;; change cursor color according to mode
    (setq evil-emacs-state-cursor '("#ff0000" box))
    (setq evil-motion-state-cursor '("#FFFFFF" box))
    (setq evil-normal-state-cursor '("#00ff00" box))
    (setq evil-visual-state-cursor '("#abcdef" box))
    (setq evil-insert-state-cursor '("#e2f00f" bar))
    (setq evil-replace-state-cursor '("red" hbar))
    (setq evil-operator-state-cursor '("red" hollow))

  :bind
  (:map evil-normal-state-map
  (", w" . evil-window-vsplit)
  ("C-r" . undo-tree-redo)
  )
  (:map evil-insert-state-map
  ;; this is also defined globally above in the config
  ("C-S-<tab>" . er/expand-region)
  )
  (:map evil-visual-state-map
  ;; this is also defined globally above in the config
  ("<tab>" . indent-region)
  ("C-/" . comment-line)
  ("C-S-/" . comment-region)
  ("C-S-M-/" . comment-box)
  ("M-=" . #'align-values)
  )

;; check if global-set-key also maps to evil insert mode; if yes delete bellow snippets
  :config
  (evil-mode)

;; unset evil bindings that conflits with other stuff
  (define-key evil-insert-state-map (kbd "<tab>") nil)
  (define-key evil-normal-state-map (kbd "<tab>") nil)
  (define-key evil-visual-state-map (kbd "<tab>") nil)

  ;; vim-like navigation with C-w hjkl
  (define-prefix-command 'evil-window-map)
  (define-key evil-window-map (kbd "h") 'evil-window-left)
  (define-key evil-window-map (kbd "j") 'evil-window-down)
  (define-key evil-window-map (kbd "k") 'evil-window-up)
  (define-key evil-window-map (kbd "l") 'evil-window-right)
  (define-key evil-window-map (kbd "b") 'evil-window-bottom-right)
  (define-key evil-window-map (kbd "c") 'evil-window-delete)
  (define-key evil-motion-state-map (kbd "M-w") 'evil-window-map)

  ;; make esc quit or cancel everything in Emacs
  (define-key evil-normal-state-map [escape] 'keyboard-quit)
  (define-key evil-visual-state-map [escape] 'keyboard-quit)
  (define-key minibuffer-local-map [escape] 'minibuffer-keyboard-quit)
  (define-key minibuffer-local-ns-map [escape] 'minibuffer-keyboard-quit)
  (define-key minibuffer-local-completion-map [escape] 'minibuffer-keyboard-quit)
  (define-key minibuffer-local-must-match-map [escape] 'minibuffer-keyboard-quit)
  (define-key minibuffer-local-isearch-map [escape] 'minibuffer-keyboard-quit)

  ;; recover native emacs commands that are overriden by evil
  ;; this gives priority to native emacs behaviour rathen than Vim's
  (define-key evil-normal-state-map (kbd "SPC") 'ace-jump-mode)
  (define-key evil-visual-state-map (kbd "SPC") 'ace-jump-mode)
  (define-key evil-normal-state-map (kbd "C-e") 'evil-end-of-line)
  (define-key evil-insert-state-map (kbd "C-e") 'move-end-of-line)
  (define-key evil-visual-state-map (kbd "C-e") 'evil-end-of-line)
  (define-key evil-motion-state-map (kbd "C-e") 'evil-end-of-line)
  (define-key evil-insert-state-map (kbd "C-d") 'evil-delete-char)
  (define-key evil-normal-state-map (kbd "C-d") 'evil-delete-char)
  (define-key evil-visual-state-map (kbd "C-d") 'evil-delete-char)
  (define-key evil-normal-state-map (kbd "C-k") 'kill-line)
  (define-key evil-insert-state-map (kbd "C-k") 'kill-line)
  (define-key evil-visual-state-map (kbd "C-k") 'kill-line)
  (define-key evil-insert-state-map (kbd "C-w") 'kill-region)
  (define-key evil-normal-state-map (kbd "C-w") 'kill-region)
  (define-key evil-visual-state-map (kbd "C-w") 'kill-region)
  (define-key evil-normal-state-map (kbd "C-w") 'evil-delete)
  (define-key evil-insert-state-map (kbd "C-w") 'evil-delete)
  (define-key evil-visual-state-map (kbd "C-w") 'evil-delete)
  (define-key evil-normal-state-map (kbd "C-y") 'yank)
  (define-key evil-insert-state-map (kbd "C-y") 'yank)
  (define-key evil-visual-state-map (kbd "C-y") 'yank)
  (define-key evil-normal-state-map (kbd "C-f") 'evil-forward-char)
  (define-key evil-insert-state-map (kbd "C-f") 'evil-forward-char)
  (define-key evil-insert-state-map (kbd "C-f") 'evil-forward-char)
  (define-key evil-normal-state-map (kbd "C-b") 'evil-backward-char)
  (define-key evil-insert-state-map (kbd "C-b") 'evil-backward-char)
  (define-key evil-visual-state-map (kbd "C-b") 'evil-backward-char)
  (define-key evil-normal-state-map (kbd "C-n") 'evil-next-line)
  (define-key evil-insert-state-map (kbd "C-n") 'evil-next-line)
  (define-key evil-visual-state-map (kbd "C-n") 'evil-next-line)
  (define-key evil-normal-state-map (kbd "C-p") 'evil-previous-line)
  (define-key evil-insert-state-map (kbd "C-p") 'evil-previous-line)
  (define-key evil-visual-state-map (kbd "C-p") 'evil-previous-line)
  (define-key evil-normal-state-map (kbd "Q") 'call-last-kbd-macro)
  (define-key evil-visual-state-map (kbd "Q") 'call-last-kbd-macro)
  (define-key evil-insert-state-map (kbd "C-r") 'search-backward)
)
\end{verbatim}


\section*{Evil packages / plugins}
\label{sec:org3371b48}

\subsection*{evil-numbers}
\label{sec:orgea45698}
\begin{verbatim}
(use-package evil-numbers
  :ensure t
  :after evil
  :bind
  (:map evil-normal-state-map
  ("C-c +" . evil-numbers/inc-at-pt)
  ("C-c -" . evil-numbers/dec-at-pt)
  ("<kp-add>" . evil-numbers/inc-at-pt)
  ("<kp-subtract>" . evil-numbers/dec-at-pt))
  :config
  (global-set-key (kbd "C-c +") 'evil-numbers/inc-at-pt)
  (global-set-key (kbd "C-c -") 'evil-numbers/dec-at-pt)
)
\end{verbatim}

\subsection*{evil-leader}
\label{sec:orgb03f243}

\begin{verbatim}
(use-package evil-leader
  :config
  (global-evil-leader-mode)
  (evil-leader/set-leader ",")
  (evil-leader/set-key
    "e" 'find-file
    "q" 'evil-quit
    "w" 'save-buffer
    "d" 'delete-frame
    "k" 'kill-buffer
    "b" 'switch-to-buffer
    "-" 'split-window-bellow
    "|" 'split-window-right)
)
\end{verbatim}

\subsection*{Evil Surround}
\label{sec:org15faf2e}

\begin{verbatim}
(use-package evil-surround
  :config
  (global-evil-surround-mode 1)
)
\end{verbatim}

\begin{verbatim}
(defun evil-surround-prog-mode-hook-setup ()
  "Documentation string, idk, put something here later."
  (push '(47 . ("/" . "/")) evil-surround-pairs-alist)
  (push '(40 . ("(" . ")")) evil-surround-pairs-alist)
  (push '(41 . ("(" . ")")) evil-surround-pairs-alist)
  (push '(91 . ("[" . "]")) evil-surround-pairs-alist)
  (push '(93 . ("[" . "]")) evil-surround-pairs-alist)
)
(add-hook 'prog-mode-hook 'evil-surround-prog-mode-hook-setup)
\end{verbatim}

\begin{verbatim}
(defun evil-surround-js-mode-hook-setup ()
  "ES6." ;  this is a documentation string, a feature in Lisp
  ;; I believe this is for auto closing pairs
  (push '(?1 . ("{`" . "`}")) evil-surround-pairs-alist)
  (push '(?2 . ("${" . "}")) evil-surround-pairs-alist)
  (push '(?4 . ("(e) => " . "(e)")) evil-surround-pairs-alist)
  ;; ReactJS
  (push '(?3 . ("classNames(" . ")")) evil-surround-pairs-alist)
)
(add-hook 'js2-mode-hook 'evil-surround-js-mode-hook-setup)
\end{verbatim}

\begin{verbatim}
(defun evil-surround-emacs-lisp-mode-hook-setup ()
  (push '(?` . ("`" . "'")) evil-surround-pairs-alist)
)
(add-hook 'emacs-lisp-mode-hook 'evil-surround-emacs-lisp-mode-hook-setup)

(defun evil-surround-org-mode-hook-setup ()
  (push '(91 . ("[" . "]")) evil-surround-pairs-alist)
  (push '(93 . ("[" . "]")) evil-surround-pairs-alist)
  (push '(?= . ("=" . "=")) evil-surround-pairs-alist)
)
(add-hook 'org-mode-hook 'evil-surround-org-mode-hook-setup)
\end{verbatim}

\subsection*{evil-commentary}
\label{sec:orgebbf974}

\begin{verbatim}
(use-package evil-commentary
  :config
  (evil-commentary-mode)
)
\end{verbatim}

\subsection*{Evil-Matchit}
\label{sec:orgd6bf0a4}
\begin{verbatim}
(use-package evil-matchit
  :config
  (global-evil-matchit-mode 1)
)
\end{verbatim}

\subsection*{evil-paredit}
\label{sec:org02a6e16}

\begin{verbatim}
(use-package evil-paredit
  :ensure t
  :hook
  (emacs-lisp-mode . evil-paredit-mode)
)
\end{verbatim}

\subsection*{evil-mc}
\label{sec:org93a0a77}

Multiple cursors for evil mode

\begin{center}
\begin{tabular}{ll}
\hline
Key & action\\
\hline
C-t or grn & skip creating a cursor forward\\
grp & skip creating a cursor backward\\
gru & undo last addded cursor\\
grq & remove all cursors\\
\hline
\end{tabular}
\end{center}

\begin{verbatim}
(use-package evil-mc
  :ensure t
  :after evil
  :bind
  (:map evil-visual-state-map
  ("C-d" . evil-mc-make-and-goto-next-match) ;; Make a cursor at point and go to the next match of the selected region or the symbol under cursor.
  ("C-a" . evil-mc-make-all-cursors) ;; Create cursors for all strings that match the selected region or the symbol under cursor.
  ("C-q" . evil-mc-undo-all-cursors)  ;; Remove all cursors.
  )
  :config
  (global-evil-mc-mode  1)
)
\end{verbatim}

\subsection*{evil-goggles}
\label{sec:org0653787}

\begin{verbatim}
(use-package evil-goggles
  :ensure t
  :config
  (evil-goggles-mode)
  (setq evil-goggles-pulse t) ;; default is to pulse when running in a graphic display
  (setq evil-goggles-duration 0.100) ;; default is 0.200

;; list of all on/off variables, their default value is `t`:

  (setq evil-goggles-enable-paste nil) ;; to disable the hint when pasting
;;(setq  evil-goggles-enable-delete t)
;;(setq  evil-goggles-enable-change t)
;;(setq evil-goggles-enable-indent t)
;;(setq  evil-goggles-enable-yank t)
;;(setq  evil-goggles-enable-join t)
;;(setq evil-goggles-enable-fill-and-move t)
;;(setq evil-goggles-enable-paste t)
;;(setq evil-goggles-enable-shift t)
;;(setq evil-goggles-enable-surround t)
;;(setq evil-goggles-enable-commentary)
;;(setq evil-goggles-enable-nerd-commenter t)
;;(setq evil-goggles-enable-replace-with-register t)
;;(setq evil-goggles-enable-set-marker t)
;;(setq evil-goggles-enable-undo t)
;;(setq evil-goggles-enable-redo t)
;;(setq evil-goggles-enable-record-macro t)

  ;; optionally use diff-mode's faces; as a result, deleted text
  ;; will be highlighed with `diff-removed` face which is typically
  ;; some red color (as defined by the color theme)
  ;; other faces such as `diff-added` will be used for other actions
  (evil-goggles-use-diff-faces)
)
\end{verbatim}

\subsection*{evil-lion}
\label{sec:org927918d}

Align by operators

\begin{verbatim}
(use-package evil-lion
  :ensure t
  :bind
  (:map evil-normal-state-map
  ("g l " . evil-lion-left)
  ("g L " . evil-lion-right)
  :map evil-visual-state-map
  ("g l " . evil-lion-left)
  ("g L " . evil-lion-right))
  :config
  (setq evil-lion-squeeze-spaces t) ;; default t
  (evil-lion-mode)
)
\end{verbatim}

\section*{org-mode}
\label{sec:org810bc70}

\subsection*{org-mode setup}
\label{sec:org5a460cc}
\begin{verbatim}
(use-package org
  :ensure org-plus-contrib
  :defer t
  :bind
  ("C-c l" . org-store-link)
  ("C-c a" . org-agenda)
  ("C-c c" . org-capture)
  ("C-c b" . org-switch)
  ;; this map is to delete de bellow commented lambda that does the same thing
  ;; Resolve issue with Tab not working with ORG only in Normal VI Mode in terminal
  ;; (something with TAB on terminals being related to C-i...)
  (:map evil-normal-state-map
  ("<tab>" . org-cycle)
  )
  :init
  ;; general org config variables
  (setq org-log-done 'time)
  (setq org-export-backends (quote (ascii html icalendar latex md odt)))
  (setq org-use-speed-commands t)

  ;; dont display atual width for images inline. set per-file with
  ;; #+ATTR_HTML: :width 600px :height: auto
  ;; #+ATTR_ORG: :width 600
  ;; #+ATTR_LATEX: :width 5in
  (setq org-image-actual-width nil)
  (setq org-startup-with-inline-images t)

  ;; make tab behave like it usually do (ie: indent) inside org source blocks
  (setq org-src-tab-acts-natively t)

  (setq org-confirm-babel-evaluate 'nil)
  (setq org-todo-keywords
   '((sequence "TODO" "IN-PROGRESS" "REVIEW" "|" "DONE")))
  (setq org-agenda-window-setup 'other-window)
  (setq org-log-done 'time) ;; Show CLOSED tag line in closed TODO items
  (setq org-log-done 'note) ;; Prompt to leave a note when closing an item
  (setq org-hide-emphasis-markers nil)

  ;;ox-twbs (exporter to twitter bootstrap html)
  (setq org-enable-bootstrap-support t)
  :config
  ;; org-capture - needs to be in :config because it assumes a variable is already defined: `org-directory'
  (setq org-default-notes-file (concat org-directory "/notes.org"))

  ;;(add-hook 'org-mode-hook
  ;;          (lambda ()
  ;;        (define-key evil-normal-state-map (kbd "TAB") 'org-cycle)))

  (defun org-export-turn-on-syntax-highlight()
    "Setup variables to turn on syntax highlighting when calling `org-latex-export-to-pdf'"
    (interactive)
    (setq org-latex-listings 'minted
          org-latex-packages-alist '(("" "minted"))
          (setq org-latex-pdf-process
          '("latexmk -pdflatex='pdflatex -interaction nonstopmode' -pdf -bibtex -f %f"))))

  (setq org-latex-pdf-process
    '("latexmk -pdflatex='pdflatex -interaction nonstopmode' -pdf -bibtex -f %f"))

  (require 'org-habit)
  '(org-emphasis-alist
   (quote
    (
     ("!" org-habit-overdue-face)
     ("%" org-habit-alert-face)
     ("*" bold)
     ("/" italic)
     ("_" underline)
     ("=" org-verbatim verbatim)
     ("~" org-code verbatim)
     ("+" (:strike-through t))
     )))
)
\end{verbatim}

\subsection*{ox-extra (org-plus-contrib)}
\label{sec:org55493a6}

ox-extras
add suport for the ignore tag (ignores a headline without ignoring its content)

\begin{verbatim}
(use-package ox-extra
  :ensure nil
  :config
  (ox-extras-activate '(ignore-headlines))
  (ox-extras-activate '(latex-header-blocks ignore-headlines))
)
\end{verbatim}

\subsection*{add more custom emacs emphasis characters}
\label{sec:org9a8e811}


first test
\begin{verbatim}
(require 'org-habit nil t)

(defun org-add-my-extra-fonts ()
  "Add alert and overdue fonts."
  (add-to-list 'org-font-lock-extra-keywords '("\\(!\\)\\([^\n\r\t]+\\)\\(!\\)" (1 '(face org-habit-alert-face invisible t)) (2 'org-habit-alert-face) (3 '(face org-habit-alert-face invisible t))))
  (add-to-list 'org-font-lock-extra-keywords '("\\(%\\)\\([^\n\r\t]+\\)\\(%\\)" (1 '(face org-habit-overdue-face invisible t)) (2 'org-habit-overdue-face) (3 '(face org-habit-overdue-face invisible t)))))

(add-hook 'org-font-lock-set-keywords-hook #'org-add-my-extra-fonts)
\end{verbatim}


\subsection*{Evil-ORG}
\label{sec:orgc0b7878}

\begin{verbatim}
(use-package evil-org
  :after org
  :hook
  (org-mode . evil-org-mode)
  :config
  (lambda ()
    (evil-org-set-key-theme))
)
\end{verbatim}

\subsection*{ox-pandoc}
\label{sec:org92298f7}

\begin{NOTE}
As pandoc supports many number of formats, initial org-export-dispatch
shortcut menu does not show full of its supported formats. You can customize
org-pandoc-menu-entry variable (and probably restart Emacs) to change its
default menu entries.
If you want delayed loading of `ox-pandoc’ when org-pandoc-menu-entry
is customized, please consider the following settings in your init file"
\end{NOTE}

\begin{verbatim}
(use-package ox-pandoc
  :after (org ox)
  :config
  ;; default options for all output formats
  (setq org-pandoc-options '((standalone . t)))
  ;; cancel above settings only for 'docx' format
  (setq org-pandoc-options-for-docx '((standalone . nil)))
  ;; special settings for beamer-pdf and latex-pdf exporters
  (setq org-pandoc-options-for-beamer-pdf '((pdf-engine . "xelatex")))
  (setq org-pandoc-options-for-latex-pdf '((pdf-engine . "luatex")))
  ;; special extensions for markdown_github output
  (setq org-pandoc-format-extensions '(markdown_github+pipe_tables+raw_html))
)
\end{verbatim}

\subsection*{UTF8 pretty bullets in org mode}
\label{sec:org02fa59f}
\begin{verbatim}
(use-package org-bullets
  :config
  (add-hook 'org-mode-hook (lambda () (org-bullets-mode 1)))
)
\end{verbatim}

\subsection*{org-jira}
\label{sec:org7b1e877}

Org Mode Integration with Jira Projects

\begin{verbatim}
(use-package org-jira
  :ensure t
  :defer 3
  :after org
  :custom
  (jiralib-url "https://stairscreativestudio.atlassian.net")
)
\end{verbatim}

\subsubsection*{ox-jira exporter}
\label{sec:org1319f35}
\begin{verbatim}
(use-package ox-jira
  :defer 3
  :after org
)
\end{verbatim}


\subsection*{ReveaJS org-reveal:}
\label{sec:orgec86ce8}

\begin{verbatim}
This delay makes the options to export to RevealJS appear on the exporter menu (C-c C-e)
\end{verbatim}


\begin{verbatim}
(use-package ox-reveal
  :after ox
  :config
  (setq org-reveal-root "https://cdn.jsdelivr.net/reveal.js/3.0.0/")
)
\end{verbatim}

\subsection*{Org Exporters}
\label{sec:orgc150b4a}

\subsubsection*{markdown}
\label{sec:org6f812fd}
\begin{verbatim}
(use-package ox-md
  :defer t
  :after org
)
\end{verbatim}

\subsubsection*{github-flavored markdown}
\label{sec:orged70e2b}
\begin{verbatim}
(use-package ox-gfm
  :ensure t
  :defer t
  :after org
)
\end{verbatim}

\section*{Shell}
\label{sec:orgdb4ff86}

\subsection*{shell-pop}
\label{sec:org6c1f33d}

\begin{verbatim}
(use-package shell-pop
  :init
  (setq shell-pop-full-span t)
  (setq shell-pop-default-directory "~/code")
  (setq shell-pop-shell-type (quote ("ansi-term" "*ansi-term*" (lambda nil (ansi-term shell-pop-term-shell)))))
  (setq shell-pop-term-shell "/bin/zsh")
  (setq shell-pop-universal-key "C-c s")
  (setq shell-pop-window-size 30)
  (setq shell-pop-full-span t)
  (setq shell-pop-window-position "bottom")
  :bind
  ("C-c s" . shell-pop)
)
\end{verbatim}

\subsection*{System Shell}
\label{sec:orgab4e8a2}
\subsubsection*{Make system shell open in a split-window buffer at the bottom of the screen}
\label{sec:org0769584}

\begin{verbatim}
(defun /shell/new-window ()
    "Opens up a new shell in the directory associated with the current buffer's file."
    (interactive)
    (let* ((parent (if (buffer-file-name)
                        (file-name-directory (buffer-file-name))
                    default-directory))
            (height (/ (window-total-height) 3))
            (name   (car (last (split-string parent "/" t)))))
        (split-window-vertically (- height))
        (other-window 1)
        (shell "new")
        (rename-buffer (concat "*shell: " name "*"))
        (insert (concat "ls"))
    )
)

; Pull system shell in a new bottom window
(define-key evil-normal-state-map (kbd "\"") #'/shell/new-window)
(define-key evil-visual-state-map (kbd "\"") #'/shell/new-window)
(define-key evil-motion-state-map (kbd "\"") #'/shell/new-window)
\end{verbatim}


\subsection*{Eshell}
\label{sec:orgefd4be0}

\subsubsection*{Make eshell open in a split-window buffer at the bottom of the screen}
\label{sec:orgba4240a}

\begin{verbatim}
(defun /eshell/new-window ()
    "Opens up a new eshell in the directory associated with the current buffer's file.  The eshell is renamed to match that directory to make multiple eshell windows easier."
    (interactive)
    (let* ((parent (if (buffer-file-name)
                       (file-name-directory (buffer-file-name))
                     default-directory))
           (height (/ (window-total-height) 3))
           (name   (car (last (split-string parent "/" t)))))
      (split-window-vertically (- height))
      (other-window 1)
      (eshell "new")
      (rename-buffer (concat "*eshell: " name "*"))

      (insert (concat "ls"))
      (eshell-send-input)))

; Pull eshell in a new bottom window
(define-key evil-normal-state-map (kbd "!") #'/eshell/new-window)
(define-key evil-visual-state-map (kbd "!") #'/eshell/new-window)
(define-key evil-motion-state-map (kbd "!") #'/eshell/new-window)
\end{verbatim}

\section*{Helm}
\label{sec:org717f631}

\begin{verbatim}
(use-package helm
  :ensure t
  :bind
  ("M-x" . helm-M-x)
  ("M-x" . helm-M-x)
  ("C-c h" . helm-command-prefix)
  ("C-x b" . helm-buffers-list)
  ("C-x C-b" . helm-mini)
  ("C-x C-f" . helm-find-files)
  ("C-x r b" . helm-bookmarks)
  ("M-y" . helm-show-kill-ring)
  ("M-:" . helm-eval-expression-with-eldoc)
  (:map helm-map
  ("C-z" . helm-select-action)
  ("C-h a" . helm-apropos)
  ("C-c h" . helm-execute-persistent-action)
  ("<tab>" . helm-execute-persistent-action)
  )
  :init
  (setq helm-autoresize-mode t)
  (setq helm-buffer-max-length 40)
  (setq helm-bookmark-show-location t)
  (setq helm-buffer-max-length 40)
  (setq helm-split-window-inside-p t)

  ;; turn on helm fuzzy matching
  (setq helm-M-x-fuzzy-match t)
  (setq helm-mode-fuzzy-match t)

  (setq helm-ff-file-name-history-use-recentf t)
  (setq helm-ff-skip-boring-files t)
  (setq helm-follow-mode-persistent t)
  ;; take between 10-30% of screen space
  (setq helm-autoresize-min-height 10)
  (setq helm-autoresize-max-height 30)
  :config
  (require 'helm-config)
  (helm-mode 1)
  ;; Make helm replace the default Find-File and M-x
  (global-set-key [remap execute-extended-command] #'helm-M-x)
  (global-set-key [remap find-file] #'helm-find-files)
  ;; helm bindings
  (global-unset-key (kbd "C-x c"))
)
\end{verbatim}

\section*{helm-ag}
\label{sec:org570535f}

\begin{verbatim}
(use-package helm-ag
  :ensure helm-ag
  :bind ("M-p" . helm-projectile-ag)
  :commands (helm-ag helm-projectile-ag)
  :init (setq helm-ag-insert-at-point 'symbol
        helm-ag-command-option "--path-to-ignore ~/.agignore"))
\end{verbatim}

\section*{helm-rg}
\label{sec:org88c42e4}

\begin{verbatim}
(use-package helm-rg
  :ensure t
  :defer t
)
\end{verbatim}

\section*{ripgrep}
\label{sec:org0fa1f23}

\begin{verbatim}
(use-package rg
  :ensure t
  :defer t
  :ensure-system-package
  (rg . ripgrep)
  :config
  ;; choose between default keybindings or magit like menu interface.
  ;; both options are mutually exclusive
  (rg-enable-default-bindings)
  ;;(rg-enable-menu)

)
\end{verbatim}

\section*{helm-fuzzier}
\label{sec:orga18c405}
\begin{verbatim}
supposed better fuzzy matching for helm
for instance, plp, plpa, paclp, should all match package-list-packages
\end{verbatim}



\begin{verbatim}
(use-package helm-fuzzier
  :disabled nil
  :ensure t
  :after helm
  :config
  (helm-fuzzier-mode 1)
)
\end{verbatim}

\section*{FlyCheck linter}
\label{sec:org9616451}

\begin{verbatim}
(use-package flycheck
    :ensure t
    :defer t
    :hook
    (prog-mode . flycheck-mode)
    :custom
    (flycheck-display-errors-delay 1)
    :config
    (global-flycheck-mode)

    ;; add eslint to list of flycheck checkers
    (setq flycheck-checkers '(javascript-eslint))
    ;; disable jshint since we prefer eslint checking
    (setq-default flycheck-disabled-checkers (append flycheck-disabled-checkers '(javascript-jshint)))
    ;; force flycheck to use its own xml parser instead of libxml32 (was giving me errors)
    (setq flycheck-xml-parser 'flycheck-parse-xml-region)
    ;; set modes that will use ESLint
    (flycheck-add-mode 'javascript-eslint 'web-mode)
    (flycheck-add-mode 'javascript-eslint 'js2-mode)
    (flycheck-add-mode 'javascript-eslint 'js-mode)

    ;; customize flycheck temp file prefix
    (setq-default flycheck-temp-prefix ".flycheck")

    ;; disable json-jsonlist checking for json files
    (setq-default flycheck-disabled-checkers (append flycheck-disabled-checkers '(json-jsonlist)))

    ;; Workaround for eslint loading slow
    ;; https://github.com/flycheck/flycheck/issues/1129#issuecomment-319600923
    (advice-add 'flycheck-eslint-config-exists-p :override (lambda() t))
)

\end{verbatim}


\subsection*{flycheck inline}
\label{sec:orgd43be8a}

\begin{verbatim}
Quick peek is an extension that embelishes flycheck inline messages
\end{verbatim}


\begin{verbatim}
(use-package quick-peek
  :ensure t
)
\end{verbatim}

\begin{verbatim}
(use-package flycheck-inline
  :ensure t
  :hook
  (flycheck-mode . flycheck-inline-mode)
  :config
  ;; Set fringe style
  (setq flycheck-indication-mode 'right-fringe)

  (setq flycheck-mode-line-prefix "Syntax")

  ;; (global-flycheck-inline-mode)
  (setq flycheck-inline-display-function
        (lambda (msg pos)
          (let* ((ov (quick-peek-overlay-ensure-at pos))
                 (contents (quick-peek-overlay-contents ov)))
            (setf (quick-peek-overlay-contents ov)
                  (concat contents (when contents "\n") msg))
            (quick-peek-update ov)))
        flycheck-inline-clear-function #'quick-peek-hide))
\end{verbatim}


\subsection*{flycheck-pos-tip (show flycheck messages in tooltip)}
\label{sec:org1a0439b}

;;; Show Flycheck errors in tooltip

\begin{verbatim}
appearently depends on cask
\end{verbatim}


\begin{verbatim}
(use-package flycheck-pos-tip
  :ensure t
  :disabled
  :defines flycheck-pos-tip-timeout
  :after flycheck
  :hook
  (global-flycheck-mode . flycheck-pos-tip-mode)
  :config
  (setq flycheck-pos-tip-timeout 30)
  (flycheck-pos-tip-mode)
)
\end{verbatim}

\subsection*{flycheck-popup-tip (show flycheck messages in tooltip)}
\label{sec:orgd428ea3}

There is another official flycheck-pos-tip extension for displaying errors under point. However, it does not display popup if you run Emacs under TTY. It displays message on echo area and that is often used for ELDoc. Also, popups made by pos-tip library does not always look good, especially on macOS and Windows.
\begin{verbatim}
appearently depends on cask
\end{verbatim}


\begin{verbatim}
;;; Show Flycheck errors in tooltip
(use-package flycheck-popup-tip
  :ensure t
  :after flycheck
  :hook
  (flycheck-mode . flycheck-popup-tip-mode)
  :config
  (setq flycheck-popup-tip-error-prefix "\u27a4") ;;  display arrow like this: `➤'
  ;; (setq flycheck-popup-tip-error-prefix "* ")
)
\end{verbatim}


\section*{Version Control}
\label{sec:org7ab373e}

Always try to make bindings like this:
M-g for magit
C-c g for any git related stuff other than magit's

\subsection*{Magit}
\label{sec:org8e4c88d}

\begin{verbatim}
(use-package magit
  :ensure t
  :custom
  (magit-auto-revert-mode nil)
  :bind
  ("<tab>" . magit-section-toggle)
  ("M-g s" . magit-status)
  ("C-x g" . magit-status)
)
\end{verbatim}

\subsection*{evil-magit}
\label{sec:orge8458d2}
\begin{verbatim}
(use-package evil-magit
  :ensure t
  :init
  (setq evil-magit-state 'normal)
  (setq evil-magit-use-y-for-yank nil)
  :config
  (evil-magit-init)
  (evil-define-key evil-magit-state magit-mode-map "<tab>" 'magit-section-toggle)
  (evil-define-key evil-magit-state magit-mode-map "l" 'magit-log-popup)
  (evil-define-key evil-magit-state magit-mode-map "j" 'evil-next-visual-line)
  (evil-define-key evil-magit-state magit-mode-map "k" 'evil-previous-visual-line)
  ;(evil-define-key evil-magit-state magit-diff-map "k" 'evil-previous-visual-line)
  (evil-define-key evil-magit-state magit-staged-section-map "K" 'magit-discard)
  (evil-define-key evil-magit-state magit-unstaged-section-map "K" 'magit-discard)
  (evil-define-key evil-magit-state magit-unstaged-section-map "K" 'magit-discard)
  (evil-define-key evil-magit-state magit-branch-section-map "K" 'magit-branch-delete)
  (evil-define-key evil-magit-state magit-remote-section-map "K" 'magit-remote-remove)
  (evil-define-key evil-magit-state magit-stash-section-map "K" 'magit-stash-drop)
  (evil-define-key evil-magit-state magit-stashes-section-map "K" 'magit-stash-clear)
)
\end{verbatim}

\subsection*{magit-todo}
\label{sec:org1be0583}

\begin{verbatim}
(use-package magit-todos
  :ensure t
  :after magit
  :after hl-todo
  :bind
  ("M-g t" . magit-todos-list)
  :config
  (magit-todos-mode)
)
\end{verbatim}

\subsection*{diff-hl (highlights uncommited diffs in bar aside from the line numbers)}
\label{sec:org72d96e1}

\begin{verbatim}
(use-package diff-hl
  :ensure t
    :custom-face (diff-hl-change ((t (:foreground ,(face-background 'highlight)))))
  :hook
  (prog-mode . diff-hl-mode)
  (org-mode . diff-hl-mode)
  (dired-mode . diff-hl-mode)
  (magit-post-refresh . diff-hl-mode)
  :init
  ;; (add-hook 'prog-mode-hook #'diff-hl-mode)
  ;; (add-hook 'org-mode-hook #'diff-hl-mode)
  ;; (add-hook 'dired-mode-hook 'diff-hl-dired-mode)
  ;; (add-hook 'magit-post-refresh-hook 'diff-hl-magit-post-refresh)

  ;; Better looking colours for diff indicators
  (custom-set-faces
    '(diff-hl-change ((t (:background "#3a81c3"))))
    '(diff-hl-insert ((t (:background "#7ccd7c"))))
    '(diff-hl-delete ((t (:background "#ee6363"))))
  )

  :config

  (setq diff-hl-fringe-bmp-function 'diff-hl-fringe-bmp-from-type)
  (setq diff-hl-side 'left)
  (setq diff-hl-margin-side 'left)
  ;; Set fringe style
  (setq-default fringes-outside-margins t)


  (diff-hl-margin-mode 1) ;; show the indicators in the margin
  (diff-hl-flydiff-mode 1) ;;  ;; On-the-fly diff updates


  (unless (display-graphic-p)
  (setq diff-hl-margin-symbols-alist
        '((insert . " ") (delete . " ") (change . " ")
          (unknown . " ") (ignored . " ")))
  ;; Fall back to the display margin since the fringe is unavailable in tty
  (diff-hl-margin-mode 1)
  ;; Avoid restoring `diff-hl-margin-mode'
  (with-eval-after-load 'desktop
    (add-to-list 'desktop-minor-mode-table
                 '(diff-hl-margin-mode nil))))

    ;; Integration with magit
  (with-eval-after-load 'magit
    (add-hook 'magit-post-refresh-hook #'diff-hl-magit-post-refresh))

  (global-diff-hl-mode 1) ;; Enable diff-hl globally
)
\end{verbatim}

\subsection*{git-messenger}
\label{sec:orgef0963e}

\begin{verbatim}
(use-package git-messenger
  :ensure t
  :bind
  ("C-c g p" . git-messenger:popup-message)
  :init
  (setq git-messenger:show-detail t)
  (setq git-messenger:use-magit-popup t)
  :config
  (progn
    (define-key git-messenger-map (kbd "RET") 'git-messenger:popup-close))
)
\end{verbatim}

\subsection*{git-ignore mode}
\label{sec:orgbf26fad}


\subsection*{git-modes}
\label{sec:orgaea4829}

Major modes for git related files

\begin{verbatim}
;; Mode for .gitignore files.
(use-package gitignore-mode :ensure t :defer t)
(use-package gitconfig-mode :ensure t :defer t)
(use-package gitattributes-mode :ensure t :defer t)
\end{verbatim}

\subsection*{git-time-machine}
\label{sec:org3362177}

Navigation through the history of files

\begin{verbatim}
(use-package git-timemachine
  :ensure t
  :bind
  ("C-c g t" . git-timemachine-toggle)
)
\end{verbatim}


\begin{verbatim}
  (use-package forge
    :ensure t
    :after magit
    :config
    (setq forge-topic-list-limit '(30 . 5)
          forge-pull-notifications t)
)
\end{verbatim}


\section*{Projectile}
\label{sec:org3693f62}
\begin{verbatim}
(use-package projectile
  :ensure t
  :bind
  (:map projectile-mode-map
  ("s-p" . projectile-command-map)
  ("C-c p" . projectile-command-map)
  )
  :config
  (projectile-mode +1)
  (setq projectile-globally-ignored-files
        (append '("~"
                  ".swp"
                  ".pyc")
                projectile-globally-ignored-files))
)
\end{verbatim}

\begin{verbatim}
(use-package helm-projectile
  :ensure t
;  :after projectile
;  :demand t
  :config
  (helm-projectile-on)
)
\end{verbatim}

\section*{restclient}
\label{sec:orgb529589}

\begin{verbatim}
(use-package restclient
  :ensure t
  :mode
  ("\\.rest$\\'" "\\.http$\\'")
  :config
  (setq restclient-same-buffer-response t)
  (progn
    ;; Add hook to override C-c C-c in this mode to stay in window
    (add-hook 'restclient-mode-hook
              '(lambda ()
                 (local-set-key
                  (kbd "C-c C-c")
                  'restclient-http-send-current-stay-in-window))))
)
\end{verbatim}

\subsection*{ob-restclient}
\label{sec:org3308ef6}

Org source blocks and exporter for restclient

\begin{verbatim}
(use-package ob-restclient
  :ensure t
  :mode "\\.rest$"
  :config
  ;; add restclient to org-babel languages
  (org-babel-do-load-languages
   'org-babel-load-languages
   '((restclient . t)))
)
\end{verbatim}

\section*{undo-tree}
\label{sec:org427756e}
\begin{verbatim}
(use-package undo-tree
  :ensure t
  :init
  (global-undo-tree-mode)
;;  (undo-tree-mode)
)
\end{verbatim}

\section*{corral - intelligent surround text with auto-guess suggestions}
\label{sec:org7c54508}
\begin{verbatim}
(use-package corral
  :bind
  ("M-9" . corral-parentheses-backward)
  :config
  (setq corral-preserve-point t)
  ;;(global-set-key (kbd "M-9") 'corral-parentheses-backward)
  (global-set-key (kbd "M-0") 'corral-parentheses-forward)
  (global-set-key (kbd "M-[") 'corral-brackets-backward)
  (global-set-key (kbd "M-]") 'corral-brackets-forward)
  (global-set-key (kbd "M-{") 'corral-braces-backward)
  (global-set-key (kbd "M-}") 'corral-braces-forward)
  (global-set-key (kbd "M-\"") 'corral-double-quotes-backward)
)
\end{verbatim}

\section*{helpfull - a better replacement for emacs help system}
\label{sec:org07d39a4}

\begin{verbatim}
(use-package helpful
  :ensure t
  :config
  (global-set-key (kbd "C-h f") #'helpful-callable)
  (global-set-key (kbd "C-h v") #'helpful-variable)
  (global-set-key (kbd "C-h k") #'helpful-key)

  ;; Lookup the current symbol at point. C-c C-d is a common keybinding
  ;; for this in lisp modes.
  (global-set-key (kbd "C-c C-d") #'helpful-at-point)

  ;; Look up *F*unctions (excludes macros).
  ;;
  ;; By default, C-h F is bound to `Info-goto-emacs-command-node'. Helpful
  ;; already links to the manual, if a function is referenced there.
  (global-set-key (kbd "C-h F") #'helpful-function)

  ;; Look up *C*ommands.
  ;;
  ;; By default, C-h C is bound to describe `describe-coding-system'. I
  ;; don't find this very useful, but it's frequently useful to only
  ;; look at interactive functions.
  (global-set-key (kbd "C-h C") #'helpful-command)
)
\end{verbatim}
\section*{paredit}
\label{sec:orgebe15b5}

\begin{verbatim}
(use-package paredit
  :ensure t
  :config
  (autoload 'enable-paredit-mode "paredit" "Turn on pseudo-structural editing of Lisp code." t)
  (add-hook 'emacs-lisp-mode-hook       #'enable-paredit-mode)
  (add-hook 'eval-expression-minibuffer-setup-hook #'enable-paredit-mode)
  (add-hook 'ielm-mode-hook             #'enable-paredit-mode)
  (add-hook 'lisp-mode-hook             #'enable-paredit-mode)
  (add-hook 'lisp-interaction-mode-hook #'enable-paredit-mode)
  (add-hook 'scheme-mode-hook           #'enable-paredit-mode)
)
\end{verbatim}



\section*{parinfer-mode}
\label{sec:orgbc15cbd}

\begin{verbatim}
(use-package parinfer
  :ensure t
  :bind
  ("C-," . parinfer-toggle-mode)
  :init
  (progn
    (setq parinfer-extensions
          '(defaults       ; should be included.
            pretty-parens  ; different paren styles for different modes.
            evil           ; If you use Evil.
            ;lispy          ; If you use Lispy. With this extension, you should install Lispy and do not enable lispy-mode directly.
            paredit        ; Introduce some paredit commands.
            smart-tab      ; C-b & C-f jump positions and smart shift with tab & S-tab.
            smart-yank))   ; Yank behavior depend on mode.
    (add-hook 'clojure-mode-hook #'parinfer-mode)
    (add-hook 'emacs-lisp-mode-hook #'parinfer-mode)
    (add-hook 'common-lisp-mode-hook #'parinfer-mode)
    (add-hook 'scheme-mode-hook #'parinfer-mode)
    (add-hook 'lisp-mode-hook #'parinfer-mode))
    :config
    ;; auto switch to Indent Mode whenever parens are balance in Paren Mode
    (setq parinfer-auto-switch-indent-mode nil)  ;; default is nil
    (setq parinfer-lighters '(" Parinfer:Indent" . "Parinfer:Paren"))

)
\end{verbatim}

\section*{elisp-format}
\label{sec:org1c31439}

\begin{verbatim}
(use-package elisp-format
  :ensure t
)
\end{verbatim}

\section*{Company}
\label{sec:orgbc740a0}

\begin{verbatim}
(use-package company
  :ensure t
  :defer t
  :init
  (global-company-mode)
  :bind
  (:map evil-insert-state-map
  ;; ("<tab>" . company-indent-or-complete-common)
  ("C-SPC" . company-indent-or-complete-common)
  )
  (:map company-active-map
  ("M-n" . nil)
  ("M-p" . nil)
  ("C-n" . company-select-next)
  ("C-p" . company-select-previous)
  ("<tab>" . company-complete-common-or-cycle)
  ("S-<tab>" . company-select-previous)
  ("<backtab>" . company-select-previous)
  ("C-d" . company-show-doc-buffer)
  )
  (:map company-search-map
   ("C-p" . company-select-previous)
   ("C-n" . company-select-next)
  )
  :config
  ;; define company appearance
  (custom-set-faces
  '(company-preview-common ((t (:foreground unspecified :background "#111111"))))
  '(company-scrollbar-bg ((t (:background "#111111"))))
  '(company-scrollbar-fg ((t (:background "#555555"))))
  '(company-tooltip ((t (:inherit default :background "#222222"))))
  '(company-tooltip-common ((t (:inherit font-lock-constant-face))))
  '(company-tooltip-selection ((t (:inherit company-tooltip-common :background "#2a2a2a" )))))

  ;; Use Company for completion
  (progn
    (bind-key [remap completion-at-point] #'company-complete company-mode-map))
  (setq company-tooltip-limit 20)                      ; bigger popup window
  (setq company-minimum-prefix-length 1)               ; start completing after 1st char typed
  (setq company-idle-delay 0)                         ; decrease delay before autocompletion popup shows
  (setq company-echo-delay 0)                          ; remove annoying blinking
  (setq company-begin-commands '(self-insert-command)) ; start autocompletion only after typing
  ;; company-dabbrev
  (setq company-dabbrev-downcase nil)                  ;; Do not downcase completions by default.
  (setq company-dabbrev-ignore-case t)  ;; Even if I write something with the ‘wrong’ case, provide the ‘correct’ casing.
  (setq company-dabbrev-code-everywhere t)
  (setq company-dabbrev-other-buffers t)
  (setq company-dabbrev-code-other-buffers t)
  (setq company-selection-wrap-around t)               ; continue from top when reaching bottom
  (setq company-auto-complete 'company-explicit-action-p)
  (setq company-require-match nil)
  (setq company-tooltip-align-annotations t)
  (setq company-complete-number t)                     ;; Allow (lengthy) numbers to be eligible for completion.
  (setq company-show-numbers t)  ;; M-⟪num⟫ to select an option according to its number.
  (setq company-transformers '(company-sort-by-occurrence)) ; weight by frequency
  ;; (setq company-tooltip-flip-when-above t)
  ;; DELETE THIS PART IF USE PACKAGE :MAP WORKS
  ;; (define-key company-active-map (kbd "M-n") nil)
  ;; (define-key company-active-map (kbd "M-p") nil)
  ;; (define-key company-active-map (kbd "C-n") 'company-select-next)
  ;; (define-key company-active-map (kbd "C-p") 'company-select-previous)
  ;; (define-key company-active-map (kbd "TAB") 'company-complete-common-or-cycle)
  ;; (define-key company-active-map (kbd "<tab>") 'company-complete-common-or-cycle)
  ;; (define-key company-active-map (kbd "S-TAB") 'company-select-previous)
  ;; (define-key company-active-map (kbd "<backtab>") 'company-select-previous)
)
\end{verbatim}

\subsection*{Company emoji suport}
\label{sec:org62a4b91}

\begin{verbatim}
use `:` to use emojis
\end{verbatim}

\begin{verbatim}
(use-package company-emoji
  :ensure t
  :config
  (add-to-list 'company-backends 'company-emoji)
)
\end{verbatim}

\subsection*{Company-QuickHelp}
\label{sec:org381139e}
\begin{verbatim}
(use-package company-quickhelp          ; Documentation popups for Company
   :ensure t
   ;; :defer t
   :hook
   (global-company-mode . company-quickhelp-mode)
   :bind
   (:map company-active-map
   ("M-h" . company-quickhelp-manual-begin)
   )
   :config
   (setq company-quickhelp-delay 0.7)
   (company-quickhelp-mode)
)
\end{verbatim}

\subsection*{Company postframe}
\label{sec:orgd51a39c}

\begin{verbatim}
PS: this looks exactly the same as the usual company popup, except it doesn't disturb other overlays (like line numbers) in the buffer.
\end{verbatim}

\begin{verbatim}
(use-package company-posframe
  :ensure t
  :hook
  (company-mode . company-posframe-mode)
  (global-company-mode . company-posframe-mode)
)
\end{verbatim}

\subsection*{Company Box (icons in suggestions)}
\label{sec:org8a7406a}
\begin{verbatim}
  ;; (use-package company-box
  ;;   :ensure t
  ;;   :hook
  ;;   (company-mode . company-box-mode)
  ;;   (global-company-mode . company-box-mode)
  ;; )
(use-package company-box
  :ensure t
  :hook
  (company-mode . company-box-mode)
  :init
  (setq company-box-icons-alist 'company-box-icons-all-the-icons)
  :config
  (setq company-box-backends-colors nil)
  (setq company-box-show-single-candidate t)
  (setq company-box-max-candidates 50)

  (defun company-box-icons--elisp (candidate)
    (when (derived-mode-p 'emacs-lisp-mode)
      (let ((sym (intern candidate)))
        (cond ((fboundp sym) 'Function)
              ((featurep sym) 'Module)
              ((facep sym) 'Color)
              ((boundp sym) 'Variable)
              ((symbolp sym) 'Text)
              (t . nil)))))

  (with-eval-after-load 'all-the-icons
    (declare-function all-the-icons-faicon 'all-the-icons)
    (declare-function all-the-icons-fileicon 'all-the-icons)
    (declare-function all-the-icons-material 'all-the-icons)
    (declare-function all-the-icons-octicon 'all-the-icons)
    (setq company-box-icons-all-the-icons
          `((Unknown . ,(all-the-icons-material "find_in_page" :height 0.7 :v-adjust -0.15))
            (Text . ,(all-the-icons-faicon "book" :height 0.68 :v-adjust -0.15))
            (Method . ,(all-the-icons-faicon "cube" :height 0.7 :v-adjust -0.05 :face 'font-lock-constant-face))
            (Function . ,(all-the-icons-faicon "cube" :height 0.7 :v-adjust -0.05 :face 'font-lock-constant-face))
            (Constructor . ,(all-the-icons-faicon "cube" :height 0.7 :v-adjust -0.05 :face 'font-lock-constant-face))
            (Field . ,(all-the-icons-faicon "tags" :height 0.65 :v-adjust -0.15 :face 'font-lock-warning-face))
            (Variable . ,(all-the-icons-faicon "tag" :height 0.7 :v-adjust -0.05 :face 'font-lock-warning-face))
            (Class . ,(all-the-icons-faicon "clone" :height 0.65 :v-adjust 0.01 :face 'font-lock-constant-face))
            (Interface . ,(all-the-icons-faicon "clone" :height 0.65 :v-adjust 0.01))
            (Module . ,(all-the-icons-octicon "package" :height 0.7 :v-adjust -0.15))
            (Property . ,(all-the-icons-octicon "package" :height 0.7 :v-adjust -0.05 :face 'font-lock-warning-face)) ;; Golang module
            (Unit . ,(all-the-icons-material "settings_system_daydream" :height 0.7 :v-adjust -0.15))
            (Value . ,(all-the-icons-material "format_align_right" :height 0.7 :v-adjust -0.15 :face 'font-lock-constant-face))
            (Enum . ,(all-the-icons-material "storage" :height 0.7 :v-adjust -0.15 :face 'all-the-icons-orange))
            (Keyword . ,(all-the-icons-material "filter_center_focus" :height 0.7 :v-adjust -0.15))
            (Snippet . ,(all-the-icons-faicon "code" :height 0.7 :v-adjust 0.02 :face 'font-lock-variable-name-face))
            (Color . ,(all-the-icons-material "palette" :height 0.7 :v-adjust -0.15))
            (File . ,(all-the-icons-faicon "file-o" :height 0.7 :v-adjust -0.05))
            (Reference . ,(all-the-icons-material "collections_bookmark" :height 0.7 :v-adjust -0.15))
            (Folder . ,(all-the-icons-octicon "file-directory" :height 0.7 :v-adjust -0.05))
            (EnumMember . ,(all-the-icons-material "format_align_right" :height 0.7 :v-adjust -0.15 :face 'all-the-icons-blueb))
            (Constant . ,(all-the-icons-faicon "tag" :height 0.7 :v-adjust -0.05))
            (Struct . ,(all-the-icons-faicon "clone" :height 0.65 :v-adjust 0.01 :face 'font-lock-constant-face))
            (Event . ,(all-the-icons-faicon "bolt" :height 0.7 :v-adjust -0.05 :face 'all-the-icons-orange))
            (Operator . ,(all-the-icons-fileicon "typedoc" :height 0.65 :v-adjust 0.05))
            (TypeParameter . ,(all-the-icons-faicon "hashtag" :height 0.65 :v-adjust 0.07 :face 'font-lock-const-face))
            (Template . ,(all-the-icons-faicon "code" :height 0.7 :v-adjust 0.02 :face 'font-lock-variable-name-face)))))
)
\end{verbatim}


\subsection*{company-go}
\label{sec:org4a609c9}
\begin{verbatim}
(use-package company-go
  :ensure t
  :defer t
  :init
  (with-eval-after-load 'company
    (add-to-list 'company-backends 'company-go))
)
\end{verbatim}

\subsection*{Company TabNine}
\label{sec:orgd966fe9}
\begin{verbatim}
this part is commented because TabNine is paid software
and not sure if i want to use it
\end{verbatim}

\begin{verbatim}
;; (use-package company-tabnine
;;   :demand
;;   :custom
;;   (company-tabnine-max-num-results 9)
;;   :bind
;;   (("M-q" . company-other-backend)
;;    ("C-z t" . company-tabnine))
;;   :config
;;   ;; Enable TabNine on default
;;   (add-to-list 'company-backends #'company-tabnine)

;;   ;; Integrate company-tabnine with lsp-mode
;;   (defun company//sort-by-tabnine (candidates)
;;     (if (or (functionp company-backend)
;;             (not (and (listp company-backend) (memq 'company-tabnine company-backend))))
;;         candidates
;;       (let ((candidates-table (make-hash-table :test #'equal))
;;             candidates-lsp
;;             candidates-tabnine)
;;         (dolist (candidate candidates)
;;           (if (eq (get-text-property 0 'company-backend candidate)
;;                   'company-tabnine)
;;               (unless (gethash candidate candidates-table)
;;                 (push candidate candidates-tabnine))
;;             (push candidate candidates-lsp)
;;             (puthash candidate t candidates-table)))
;;         (setq candidates-lsp (nreverse candidates-lsp))
;;         (setq candidates-tabnine (nreverse candidates-tabnine))
;;         (nconc (seq-take candidates-tabnine 3)
;;                (seq-take candidates-lsp 6)))))
;;   (add-hook 'lsp-after-open-hook
;;             (lambda ()
;;               (setq company-tabnine-max-num-results 3)
;;               (add-to-list 'company-transformers 'company//sort-by-tabnine t)
;;               (add-to-list 'company-backends '(company-lsp :with company-tabnine :separate)))))
\end{verbatim}

\section*{LSP}
\label{sec:orgc32415b}

\subsection*{LSP (language server protocol implementation for emacs)}
\label{sec:orgef7fddc}

\begin{verbatim}
(use-package lsp-mode
  :ensure t
  :commands lsp
  :init
  (setq lsp-inhibit-message nil) ;; was `t`, changed to nil to see what it does
  (setq lsp-eldoc-render-all t)  ;; was `nil`, changed to nil to see what it does
  (setq lsp-highlight-symbol-at-point t)  ;; was `nil`, changed to nil to see what it does
  (setq lsp-print-io nil)
  (setq lsp-trace nil)
  (setq lsp-print-performance nil)
  (setq lsp-prefer-flymake t) ;; t(flymake), nil(lsp-ui), or :none

  :hook
  ;; disabled lsp for javascript and typescript to use Tide-mode only
  ;; disabled lsp for typescript to use Tide-mode only
  ;;(typescript-mode . lsp)
  ;;(js2-mode . lsp)
  (js2-jsx-mode . lsp)
  (enh-ruby-mode . lsp)
)

\end{verbatim}

\begin{verbatim}
(use-package company-lsp
  :ensure t
  :custom
  ;; debug
  (lsp-print-io nil)
  (lsp-trace nil)
  (lsp-print-performance nil)
  ;; general
  (lsp-auto-guess-root t)
  (lsp-document-sync-method 'incremental) ;; none, full, incremental, or nil
  (lsp-response-timeout 10)
  (lsp-prefer-flymake nil) ;; t(flymake), nil(lsp-ui), or :none
  :config
  (setq company-lsp-enable-snippet t)
  (setq company-lsp-async t)
  ;; When `company-lsp-cache-candidates' is setting is auto, company-lsp will filter the candidates client side without retrieving them from the server when you type. This means that the candidates might not be the same(e. g. might be sorted in a different order) but from a functional standpoint of view you should not notice any difference.
  ;;(setq company-lsp-cache-candidates t)
  (setq company-lsp-cache-candidates 'auto)
  (setq company-lsp-enable-recompletion t)
)
\end{verbatim}

\begin{verbatim}
(use-package lsp-ui
  :ensure t
  :hook
  (lsp-mode . lsp-ui-mode)
  :preface
  (defun ladicle/toggle-lsp-ui-doc ()
    (interactive)
    (if lsp-ui-doc-mode
      (progn
        (lsp-ui-doc-mode -1)
        (lsp-ui-doc--hide-frame)
      )
    (lsp-ui-doc-mode 1)
    )
  )
  :config
  ;; lsp-ui appearance
  (set-face-attribute 'lsp-ui-doc-background  nil :background "#f9f2d9")
  (add-hook 'lsp-ui-doc-frame-hook
    (lambda (frame _w)
      (set-face-attribute 'default frame :font "Overpass Mono 11")
    )
  )
  (set-face-attribute 'lsp-ui-sideline-global nil
                      :inherit 'shadow
                      :background "#f9f2d9")
  (setq ;; lsp-ui-doc
        lsp-ui-doc-enable t
        lsp-ui-doc-header t
        lsp-ui-doc-include-signature nil
        lsp-ui-doc-position 'at-point ;; top, bottom, or at-point
        lsp-ui-doc-max-width 100
        lsp-ui-doc-max-height 30
        lsp-ui-doc-use-childframe t
        lsp-ui-doc-use-webkit t
        ;; lsp-ui-flycheck
        lsp-ui-flycheck-enable t
        lsp-ui-flycheck-list-position 'right
        lsp-ui-flycheck-live-reporting t
        ;; lsp-ui-sideline
        lsp-ui-sideline-enable t
        lsp-ui-sideline-ignore-duplicate t
        lsp-ui-sideline-show-symbol t
        lsp-ui-sideline-show-hover t
        lsp-ui-sideline-show-diagnostics nil
        lsp-ui-sideline-show-code-actions t
        lsp-ui-sideline-code-actions-prefix ""
        lsp-ui-sideline-update-mode 'point
        ;; lsp-ui-imenu
        lsp-ui-imenu-enable t
        lsp-ui-imenu-kind-position 'top
        ;; lsp-ui-peek
        lsp-ui-peek-enable t
        lsp-ui-peek-peek-height 20
        lsp-ui-peek-list-width 40
        lsp-ui-peek-fontify 'on-demand) ;; never, on-demand, or always
  :bind
    (:map lsp-mode-map
      ("C-c C-r" . lsp-ui-peek-find-references)
      ("C-c C-j" . lsp-ui-peek-find-definitions)
      ("C-c g d" . lsp-goto-type-definition)
      ("C-c f d" . lsp-find-definition)
      ("C-c g i" . lsp-goto-implementation)
      ("C-c f i" . lsp-find-implementation)
      ("C-c i"   . lsp-ui-peek-find-implementation)
      ("C-c m"   . lsp-ui-imenu)
      ("C-c s"   . lsp-ui-sideline-mode)
      ("C-c d"   . ladicle/toggle-lsp-ui-doc)
    )
    ;; remap native find-definitions and references to use lsp-ui
    (:map lsp-ui-mode-map
      ([remap xref-find-definitions] . lsp-ui-peek-find-definitions)
      ([remap xref-find-references] . lsp-ui-peek-find-references)
      ("C-c u" . lsp-ui-imenu)
    )
)
\end{verbatim}

\subsection*{Disable <RET> for autocomplete and leave on TAB}
\label{sec:orgd54f4b0}
\begin{verbatim}
;; (define-key ac-completing-map [return] nil)
;; (define-key ac-completing-map "\r" nil)
\end{verbatim}


\subsection*{enable autocompletion engine}
\label{sec:org9ab53f9}
\begin{verbatim}
(require 'auto-complete)
(global-auto-complete-mode t)
\end{verbatim}

\section*{Yasnippets}
\label{sec:org25509b2}

\begin{verbatim}
(use-package yasnippet
  :ensure t
  :hook
  (prog-mode . yas-minor-mode)
  (text-mode . yas-minor-mode)
  :bind
  ;; ("<tab>" . yas-maybe-expand)
  ("C-<tab>" . yas-maybe-expand)
  (:map yas-minor-mode-map
  ;; yas-maybe-expand only expands if there are candidates.
  ;; if not, acts like binding is unbound and run whatever command is bound to that key normally
  ;; ("<tab>" . yas-maybe-expand)
  ;; Bind `C-c y' to `yas-expand' ONLY.
  ("C-c y" . yas-expand)
  ("C-SPC" . yas-expand)
  )
  :config
  ;; set snippets directory
  ;; (with-eval-after-load 'yasnippet
  ;;  (setq yas-snippet-dirs '(yasnippet-snippets-dir)))
  (setq yas-verbosity 1)                      ; No need to be so verbose
  (setq yas-wrap-around-region t)
  (yas-reload-all)
  ;; disabled global mode in favor or hooks in prog and text modes only
  ;; (yas-global-mode 1)
)
\end{verbatim}

\begin{verbatim}
(use-package yasnippet-snippets         ; Collection of snippets
  :ensure t
)
\end{verbatim}


\section*{File Explorers}
\label{sec:orgf884f24}
\subsection*{Dired}
\label{sec:org4410698}

\begin{verbatim}
(use-package dired-k
  :after dired
  :config
  (setq dired-k-style 'git)
  (setq dired-k-human-readable t)
  (setq dired-dwin-target t)
  (add-hook 'dired-initial-position-hook #'dired-k)
)
\end{verbatim}

\begin{verbatim}
 (use-package all-the-icons-dired
  :ensure t
  :defer t
  :hook
  (dired-mode . all-the-icons-dired-mode)
)
\end{verbatim}

\subsection*{Treemacs (neotree like navigation)}
\label{sec:orgdc99c63}


\subsubsection*{Treemacs itself}
\label{sec:orgd23273f}
\begin{verbatim}
 (use-package treemacs
   :ensure t
   :defer t
   :hook
   (after-init . treemacs)
   :bind
   (:map global-map
         ("<f8>"       . treemacs)
         ("M-0"       . treemacs-select-window)
         ("C-x t 1"   . treemacs-delete-other-windows)
         ("C-x t t"   . treemacs)
         ("C-x t B"   . treemacs-bookmark)
         ("C-x t C-t" . treemacs-find-file)
         ("C-x t M-t" . treemacs-find-tag))
   :init
   ;; general variables
   (setq treemacs-no-png-images nil)
   (setq treemacs-deferred-git-apply-delay 0.5)
   (setq treemacs-display-in-side-window t)
   (setq treemacs-eldoc-display t)
   (setq treemacs-file-event-delay 5000)
   (setq treemacs-file-follow-delay 0.2)
   (setq treemacs-follow-after-init t)
   (setq treemacs-git-command-pipe "")
   (setq treemacs-goto-tag-strategy 'refetch-index)
   (setq treemacs-indentation 2)
   (setq treemacs-indentation-string " ")
   (setq treemacs-is-never-other-window nil)
   (setq treemacs-max-git-entries 5000)
   (setq treemacs-missing-project-action 'ask)
   (setq treemacs-no-delete-other-windows t)
   (setq treemacs-project-follow-cleanup nil)
   (setq treemacs-persist-file (expand-file-name ".cache/treemacs-persist" user-emacs-directory))
   (setq treemacs-position 'left)
   (setq treemacs-recenter-distance 0.1)
   (setq treemacs-recenter-after-file-follow nil)
   (setq treemacs-recenter-after-tag-follow nil)
   (setq treemacs-recenter-after-project-jump 'always)
   (setq treemacs-recenter-after-project-expand 'on-distance)
   (setq treemacs-show-cursor nil)
   (setq treemacs-show-hidden-files t)
   (setq treemacs-silent-filewatch nil)
   (setq treemacs-silent-refresh nil)
   (setq treemacs-sorting 'alphabetic-desc)
   (setq treemacs-space-between-root-nodes t)
   (setq treemacs-tag-follow-cleanup t)
   (setq treemacs-tag-follow-delay 1.5)
   (setq treemacs-width 35)
   :config

   (setq treemacs-collapse-dirs (if treemacs-python-executable 3 0))
   (add-hook 'treemacs-mode-hook #'hide-mode-line-mode)
   (add-hook 'treemacs-mode-hook (lambda ()
                                   (linum-mode -1)
                                   (fringe-mode 0)
                                   ;; (setq buffer-face-mode-face `(:background "#211C1C"))
                                   (buffer-face-mode 1)))


   (treemacs-follow-mode t)
   (treemacs-filewatch-mode t)
   (treemacs-fringe-indicator-mode t)
   (pcase (cons (not (null (executable-find "git")))
                (not (null treemacs-python-executable)))
     (`(t . t)
      (treemacs-git-mode 'deferred))
     (`(t . _)
      (treemacs-git-mode 'simple)))

  ;; my vscode icon theme, using vscode-icons
  ;; i just placed the vscode-icons folder inside the treemacs icons vfolder and changed paths
  (defcustom tau-themes-treemacs-theme "vscode"
  "Default treemacs theme."
  :type '(radio (const :doc "A minimalistic atom-inspired icon theme" "doom-atom")
                (const :doc "A colorful icon theme leveraging all-the-icons" "doom-colors"))
  :group 'tau-themes-treemacs)

  ;; warn if all-the-icons isnt sinstalled
  (with-eval-after-load 'treemacs
  (unless (require 'all-the-icons nil t)
    (error "all-the-icons isn't installed"))

  (let ((face-spec '(:inherit font-lock-doc-face :slant normal)))  ;; taken from doom-treemacs -theme to use all theicons in some parts
  (treemacs-create-theme "vscode"
    :icon-directory (f-join treemacs-dir "icons/default")
    :config
    (progn
      ;; directory and other icons
      ;; (treemacs-create-icon :file "root.png"        :extensions (root)       :fallback "")
       (treemacs-create-icon
         :icon (format " %s\t" (all-the-icons-octicon "repo" :height 1.2 :v-adjust -0.1 :face face-spec))
         :extensions (root))
      (treemacs-create-icon :file "vscode/default_folder.png"  :extensions (dir-closed) :fallback (propertize "+ " 'face 'treemacs-term-node-face))
      (treemacs-create-icon :file "vscode/default_folder_opened.png"    :extensions (dir-open)   :fallback (propertize "- " 'face 'treemacs-term-node-face))
      (treemacs-create-icon :file "tags-leaf.png"   :extensions (tag-leaf)   :fallback (propertize "• " 'face 'font-lock-constant-face))
      (treemacs-create-icon :file "tags-open.png"   :extensions (tag-open)   :fallback (propertize "▸ " 'face 'font-lock-string-face))
      (treemacs-create-icon :file "tags-closed.png" :extensions (tag-closed) :fallback (propertize "▾ " 'face 'font-lock-string-face))
      (treemacs-create-icon :file "error.png"       :extensions (error)      :fallback (propertize "• " 'face 'font-lock-string-face))
      (treemacs-create-icon :file "warning.png"     :extensions (warning)    :fallback (propertize "• " 'face 'font-lock-string-face))
      (treemacs-create-icon :file "info.png"        :extensions (info)       :fallback (propertize "• " 'face 'font-lock-string-face))

      ;; common file types icons
      (treemacs-create-icon :file "vscode/default_file.png"         :extensions (fallback))
      (treemacs-create-icon :file "vscode/image.png"       :extensions ("jpg" "jpeg" "bmp" "svg" "png" "xpm" "gif"))
      (treemacs-create-icon :file "vscode/video.png"       :extensions ("webm" "mp4" "avi" "mkv" "flv" "mov" "wmv" "mpg" "mpeg" "mpv"))
      (treemacs-create-icon :file "vscode/pdf.png"         :extensions ("pdf"))
      (treemacs-create-icon :file "vscode/emacs.png"       :extensions ("el" "elc"))
      (treemacs-create-icon :file "ledger.png"      :extensions ("ledger"))
      (treemacs-create-icon :file "vscode/config.png"        :extensions ("properties" "conf" "config" "cfg" "ini" "xdefaults" "xresources" "terminalrc" "ledgerrc"))
      (treemacs-create-icon :file "vscode/shell.png"       :extensions ("sh" "zsh" "fish"))
      (treemacs-create-icon :file "asciidoc.png"    :extensions ("adoc" "asciidoc"))
      ;; git
      (treemacs-create-icon :file "vscode/git.png"         :extensions ("git" "gitignore" "gitconfig" "gitmodules" "gitattributes"))
      ;; dev lib
      (treemacs-create-icon :file "vscode/editorconfig.png"         :extensions ("editorconfig"))
      ;; frontend universe
      (treemacs-create-icon :file "vscode/json.png"        :extensions ("json"))
      (treemacs-create-icon :file "vscode/html.png"        :extensions ("html" "htm"))
      (treemacs-create-icon :file "vscode/css.png"         :extensions ("css"))
      (treemacs-create-icon :file "vscode/scss.png"         :extensions ("scss"))
      (treemacs-create-icon :file "vscode/js_official.png"          :extensions ("js" "jsx"))
      (treemacs-create-icon :file "vscode/typescript.png"          :extensions ("ts" "tsx"))
      (treemacs-create-icon :file "vscode/typescriptdef.png"          :extensions ("spec"))
      (treemacs-create-icon :file "vscode/tslint.png"          :extensions ("tslint"))
      (treemacs-create-icon :file "vscode/tsconfig.png"          :extensions ("tsconfig"))
      (treemacs-create-icon :file "vscode/vue.png"         :extensions ("vue"))
      (treemacs-create-icon :file "vscode/elm.png"         :extensions ("elm"))
      ;; markupgs
      (treemacs-create-icon :file "vscode/org.png"     :extensions ("org"))
      (treemacs-create-icon :file "vscode/markdown.png"    :extensions ("md"))
      (treemacs-create-icon :file "vscode/tex.png"         :extensions ("tex"))
      (treemacs-create-icon :file "vscode/yaml.png"        :extensions ("yml" "yaml"))
      (treemacs-create-icon :file "vscode/toml.png"        :extensions ("toml"))
      (treemacs-create-icon :file "vscode/dartlang.png"        :extensions ("dart"))
      (treemacs-create-icon :file "vscode/julia.png"       :extensions ("jl"))
      ;; erlang / elixir
      (treemacs-create-icon :file "vscode/erlang2.png"      :extensions ("erl" "hrl"))
      (treemacs-create-icon :file "vscode/elixir.png"         :extensions ("ex"))
      (treemacs-create-icon :file "elx-light.png"   :extensions ("exs" "eex"))
      ;; ruby
      (treemacs-create-icon :file "ruby.png"   :extensions ("rb"))
      (treemacs-create-icon :file "erb.png"   :extensions ("erb"))
      ;; backend languages file types
      (treemacs-create-icon :file "vscode/rust.png"        :extensions ("rs"))
      (treemacs-create-icon :file "vscode/clojure.png"     :extensions ("clj" "cljs" "cljc"))
      (treemacs-create-icon :file "vscode/java.png"        :extensions ("java"))
      (treemacs-create-icon :file "vscode/kotlin.png"      :extensions ("kt"))
      (treemacs-create-icon :file "vscode/scala.png"       :extensions ("scala"))
      (treemacs-create-icon :file "sbt.png"         :extensions ("sbt"))
      (treemacs-create-icon :file "vscode/go.png"          :extensions ("go"))
      (treemacs-create-icon :file "vscode/php.png"         :extensions ("php"))
      (treemacs-create-icon :file "vscode/c.png"           :extensions ("c" "h"))
      (treemacs-create-icon :file "vscode/cpp.png"         :extensions ("cpp" "cxx" "hpp" "tpp" "cc" "hh"))
      ;; lisp ecosystem
      (treemacs-create-icon :file "racket.png"      :extensions ("racket" "rkt" "rktl" "rktd" "scrbl" "scribble" "plt"))
      ;; haskell
      (treemacs-create-icon :file "vscode/haskell.png"     :extensions ("hs" "lhs"))
      (treemacs-create-icon :file "cabal.png"       :extensions ("cabal"))
      ;; python
      (treemacs-create-icon :file "python.png"      :extensions ("py" "pyc"))
      (treemacs-create-icon :file "hy.png"          :extensions ("hy"))
      (treemacs-create-icon :file "ocaml.png"       :extensions ("ml" "mli"))
      (treemacs-create-icon :file "puppet.png"      :extensions ("pp"))
      ;; devops tools
      (treemacs-create-icon :file "vscode/docker.png"      :extensions ("dockerfile"))
      (treemacs-create-icon :file "vagrant.png"     :extensions ("vagrantfile"))
      (treemacs-create-icon :file "jinja2.png"      :extensions ("j2" "jinja2"))
      (treemacs-create-icon :file "purescript.png"  :extensions ("purs"))
      (treemacs-create-icon :file "nix.png"         :extensions ("nix"))
      (treemacs-create-icon :file "scons.png"       :extensions ("sconstruct" "sconstript"))
      (treemacs-create-icon :file "vscode/make.png"    :extensions ("makefile"))
      (treemacs-create-icon :file "vscode/license.png" :extensions ("license"))
      (treemacs-create-icon :file "vscode/zip.png"     :extensions ("zip" "7z" "tar" "gz" "rar"))
      (treemacs-create-icon :file "vscode/elm.png"     :extensions ("elm"))
      (treemacs-create-icon :file "vscode/xml.png"     :extensions ("xml" "xsl"))
      (treemacs-create-icon :file "vscode/binary.png"  :extensions ("exe" "dll" "obj" "so" "o"))
      (treemacs-create-icon :file "vscode/ruby.png"    :extensions ("rb"))
      (treemacs-create-icon :file "vscode/scss.png"    :extensions ("scss"))
      (treemacs-create-icon :file "vscode/lua.png"     :extensions ("lua"))
      (treemacs-create-icon :file "vscode/log.png"     :extensions ("log"))
      (treemacs-create-icon :file "vscode/lisp.png"    :extensions ("lisp"))
      (treemacs-create-icon :file "vscode/sql.png"     :extensions ("sql"))
      (treemacs-create-icon :file "vscode/nim.png"     :extensions ("nim"))
      (treemacs-create-icon :file "vscode/perl.png"    :extensions ("pl" "pm" "perl"))
      (treemacs-create-icon :file "vscode/vim.png"     :extensions ("vimrc" "tridactylrc" "vimperatorrc" "ideavimrc" "vrapperrc"))
      (treemacs-create-icon :file "vscode/deps.png"    :extensions ("cask"))
      (treemacs-create-icon :file "vscode/r.png"       :extensions ("r"))
      (treemacs-create-icon :file "vscode/reason.png"  :extensions ("re" "rei")))))

  ;; finally apply the custom theme
  (treemacs-load-theme tau-themes-treemacs-theme))

   ;;apply treemacs icon theme
   (treemacs-load-theme "vscode")
   (treemacs-resize-icons 18) ;; usefull on high dpi monitors.  default icon size is 22

)
\end{verbatim}

\subsubsection*{Treemacs Evil}
\label{sec:orgeab1d9c}
\begin{verbatim}
(use-package treemacs-evil
  :after treemacs evil
  :ensure t)
\end{verbatim}

\subsubsection*{Treemacs Projectile}
\label{sec:org653b661}
\begin{verbatim}
(use-package treemacs-projectile
  :after treemacs projectile
  :ensure t)
\end{verbatim}

\subsubsection*{Treemacs Dired}
\label{sec:orgebc4169}
\begin{verbatim}
(use-package treemacs-icons-dired
  :after treemacs dired
  :ensure t
  :config (treemacs-icons-dired-mode)
)
\end{verbatim}

\subsubsection*{Treemacs Magit}
\label{sec:org40803f9}
\begin{verbatim}
(use-package treemacs-magit
  :after treemacs magit
  :ensure t)
\end{verbatim}

\subsection*{Neotree}
\label{sec:orge2d4e7b}

\begin{verbatim}
(use-package neotree
  :bind
  ("<f7>" . neotree-toggle)
  :config
  (progn
    (setq neo-smart-open t)
    (setq neo-window-fixed-size nil)
    (evil-leader/set-key
      "tt" 'neotree-toggle
      "tp" 'neotree-projectile-action))

  ;; neotree 'icons' theme, which supports filetype icons
  (setq neo-theme (if (display-graphic-p) 'icons))
  (setq neo-theme 'icons)
  (setq neo-window-width 32)

  ;; Neotree bindings
  (add-hook 'neotree-mode-hook
            (lambda ()
              ; default Neotree bindings
              (define-key evil-normal-state-local-map (kbd "<tab>") 'neotree-enter)
              (define-key evil-normal-state-local-map (kbd "SPC") 'neotree-quick-look)
              (define-key evil-normal-state-local-map (kbd "q") 'neotree-hide)
              (define-key evil-normal-state-local-map (kbd "RET") 'neotree-enter)
              (define-key evil-normal-state-local-map (kbd "g") 'neotree-refresh)
              (define-key evil-normal-state-local-map (kbd "n") 'neotree-next-line)
              (define-key evil-normal-state-local-map (kbd "p") 'neotree-previous-line)
              (define-key evil-normal-state-local-map (kbd "A") 'neotree-stretch-toggle)
              (define-key evil-normal-state-local-map (kbd "H") 'neotree-hidden-file-toggle)
              (define-key evil-normal-state-local-map (kbd "|") 'neotree-enter-vertical-split)
              (define-key evil-normal-state-local-map (kbd "-") 'neotree-enter-horizontal-split)
              ; simulating NERDTree bindings in Neotree
              (define-key evil-normal-state-local-map (kbd "R") 'neotree-refresh)
              (define-key evil-normal-state-local-map (kbd "r") 'neotree-refresh)
              (define-key evil-normal-state-local-map (kbd "u") 'neotree-refresh)
              (define-key evil-normal-state-local-map (kbd "C") 'neotree-change-root)
              (define-key evil-normal-state-local-map (kbd "c") 'neotree-create-node))))
\end{verbatim}

\subsection*{dired-sidebar}
\label{sec:orge04e152}

\begin{verbatim}
(use-package dired-sidebar
  :ensure t
  :commands (dired-sidebar-toggle-sidebar)
  :bind
  ("<f6>" . dired-sidebar-toggle-sidebar)
  :init
  (add-hook 'dired-sidebar-mode-hook
         (lambda ()
           (unless (file-remote-p default-directory)
             (auto-revert-mode))))
  :config
  (push 'toggle-window-split dired-sidebar-toggle-hidden-commands)
  (push 'rotate-windows dired-sidebar-toggle-hidden-commands)

  (setq dired-sidebar-subtree-line-prefix "__")
  (setq dired-sidebar-theme 'vscode)
  (setq dired-sidebar-use-term-integration t)
  (setq dired-sidebar-use-custom-font t)
)
\end{verbatim}

\subsection*{ranger}
\label{sec:orgee0f0c1}

\begin{verbatim}
(use-package ranger
  :ensure t
  :bind
  ("C-x C-j" . ranger)
  :config
  (setq ranger-show-hidden t) ;; show hidden files
)
\end{verbatim}


\section*{eyebrowse}
\label{sec:org389f4c6}
\begin{verbatim}
(use-package eyebrowse
  :bind
  (:map eyebrowse-mode-map
  ("M-1" . eyebrowse-switch-to-window-config-1)
  ("M-2" . eyebrowse-switch-to-window-config-2)
  ("M-3" . eyebrowse-switch-to-window-config-3)
  ("M-4" . eyebrowse-switch-to-window-config-4)
  ("H-<right>" . eyebrowse-next-window-config)
  ("H-<left>" . eyebrowse-prev-window-config))
  :config
  ;;(define-key eyebrowse-mode-map (kbd "M-1") 'eyebrowse-switch-to-window-config-1)
  ;;(define-key eyebrowse-mode-map (kbd "M-2") 'eyebrowse-switch-to-window-config-2)
  ;;(define-key eyebrowse-mode-map (kbd "M-3") 'eyebrowse-switch-to-window-config-3)
  ;;(define-key eyebrowse-mode-map (kbd "M-4") 'eyebrowse-switch-to-window-config-4)
  ;;(define-key eyebrowse-mode-map (kbd "H-<right>") 'eyebrowse-next-window-config)
  ;;(define-key eyebrowse-mode-map (kbd "H-<left>") 'eyebrowse-prev-window-config)
  (eyebrowse-mode t)
  (setq eyebrowse-new-workspace t)
)
\end{verbatim}

\section*{windmove}
\label{sec:org9366841}
\begin{verbatim}
(use-package windmove
  :ensure t
  :config
  ;; use shift + arrow keys to switch between visible buffers
  ;; (windmove-default-keybindings)
  (windmove-default-keybindings 'control)
  (global-set-key (kbd "C-S-H") 'windmove-left)
  (global-set-key (kbd "C-S-L") 'windmove-right)
  (global-set-key (kbd "C-S-K") 'windmove-up)
  (global-set-key (kbd "C-S-J") 'windmove-down)
)
\end{verbatim}

\section*{emacs-rotate}
\label{sec:org9f3cfab}

\begin{verbatim}
(use-package rotate
  :ensure t
  :bind
  ("C-c C-r w" . rotate-window)
  ("C-c C-r l" . rotate-layout)
)
\end{verbatim}

\section*{ace jump mode}
\label{sec:org002094c}

\begin{verbatim}
(use-package ace-jump-mode
  :ensure t
  :bind
  ("C-." . ace-jump-mode)
)
\end{verbatim}

\section*{PDF Tools}
\label{sec:org47a17c1}

\subsection*{Install pdf-tools if its not already installed}
\label{sec:org2cf7ed2}
\begin{verbatim}
;; (pdf-tools-install)
;; the docs say if i care about startup time, i should use pdf-loader-install instead of pdf-tools-install, but doenst say why
;; (pdf-loader-install)
\end{verbatim}

\subsection*{Make buffer refresh every 1 second to PDF-tools updates the changed pdf}
\label{sec:org66bbd17}
\begin{verbatim}
(add-hook 'TeX-after-compilation-finished-functions #'TeX-revert-document-buffer)
;; (add-hook 'pdf-view-mode-hook 'auto-revert-mode)
;; (add-hook 'doc-view-mode-hook 'auto-revert-mode)
\end{verbatim}

\subsection*{PDF tools evil keybindings}
\label{sec:org0c559a9}
\begin{verbatim}
(evil-define-key 'normal pdf-view-mode-map
  "h" 'pdf-view-previous-page-command
  "j" (lambda () (interactive) (pdf-view-next-line-or-next-page 5))
  "k" (lambda () (interactive) (pdf-view-previous-line-or-previous-page 5))
  "l" 'pdf-view-next-page-command)
\end{verbatim}



\section*{Browsers integrations}
\label{sec:orga03591f}

\subsection*{edit-server}
\label{sec:org1cf89ef}

Edit stuff in browsers with emacs
Used in adition with the Edit in Emacs plugin for Chrome
\begin{verbatim}
(use-package edit-server
  :if (and window-system
           (not alternate-emacs))
  ;; :if window-system
  :ensure t
  :defer 5
  :disabled
  :config
  (edit-server-start)
)
\end{verbatim}

\subsection*{atomic-chrome}
\label{sec:org2631ac8}

\begin{verbatim}
(use-package atomic-chrome
  :ensure t
  :disabled
  :config
  (atomic-chrome-start-server)
)
\end{verbatim}


\section*{Appearance}
\label{sec:orgd4dcdf9}


\subsection*{startup stuff (splash and start screen, scratch message, etc\ldots{})}
\label{sec:org2d2ca91}

\begin{verbatim}
;; appearantly the `inhibit-splash-screen' was deprecaded. uses `inhibit-startup-screen' now
(setq inhibit-splash-screen t)
(setq inhibit-startup-screen t)

(setq inhibit-startup-screen t)
(setq inhibit-startup-message t)
(setq initial-buffer-choice nil)
;; Makes *scratch* empty.
(setq initial-scratch-message nil)
;; Don't show *Buffer list* when opening multiple files at the same time.
(setq inhibit-startup-buffer-menu t)
;; Make the buffer that opens on startup your init file ("~/.emacs" or
;; "~/.emacs.d/init.el").
;;(setq initial-buffer-choice user-init-file)

(blink-cursor-mode t)
(setq blink-cursor-blinks 0) ;; blink forever
(setq-default indicate-empty-lines t)

(setq frame-title-format '("Emacs"))
\end{verbatim}

\subsection*{scroll bars from frames}
\label{sec:orgb562bde}
\begin{verbatim}
(scroll-bar-mode -1)
\end{verbatim}

\subsection*{Remove menu bar and tool bar}
\label{sec:org9799607}
\begin{verbatim}
(tool-bar-mode -1)
(menu-bar-mode -1)
\end{verbatim}

\subsection*{set background and foreground color}
\label{sec:org011c495}

\begin{verbatim}
(set-background-color "#111111")
(set-foreground-color "#dddddd")
\end{verbatim}

\subsection*{Applying my theme}
\label{sec:org048232f}

\begin{verbatim}
(add-to-list 'custom-theme-load-path "~/dotfiles/emacs.d/themes/")
  ; theme options:
  ; atom-one-dark (doenst work well with emacsclient, ugly blue bg)
  ; dracula
  ; darktooth
  ; gruvbox-dark-hard
  ; gruvbox-dark-light
  ; gruvbox-dark-medium
  ; base16-default-dark-theme -- this one is good

(setq my-theme 'darkplus)
\end{verbatim}

Load the theme

\begin{verbatim}
(load-theme my-theme t)
\end{verbatim}


\begin{verbatim}

(defun load-my-theme (frame)
  "Function to load the theme in current FRAME.
  sed in conjunction
  with bellow snippet to load theme after the frame is loaded
  to avoid terminal breaking theme."
  (select-frame frame)
  (load-theme my-theme t))

; make emacs load the theme after loading the frame
; resolves issue with the theme not loading properly in terminal mode on emacsclient
;; NOTE this if was breaking my emacs!!!!!
;; (add-hook 'after-make-frame-functions #'load-my-theme)
\end{verbatim}


\subsection*{doom themes}
\label{sec:org507e85b}
\begin{verbatim}
(use-package doom-themes
  :ensure t
  :disabled
  :init (load-theme 'doom-tomorrow-night t)
  :config
  ;; Enable flashing mode-line on errors
  (doom-themes-visual-bell-config)

  ;; Corrects (and improves) org-mode's native fontification.
  (doom-themes-org-config)

  ;; Enable custom treemacs theme (all-the-icons must be installed!)
  (doom-themes-treemacs-config)
)
\end{verbatim}

\subsection*{Uniquify (unique files names in buffers)}
\label{sec:orga4ace43}

This package is included in emacs, so `:ensure nil` prevents use-package from trying to download it on Melpa

\begin{verbatim}
(use-package uniquify
  :defer 1
  :ensure nil
  :custom
  (uniquify-after-kill-buffer-p t)
  (uniquify-buffer-name-style 'post-forward)
  (uniquify-strip-common-suffix t)
)
\end{verbatim}

\subsection*{All The Icons - Icon package}
\label{sec:org28af1a0}

\begin{verbatim}
(use-package all-the-icons
  :ensure t
)
\end{verbatim}

\subsection*{vscode-icons}
\label{sec:org954da37}

\begin{verbatim}
(use-package vscode-icon
  :ensure t
  :commands (vscode-icon-for-file)
)
\end{verbatim}

\subsection*{parrot-mode}
\label{sec:org60d5ae2}

Type of parrots available:

\begin{itemize}
\item default
\item confused
\item emacs
\item nyan
\item rotating
\item science
\item thumbsup
\end{itemize}

\begin{verbatim}
(use-package parrot
  :ensure t
  :config
  ;; To see the party parrot in the modeline, turn on parrot mode:
  (parrot-mode)
  (parrot-set-parrot-type 'default)
  ;; Rotate the parrot when clicking on it (this can also be used to execute any function when clicking the parrot, like 'flyspell-buffer)
  (add-hook 'parrot-click-hook #'parrot-start-animation)
  ;; Rotate parrot when buffer is saved
  (add-hook 'after-save-hook #'parrot-start-animation)
  ;;/Rotation function keybindings for evil users
  (define-key evil-normal-state-map (kbd "[r") 'parrot-rotate-prev-word-at-point)
  (define-key evil-normal-state-map (kbd "]r") 'parrot-rotate-next-word-at-point)
  (add-hook 'mu4e-index-updated-hook #'parrot-start-animation)
)
\end{verbatim}

\subsection*{nyan-mode}
\label{sec:orge35b057}

\begin{verbatim}
(use-package nyan-mode
   :if window-system
   :hook
   (after-init . nyan-mode)
   :config
   (setq nyan-cat-face-number 4)
   (setq nyan-animate-nyancat t)
   (setq nyan-wavy-trail t)
   (nyan-start-animation))
\end{verbatim}

\subsection*{solaire-mode}
\label{sec:orgada7d9b}

solaire-mode is an aesthetic plugin that helps visually distinguish
file-visiting windows from other types of windows (like popups or sidebars)
by giving them a slightly different -- often brighter -- background.

\begin{verbatim}
;; (use-package solaire-mode
;;   :config
;;   (solaire-mode)
;;   :hook
;;   (after-init . solaire-global-mode +1)
;;   ;; To enable solaire-mode unconditionally for certain modes:
;;   (ediff-prepare-buffer . solaire-mode)
;;   ;; if you use auto-revert-mode, this prevents solaire-mode from turning itself off every time Emacs reverts the file
;;   (after-revert- . turn-on-solaire-mode)
;;   ;; highlight the minibuffer when it is activated:
;;   (minibuffer-setup . solaire-mode-in-minibuffer)
;;   (after-change-major-mode . turn-on-solaire-mode)
;;   :config
;;   ;; if the bright and dark background colors are the wrong way around, use this
;;   ;; to switch the backgrounds of the `default` and `solaire-default-face` faces.
;;   ;; This should be used *after* you load the active theme!
;;   ;;  NOTE: This is necessary for themes in the doom-themes package!
;;   (solaire-mode-swap-bg))
\end{verbatim}

\subsection*{fancy-battery-mode}
\label{sec:org1e3774d}
\begin{verbatim}
(use-package fancy-battery
  :ensure t
  :config
  (add-hook 'after-init-hook #'fancy-battery-mode)
)
\end{verbatim}
\subsection*{dimmer}
\label{sec:org10ebee5}

Dim unused frames and windows
Focus current window and dim unused ones

\begin{verbatim}
(use-package dimmer
  :disabled
  :ensure t
  :config (dimmer-mode)
)
\end{verbatim}

\section*{mode line}
\label{sec:orgf655e29}
\begin{verbatim}
(line-number-mode t)
(column-number-mode t)
(size-indication-mode t)
\end{verbatim}

\subsection*{doom-modeline}
\label{sec:org6b38f26}

Require and enable the doom-modeline
\begin{verbatim}
(require 'doom-modeline)
(doom-modeline-mode 1)
\end{verbatim}

Don’t compact font caches during GC (garbage collection).
\begin{verbatim}
;; (setq inhibit-compacting-font-caches t)
\end{verbatim}

Customize the doom-modeline (convert the comments to org later)
\begin{verbatim}
;; How tall the mode-line should be. It's only respected in GUI.
;; If the actual char height is larger, it respects the actual height.
(setq doom-modeline-height 23)

;; How wide the mode-line bar should be. It's only respected in GUI.
(setq doom-modeline-bar-width 3)

;; Determines the style used by `doom-modeline-buffer-file-name'.
;;
;; Given ~/Projects/FOSS/emacs/lisp/comint.el
;;   truncate-upto-project = ~/P/F/emacs/lisp/comint.el
;;   truncate-from-project = ~/Projects/FOSS/emacs/l/comint.el
;;   truncate-with-project = emacs/l/comint.el
;;   truncate-except-project = ~/P/F/emacs/l/comint.el
;;   truncate-upto-root = ~/P/F/e/lisp/comint.el
;;   truncate-all = ~/P/F/e/l/comint.el
;;   relative-from-project = emacs/lisp/comint.el
;;   relative-to-project = lisp/comint.el
;;   file-name = comint.el
;;   buffer-name = comint.el<2> (uniquify buffer name)
;;
;; If you are expereicing the laggy issue, especially while editing remote files
;; with tramp, please try `file-name' style.
;; Please refer to https://github.com/bbatsov/projectile/issues/657.
(setq doom-modeline-buffer-file-name-style 'truncate-upto-project)

;; Whether display icons in mode-line or not.
(setq doom-modeline-icon t)

;; Whether display the icon for major mode. It respects `doom-modeline-icon'.
(setq doom-modeline-major-mode-icon t)

;; Whether display color icons for `major-mode'. It respects
;; `doom-modeline-icon' and `all-the-icons-color-icons'.
(setq doom-modeline-major-mode-color-icon t)

;; Whether display icons for buffer states. It respects `doom-modeline-icon'.
(setq doom-modeline-buffer-state-icon t)

;; Whether display buffer modification icon. It respects `doom-modeline-icon'
;; and `doom-modeline-buffer-state-icon'.
(setq doom-modeline-buffer-modification-icon t)

;; Whether display minor modes in mode-line or not.
(setq doom-modeline-minor-modes nil)

;; If non-nil, a word count will be added to the selection-info modeline segment.
(setq doom-modeline-enable-word-count nil)

;; Whether display buffer encoding.
(setq doom-modeline-buffer-encoding t)

;; Whether display indentation information.
(setq doom-modeline-indent-info nil)

;; If non-nil, only display one number for checker information if applicable.
(setq doom-modeline-checker-simple-format t)

;; The maximum displayed length of the branch name of version control.
(setq doom-modeline-vcs-max-length 12)

;; Whether display perspective name or not. Non-nil to display in mode-line.
(setq doom-modeline-persp-name t)

;; Whether display icon for persp name. Nil to display a # sign. It respects `doom-modeline-icon'
(setq doom-modeline-persp-name-icon nil)

;; Whether display `lsp' state or not. Non-nil to display in mode-line.
(setq doom-modeline-lsp t)

;; Whether display github notifications or not. Requires `ghub` package.
(setq doom-modeline-github nil)

;; The interval of checking github.
(setq doom-modeline-github-interval (* 30 60))

;; Whether display environment version or not
(setq doom-modeline-env-version t)
;; Or for individual languages
;; (setq doom-modeline-env-enable-python t)
;; (setq doom-modeline-env-enable-ruby t)
;; (setq doom-modeline-env-enable-perl t)
;; (setq doom-modeline-env-enable-go t)
;; (setq doom-modeline-env-enable-elixir t)
;; (setq doom-modeline-env-enable-rust t)

;; Change the executables to use for the language version string
(setq doom-modeline-env-python-executable "python")
(setq doom-modeline-env-ruby-executable "ruby")
(setq doom-modeline-env-perl-executable "perl")
(setq doom-modeline-env-go-executable "go")
(setq doom-modeline-env-elixir-executable "iex")
(setq doom-modeline-env-rust-executable "rustc")

;; Whether display mu4e notifications or not. Requires `mu4e-alert' package.
(setq doom-modeline-mu4e t)

;; Whether display irc notifications or not. Requires `circe' package.
(setq doom-modeline-irc t)

;; Function to stylize the irc buffer names.
(setq doom-modeline-irc-stylize 'identity)
\end{verbatim}


this was commented with C-c ; so it doenst get exported in favor of doom-modeline
\section*{Interface Enhancement}
\label{sec:org7a3ef4f}

\subsection*{discover}
\label{sec:org25b119e}

\begin{verbatim}
(require 'discover)
(when (featurep 'discover)
  (discover-add-context-menu
    :context-menu '(isearch
              (description "Isearch, occur and highlighting")
              (lisp-switches
               ("-cf" "Case should fold search" case-fold-search t nil))
              (lisp-arguments
               ("=l" "context lines to show (occur)"
                "list-matching-lines-default-context-lines"
                (lambda (dummy) (interactive) (read-number "Number of context lines to show: "))))
              (actions
               ("Isearch"
                ("_" "isearch forward symbol" isearch-forward-symbol)
                ("w" "isearch forward word" isearch-forward-word))
               ("Occur"
                ("o" "occur" occur))
               ("More"
                ("h" "highlighters ..." makey-key-mode-popup-isearch-highlight))))
    :bind "M-s"
  )

  (discover-add-context-menu
    :context-menu '(dired)
    :bind "?"
    :mode 'dired-mode
    :mode-hook 'dired-mode-hook
  )
)
\end{verbatim}

\subsection*{auto balance windows area}
\label{sec:org0e282ce}

\begin{verbatim}
(global-set-key (kbd "C-M-+") 'balance-windows-area)
\end{verbatim}

\subsection*{zoom}
\label{sec:orgd1e3d19}

Zoom in selected window and resize othersAuto balance windows and frames

\begin{verbatim}
(use-package zoom
  :ensure t
  :bind
  ("C-M-z" . zoom)
  :init
  (setq zoom-size '(0.618 . 0.618))
  (setq zoom-ignored-major-modes '(treemacs dired-mode neotree dired-sidebar))
  (setq zoom-ignored-buffer-names '("zoom.el" "init.el"))
  (setq zoom-ignored-buffer-name-regexps '("^*calc"))
  :config
  (zoom-mode t)
)
\end{verbatim}

\subsection*{sublimity}
\label{sec:org2378677}

\begin{verbatim}
(use-package sublimity
  :ensure t
  :config
  ;;(setq sublimity-scroll-weight 10
  ;;    sublimity-scroll-drift-length 5)
  ;; this is the only part of the config where i use `use-package' inside another package config.
  ;; the oficial docs appears to suggest this way
  (sublimity-mode 1)
)
\end{verbatim}

\subsubsection*{sublimity-scroll}
\label{sec:org78513e2}
\begin{verbatim}
(use-package sublimity-scroll
  :ensure nil
  :config
  (setq sublimity-scroll-weight 10)  ;; default 10
  (setq sublimity-scroll-drift-length 5)  ;; default 5
  (setq sublimity-scroll-hide-cursor t) ;; default t
)
\end{verbatim}

\subsubsection*{sublimity-map (experimental)}
\label{sec:org9002bc8}
This package ruins the scrolling from either sublimity-scroll or the smooth-scrolling package
\begin{verbatim}
(use-package sublimity-map
  :disabled
  :ensure nil
  :config
  (setq sublimity-map-size 14)  ;; minimap width
  (setq sublimity-map-fraction 0.3)
  (setq sublimity-map-text-scale -5)
  (sublimity-map-set-delay nil) ;; minimap is displayed after 5 seconds of idle time

  ;; document this snippet better, not sure what it does, but it defines the font-family
;;  (add-hook 'sublimity-map-setup-hook
;;          (lambda ()
;;            (setq buffer-face-mode-face '(:family "Monospace"))
;;            (buffer-face-mode)))

)
\end{verbatim}

\subsubsection*{sublimity-attractive}
\label{sec:org4e9d67a}
\begin{verbatim}
(use-package sublimity-attractive
  :disabled
  :ensure nil
  :config
  (setq sublimity-attractive-centering-width 110)

  ;; these are functions (NOT variables) to configure some UI parts
  ;; (sublimity-attractive-hide-bars)
  ;; (sublimity-attractive-hide-vertical-border)
  ;; (sublimity-attractive-hide-fringes)
  ;; (sublimity-attractive-hide-modelines)
)

\end{verbatim}


\subsection*{goto-line-preview}
\label{sec:orgbcf1969}

\begin{verbatim}
(use-package goto-line-preview
  :ensure t
  :config
  (global-set-key [remap goto-line] 'goto-line-preview)
)
\end{verbatim}

\subsection*{dashboard}
\label{sec:org3f70116}

\begin{verbatim}
(use-package dashboard
  :ensure t
  :preface
  (defun tau/dashboard-banner ()
    "Sets a dashboard banner including information on package initialization
     time and garbage collections."
    (setq dashboard-banner-logo-title
          (format "Emacs ready in %.2f seconds with %d garbage collections."
                  (float-time
                   (time-subtract after-init-time before-init-time)) gcs-done)))
  :custom-face
  (dashboard-heading ((t (:foreground "#f1fa8c" :weight bold))))
  :init

  ;; set widgets to show
  (setq dashboard-items '((recents  . 5)
                         (bookmarks . 5)
                         (projects . 5)
                         (agenda . 5)
                         (fireplace . 1)
                         (registers . 5))
  )

  ;; sets dashboard as emacs initial buffer on startup
  (setq initial-buffer-choice (lambda () (get-buffer "*dashboard*")))

  ;; Set the title
  (setq dashboard-banner-logo-title "Hi 😊 ")
  (setq dashboard-banner-logo-title
          (message " ★ Emacs initialized in %.2fs ★ "
                   (float-time (time-subtract (current-time) my-init-el-start-time))))

  ;; Set the banner
  (setq dashboard-startup-banner 'logo) ;; values: ('oficial, 'logo, 1, 2, 3, or "path/to/image.png")

  ;; Content is not centered by default. To center, set
  (setq dashboard-center-content t)

  ;; To disable shortcut "jump" indicators for each section, set
  (setq dashboard-show-shortcuts t)

  ;; To add icons to the widget headings and their items:
  (setq dashboard-set-heading-icons t)
  (setq dashboard-set-file-icons t)

  ;; To show navigator below the banner:
  (setq dashboard-set-navigator t)

  ;;To show info about the packages loaded and the init time:
  (setq dashboard-set-init-info t)

  ;; A randomly selected footnote will be displayed. To disable it:
  ;;(setq dashboard-set-footer nil)

  :config
  ;; Org mode’s agenda
  ;; To display today’s agenda items on the dashboard, add agenda to dashboard-items:
  (add-to-list 'dashboard-items '(agenda) t)
  ;; To show agenda for the upcoming seven days set the variable show-week-agenda-p to t.
  (setq show-week-agenda-p t)
  ;; Note that setting list-size for the agenda list is intentionally ignored; all agenda items for the current day will be displayed.
  ;; To customize which categories from the agenda items should be visible in the dashboard set the dashboard-org-agenda-categories to the list of categories you need.
  (setq dashboard-org-agenda-categories '("Tasks" "Appointments"))

  ;; adds fireplace as a widget
;;  (defun dashboard-insert-custom (list-size)
;;    (fireplace))
;;  (add-to-list 'dashboard-item-generators  '(fireplace . dashboard-insert-custom))
;;  (add-to-list 'dashboard-items '(fireplace) t)

  (dashboard-setup-startup-hook)
)
\end{verbatim}

\subsubsection*{add fireplace as a widged in the dashboard}
\label{sec:org528a152}
\subsection*{toggle window transparency}
\label{sec:org0e7d506}

\begin{verbatim}
(defun tau/toggle-window-transparency ()
  "Cycle the frame transparency from default to transparent."
  (interactive)
  (let ((transparency 85)
        (opacity 100))
    (if (and (not (eq (frame-parameter nil 'alpha) nil))
             (< (frame-parameter nil 'alpha) opacity))
        (set-frame-parameter nil 'alpha opacity)
      (set-frame-parameter nil 'alpha transparency))))

(global-set-key (kbd "M-<f12> t") 'tau/toggle-window-transparency)
\end{verbatim}

\subsection*{minimap}
\label{sec:org6c9f088}

\begin{verbatim}
(use-package minimap
  :ensure t
  :disabled t
  :commands
  (minimap-bufname minimap-create minimap-kill)
  :custom
  (minimap-major-modes '(prog-mode))
  (minimap-window-location 'right)
  (minimap-update-delay 0.2)
  (minimap-minimum-width 20)
  :bind
  ("M-<f12> m" . tau/toggle-minimap)
  :preface
  (defun tau/toggle-minimap ()
    "Toggle minimap for current buffer."
    (interactive)
    (if (null minimap-bufname)
        (minimap-create)
      (minimap-kill)))
  :config
  (custom-set-faces
   '(minimap-active-region-background
    ((((background dark)) (:background "#555555555555"))
      (t (:background "#C847D8FEFFFF"))) :group 'minimap))
)
\end{verbatim}

\subsection*{pulse}
\label{sec:org2861b3c}

\begin{verbatim}
;; Visualize TAB, (HARD) SPACE, NEWLINE
;; Pulse current line
(use-package pulse
  :ensure nil
  :preface
  (defun my-pulse-momentary-line (&rest _)
    "Pulse the current line."
    (pulse-momentary-highlight-one-line (point) 'next-error))

  (defun my-pulse-momentary (&rest _)
    "Pulse the current line."
    (if (fboundp 'xref-pulse-momentarily)
        (xref-pulse-momentarily)
      (my-pulse-momentary-line)))

  (defun my-recenter-and-pulse(&rest _)
    "Recenter and pulse the current line."
    (recenter)
    (my-pulse-momentary))

  (defun my-recenter-and-pulse-line (&rest _)
    "Recenter and pulse the current line."
    (recenter)
    (my-pulse-momentary-line))
  :hook (((dumb-jump-after-jump
           imenu-after-jump) . my-recenter-and-pulse)
         ((bookmark-after-jump
           magit-diff-visit-file
           next-error) . my-recenter-and-pulse-line))
  :init
  (dolist (cmd '(recenter-top-bottom
                 other-window ace-window windmove-do-window-select
                 pager-page-down pager-page-up
                 symbol-overlay-basic-jump))
    (advice-add cmd :after #'my-pulse-momentary-line))
  (dolist (cmd '(pop-to-mark-command
                 pop-global-mark
                 goto-last-change))
    (advice-add cmd :after #'my-recenter-and-pulse))
)
\end{verbatim}

\subsection*{prettify symbols}
\label{sec:org97239cc}

\begin{verbatim}
this is built-in with emacs >= v24
\end{verbatim}

\begin{verbatim}
(global-prettify-symbols-mode 1)
(defun add-pretty-lambda ()
  "Make some word or string show as pretty Unicode symbols. See https://unicodelookup.com for more."
  (setq prettify-symbols-alist
        '(
          ("lambda" . 955)
          ("delta" . 120517)
          ("epsilon" . 120518)
          ("->" . 8594)
          ("<=" . 8804)
          (">=" . 8805)
          )))
(add-hook 'prog-mode-hook 'add-pretty-lambda)
(add-hook 'org-mode-hook 'add-pretty-lambda)
\end{verbatim}

\subsection*{centaur-tabs}
\label{sec:orga0721c8}

\begin{verbatim}
(use-package centaur-tabs
 :ensure t
 :hook
 (after-init . centaur-tabs-mode)
 (dashboard-mode . centaur-tabs-local-mode)
 (term-mode . centaur-tabs-local-mode)
 (calendar-mode . centaur-tabs-local-mode)
 (org-agenda-mode . centaur-tabs-local-mode)
 (helpful-mode . centaur-tabs-local-mode)
 :bind
 ("C-<prior>" . centaur-tabs-backward)
 ("C-<next>" . centaur-tabs-forward)
 ("C-c t s" . centaur-tabs-counsel-switch-group)
 ("C-c t p" . centaur-tabs-group-by-projectile-project)
 ("C-c t g" . centaur-tabs-group-buffer-groups)
 (:map evil-normal-state-map
 ("g t" . centaur-tabs-forward)
 ("g T" . centaur-tabs-backward))
 :init
 :config
 ;; appearantly these dont work if put in :init
 (setq centaur-tabs-style "box") ; types available: (alternative, bar, box, chamfer, rounded, slang, wave, zigzag)
 (setq centaur-tabs-height 25)
 (setq centaur-tabs-set-icons t) ;; display themed icons from all the icons
 (setq centaur-tabs-set-modified-marker t) ;; display a marker indicating that a buffer has been modified (atom-style)
 (setq centaur-tabs-modified-marker "*")
 (centaur-tabs-headline-match)
 (centaur-tabs-mode t)

 (setq centaur-tabs-set-bar 'over) ;; in previous config value was 'over
 ;; (setq centaur-tabs-gray-out-icons 'buffer)
 ;; (centaur-tabs-enable-buffer-reordering)
 ;; (setq centaur-tabs-adjust-buffer-order t)
 (setq uniquify-separator "/")
 (setq uniquify-buffer-name-style 'forward)
 ;; (centaur-tabs-change-fonts "arial" 160)
)
\end{verbatim}

\section*{Highlights}
\label{sec:orgb4206d6}

\subsection*{show paren mode}
\label{sec:orgdee0e80}

Highlight matching parenthesis

\begin{verbatim}
(use-package paren
  :ensure nil
  :hook
  (after-init . show-paren-mode)
  :custom-face
  (show-paren-match ((nil (:background "#44475a" :foreground "#f1fa8c")))) ;; :box t
  :config
  (setq show-paren-delay 0)
  (setq show-paren-style 'mixed)
  (setq show-paren-when-point-inside-paren t)
  (setq show-paren-when-point-in-periphery t)
  (show-paren-mode +1)
)
\end{verbatim}

\subsection*{Highlighting numbers}
\label{sec:orgbc71d21}

\begin{verbatim}
(use-package highlight-numbers
    :ensure t
    :hook
    (prog-mode . highlight-numbers-mode)
)
\end{verbatim}

\subsection*{Highlighting operators}
\label{sec:org6a67b12}

\begin{verbatim}
(use-package highlight-operators
  :ensure t
  :hook
  (prog-mode . highlight-operators-mode)
)
\end{verbatim}

\subsection*{Highlighting escape sequences}
\label{sec:org34185a5}

\begin{verbatim}
(use-package highlight-escape-sequences
  :ensure t
  :hook
  (prog-mode . hes-mode)
)
\end{verbatim}

\subsection*{Highlighting parentheses}
\label{sec:org5f1a662}

\begin{verbatim}
(use-package highlight-parentheses
  :ensure t
  :hook
  (prog-mode . highlight-parentheses-mode)
)
\end{verbatim}

\subsection*{Highlight TODO}
\label{sec:orgb21c940}

Basic support todos.
By default these include:
TODO NEXT THEM PROG OKAY DONT FAIL DONE NOTE KLUDGE HACK TEMP FIXME
and any sequence of X's or ?'s of length at least 3: XXX, XXXX, XXXXX, …, ???, ????, ????, ….

\begin{verbatim}
;; NOTE that the highlighting works even in comments.
(use-package hl-todo
  :ensure t
  :hook
  (prog-mode . hl-todo-mode)
  (text-mode . hl-todo-mode)
  :init
  ;; (add-hook 'text-mode-hook (lambda () (hl-todo-mode t)))
  :config
  ;; Adding a new keyword: TEST.
  (add-to-list 'hl-todo-keyword-faces '("TODO" . "#ff3300"))
  (add-to-list 'hl-todo-keyword-faces '("TEST" . "#dc8cc3"))
  (add-to-list 'hl-todo-keyword-faces '("NOTE" . "#ffff00"))
  (add-to-list 'hl-todo-keyword-faces '("DONE" . "#00ff00"))
)
\end{verbatim}

\subsection*{hi-lock mode}
\label{sec:org1439661}

Highlight regexp

From Mastering Emacs:
There is a mechanism for storing and restoring the Hi-Locks you’ve created. If you create highlights interactively you can tell Emacs to insert those patterns into the active buffer by running M-s h w. Emacs will wrap the elisp patterns in the comment format used by the buffer (if one is defined) or ask if you no comment format is defined.

The patterns should be added to the top of the file, as Emacs will only search the first 10,000 characters (customize hi-lock-file-patterns-range to change that amount) for the patterns before giving up.

Emacs will not highlight patterns found in a file automatically. You must explicitly tell it to do so by manually invoking M-x hi-lock-mode or globally with global-hi-lock-mode.

\begin{verbatim}
(use-package hi-lock
  :init
  (global-hi-lock-mode 1)
  :config
  (add-hook 'hi-lock-mode-hook
          (lambda nil
            (highlight-regexp "FIXME" 'hi-red-b)
            (highlight-regexp "NOTE" 'hi-red-b)
            (highlight-regexp "TODO" 'hi-red-b))
  )
  ;; always highlight patterns found in files without confirmation
  (setq hi-lock-file-patterns-policy #'(lambda (dummy) t))
)
\end{verbatim}

\subsection*{hl-anything}
\label{sec:orga408c36}

Highlight portions of text

\begin{verbatim}
(use-package hl-anything
  :ensure t
  :after evil
;;  :hook
;;  (kill-emacs . hl-save-highlights)
  :bind
  ("C-<f8> h" . hl-highlight-thingatpt-local)
  ("C-<f8> S-h" . hl-highlight-thingatpt-global)
  ("C-<f8> u l" . hl-unhighlight-all-local)
  ("C-<f8> u g" . hl-unhighlight-all-global)
  ("C-<f8> n" . hl-find-next-thing)
  ("C-<f8> p" . hl-find-prev-thing)
  ("C-<f8> s" . hl-save-highlights)
  ("C-<f8> r" . hl-restore-highlights)
  :config
  (hl-highlight-mode 1)

  ;; evil leader key bindings for hl-anything
  (evil-leader/set-key
    "hul"  'hl-unhighlight-all-local
    "hug" 'hl-unhighlight-all-global
    "htg" 'hl-highlight-thingatpt-global
    "htl"  'hl-highlight-thingatpt-local
    "hn"  'hl-find-next-thing
    "hp"  'hl-find-prev-thing
    "hr"  'hl-restore-highlights
    "hs"  'hl-save-highlights)
)
\end{verbatim}

\subsection*{beacon - flash light where cursor is}
\label{sec:org3d91ae5}
\begin{verbatim}
(use-package beacon
  :ensure t
  :init
  (setq beacon-blink-when-point-moves-vertically nil) ; default nil
  (setq beacon-blink-when-point-moves-horizontally nil) ; default nil
  (setq beacon-blink-when-buffer-changes t) ; default t
  (setq beacon-blink-when-window-scrolls t) ; default t
  (setq beacon-blink-when-window-changes t) ; default t
  (setq beacon-blink-when-focused nil) ; default nil
  (setq beacon-blink-duration 0.3) ; default 0.3
  (setq beacon-blink-delay 0.3) ; default 0.3
  (setq beacon-size 20) ; default 40
  ;; (setq beacon-color "yellow") ; default 0.5
  (setq beacon-color 0.6) ; default 0.5
  :config
  (beacon-mode 1)
)
\end{verbatim}

\subsection*{Rainbow Delimiters}
\label{sec:org3e798bc}

This highlights matching parentheses acording to their depth
Helps editing lisp code

\begin{verbatim}
(use-package rainbow-delimiters
  :ensure t
  ;;:hook
  ;;(emacs-lisp-mode . rainbow-delimiters-mode)
  ;;(prog-mode . rainbow-delimiters-mode)
)
\end{verbatim}

\subsection*{rainbow mode}
\label{sec:org3148cc7}

\begin{verbatim}
Colorize hex, rgb and named color codes
\end{verbatim}


\begin{verbatim}
(use-package rainbow-mode
  :ensure t
  :hook
  (org-mode . rainbow-mode)
  (css-mode . rainbow-mode)
  (scss-mode . rainbow-mode)
  (php-mode . rainbow-mode)
  (html-mode . rainbow-mode)
  (web-mode . rainbow-mode)
  (js2-mode . rainbow-mode))
\end{verbatim}

\subsection*{Highlight lines}
\label{sec:org100260c}

\begin{verbatim}
built-in package
\end{verbatim}


\begin{verbatim}
(use-package hl-line
  :ensure nil
  :defer nil
  :config
  (global-hl-line-mode)
)
\end{verbatim}

\subsection*{{\bfseries\sffamily TODO} Highlight columns}
\label{sec:org72dd254}

\begin{verbatim}
(use-package col-highlight
  :disabled
  :defer nil
  :config
  (col-highlight-toggle-when-idle)
  (col-highlight-set-interval 2)
)
\end{verbatim}

\subsection*{Highlight crosshair}
\label{sec:orgddf568c}

Highlight crosshair (combination of hl-lines and hl-columns

\begin{verbatim}
(use-package crosshairs
  :disabled
  :defer nil
  :config
  (crosshairs-mode)
)
\end{verbatim}

\subsection*{volatile-highlights}
\label{sec:org080a0df}

\begin{verbatim}
(use-package volatile-highlights
  :ensure t
  :disabled
  :hook
  (after-init . volatile-highlights-mode)
  :custom-face
  (vhl/default-face ((nil (:foreground "#FF3333" :background "#FFCDCD"))))
  :config
  ;;-----------------------------------------------------------------------------
  ;; Supporting evil-mode.
  ;;-----------------------------------------------------------------------------
  (vhl/define-extension 'evil 'evil-paste-after 'evil-paste-before
                        'evil-paste-pop 'evil-move)
  (vhl/install-extension 'evil)
  ;;-----------------------------------------------------------------------------
  ;; Supporting undo-tree.
  ;;-----------------------------------------------------------------------------
  (vhl/define-extension 'undo-tree 'undo-tree-yank 'undo-tree-move)
  (vhl/install-extension 'undo-tree)
)
\end{verbatim}


\begin{verbatim}
(use-package highlight-indent-guides
  :ensure t
  :hook
  ((prog-mode yaml-mode) . highlight-indent-guides-mode)
  :custom
  (highlight-indent-guides-auto-enabled t)
  (highlight-indent-guides-responsive t)
  (highlight-indent-guides-method 'character) ; column
)
\end{verbatim}


\section*{Minor modes}
\label{sec:orgfcd4d6a}

\subsection*{which-key}
\label{sec:orge950add}

\begin{verbatim}
(use-package which-key
  :hook (after-init . which-key-mode))
  :config
  (setq which-key-idle-delay 0.2)
  (setq which-key-min-display-lines 3)
  (setq which-key-max-description-length 20)
  (setq which-key-max-display-columns 6)
\end{verbatim}

\subsection*{key-frequency}
\label{sec:org23afdbe}

\begin{verbatim}
(use-package keyfreq
  :ensure t
  :init
  (keyfreq-mode 1)
  (keyfreq-autosave-mode 1)
)
\end{verbatim}

\subsection*{smartparens}
\label{sec:org84784eb}
\begin{verbatim}
(use-package smartparens
  :ensure t
  :hook
  (after-init . smartparens-global-mode)
  :config
  (require 'smartparens-config)
  (sp-pair "=" "=" :actions '(wrap))
  (sp-pair "+" "+" :actions '(wrap))
  (sp-pair "<" ">" :actions '(wrap))
  (sp-pair "$" "$" :actions '(wrap)))
\end{verbatim}

\subsubsection*{evil-smartparens helps avoid conflicts between evil and smartparens}
\label{sec:org3d72e9b}

\begin{verbatim}
(use-package evil-smartparens
  :ensure t
  :hook
  (smartparens-enabled . evil-smartparens-mode)
)
\end{verbatim}

\subsection*{Smartscan mode}
\label{sec:orgd51defa}
\begin{verbatim}
Usage:
M-n and M-p move between symbols
M-' to replace all symbols in the buffer matching the one under point
C-u M-' to replace symbols in your current defun only (as used by narrow-to-defun.)
\end{verbatim}


\begin{verbatim}
(smartscan-mode 1)
\end{verbatim}

\subsection*{editorconfig}
\label{sec:orgc83b443}

\begin{verbatim}
(use-package editorconfig
  :ensure t
  :config
  (editorconfig-mode 1)
)
\end{verbatim}

\subsection*{hide mode line mode}
\label{sec:orgdcb1684}

\begin{verbatim}
(use-package hide-mode-line
  :ensure t
  :hook
  (completion-list-mode . hide-mode-line-mode)
  (neotree-mode . hide-mode-line-mode)
  (treemacs-mode . hide-mode-line-mode)
)
\end{verbatim}


\begin{verbatim}
(use-package saveplace
  :ensure t
  :hook
  (after-init . save-place-mode)
  :init
  (setq-default save-place t)
  (setq save-place-file (expand-file-name ".places" user-emacs-directory))
)
\end{verbatim}

\section*{multiple cursors}
\label{sec:orgbfaf68b}

\begin{verbatim}
(use-package multiple-cursors
  :after evil
  ;; step 1, select thing in visual-mode (OPTIONAL)
  ;; step 2, `mc/mark-all-like-dwim' or `mc/mark-all-like-this-in-defun'
  ;; step 3, `ace-mc-add-multiple-cursors' to remove cursor, press RET to confirm
  ;; step 4, press s or S to start replace
  ;; step 5, press C-g to quit multiple-cursors
  :bind
  ("M-u" . hydra-multiple-cursors/body)
  (:map evil-visual-state-map
  ("C-d" . mc/mark-next-like-this)
  ("C-a" . mc/mark-all-like-this)
  )
  :config
  (define-key evil-visual-state-map (kbd "mn") 'mc/mark-next-like-this)
  (define-key evil-visual-state-map (kbd "ma") 'mc/mark-all-like-this-dwim)
  (define-key evil-visual-state-map (kbd "md") 'mc/mark-all-like-this-in-defun)
  (define-key evil-visual-state-map (kbd "mm") 'ace-mc-add-multiple-cursors)
  (define-key evil-visual-state-map (kbd "ms") 'ace-mc-add-single-cursor)
)
\end{verbatim}


\section*{quickrun (compile and execute code)}
\label{sec:org537525c}
\begin{verbatim}
(use-package quickrun
  :bind
  ("C-<f5>" . quickrun)
  ("M-<f5>" . quickrun-shell)
)
\end{verbatim}


\section*{\LaTeX{}}
\label{sec:orgc35276f}

\begin{verbatim}
(use-package auctex-latexmk
  :defer t
  :init
  (add-hook 'LaTeX-mode-hook 'auctex-latexmk-setup)
)
\end{verbatim}


\begin{verbatim}
(use-package company-auctex
  :ensure t
  :defer t
  :init
  (add-hook 'LaTeX-mode-hook 'company-auctex-init)
)
\end{verbatim}

\begin{verbatim}
(use-package tex
  :ensure auctex
  :defer t
  :hook
  (LaTeX-mode . visual-line-mode)
  (LaTeX-mode . flyspell-mode)
  (LaTeX-mode . LaTeX-math-mode)
  :custom
  (TeX-auto-save t)
  (TeX-parse-self t)
  (TeX-master nil)
  ;; to use pdfview with auctex
  (TeX-view-program-selection '((output-pdf "pdf-tools"))
    TeX-source-correlate-start-server t)
  (TeX-view-program-list '(("pdf-tools" "TeX-pdf-tools-sync-view")))
  (TeX-after-compilation-finished-functions #'TeX-revert-document-buffer)
  :config
  (setq TeX-PDF-mode t) ;; compile to PDF by default
  (setq org-export-with-smart-quotes t) ;; convert quotes to LaTeX smartquotes on export

  (add-hook 'LaTeX-mode-hook
    (lambda ()
        (turn-on-reftex)
        (setq reftex-plug-into-AUCTeX t)
        (reftex-isearch-minor-mode)
        (setq TeX-PDF-mode t)
        (setq TeX-source-correlate-method 'synctex)
        (setq TeX-source-correlate-start-server t)))
)

\end{verbatim}

\subsection*{Add the beamer presentation class template to org}
\label{sec:org06fea09}
\begin{verbatim}
(add-to-list 'org-latex-classes
        '("beamer"
          "\\documentclass\[presentation\]\{beamer\}"
          ("\\section\{%s\}" . "\\section*\{%s\}")
          ("\\subsection\{%s\}" . "\\subsection*\{%s\}")
          ("\\subsubsection\{%s\}" . "\\subsubsection*\{%s\}"))
)
\end{verbatim}


\subsection*{Add the memoir class template to org}
\label{sec:orga3841fc}

The Sections and Heading Levels gets configured as follows:

\begin{center}
\begin{tabular}{lrl}
Division & <c>Level & <c>org-equivalent\\
\book & -2 & *\\
\part & -1 & **\\
\chapter & 0 & \textbf{*}\\
\section & 1 & \textbf{**}\\
\subsection & 2 & \textbf{\textbf{*}}\\
\subsubsection & 3 & \textbf{\textbf{**}}\\
\paragraph & 4 & \textbf{\textbf{\textbf{*}}}\\
\subparagraph & 5 & \textbf{\textbf{\textbf{**}}}\\
\end{tabular}
\end{center}


\begin{verbatim}
;(add-to-list 'org-latex-classes
;        '("memoir"
;          "\\documentclass\[a4paper\]\{memoir\}"
;          ("\\book\{%s\}" . "\\book*\{%s\}")
;          ("\\part\{%s\}" . "\\part*\{%s\}")
;          ("\\chapter\{%s\}" . "\\chapter*\{%s\}")
;          ("\\section\{%s\}" . "\\section*\{%s\}")
;          ("\\subsection\{%s\}" . "\\subsection*\{%s\}")
;          ("\\subsubsection\{%s\}" . "\\subsubsection*\{%s\}"))
;)
\end{verbatim}

\subsection*{{\bfseries\sffamily TODO} Add abntex2 class to org list of latex classes}
\label{sec:org889ce80}
This class is based on the Memoir class
The Sections and Heading Levels gets configured as follows:

\begin{center}
\begin{tabular}{lrl}
Division & <c>Level & <c>org-equivalent\\
\part & -1 & *\\
\chapter & 0 & **\\
\section & 1 & \textbf{*}\\
\subsection & 2 & \textbf{**}\\
\subsubsection & 3 & \textbf{\textbf{*}}\\
\paragraph & 4 & \textbf{\textbf{**}}\\
\subparagraph & 5 & \textbf{\textbf{\textbf{*}}}\\
\end{tabular}
\end{center}
\begin{verbatim}
;(add-to-list 'org-latex-classes
;             '("abntex2"
;               "\\documentclass{abntex2}"
;               ("\\part{%s}" . "\\part*{%s}")
;               ("\\chapter{%s}" . "\\chapter*{%s}")
;               ("\\section{%s}" . "\\section*{%s}")
;               ("\\subsection{%s}" . "\\subsection*{%s}")
;               ("\\subsubsection{%s}" . "\\subsubsection*{%s}")
;               ("\\subsubsubsection{%s}" . "\\subsubsubsection*{%s}")
;               ("\\paragraph{%s}" . "\\paragraph*{%s}"))
;)

(add-to-list 'org-latex-classes
             '("abntex2"
               "\\documentclass{abntex2}"
               ("\\section{%s}" . "\\section*{%s}")
               ("\\subsection{%s}" . "\\subsection*{%s}")
               ("\\subsubsection{%s}" . "\\subsubsection*{%s}")
               ("\\subsubsubsection{%s}" . "\\subsubsubsection*{%s}")
               ("\\paragraph{%s}" . "\\paragraph*{%s}"))
)
\end{verbatim}

\section*{HTML}
\label{sec:orge3ab266}

\subsection*{Set HTML indentation to 4 spaces by default (only on html-mode)}
\label{sec:orge2773a1}
\begin{verbatim}
(add-hook 'html-mode-hook
  (lambda ()
    (set (make-local-variable 'sgml-basic-offset) 4)))
\end{verbatim}

\section*{CSS and SCSS}
\label{sec:orgcf9c04c}

\subsection*{CSS}
\label{sec:org8f4e220}

\begin{verbatim}
(use-package css-mode
  :ensure t
  :mode "\\.css\\'"
  :init
  (setq css-indent-offset 2)
)
\end{verbatim}

\subsection*{SCSS}
\label{sec:org3366b68}
\begin{verbatim}
(use-package scss-mode
  :ensure t
  ;; this mode doenst load using :mode from use-package, dunno why
  :mode (("\\.scss\\'" . scss-mode)
         ("\\.component.scss\\'" . scss-mode))
  :init
  (setq scss-compile-at-save 'nil)
  :config
   (add-to-list 'auto-mode-alist '("\\.scss$\\'" . scss-mode))
   (add-to-list 'auto-mode-alist '("\\.component.scss$\\'" . scss-mode))
)
\end{verbatim}

\begin{verbatim}
(use-package helm-css-scss
  :ensure t
  :bind (
  (:map isearch-mode-map
  ("s-i" . helm-css-scss-from-isearch)
  :map helm-css-scss-map
  ("s-i" . helm-css-scss-multi-from-helm-css-scss)
  :map css-mode-map
  ("s-i" . helm-css-scss)
  ("s-S-I" . helm-css-scss-back-to-last-point)
  :map scss-mode-map
  ("s-i" . helm-css-scss)
  ("s-S-I" . helm-css-scss-back-to-last-point))
  )
  :config
  (setq helm-css-scss-insert-close-comment-depth 2
        helm-css-scss-split-with-multiple-windows t
        helm-css-scss-split-direction 'split-window-vertically)
)
\end{verbatim}

\section*{YAML}
\label{sec:org6c4b7ec}
\begin{verbatim}
(use-package yaml-mode
  :ensure t
)
\end{verbatim}

\section*{TOML}
\label{sec:org8990fd7}
\begin{verbatim}
(use-package toml-mode
  :ensure t
)
\end{verbatim}

\section*{Emmet}
\label{sec:org659daf9}

\begin{verbatim}
(use-package emmet-mode
  :ensure t
  :commands emmet-mode
  :init
    (setq emmet-indentation 2)
    (setq emmet-move-cursor-between-quotes t)
  :hook
    (sgml-mode . emmet-mode) ;; Auto-start on any markup modes
    (css-mode . emmet-mode) ;; enable Emmet's css abbreviation.
    (scss-mode . emmet-mode) ;; enable Emmet's css abbreviation.
    (html-mode . emmet-mode) ;; Auto-start on HTML files
    (web-mode . emmet-mode) ;; Auto-start on web-mode
  :config
  (unbind-key "<C-return>" emmet-mode-keymap)
  (unbind-key "C-M-<left>" emmet-mode-keymap)
  (unbind-key "C-M-<right>" emmet-mode-keymap)
  (setq emmet-expand-jsx-className? nil)) ;; use emmet with JSX markup
\end{verbatim}


\section*{Ruby}
\label{sec:org93c7cdb}

\begin{verbatim}
(use-package ruby-mode
  :mode "\\.rb\\'"
  :interpreter "ruby"
  :ensure-system-package
  ((rubocop     . "gem install rubocop")
   (ruby-lint   . "gem install ruby-lint")
   (ripper-tags . "gem install ripper-tags")
   (pry         . "gem install pry"))
  :functions inf-ruby-keys
  :hook
  (ruby-mode . subword-mode)
  (ruby-mode . eldoc-mode)
  :config
  (defun my-ruby-mode-hook ()
    (require 'inf-ruby)
    (inf-ruby-keys))

  ;; Switch the compilation buffer mode with C-x C-q (useful
  ;; when interacting with a debugger)
  (add-hook 'after-init-hook 'inf-ruby-switch-setup)

  (add-hook 'ruby-mode-hook
            (lambda ()
              (hs-minor-mode 1) ;; Enables folding
              (modify-syntax-entry ?: "."))) ;; Adds ":" to the word definition
)
\end{verbatim}

\subsection*{ruby refactor}
\label{sec:org58a278a}
\begin{verbatim}
;; Functions to help with refactoring
(use-package ruby-refactor
  :ensure t
  :defer t
  :hook
  (ruby-mode . ruby.refactor-mode-launch)
)
\end{verbatim}

\subsection*{ruby hash syntax}
\label{sec:org6dc07fc}

Easily toggle ruby's hash syntax

\begin{verbatim}
(use-package ruby-hash-syntax
  :ensure t
)
\end{verbatim}

\subsection*{ruby additional}
\label{sec:org52aa48e}

Ruby rdoc helpers mostly

\begin{verbatim}
(use-package ruby-additional
  :ensure t
)
\end{verbatim}

\subsection*{ruby-tools}
\label{sec:org1c738f9}

\begin{verbatim}
(use-package ruby-tools
  :ensure t
)
\end{verbatim}

\subsection*{rspec-mode}
\label{sec:org8aafbe8}

Support for running rspec tests

\begin{verbatim}
(use-package rspec-mode
  :ensure t
)
\end{verbatim}

\subsection*{ruby-blocks}
\label{sec:org8824daf}

Highlights ruby def/end blocks
Unfortunatelly this package has to be installed manually, as its not available on MELPA

\begin{verbatim}
(use-package ruby-block
  :disabled
  :ensure t
  :init
  ;; do overlay
  (setq ruby-block-highlight-toggle 'overlay)
  ;; display to minibuffer
  (setq ruby-block-highlight-toggle 'minibuffer)
  ;; display to minibuffer and do overlay
  (setq ruby-block-highlight-toggle t)
  :config
  (ruby-block-mode t)
)
\end{verbatim}

\subsection*{ruby-extra-highlights}
\label{sec:orgf4d9566}

\begin{verbatim}
(use-package ruby-extra-highlight
  :ensure t
  :hook
  (ruby-mode . ruby-extra-highlight-mode)
)
\end{verbatim}


\section*{Go}
\label{sec:org8aaba47}

\begin{verbatim}
(use-package go-mode
  :mode "\\.go\\'"
  :hook (before-save . gofmt-before-save)
)
\end{verbatim}


\section*{Web-Mode}
\label{sec:org38d57da}
\begin{verbatim}
(use-package web-mode
  :custom-face
  (css-selector ((t (:inherit default :foreground "#66CCFF"))))
  (font-lock-comment-face ((t (:foreground "#828282"))))
  :mode
  ("\\.phtml\\'" "\\.tpl\\.php\\'" "\\.[agj]sp\\'" "\\.as[cp]x\\'"
  "\\.erb\\'" "\\.mustache\\'" "\\.djhtml\\'" "\\.[t]?html?\\'")
  :config
  ;; web-mode indentation
  (setq web-mode-markup-indent-offset 4)
  (setq web-mode-css-indent-offset 2)
  (setq web-mode-code-indent-offset 2)

  ;; Use tidy to check HTML buffers with web-mode.
  (eval-after-load 'flycheck
     '(flycheck-add-mode 'html-tidy 'web-mode))
)
\end{verbatim}

\section*{JavaScript JS2-mode}
\label{sec:orgf66f5d1}

\begin{verbatim}
;; js2-mode: enhanced JavaScript editing mode
;; https://github.com/mooz/js2-mode
(use-package js2-mode
  :mode
  ("\\.js$" . js2-mode)

  :hook
  (js2-mode . flycheck-mode)
  (js2-mode . company-mode)
  (js2-mode . tide-mode)
  (js2-mode . add-node-modules-path)
  :config
  ;; have 2 space indentation by default
  (setq js-indent-level 2)
  (setq js2-basic-offset 2)
  (setq js-chain-indent t)

  ;; use eslint_d insetad of eslint for faster linting
  ;; (setq flycheck-javascript-eslint-executable "eslint_d")

  ;; Try to highlight most ECMA built-ins
  (setq js2-highlight-level 3)

  ;; turn off all warnings in js2-mode
  (setq js2-mode-show-parse-errors t)
  (setq js2-mode-show-strict-warnings nil)
  (setq js2-strict-missing-semi-warning nil)
)
\end{verbatim}

\section*{PrettierJS}
\label{sec:org93e512b}

\begin{verbatim}
;; prettier-emacs: minor-mode to prettify javascript files on save
;; https://github.com/prettier/prettier-emacs
(use-package prettier-js
  :mode
  ("\\.js$" . prettier-js-mode)
  ("\\.scss$" . prettier-js-mode)
  :ensure-system-package
  (prettier . "npm install -g prettier")
  :hook
  (js2-mode . prettier-js-mode)
  (web-mode . prettier-js-mode)
  (rjsx-mode . prettier-js-mode)
  (css-mode . prettier-js-mode)
  (scss-mode . prettier-js-mode)
  (json-mode . prettier-js-mode)
  :init
  (setq prettier-js-show-errors 'buffer) ;; options: 'buffer, 'echo or nil
  :config
  (setq prettier-js-args '("--bracket-spacing" "false"
                           "--print-width" "80"
                           "--tab-width" "2"
                           "--use-tabs" "false"
                           "--no-semi" "true"
                           "--single-quote" "true"
                           "--trailing-comma" "none"
                           "--no-bracket-spacing" "true"
                           "--jsx-bracket-same-line" "false"
                           "--arrow-parens" "avoid"))

  ;; use prettier from local `node_modules' folder if available
  (defun tau/use-prettier-if-in-node-modules ()
    "Enable prettier-js-mode iff prettier was found installed locally in project"
    (interactive)
    (let* ((file-name (or (buffer-file-name) default-directory))
           (root (locate-dominating-file file-name "node_modules"))
           (prettier (and root
                          (expand-file-name "node_modules/prettier/bin-prettier.js" root))))
      (if (and prettier (file-executable-p prettier))
          (progn
            (message "Found local prettier executable at %s. Enabling prettier-js-mode" prettier)
            (setq prettier-js-command prettier)
            (make-variable-buffer-local 'prettier-js-command)
            (prettier-js-mode)
            (message "Disabling aggressive-indent-mode in favour of prettier")
            (aggressive-indent-mode -1))
        (progn
          (message "Prettier not found in %s. Not enabling prettier-js-mode" root)
          (message "Falling back to aggressive-indent-mode")
          (aggressive-indent-mode 1)))))
  (add-hook 'prettier-js-mode-hook #'tau/use-prettier-if-in-node-modules)

)
\end{verbatim}


\section*{JSON Mode}
\label{sec:org21805da}

\begin{verbatim}
;; json-mode: Major mode for editing JSON files with emacs
;; https://github.com/joshwnj/json-mode
(use-package json-mode
  :mode "\\.js\\(?:on\\|[hl]int\\(rc\\)?\\)\\'"
  :config
  (add-hook 'json-mode-hook #'prettier-js-mode)
  (setq json-reformat:indent-width 2)
  (setq json-reformat:pretty-string? t)
  (setq js-indent-level 2))
\end{verbatim}

\section*{ESLint}
\label{sec:org780a90f}

eslintd-fix: Emacs minor-mode to automatically fix javascript with eslint\textsubscript{d}.
\url{https://github.com/aaronjensen/eslintd-fix/tree/master}

\begin{verbatim}
(use-package eslintd-fix
  :ensure t
  :ensure-system-package
  (eslint . "npm i -g eslint")
  :config
  ;; Grab eslint executable from node_modules instead of global
  ;; Taken from https://github.com/flycheck/flycheck/issues/1087#issuecomment-246514860
  ;; Gist: https://github.com/lunaryorn/.emacs.d/blob/master/lisp/lunaryorn-flycheck.el#L62
  (defun lunaryorn-use-js-executables-from-node-modules ()
    "Set executables of JS and TS checkers from local node modules."
    (-when-let* ((file-name (buffer-file-name))
                 (root (locate-dominating-file file-name "node_modules"))
                 (module-directory (expand-file-name "node_modules" root)))
      (pcase-dolist (`(,checker . ,module) '((javascript-jshint . "jshint")
                                             (javascript-eslint . "eslint")
                                             (typescript-tslint . "tslint")
                                             (javascript-jscs   . "jscs")))
        (let ((package-directory (expand-file-name module module-directory))
              (executable-var (flycheck-checker-executable-variable checker)))
          (when (file-directory-p package-directory)
            (set (make-local-variable executable-var)
                 (expand-file-name (if (string= module "tslint")
                                       (concat "bin/" module)
                                     (concat "bin/" module ".js"))
                                   package-directory)))))))
)
\end{verbatim}

\section*{RJSX Mode}
\label{sec:org5f19f57}

\url{https://github.com/felipeochoa/rjsx-mode}

\begin{verbatim}
(use-package rjsx-mode
    :after js2-mode
    :mode
    ("\\.jsx$" . rjsx-mode)
    ("components/.+\\.js$" . rjsx-mode)

    :config
    ;; auto register for JS files that are inside a `components' folder
    (add-to-list 'auto-mode-alist '("components\\/.*\\.js\\'" . rjsx-mode))

    ;; for better jsx syntax-highlighting in web-mode
    ;; - courtesy of Patrick @halbtuerke
    (defadvice web-mode-highlight-part (around tweak-jsx activate)
      (if (equal web-mode-content-type "jsx")
        (let ((web-mode-enable-part-face nil))
          ad-do-it)
        ad-do-it))
)
\end{verbatim}

\section*{Typescript}
\label{sec:org8f6a14d}

\subsection*{Typescript mode}
\label{sec:org8b10feb}

\begin{verbatim}
(use-package typescript-mode
  :ensure t
  :mode (("\\.ts\\'" . typescript-mode)
         ("\\.tsx\\'" . typescript-mode))
)
\end{verbatim}

\begin{verbatim}
(defun setup-tide-mode ()
  (interactive)
  (tide-setup)
  (flycheck-mode +1)
  (setq flycheck-check-syntax-automatically '(save mode-enabled))
  (eldoc-mode +1)
  (tide-hl-identifier-mode +1)
  (company-mode +1)
)
\end{verbatim}

\subsection*{Tide}
\label{sec:org37b014d}
\begin{verbatim}
(use-package tide
  :ensure t
  :config
  (progn
    (add-hook 'before-save-hook 'tide-format-before-save)
    (add-hook 'typescript-mode-hook #'setup-tide-mode)
    (add-hook 'js2-mode-hook #'setup-tide-mode)
  )
)

\end{verbatim}

\section*{Angular}
\label{sec:org00ce184}
\begin{verbatim}
(use-package ng2-mode
  :defer
  :hook
  (ng2-mode . prettier-js-mode)
)
\end{verbatim}

\section*{Docker}
\label{sec:org9682473}

\begin{verbatim}
(use-package dockerfile-mode
  :mode "\\Dockerfile\\'"
)
\end{verbatim}


\section*{fun packages}
\label{sec:org20398ad}

\subsection*{xkcd}
\label{sec:orgc98f017}

\begin{verbatim}
(use-package xkcd
:ensure t
)
\end{verbatim}

\subsection*{fireplaces}
\label{sec:org8d0d17d}

\begin{verbatim}
(use-package fireplace
  :ensure t
  :init (defvar fireplace-mode-map)
  :bind (:map fireplace-mode-map
              ("d" . fireplace-down)
              ("s" . fireplace-toggle-smoke)
              ("u" . fireplace-up))
  :config
  (setq fireplace-toggle-smoke t)
  ;; (fireplace)
)
\end{verbatim}

\section*{Copy/Paste To/From System's Clipboard =D}
\label{sec:orgda0ba5b}
this was supposed to be on the helper functions and macro section at the beggining of the file
but it has evil defined keybindings and had to be put after the evil section or emacs would complain it didnt know what evil is

\subsubsection*{Copy to system clipboard}
\label{sec:orgd70597a}

\begin{verbatim}
(defun copy-to-clipboard ()
"Make F8 and F9 Copy and Paste to/from OS Clipboard.  Super usefull."
(interactive)
(if (display-graphic-p)
    (progn
        (message "Yanked region to x-clipboard!")
        (call-interactively 'clipboard-kill-ring-save)
        )
    (if (region-active-p)
        (progn
        (shell-command-on-region (region-beginning) (region-end) "xsel -i -b")
        (message "Yanked region to clipboard!")
        (deactivate-mark))
    (message "No region active; can't yank to clipboard!")))
)
\end{verbatim}

\subsubsection*{Paste}
\label{sec:org9e71dd3}

\begin{verbatim}
(evil-define-command paste-from-clipboard()
(if (display-graphic-p)
    (progn
        (clipboard-yank)
        (message "graphics active")
        )
    (insert (shell-command-to-string "xsel -o -b")) ) )
\end{verbatim}

\begin{verbatim}
(global-set-key [f9] 'copy-to-clipboard)
(global-set-key [f10] 'paste-from-clipboard)
\end{verbatim}


\section*{End init.el file}
\label{sec:org10cc525}
\begin{verbatim}
;; Local Variables:
;; coding: utf-8
;; no-byte-compile: t
;; End:


(provide 'init)
;;; .emacs ends here

\end{verbatim}
\end{document}
