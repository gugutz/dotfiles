% Created 2019-10-13 dom 03:22
% Intended LaTeX compiler: pdflatex
\documentclass[11pt]{article}
\usepackage[utf8]{inputenc}
\usepackage[T1]{fontenc}
\usepackage{graphicx}
\usepackage{grffile}
\usepackage{longtable}
\usepackage{wrapfig}
\usepackage{rotating}
\usepackage[normalem]{ulem}
\usepackage{amsmath}
\usepackage{textcomp}
\usepackage{amssymb}
\usepackage{capt-of}
\usepackage{hyperref}
\author{Gustavo P Borges}
\date{\today}
\title{gugutz emacs config\\\medskip
\large ORGfied configuration for Emacs}
\hypersetup{
 pdfauthor={Gustavo P Borges},
 pdftitle={gugutz emacs config},
 pdfkeywords={},
 pdfsubject={This file is compiled to init.el automatically on every save},
 pdfcreator={Emacs 26.3 (Org mode 9.2.6)},
 pdflang={English}}
\begin{document}

\maketitle
\setcounter{tocdepth}{0}
\tableofcontents

\begin{verbatim}
;; (setq user-full-name "Gustavo P Borges")
;; (setq user-mail-address "gugutz@gmail.com")
\end{verbatim}
\section*{Observations about this config}
\label{sec:org2550b35}
\begin{verbatim}
Keybindings use the <kbd> macro, as recommended in Mastering Emacs:
\end{verbatim}

\url{https://www.masteringemacs.org/article/mastering-key-bindings-emacs}

\begin{verbatim}
For native packages that come with emacs, `:ensure nil` must be set or use-package will try to download those packages from melpa and break
\end{verbatim}

\section*{Recompile init.el everytime emacs.org is changed and saved}
\label{sec:org722fef5}

\begin{verbatim}
Moved this to beggining of the file to avoid it not being parsed when theres an error in the middle of the file
It was being recompiled without this function so i had to manually re-copy first-init.el to make it compile first time again and again
\end{verbatim}



\begin{verbatim}
(defun /util/tangle-init ()
  (interactive)
  "If the current buffer is init.org' the code-blocks are
tangled, and the tangled file is compiled."
  (when (equal (buffer-file-name)
               (expand-file-name (concat user-emacs-directory "emacs.org")))
    ;; Avoid running hooks when tangling.
    (let ((prog-mode-hook nil))
      (org-babel-tangle)
      (byte-compile-file (concat user-emacs-directory "init.el")))))
\end{verbatim}

\begin{verbatim}
(add-hook 'after-save-hook #'/util/tangle-init)
\end{verbatim}

\section*{Packages}
\label{sec:orge690e3a}

\subsection*{package repositories}
\label{sec:orge7d3de9}

\begin{verbatim}
(require 'package)
;; add melpa stable emacs package repository
(add-to-list 'package-archives '("melpa" . "https://melpa.org/packages/"))
(add-to-list 'package-archives '("gnu" . "https://elpa.gnu.org/packages/"))
(add-to-list 'package-archives '("org" . "http://orgmode.org/elpa/") t) ; Org-mode's repository
\end{verbatim}

\subsection*{initialize packages}
\label{sec:org88cb0f3}
\begin{verbatim}
(package-initialize)
\end{verbatim}

moved this part to beggining of the file because if the
custom-safe-themes variable is not set before smart-mode-line (sml) activates
emacs asks 2 annoying confirmations on every startup before actually starting

\begin{verbatim}
(custom-set-variables
;; custom-set-variables was added by Custom.
;; If you edit it by hand, you could mess it up, so be careful.
;; Your init file should contain only one such instance.
;; If there is more than one, they won't work right.
'(custom-safe-themes
   (quote
   ("84d2f9eeb3f82d619ca4bfffe5f157282f4779732f48a5ac1484d94d5ff5b279" "57f95012730e3a03ebddb7f2925861ade87f53d5bbb255398357731a7b1ac0e0" "3c83b3676d796422704082049fc38b6966bcad960f896669dfc21a7a37a748fa" default)))
   '(fci-rule-color "#3E4451")
   '(package-selected-packages
     (quote
     (pdf-tools ox-pandoc ox-reveal org-preview-html latex-preview-pane smart-mode-line-powerline-theme base16-theme gruvbox-theme darktooth-theme rainbow-mode smartscan restclient editorconfig prettier-js pandoc rjsx-mode js2-refactor web-mode evil-org multiple-cursors flycheck smart-mode-line ## evil-leader evil-commentary evil-surround htmlize magit neotree evil json-mode web-serverx org))))
   (custom-set-faces
   ;; custom-set-faces was added by Custom.
   ;; If you edit it by hand, you could mess it up, so be careful.
   ;; Your init file should contain only one such instance.
   ;; If there is more than one, they won't work right.
   )
\end{verbatim}

\subsection*{Add the folder 'config' to emacs load-path so i can require stuff from there}
\label{sec:org7fabfc4}

\begin{verbatim}
(add-to-list 'load-path (expand-file-name "config" user-emacs-directory))
;; (add-to-list 'load-path "~/dotfiles/emacs.d/config")
\end{verbatim}

\subsection*{require use-package}
\label{sec:org0c06f6a}

\subsubsection*{Install use-package if not already installed}
\label{sec:org3b0a608}
\begin{verbatim}
(unless (package-installed-p 'use-package)
  (package-refresh-contents)
  (package-install 'use-package)
)
\end{verbatim}

\subsubsection*{load use-package}
\label{sec:org86e2279}
\begin{verbatim}
(eval-when-compile
  (require 'use-package))
\end{verbatim}

\subsubsection*{Enable use-package extension `ensure-system-package`}
\label{sec:org0b7bdc7}
\begin{verbatim}
(use-package use-package-ensure-system-package
  :ensure t
  :init
  ;; use sudo when needed
  (setq system-packages-use-sudo t)
)
\end{verbatim}

\subsubsection*{Set `:ensure t` globally for all packages using use-package}
\label{sec:orgfc8eb58}

\begin{verbatim}
this is disabled for now as i preffer to specify for each package
\end{verbatim}

\begin{verbatim}
(require 'use-package-ensure)
;; (setq use-package-always-ensure t)
\end{verbatim}

\subsubsection*{Auto update packages}
\label{sec:orgdb00c6d}
\begin{verbatim}
(use-package auto-package-update
  :config
  (setq auto-package-update-interval 7) ;; in days
  (setq auto-package-update-prompt-before-update t)
  (setq auto-package-update-delete-old-versions t)
  (setq auto-package-update-hide-results t)
  (auto-package-update-maybe)
)
\end{verbatim}


\section*{General editor settings}
\label{sec:org0de5f67}

\subsection*{Emacs Server}
\label{sec:org7679eeb}
Allow access from emacsclient
\begin{verbatim}
;; (use-package server
;;   :ensure nil
;;   :init
;;   (unless (or (daemonp) (server-running-p))
;;     (server-start))
;;   :hook (after-init . server-mode))
\end{verbatim}

\begin{verbatim}
(require 'server)
(unless (or (daemonp) (server-running-p))
  (server-start))
\end{verbatim}

\subsection*{enable visual-line-mode globally (word wrap)}
\label{sec:org3542815}
\begin{verbatim}
(use-package visual-line-mode
  :ensure nil
  :hook
  (after-init . visual-line-mode)
  (prog-mode . visual-line-mode)
  (text-mode . visual-line-mode)
)
\end{verbatim}

\subsection*{Prevent emacs to create lockfiles (.\#files\#).}
\label{sec:org6c511cc}

PS: this also stops preventing editing colisions, so watch out
\begin{verbatim}
(setq create-lockfiles nil)
\end{verbatim}

\subsection*{Use the system clipboard}
\label{sec:org6553b2d}
\begin{verbatim}
(setq x-select-enable-clipboard t)
\end{verbatim}

\subsection*{Always follow symbolic links to edit the 'actual' file it points to}
\label{sec:orgd6666e0}

\begin{verbatim}
(setq vc-follow-symlinks t)
\end{verbatim}

\subsection*{Save all tempfiles in \$TMPDIR/emacs\$UID/}
\label{sec:org191a788}

\begin{verbatim}
(defconst emacs-tmp-dir (expand-file-name (format "emacs%d" (user-uid)) temporary-file-directory))
(setq backup-directory-alist
    `((".*" . ,emacs-tmp-dir)))
(setq auto-save-file-name-transforms
    `((".*" ,emacs-tmp-dir t)))
(setq auto-save-list-file-prefix
    emacs-tmp-dir)
\end{verbatim}

\subsection*{Disable the annoying Emacs bell ring (beep)}
\label{sec:org5f572a8}

\begin{verbatim}
(setq ring-bell-function 'ignore)
\end{verbatim}

\subsection*{Disable initial scratch message}
\label{sec:orgfb70ed7}

\begin{verbatim}
(setq initial-scratch-message nil)
\end{verbatim}
\subsection*{Create alias to yes-or-no anwsers (y-or-n-p}
\label{sec:org9189f81}
\begin{verbatim}
(defalias 'yes-or-no-p 'y-or-n-p)
(fset 'yes-or-no-p 'y-or-n-p)
\end{verbatim}

\subsection*{show line numbers}
\label{sec:org46261eb}
\begin{verbatim}
(when (version<= "26.0.50" emacs-version )
  (global-display-line-numbers-mode))
\end{verbatim}

\subsection*{line number : pretty format}
\label{sec:org2111c66}
\begin{verbatim}
(setq linum-format " %d ")
\end{verbatim}

\subsection*{Turn on auto-revert mode (auto updates files changed on disk)}
\label{sec:org910f164}

\begin{verbatim}
(global-auto-revert-mode 1)
(setq auto-revert-interval 0.5)
\end{verbatim}

\subsection*{C-n insert newlines if the point is at the end of the buffer.}
\label{sec:org2f3e86b}

\begin{verbatim}
Useful, as it means you won’t have to reach for the return key to add newlines!
\end{verbatim}

\begin{verbatim}
(setq next-line-add-newlines t)
\end{verbatim}

\subsection*{Remove the \^{}M characters from files that contains Unix and DOS line endings}
\label{sec:org34f0739}

\begin{verbatim}
(defun remove-dos-eol ()
  "Do not show ^M in files containing mixed UNIX and DOS line endings."
  (interactive)
  (setq buffer-display-table (make-display-table))
  (aset buffer-display-table ?\^M [])
)
\end{verbatim}

\subsubsection*{Hook it to text-mode and prog-mode}
\label{sec:orgc0bf3b9}
\begin{verbatim}
(add-hook 'text-mode-hook 'remove-dos-eol)
(add-hook 'prog-mode-hook 'remove-dos-eol)
\end{verbatim}

\subsection*{Increase, decrease and adjust font size}
\label{sec:org768151b}

\begin{verbatim}
(global-set-key (kbd "C-S-+") #'text-scale-increase)
(global-set-key (kbd "C-S-_") #'text-scale-decrease)
(global-set-key (kbd "C-S-)") #'text-scale-adjust)
\end{verbatim}

\subsection*{expand-region}
\label{sec:org697d647}
\begin{verbatim}
;; (require 'expand-region)
(global-set-key (kbd "C-S-<tab>") 'er/expand-region)
\end{verbatim}

\subsection*{refresh buffer with F5}
\label{sec:orgefd29e2}
\begin{verbatim}
(global-set-key [f5] '(lambda () (interactive) (revert-buffer nil t nil)))
\end{verbatim}
\subsection*{C-k kills current buffer without having to select which buffer}
\label{sec:orge7ef2c3}

By default C-x k prompts to select which buffer should be selected.
I almost always want to kill the current buffer, so this snippet helps in that.
\begin{verbatim}
;; Kill current buffer; prompt only if
;; there are unsaved changes.
(global-set-key (kbd "C-x k")
  '(lambda () (interactive) (kill-buffer (current-buffer)))
)
\end{verbatim}

\subsection*{smooth scrolling}
\label{sec:org6c83eba}

\begin{verbatim}
;; Vertical Scroll
(setq scroll-step 1)
(setq scroll-margin 1)
(setq scroll-conservatively 101)
(setq scroll-up-aggressively 0.01)
(setq scroll-down-aggressively 0.01)
(setq auto-window-vscroll nil)
(setq fast-but-imprecise-scrolling nil)
(setq mouse-wheel-scroll-amount '(1 ((shift) . 1)))
(setq mouse-wheel-progressive-speed nil)
;; Horizontal Scroll
(setq hscroll-step 1)
(setq hscroll-margin 1)
\end{verbatim}


\section*{Code editing settings}
\label{sec:orge8e07ad}
\subsection*{subword-mode}
\label{sec:org52e6b24}

\begin{verbatim}
Alt+x subword-mode. It change all cursor movement/edit commands to stop in-between the “camelCase” words.
subword-mode and superword-mode are mutally exclusive. Turning one on turns off the other.
\end{verbatim}


\begin{verbatim}
(use-package subword
  :ensure nil
  :hook
  (clojure-mode . subword-mode)
  (ruby-mode . subword-mode)
  (enh-ruby-mode . subword-mode)
  (elixir-mode . subword-mode)
)
\end{verbatim}

\subsection*{superword-mode}
\label{sec:org4e53c38}

\begin{verbatim}
Alt+x superword-mode (emacs 24.4) is similar. It treats text like “x_y” as one word. Useful for “snake_case”.
subword-mode and superword-mode are mutally exclusive. Turning one on turns off the other.
\end{verbatim}


\begin{verbatim}
(use-package superword
  :ensure nil
  :hook
  (js2-mode . superword-mode)
)
\end{verbatim}

\subsection*{show matching parenthesis}
\label{sec:org6ce399f}
\begin{verbatim}
; parentheses
(show-paren-mode t)
\end{verbatim}

\subsection*{default indentation}
\label{sec:org5709edf}
\begin{verbatim}
(setq-default indent-tabs-mode nil)
;; C e C-like langs default indent size
(setq-default tab-width 2)
;; Perl default indent size
(setq-default cperl-basic-offset 2)
(setq-default c-basic-offset 2)
\end{verbatim}

\subsection*{Use unix-conf-mode for .*rc files}
\label{sec:org5ba23c2}
\begin{verbatim}
(add-to-list 'auto-mode-alist '("\\.*rc$" . conf-unix-mode))
\end{verbatim}

\subsection*{iedit}
\label{sec:orgf4bbf9e}
\begin{verbatim}
(use-package iedit
  :config
  (set-face-background 'iedit-occurrence "Magenta")
  :bind
  ("C-;" . iedit-mode)
)
\end{verbatim}

\subsection*{eldoc}
\label{sec:org6370ecf}

Enable documentation for programming languages

\begin{verbatim}
(use-package eldoc
  :diminish
  :hook
  (prog-mode       . turn-on-eldoc-mode)
  (cider-repl-mode . turn-on-eldoc-mode)
)
\end{verbatim}

\subsection*{Highlighting numbers}
\label{sec:org00a8a31}
\begin{verbatim}
(use-package highlight-numbers
    :ensure t
    :hook
    (prog-mode . highlight-numbers-mode)
)
\end{verbatim}
\subsection*{Highlighting operators}
\label{sec:orgbea0089}
\begin{verbatim}
(use-package highlight-operators
  :ensure t
  :hook
  (prog-mode . highlight-operators-mode)
)
\end{verbatim}
\subsection*{Highlighting escape sequences}
\label{sec:org4113a14}

\begin{verbatim}
(use-package highlight-escape-sequences
  :ensure t
  :hook
  (prog-mode . hes-mode)
)
\end{verbatim}
\subsection*{Highlighting parentheses}
\label{sec:org6f205bd}

\begin{verbatim}
(use-package highlight-parentheses
  :ensure t
  :hook
  (prog-mode . highlight-parentheses-mode)
)
\end{verbatim}


\section*{Text editing settings}
\label{sec:org244ae89}

\subsection*{Helper functions for casing words}
\label{sec:org423774f}

\begin{verbatim}
(defun upcase-backward-word (arg)
  (interactive "p")
  (upcase-word (- arg))
)
\end{verbatim}

\begin{verbatim}
(defun downcase-backward-word (arg)
  (interactive "p")
  (downcase-word (- arg))
)
\end{verbatim}

\begin{verbatim}
(defun capitalize-backward-word (arg)
  (interactive "p")
  (capitalize-word (- arg))
)
\end{verbatim}

\begin{verbatim}
(global-set-key (kbd "C-M-u")	 'upcase-backward-word)
(global-set-key (kbd "C-M-l")	 'downcase-backward-word)
;; this replaces native capitlize word!
(global-set-key (kbd "M-c")	 'capitalize-backward-word)
\end{verbatim}

\subsection*{Spellchecking}
\label{sec:orgf3109d5}

\begin{verbatim}
(defconst *spell-check-support-enabled* t) ;; Enable with t if you prefer
\end{verbatim}

\subsection*{Flyspell}
\label{sec:org328c4de}

Change dictionaries with F12

\begin{verbatim}
;(defun fd-switch-dictionary()
;(interactive)
;(let* ((dic ispell-current-dictionary)
;    (change (if (string= dic "deutsch8") "english" "deutsch8")))
;  (ispell-change-dictionary change)
;  (message "Dictionary switched from %s to %s" dic change)
;  ))

;(global-set-key (kbd "<f12>")   'fd-switch-dictionary)
\end{verbatim}

\begin{verbatim}
;; Change dictionaries with F12 (teste pt-br)
(let ((langs '("american" "brasileiro")))
  (setq lang-ring (make-ring (length langs)))
  (dolist (elem langs) (ring-insert lang-ring elem))
)

(defun cycle-ispell-languages ()
   (interactive)
   (let ((lang (ring-ref lang-ring -1)))
     (ring-insert lang-ring lang)
     (ispell-change-dictionary lang))
)

(global-set-key (kbd "<f12>")   'cycle-ispell-languages)
\end{verbatim}

\begin{verbatim}
(use-package flyspell
  :defer 1
  :hook
  (text-mode . flyspell-mode)
  :config
  ;; ignore org source blocks from spellchecking
  (add-to-list 'ispell-skip-region-alist '(":\\(PROPERTIES\\|LOGBOOK\\):" . ":END:"))
  (add-to-list 'ispell-skip-region-alist '("^#+BEGIN_SRC" . "^#+END_SRC"))

  ;; global ispell settings (disabled in favor of conditional hunspell setup bellow)
  ;; (setenv "LANG" "en_US.UTF-8")
  ;; (setq ispell-program-name "aspell")
  ;; (setq ispell-program-name "hunspell")
  ;; (setq ispell-dictionary "en_US")
  ;; (setq ispell-local-dictionary "pt_BR")
  ;; (setq ispell-local-dictionary "en_US")

  ;; Hunspell settings
  ;; find aspell and hunspell automatically
  (cond
   ;; try hunspell at first
   ;; if hunspell does NOT exist, use aspell
    ((executable-find "hunspell")
      (setq ispell-program-name "hunspell")
      ;; i could set `ispell-dictionary' instead but `ispell-local-dictionary' has higher priority
      (setq ispell-local-dictionary "en_US")
      ;; setup both en_US and pt_BR dictionaries in hunspell
      ;; (ispell-hunspell-add-multi-dic "en_GB,en_US-med")

      (setq ispell-local-dictionary-alist
         ;; Please note the list `("-d" "en_US")` contains ACTUAL parameters passed to hunspell
         ;; You could use `("-d" "en_US,en_US-med")` to check with multiple dictionaries
         '(("en_US" "[[:alpha:]]" "[^[:alpha:]]" "[']" nil ("-d" "en_US,pt_BR") nil utf-8)
           ))
    )
    ;; hunspell was not found, use aspell
    ((executable-find "aspell")
      (setq ispell-program-name "aspell")
      ;; Please note `ispell-extra-args' contains ACTUAL parameters passed to aspell
      (setq ispell-extra-args '("--sug-mode=ultra" "--lang=en_US"))
      ;;(setq ispell-local-dictionary "pt_BR")
    )
  )

)
\end{verbatim}



\section*{GPG Encryption}
\label{sec:orgef09588}

\begin{verbatim}
(require 'epa-file)
(epa-file-enable)
\end{verbatim}

\section*{Make emacs use \$PATH defined in the systems shell}
\label{sec:org0a9bc6c}

\begin{verbatim}
snippet taken from oficial use package github page
\end{verbatim}

\begin{verbatim}
(use-package exec-path-from-shell
  :if (memq window-system '(mac ns x))
  :ensure t
  :init
  ;;(setenv "SHELL" "/bin/zsh")
  ;;(setq explicit-shell-file-name "/bin/zsh")
  ;;(setq shell-file-name "zsh")
  :config
  ;; This sets $MANPATH, $PATH and exec-path from your shell, but only on OS X and Linux.
  (exec-path-from-shell-initialize)
  ;; Its possible to copy values from other SHELL variables using one of the two methods bellow
  ;; either using the `exec-path-from-shell-copy-env' functon or setting the variable `exec-path-from-shell-variables'
  ;; (exec-path-from-shell-copy-env "PYTHONPATH")
  ;; (setq exec-path-from-shell-variables '("PYTHONPATH" "GOPATH"))
)
\end{verbatim}

\section*{Mouse configuration}
\label{sec:org8868556}
\subsection*{Enable mouse support in terminal mode}
\label{sec:org62deb26}

\begin{verbatim}
(when (eq window-system nil)
  (xterm-mouse-mode 1))
\end{verbatim}

\begin{verbatim}
;; (use-package mouse3
;;     :config
;; (global-set-key (kbd "<mouse-3>") 'mouse3-popup-menu))
\end{verbatim}

\subsection*{right-click-context-menu}
\label{sec:org58b4299}

\begin{verbatim}
(use-package right-click-context
  :ensure t
  :config
  (global-set-key (kbd "<menu>") 'right-click-context-menu)
  (global-set-key (kbd "<mouse-3>") 'right-click-context-menu)
  (bind-key "C-c <mouse-3>" 'right-click-context-menu)

  ;; (setq right-click-context-mode-lighter "🐭")

  ;; customize the right-click-context-menu
  (let ((right-click-context-local-menu-tree
       (append right-click-context-global-menu-tree
             '(("Insert"
                ("Go to definition" :call (lsp-goto-type-definition)
                ("FooBar" :call (insert "FooBar"))
                )))))
  (right-click-context-menu))))
\end{verbatim}

\section*{hippie-expand (native emacs expand function)}
\label{sec:org0b949c6}

\begin{verbatim}
(use-package hippie-exp
  ;;:ensure nil
  :defer t
  :bind
  ("<tab>" . hippie-expand)
  ("<C-return>" . hippie-expand)
  ("C-M-SPC" . hippie-expand)
  (:map evil-insert-state-map
  ("<tab>" . hippie-expand)
  )
  :config
  (setq-default hippie-expand-try-functions-list
        '(yas-hippie-try-expand
          yas-expand
          company-indent-or-complete-common
          emmet-expand-line
          )
  )
)
\end{verbatim}


\section*{undo-tree}
\label{sec:org768ffc8}
\begin{verbatim}
(use-package undo-tree
  :ensure t
  :init
  (global-undo-tree-mode)
;;  (undo-tree-mode)
)
\end{verbatim}

\section*{Evil}
\label{sec:org685cc8c}

\begin{verbatim}
(use-package evil
    :ensure t
    :init
    (setq evil-ex-complete-emacs-commands nil)
    (setq evil-vsplit-window-right t)
    (setq evil-split-window-below t)
    (setq evil-shift-round nil)
    (setq evil-esc-delay 0)  ;; Don't wait for any other keys after escape is pressed.
    ;; Make Evil look a bit more like (n) vim  (??)
    (setq evil-search-module 'isearch-regexp)
    ;; (setq evil-search-module 'evil-search)
    (setq evil-magic 'very-magic)
    (setq evil-shift-width (symbol-value 'tab-width))
    (setq evil-regexp-search t)
    (setq evil-search-wrap t)
    ;; (setq evil-want-C-i-jump t)
    (setq evil-want-C-u-scroll t)
    (setq evil-want-fine-undo nil)
    (setq evil-want-integration nil)
    ;; (setq evil-want-abbrev-on-insert-exit nil)
    (setq evil-want-abbrev-expand-on-insert-exit nil)
    (setq evil-mode-line-format '(before . mode-line-front-space)) ;; move evil tag to beginning of modeline
    ;; Cursor is alway black because of evil.
    ;; Here is the workaround
    ;; (@see https://bitbucket.org/lyro/evil/issue/342/evil-default-cursor-setting-should-default)
    (setq evil-default-cursor t)
    ;; change cursor color according to mode
    (setq evil-emacs-state-cursor '("#ff0000" box))
    (setq evil-motion-state-cursor '("#FFFFFF" box))
    (setq evil-normal-state-cursor '("#00ff00" box))
    (setq evil-visual-state-cursor '("#abcdef" box))
    (setq evil-insert-state-cursor '("#e2f00f" bar))
    (setq evil-replace-state-cursor '("red" hbar))
    (setq evil-operator-state-cursor '("red" hollow))

  :bind
  (:map evil-normal-state-map
  ("g t" . centaur-tabs-forward)
  ("g T" . centaur-tabs-backward)
  (", w" . evil-window-vsplit)
  ("C-r" . undo-tree-redo)
  )
  (:map evil-insert-state-map
  ;; this is also defined globally above in the config
  ("C-S-<tab>" . er/expand-region)
  )

;; check if global-set-key also maps to evil insert mode; if yes delete bellow snippets
  :config
  (evil-mode)

;; unset evil bindings that conflits with other stuff
  (define-key evil-insert-state-map (kbd "<tab>") nil)
  (define-key evil-normal-state-map (kbd "<tab>") nil)
  (define-key evil-visual-state-map (kbd "<tab>") nil)

  ;; vim-like navigation with C-w hjkl
  (define-prefix-command 'evil-window-map)
  (define-key evil-window-map (kbd "h") 'evil-window-left)
  (define-key evil-window-map (kbd "j") 'evil-window-down)
  (define-key evil-window-map (kbd "k") 'evil-window-up)
  (define-key evil-window-map (kbd "l") 'evil-window-right)
  (define-key evil-window-map (kbd "b") 'evil-window-bottom-right)
  (define-key evil-window-map (kbd "c") 'evil-window-delete)
  (define-key evil-motion-state-map (kbd "M-w") 'evil-window-map)

  ;; make esc quit or cancel everything in Emacs
  (define-key evil-normal-state-map [escape] 'keyboard-quit)
  (define-key evil-visual-state-map [escape] 'keyboard-quit)
  (define-key minibuffer-local-map [escape] 'minibuffer-keyboard-quit)
  (define-key minibuffer-local-ns-map [escape] 'minibuffer-keyboard-quit)
  (define-key minibuffer-local-completion-map [escape] 'minibuffer-keyboard-quit)
  (define-key minibuffer-local-must-match-map [escape] 'minibuffer-keyboard-quit)
  (define-key minibuffer-local-isearch-map [escape] 'minibuffer-keyboard-quit)

  ;; recover native emacs commands that are overriden by evil
  ;; this gives priority to native emacs behaviour rathen than Vim's
  (define-key evil-normal-state-map (kbd "SPC") 'ace-jump-mode)
  (define-key evil-visual-state-map (kbd "SPC") 'ace-jump-mode)
  (define-key evil-normal-state-map (kbd "C-e") 'evil-end-of-line)
  (define-key evil-insert-state-map (kbd "C-e") 'move-end-of-line)
  (define-key evil-visual-state-map (kbd "C-e") 'evil-end-of-line)
  (define-key evil-motion-state-map (kbd "C-e") 'evil-end-of-line)
  (define-key evil-insert-state-map (kbd "C-d") 'evil-delete-char)
  (define-key evil-normal-state-map (kbd "C-d") 'evil-delete-char)
  (define-key evil-visual-state-map (kbd "C-d") 'evil-delete-char)
  (define-key evil-normal-state-map (kbd "C-k") 'kill-line)
  (define-key evil-insert-state-map (kbd "C-k") 'kill-line)
  (define-key evil-visual-state-map (kbd "C-k") 'kill-line)
  (define-key evil-insert-state-map (kbd "C-w") 'kill-region)
  (define-key evil-normal-state-map (kbd "C-w") 'kill-region)
  (define-key evil-visual-state-map (kbd "C-w") 'kill-region)
  (define-key evil-normal-state-map (kbd "C-w") 'evil-delete)
  (define-key evil-insert-state-map (kbd "C-w") 'evil-delete)
  (define-key evil-visual-state-map (kbd "C-w") 'evil-delete)
  (define-key evil-normal-state-map (kbd "C-y") 'yank)
  (define-key evil-insert-state-map (kbd "C-y") 'yank)
  (define-key evil-visual-state-map (kbd "C-y") 'yank)
  (define-key evil-normal-state-map (kbd "C-f") 'evil-forward-char)
  (define-key evil-insert-state-map (kbd "C-f") 'evil-forward-char)
  (define-key evil-insert-state-map (kbd "C-f") 'evil-forward-char)
  (define-key evil-normal-state-map (kbd "C-b") 'evil-backward-char)
  (define-key evil-insert-state-map (kbd "C-b") 'evil-backward-char)
  (define-key evil-visual-state-map (kbd "C-b") 'evil-backward-char)
  (define-key evil-normal-state-map (kbd "C-n") 'evil-next-line)
  (define-key evil-insert-state-map (kbd "C-n") 'evil-next-line)
  (define-key evil-visual-state-map (kbd "C-n") 'evil-next-line)
  (define-key evil-normal-state-map (kbd "C-p") 'evil-previous-line)
  (define-key evil-insert-state-map (kbd "C-p") 'evil-previous-line)
  (define-key evil-visual-state-map (kbd "C-p") 'evil-previous-line)
  (define-key evil-normal-state-map (kbd "Q") 'call-last-kbd-macro)
  (define-key evil-visual-state-map (kbd "Q") 'call-last-kbd-macro)
  (define-key evil-insert-state-map (kbd "C-r") 'search-backward)
)
\end{verbatim}


\subsection*{Window and buffer navigation with vim-like bindings}
\label{sec:org51bc9c9}

\subsubsection*{vim-like navigation using C- HJKL (uppercase homerow keys)}
\label{sec:org93a2d3a}
\begin{verbatim}
(use-package windmove
  :ensure t
  :config
  ;; use shift + arrow keys to switch between visible buffers
  ;; (windmove-default-keybindings)
  (windmove-default-keybindings 'control)
  (global-set-key (kbd "C-S-H") 'windmove-left)
  (global-set-key (kbd "C-S-L") 'windmove-right)
  (global-set-key (kbd "C-S-K") 'windmove-up)
  (global-set-key (kbd "C-S-J") 'windmove-down)
)
\end{verbatim}




\subsection*{evil-leader}
\label{sec:org88493c4}

\begin{verbatim}
(use-package evil-leader
  :config
  (global-evil-leader-mode)
  (evil-leader/set-leader ",")
  (evil-leader/set-key
    "e" 'find-file
    "q" 'evil-quit
    "w" 'save-buffer
    "d" 'delete-frame
    "k" 'kill-buffer
    "b" 'switch-to-buffer
    "-" 'split-window-bellow
    "|" 'split-window-right)
)
\end{verbatim}

\subsection*{Evil Surround}
\label{sec:orgd80df70}

\begin{verbatim}
(use-package evil-surround
  :config
  (global-evil-surround-mode 1)
)
\end{verbatim}

\begin{verbatim}
(defun evil-surround-prog-mode-hook-setup ()
  "Documentation string, idk, put something here later."
  (push '(47 . ("/" . "/")) evil-surround-pairs-alist)
  (push '(40 . ("(" . ")")) evil-surround-pairs-alist)
  (push '(41 . ("(" . ")")) evil-surround-pairs-alist)
  (push '(91 . ("[" . "]")) evil-surround-pairs-alist)
  (push '(93 . ("[" . "]")) evil-surround-pairs-alist)
)
(add-hook 'prog-mode-hook 'evil-surround-prog-mode-hook-setup)
\end{verbatim}

\begin{verbatim}
(defun evil-surround-js-mode-hook-setup ()
  "ES6." ;  this is a documentation string, a feature in Lisp
  ;; I believe this is for auto closing pairs
  (push '(?1 . ("{`" . "`}")) evil-surround-pairs-alist)
  (push '(?2 . ("${" . "}")) evil-surround-pairs-alist)
  (push '(?4 . ("(e) => " . "(e)")) evil-surround-pairs-alist)
  ;; ReactJS
  (push '(?3 . ("classNames(" . ")")) evil-surround-pairs-alist)
)
(add-hook 'js2-mode-hook 'evil-surround-js-mode-hook-setup)
\end{verbatim}

\begin{verbatim}
(defun evil-surround-emacs-lisp-mode-hook-setup ()
  (push '(?` . ("`" . "'")) evil-surround-pairs-alist)
)
(add-hook 'emacs-lisp-mode-hook 'evil-surround-emacs-lisp-mode-hook-setup)

(defun evil-surround-org-mode-hook-setup ()
  (push '(91 . ("[" . "]")) evil-surround-pairs-alist)
  (push '(93 . ("[" . "]")) evil-surround-pairs-alist)
  (push '(?= . ("=" . "=")) evil-surround-pairs-alist)
)
(add-hook 'org-mode-hook 'evil-surround-org-mode-hook-setup)
\end{verbatim}


\subsection*{Vim Commentary}
\label{sec:org6a27f70}

\begin{verbatim}
(use-package evil-commentary
  :config
  (evil-commentary-mode)
)
\end{verbatim}

\subsection*{Evil-Matchit}
\label{sec:orgbb6c468}
\begin{verbatim}
(use-package evil-matchit
  :config
  (global-evil-matchit-mode 1)
)
\end{verbatim}

\section*{corral - intelligent surround text with auto-guess suggestions}
\label{sec:orgc89889d}
\begin{verbatim}
(use-package corral
  :bind
  ("M-9" . corral-parentheses-backward)
  :config
  (setq corral-preserve-point t)
  ;;(global-set-key (kbd "M-9") 'corral-parentheses-backward)
  (global-set-key (kbd "M-0") 'corral-parentheses-forward)
  (global-set-key (kbd "M-[") 'corral-brackets-backward)
  (global-set-key (kbd "M-]") 'corral-brackets-forward)
  (global-set-key (kbd "M-{") 'corral-braces-backward)
  (global-set-key (kbd "M-}") 'corral-braces-forward)
  (global-set-key (kbd "M-\"") 'corral-double-quotes-backward)
)
\end{verbatim}

\section*{helpfull - a better replacement for emacs help system}
\label{sec:orga4a1798}

\begin{verbatim}
(use-package helpful
  :ensure t
  :config
  (global-set-key (kbd "C-h f") #'helpful-callable)
  (global-set-key (kbd "C-h v") #'helpful-variable)
  (global-set-key (kbd "C-h k") #'helpful-key)

  ;; Lookup the current symbol at point. C-c C-d is a common keybinding
  ;; for this in lisp modes.
  (global-set-key (kbd "C-c C-d") #'helpful-at-point)

  ;; Look up *F*unctions (excludes macros).
  ;;
  ;; By default, C-h F is bound to `Info-goto-emacs-command-node'. Helpful
  ;; already links to the manual, if a function is referenced there.
  (global-set-key (kbd "C-h F") #'helpful-function)

  ;; Look up *C*ommands.
  ;;
  ;; By default, C-h C is bound to describe `describe-coding-system'. I
  ;; don't find this very useful, but it's frequently useful to only
  ;; look at interactive functions.
  (global-set-key (kbd "C-h C") #'helpful-command)
)
\end{verbatim}
\section*{paredit}
\label{sec:org66194b2}

\begin{verbatim}
(use-package paredit
  :ensure t
  :config
  (autoload 'enable-paredit-mode "paredit" "Turn on pseudo-structural editing of Lisp code." t)
  (add-hook 'emacs-lisp-mode-hook       #'enable-paredit-mode)
  (add-hook 'eval-expression-minibuffer-setup-hook #'enable-paredit-mode)
  (add-hook 'ielm-mode-hook             #'enable-paredit-mode)
  (add-hook 'lisp-mode-hook             #'enable-paredit-mode)
  (add-hook 'lisp-interaction-mode-hook #'enable-paredit-mode)
  (add-hook 'scheme-mode-hook           #'enable-paredit-mode)
)
\end{verbatim}


\section*{evil-paredit}
\label{sec:orgba3a55b}

\begin{verbatim}
(use-package evil-paredit
  :ensure t
  :hook
  (emacs-lisp-mode . evil-paredit-mode)
)
\end{verbatim}
\section*{parinfer-mode}
\label{sec:org67cb31d}

\begin{verbatim}
(use-package parinfer
  :ensure t
  :bind
  ("C-," . parinfer-toggle-mode)
  :init
  (progn
    (setq parinfer-extensions
          '(defaults       ; should be included.
            pretty-parens  ; different paren styles for different modes.
            evil           ; If you use Evil.
            ;lispy          ; If you use Lispy. With this extension, you should install Lispy and do not enable lispy-mode directly.
            paredit        ; Introduce some paredit commands.
            smart-tab      ; C-b & C-f jump positions and smart shift with tab & S-tab.
            smart-yank))   ; Yank behavior depend on mode.
    (add-hook 'clojure-mode-hook #'parinfer-mode)
    (add-hook 'emacs-lisp-mode-hook #'parinfer-mode)
    (add-hook 'common-lisp-mode-hook #'parinfer-mode)
    (add-hook 'scheme-mode-hook #'parinfer-mode)
    (add-hook 'lisp-mode-hook #'parinfer-mode))
    :config
    ;; auto switch to Indent Mode whenever parens are balance in Paren Mode
    (setq parinfer-auto-switch-indent-mode nil)  ;; default is nil
    (setq parinfer-lighters '(" Parinfer:Indent" . "Parinfer:Paren"))

)
\end{verbatim}

\section*{elisp-format}
\label{sec:org4bbee9e}

\begin{verbatim}
(use-package elisp-format
  :ensure t
)
\end{verbatim}


\section*{Shell}
\label{sec:org9390862}

\subsection*{System Shell}
\label{sec:orgb44a819}
\subsubsection*{Make system shell open in a split-window buffer at the bottom of the screen}
\label{sec:org4becafd}

\begin{verbatim}
(defun /shell/new-window ()
    "Opens up a new shell in the directory associated with the current buffer's file."
    (interactive)
    (let* ((parent (if (buffer-file-name)
                        (file-name-directory (buffer-file-name))
                    default-directory))
            (height (/ (window-total-height) 3))
            (name   (car (last (split-string parent "/" t)))))
        (split-window-vertically (- height))
        (other-window 1)
        (shell "new")
        (rename-buffer (concat "*shell: " name "*"))
        (insert (concat "ls"))
    )
)

; Pull system shell in a new bottom window
(define-key evil-normal-state-map (kbd "\"") #'/shell/new-window)
(define-key evil-visual-state-map (kbd "\"") #'/shell/new-window)
(define-key evil-motion-state-map (kbd "\"") #'/shell/new-window)
\end{verbatim}


\subsection*{Eshell}
\label{sec:org1d7b6d3}

\subsubsection*{Make eshell open in a split-window buffer at the bottom of the screen}
\label{sec:orgf63ad59}

\begin{verbatim}
(defun /eshell/new-window ()
    "Opens up a new eshell in the directory associated with the current buffer's file.  The eshell is renamed to match that directory to make multiple eshell windows easier."
    (interactive)
    (let* ((parent (if (buffer-file-name)
                       (file-name-directory (buffer-file-name))
                     default-directory))
           (height (/ (window-total-height) 3))
           (name   (car (last (split-string parent "/" t)))))
      (split-window-vertically (- height))
      (other-window 1)
      (eshell "new")
      (rename-buffer (concat "*eshell: " name "*"))

      (insert (concat "ls"))
      (eshell-send-input)))

; Pull eshell in a new bottom window
(define-key evil-normal-state-map (kbd "!") #'/eshell/new-window)
(define-key evil-visual-state-map (kbd "!") #'/eshell/new-window)
(define-key evil-motion-state-map (kbd "!") #'/eshell/new-window)
\end{verbatim}


\section*{restclient}
\label{sec:org244ecf5}

\begin{verbatim}
(use-package restclient
  :ensure t
  :mode "\\.rest$"
  :config
  (progn
    ;; Add hook to override C-c C-c in this mode to stay in window
    (add-hook 'restclient-mode-hook
              '(lambda ()
                 (local-set-key
                  (kbd "C-c C-c")
                  'restclient-http-send-current-stay-in-window))))
)
\end{verbatim}

\section*{multiple cursors}
\label{sec:org4fd8842}

\begin{verbatim}
(use-package multiple-cursors
  :after evil
  ;; step 1, select thing in visual-mode (OPTIONAL)
  ;; step 2, `mc/mark-all-like-dwim' or `mc/mark-all-like-this-in-defun'
  ;; step 3, `ace-mc-add-multiple-cursors' to remove cursor, press RET to confirm
  ;; step 4, press s or S to start replace
  ;; step 5, press C-g to quit multiple-cursors
  :bind
  ("M-u" . hydra-multiple-cursors/body)
  :config
  (define-key evil-visual-state-map (kbd "mn") 'mc/mark-next-like-this)
  (define-key evil-visual-state-map (kbd "ma") 'mc/mark-all-like-this-dwim)
  (define-key evil-visual-state-map (kbd "md") 'mc/mark-all-like-this-in-defun)
  (define-key evil-visual-state-map (kbd "mm") 'ace-mc-add-multiple-cursors)
  (define-key evil-visual-state-map (kbd "ms") 'ace-mc-add-single-cursor)
)
\end{verbatim}


\section*{org-mode}
\label{sec:org6c8e1eb}

\subsection*{org-mode setup}
\label{sec:orgaeb7fc4}
\begin{verbatim}
(use-package org
  :ensure org-plus-contrib
  :defer t
  :bind
  ("C-c l" . org-store-link)
  ("C-c a" . org-agenda)
  ("C-c c" . org-capture)
  ("C-c b" . org-switch)
  ;; this map is to delete de bellow commented lambda that does the same thing
  ;; Resolve issue with Tab not working with ORG only in Normal VI Mode in terminal
  ;; (something with TAB on terminals being related to C-i...)
  (:map evil-normal-state-map
  ("<tab>" . org-cycle)
  )
  :config
  ;;(add-hook 'org-mode-hook
  ;;          (lambda ()
  ;;        (define-key evil-normal-state-map (kbd "TAB") 'org-cycle)))

  ;; general org config variables
  (setq org-log-done 'time)
  (setq org-export-backends (quote (ascii html icalendar latex md odt)))
  (setq org-use-speed-commands t)

  ;; dont display atual width for images inline. set per-file with
  ;; #+ATTR_HTML: :width 600px :height: auto
  ;; #+ATTR_ORG: :width 600
  ;; #+ATTR_LATEX: :width 5in
  (setq org-image-actual-width nil)

  (setq org-confirm-babel-evaluate 'nil)
  (setq org-todo-keywords
   '((sequence "TODO" "IN-PROGRESS" "REVIEW" "|" "DONE")))
  (setq org-agenda-window-setup 'other-window)
  (setq org-log-done 'time) ;; Show CLOSED tag line in closed TODO items
  (setq org-log-done 'note) ;; Prompt to leave a note when closing an item

  ;;ox-twbs (exporter to twitter bootstrap html)
  (setq org-enable-bootstrap-support t)

  (defun org-export-turn-on-syntax-highlight()
    "Setup variables to turn on syntax highlighting when calling `org-latex-export-to-pdf'"
    (interactive)
    (setq org-latex-listings 'minted
          org-latex-packages-alist '(("" "minted"))
          org-latex-pdf-process
          '("pdflatex -shell-escape -interaction nonstopmode -output-directory %o %f"
            "pdflatex -shell-escape -interaction nonstopmode -output-directory %o %f")))
)
\end{verbatim}

\subsection*{ox-extra (org-plus-contrib)}
\label{sec:orgdcfbbaa}

ox-extras
add suport for the ignore tag (ignores a headline without ignoring its content)

\begin{verbatim}
(use-package ox-extra
  :ensure nil
  :config
  (ox-extras-activate '(ignore-headlines))
  (ox-extras-activate '(latex-header-blocks ignore-headlines))
)
\end{verbatim}

\subsection*{Evil-ORG}
\label{sec:org5a5dc59}

\begin{verbatim}
(use-package evil-org
  :after org
  :hook
  (org-mode . evil-org-mode)
  :config
  (lambda ()
    (evil-org-set-key-theme))
)
\end{verbatim}

\subsection*{ox-pandoc}
\label{sec:orgafbe85e}

\begin{NOTE}
As pandoc supports many number of formats, initial org-export-dispatch
shortcut menu does not show full of its supported formats. You can customize
org-pandoc-menu-entry variable (and probably restart Emacs) to change its
default menu entries.
If you want delayed loading of `ox-pandoc’ when org-pandoc-menu-entry
is customized, please consider the following settings in your init file"
\end{NOTE}

\begin{verbatim}
(use-package ox-pandoc
  :after (org ox)
  :config
  ;; default options for all output formats
  (setq org-pandoc-options '((standalone . t)))
  ;; cancel above settings only for 'docx' format
  (setq org-pandoc-options-for-docx '((standalone . nil)))
  ;; special settings for beamer-pdf and latex-pdf exporters
  (setq org-pandoc-options-for-beamer-pdf '((pdf-engine . "xelatex")))
  (setq org-pandoc-options-for-latex-pdf '((pdf-engine . "luatex")))
  ;; special extensions for markdown_github output
  (setq org-pandoc-format-extensions '(markdown_github+pipe_tables+raw_html))
)
\end{verbatim}

\subsection*{UTF8 pretty bullets in org mode}
\label{sec:org689397b}
\begin{verbatim}
(use-package org-bullets
  :config
  (add-hook 'org-mode-hook (lambda () (org-bullets-mode 1)))
)
\end{verbatim}

\subsection*{org-jira}
\label{sec:org4714ffe}

Org Mode Integration with Jira Projects

\begin{verbatim}
(use-package org-jira
  :ensure t
  :defer 3
  :after org
  :custom
  (jiralib-url "https://stairscreativestudio.atlassian.net")
)
\end{verbatim}

\subsubsection*{ox-jira exporter}
\label{sec:orge5fd56b}
\begin{verbatim}
(use-package ox-jira
  :defer 3
  :after org
)
\end{verbatim}


\subsection*{ReveaJS org-reveal:}
\label{sec:org6458414}

\begin{verbatim}
This delay makes the options to export to RevealJS appear on the exporter menu (C-c C-e)
\end{verbatim}


\begin{verbatim}
(use-package ox-reveal
  :after ox
  :config
  (setq org-reveal-root "https://cdn.jsdelivr.net/reveal.js/3.0.0/")
)
\end{verbatim}

\subsection*{Org Exporters}
\label{sec:org25ab9be}

\subsubsection*{markdown}
\label{sec:orgf1d6c7e}
\begin{verbatim}
(use-package ox-md
  :defer t
  :after org
)
\end{verbatim}

\subsubsection*{github-flavored markdown}
\label{sec:org35557d4}
\begin{verbatim}
(use-package ox-gfm
  :ensure t
  :defer t
  :after org
)
\end{verbatim}

\section*{\LaTeX{}}
\label{sec:org8abcdcf}

\begin{verbatim}
(use-package auctex-latexmk
  :defer t
  :init
  (add-hook 'LaTeX-mode-hook 'auctex-latexmk-setup)
)
\end{verbatim}


\begin{verbatim}
(use-package company-auctex
  :ensure t
  :defer t
  :init
  (add-hook 'LaTeX-mode-hook 'company-auctex-init)
)
\end{verbatim}

\begin{verbatim}
(use-package tex
  :ensure auctex
  :defer t
  :hook
  (LaTeX-mode . visual-line-mode)
  (LaTeX-mode . flyspell-mode)
  (LaTeX-mode . LaTeX-math-mode)
  :custom
  (TeX-auto-save t)
  (TeX-parse-self t)
  (TeX-master nil)
  ;; to use pdfview with auctex
  (TeX-view-program-selection '((output-pdf "pdf-tools"))
    TeX-source-correlate-start-server t)
  (TeX-view-program-list '(("pdf-tools" "TeX-pdf-tools-sync-view")))
  (TeX-after-compilation-finished-functions #'TeX-revert-document-buffer)
  :config
  (setq TeX-PDF-mode t) ;; compile to PDF by default
  (setq org-export-with-smart-quotes t) ;; convert quotes to LaTeX smartquotes on export


  (if (version< emacs-version "26")
    (add-hook LaTeX-mode-hook #'display-line-numbers-mode))
  (add-hook 'LaTeX-mode-hook
    (lambda ()
        (turn-on-reftex)
        (setq reftex-plug-into-AUCTeX t)
        (reftex-isearch-minor-mode)
        (setq TeX-PDF-mode t)
        (setq TeX-source-correlate-method 'synctex)
        (setq TeX-source-correlate-start-server t)))
)

\end{verbatim}

\subsection*{Add the beamer presentation class template to org}
\label{sec:org68f5673}
\begin{verbatim}
(add-to-list 'org-latex-classes
        '("beamer"
          "\\documentclass\[presentation\]\{beamer\}"
          ("\\section\{%s\}" . "\\section*\{%s\}")
          ("\\subsection\{%s\}" . "\\subsection*\{%s\}")
          ("\\subsubsection\{%s\}" . "\\subsubsection*\{%s\}"))
)
\end{verbatim}


\subsection*{Add the memoir class template to org}
\label{sec:org8d9139e}

The Sections and Heading Levels gets configured as follows:

\begin{center}
\begin{tabular}{lrl}
Division & <c>Level & <c>org-equivalent\\
\book & -2 & *\\
\part & -1 & **\\
\chapter & 0 & \textbf{*}\\
\section & 1 & \textbf{**}\\
\subsection & 2 & \textbf{\textbf{*}}\\
\subsubsection & 3 & \textbf{\textbf{**}}\\
\paragraph & 4 & \textbf{\textbf{\textbf{*}}}\\
\subparagraph & 5 & \textbf{\textbf{\textbf{**}}}\\
\end{tabular}
\end{center}


\begin{verbatim}
;(add-to-list 'org-latex-classes
;        '("memoir"
;          "\\documentclass\[a4paper\]\{memoir\}"
;          ("\\book\{%s\}" . "\\book*\{%s\}")
;          ("\\part\{%s\}" . "\\part*\{%s\}")
;          ("\\chapter\{%s\}" . "\\chapter*\{%s\}")
;          ("\\section\{%s\}" . "\\section*\{%s\}")
;          ("\\subsection\{%s\}" . "\\subsection*\{%s\}")
;          ("\\subsubsection\{%s\}" . "\\subsubsection*\{%s\}"))
;)
\end{verbatim}

\subsection*{Add abntex2 class to org list of latex classes}
\label{sec:orgc6a7ad3}
This class is based on the Memoir class
The Sections and Heading Levels gets configured as follows:

\begin{center}
\begin{tabular}{lrl}
Division & <c>Level & <c>org-equivalent\\
\part & -1 & *\\
\chapter & 0 & **\\
\section & 1 & \textbf{*}\\
\subsection & 2 & \textbf{**}\\
\subsubsection & 3 & \textbf{\textbf{*}}\\
\paragraph & 4 & \textbf{\textbf{**}}\\
\subparagraph & 5 & \textbf{\textbf{\textbf{*}}}\\
\end{tabular}
\end{center}
\begin{verbatim}
;(add-to-list 'org-latex-classes
;             '("abntex2"
;               "\\documentclass{abntex2}"
;               ("\\part{%s}" . "\\part*{%s}")
;               ("\\chapter{%s}" . "\\chapter*{%s}")
;               ("\\section{%s}" . "\\section*{%s}")
;               ("\\subsection{%s}" . "\\subsection*{%s}")
;               ("\\subsubsection{%s}" . "\\subsubsection*{%s}")
;               ("\\subsubsubsection{%s}" . "\\subsubsubsection*{%s}")
;               ("\\paragraph{%s}" . "\\paragraph*{%s}"))
;)
\end{verbatim}


\section*{Magit}
\label{sec:orgdd6f9b6}

\begin{verbatim}
(use-package magit
  :ensure t
  :custom
  (magit-auto-revert-mode nil)
  :bind
  ("M-g s" . magit-status)
  ("C-x g" . magit-status)
)
\end{verbatim}

\subsection*{magit-todo}
\label{sec:orgac4c222}

\begin{verbatim}
(use-package magit-todos
  :ensure t
  :bind
  ("M-g t" . magit-todos-list)
  :config
  (magit-todos-mode)
)
\end{verbatim}

\subsection*{evil-magit}
\label{sec:orgd364aa3}
\begin{verbatim}
(use-package evil-magit
  :ensure t
  :init
;;  (evil-magit-init)
  (setq evil-magit-state 'normal)
  (setq evil-magit-use-y-for-yank nil)
  :config
  (evil-define-key evil-magit-state magit-mode-map "j" 'magit-log-popup)
  (evil-define-key evil-magit-state magit-mode-map "k" 'evil-next-visual-line)
  (evil-define-key evil-magit-state magit-mode-map "l" 'evil-previous-visual-line)
)
\end{verbatim}


\section*{Helm}
\label{sec:orgd58dc3c}

\begin{verbatim}
(use-package helm
  :ensure t
  :bind
  ("M-x" . helm-M-x)
  ("M-x" . helm-M-x)
  ("C-c h" . helm-command-prefix)
  ("C-x b" . helm-buffers-list)
  ("C-x C-b" . helm-mini)
  ("C-x C-f" . helm-find-files)
  ("C-x r b" . helm-bookmarks)
  ("M-y" . helm-show-kill-ring)
  ("M-:" . helm-eval-expression-with-eldoc)
  (:map helm-map
  ("C-z" . helm-select-action)
  ("C-h a" . helm-apropos)
  ("C-c h" . helm-execute-persistent-action)
  ("<tab>" . helm-execute-persistent-action)
  )
  :init
  (setq helm-autoresize-mode t)
  (setq helm-buffer-max-length 40)
  (setq helm-bookmark-show-location t)
  (setq helm-buffer-max-length 40)
  (setq helm-split-window-inside-p t)

  ;; turn on helm fuzzy matching
  (setq helm-M-x-fuzzy-match t)
  (setq helm-mode-fuzzy-match t)

  (setq helm-ff-file-name-history-use-recentf t)
  (setq helm-ff-skip-boring-files t)
  (setq helm-follow-mode-persistent t)
  ;; take between 10-30% of screen space
  (setq helm-autoresize-min-height 10)
  (setq helm-autoresize-max-height 30)
  :config
  (require 'helm-config)
  (helm-mode 1)
  ;; Make helm replace the default Find-File and M-x
  (global-set-key [remap execute-extended-command] #'helm-M-x)
  (global-set-key [remap find-file] #'helm-find-files)
  ;; helm bindings
  (global-unset-key (kbd "C-x c"))
)
\end{verbatim}

\section*{helm-ag}
\label{sec:org9eb8eba}

\begin{verbatim}
(use-package helm-ag
  :ensure helm-ag
  :bind ("M-p" . helm-projectile-ag)
  :commands (helm-ag helm-projectile-ag)
  :init (setq helm-ag-insert-at-point 'symbol
        helm-ag-command-option "--path-to-ignore ~/.agignore"))
\end{verbatim}

\section*{helm-rg}
\label{sec:org2eeb811}

\begin{verbatim}
(use-package helm-rg
  :ensure t
  :defer t
)
\end{verbatim}

\section*{ripgrep}
\label{sec:orge5f7363}

\begin{verbatim}
(use-package rg
  :ensure t
  :defer t
  :ensure-system-package
  (rg . ripgrep)
  :config
  ;; choose between default keybindings or magit like menu interface.
  ;; both options are mutually exclusive
  (rg-enable-default-bindings)
  ;;(rg-enable-menu)

)
\end{verbatim}

\section*{helm-fuzzier}
\label{sec:orgc0abfe5}
\begin{verbatim}
supposed better fuzzy matching for helm
for instance, plp, plpa, paclp, should all match package-list-packages
\end{verbatim}



\begin{verbatim}
(use-package helm-fuzzier
  :disabled nil
  :ensure t
  :after helm
  :config
  (helm-fuzzier-mode 1)
)
\end{verbatim}

\section*{FlyCheck linter}
\label{sec:org4839286}

\begin{verbatim}
(use-package flycheck
    :ensure t
    :defer t
    :hook
    (prog-mode . flycheck-mode)
    :custom
    (flycheck-display-errors-delay 1)
    :config
    (global-flycheck-mode)

    ;; add eslint to list of flycheck checkers
    (setq flycheck-checkers '(javascript-eslint))

    ;; disable jshint since we prefer eslint checking
    (setq-default flycheck-disabled-checkers (append flycheck-disabled-checkers '(javascript-jshint)))

    ;; set modes that will use ESLint
    (flycheck-add-mode 'javascript-eslint 'web-mode)
    (flycheck-add-mode 'javascript-eslint 'js2-mode)
    (flycheck-add-mode 'javascript-eslint 'js-mode)

    ;; customize flycheck temp file prefix
    (setq-default flycheck-temp-prefix ".flycheck")

    ;; disable json-jsonlist checking for json files
    (setq-default flycheck-disabled-checkers (append flycheck-disabled-checkers '(json-jsonlist)))

    ;; Workaround for eslint loading slow
    ;; https://github.com/flycheck/flycheck/issues/1129#issuecomment-319600923
    (advice-add 'flycheck-eslint-config-exists-p :override (lambda() t))
)

\end{verbatim}


\subsection*{Turn flycheck inline extension after flycheck starts}
\label{sec:org2514a3e}

\begin{verbatim}
Quick peek is an extension that embelishes flycheck inline messages
\end{verbatim}


\begin{verbatim}
(use-package quick-peek
  :ensure t
)
\end{verbatim}

\begin{verbatim}
(use-package flycheck-inline
  :ensure t
  :hook
  (flycheck-mode . flycheck-inline-mode)
  :config
  ;; (global-flycheck-inline-mode)
  (setq flycheck-inline-display-function
        (lambda (msg pos)
          (let* ((ov (quick-peek-overlay-ensure-at pos))
                 (contents (quick-peek-overlay-contents ov)))
            (setf (quick-peek-overlay-contents ov)
                  (concat contents (when contents "\n") msg))
            (quick-peek-update ov)))
        flycheck-inline-clear-function #'quick-peek-hide))
\end{verbatim}


\subsection*{flycheck-pos-tip (show flycheck messages in tooltip)}
\label{sec:org7eb3235}

\begin{verbatim}
;;; Show Flycheck errors in tooltip
(use-package flycheck-pos-tip
  :ensure t
  ;;:disabled t
  :after flycheck
  :config (flycheck-pos-tip-mode)
)
\end{verbatim}


\section*{Projectile}
\label{sec:org697466b}
\begin{verbatim}
(use-package projectile
  :ensure t
  :bind
  (:map projectile-mode-map
  ("s-p" . projectile-command-map)
  ("C-c p" . projectile-command-map)
  )
  :config
  (projectile-mode +1)
  (setq projectile-globally-ignored-files
        (append '("~"
                  ".swp"
                  ".pyc")
                projectile-globally-ignored-files))
)
\end{verbatim}

\begin{verbatim}
(use-package helm-projectile
  :ensure t
;  :after projectile
;  :demand t
  :config
  (helm-projectile-on)
)
\end{verbatim}


\section*{Dired}
\label{sec:org0435985}

\begin{verbatim}
(use-package dired-k
  :after dired
  :config
  (setq dired-k-style 'git)
  (setq dired-k-human-readable t)
  (setq dired-dwin-target t)
  (add-hook 'dired-initial-position-hook #'dired-k)
)
\end{verbatim}

\section*{Treemacs (neotree like navigation)}
\label{sec:org2ba9217}


\subsection*{Treemacs itself}
\label{sec:orgac3e147}
\begin{verbatim}
(use-package treemacs
  :ensure t
  :defer t
  :hook
  (after-init . treemacs)
  :config
  (progn
    (setq treemacs-collapse-dirs                 (if treemacs-python-executable 3 0)
          treemacs-deferred-git-apply-delay      0.5
          treemacs-display-in-side-window        t
          treemacs-eldoc-display                 t
          treemacs-file-event-delay              5000
          treemacs-file-follow-delay             0.2
          treemacs-follow-after-init             t
          treemacs-git-command-pipe              ""
          treemacs-goto-tag-strategy             'refetch-index
          treemacs-indentation                   2
          treemacs-indentation-string            " "
          treemacs-is-never-other-window         nil
          treemacs-max-git-entries               5000
          treemacs-missing-project-action        'ask
          treemacs-no-png-images                 nil
          treemacs-no-delete-other-windows       t
          treemacs-project-follow-cleanup        nil
          treemacs-persist-file                  (expand-file-name ".cache/treemacs-persist" user-emacs-directory)
          treemacs-position                      'left
          treemacs-recenter-distance             0.1
          treemacs-recenter-after-file-follow    nil
          treemacs-recenter-after-tag-follow     nil
          treemacs-recenter-after-project-jump   'always
          treemacs-recenter-after-project-expand 'on-distance
          treemacs-show-cursor                   nil
          treemacs-show-hidden-files             t
          treemacs-silent-filewatch              nil
          treemacs-silent-refresh                nil
          treemacs-sorting                       'alphabetic-desc
          treemacs-space-between-root-nodes      t
          treemacs-tag-follow-cleanup            t
          treemacs-tag-follow-delay              1.5
          treemacs-width                         35)

    ;; The default width and height of the icons is 22 pixels. If you are
    ;; using a Hi-DPI display, uncomment this to double the icon size.
    ;;(treemacs-resize-icons 44)

    (treemacs-follow-mode t)
    (treemacs-filewatch-mode t)
    (treemacs-fringe-indicator-mode t)
    (pcase (cons (not (null (executable-find "git")))
                 (not (null treemacs-python-executable)))
      (`(t . t)
       (treemacs-git-mode 'deferred))
      (`(t . _)
       (treemacs-git-mode 'simple))))
  :bind
  (:map global-map
        ("<f8>"       . treemacs)
        ("M-0"       . treemacs-select-window)
        ("C-x t 1"   . treemacs-delete-other-windows)
        ("C-x t t"   . treemacs)
        ("C-x t B"   . treemacs-bookmark)
        ("C-x t C-t" . treemacs-find-file)
        ("C-x t M-t" . treemacs-find-tag)))
\end{verbatim}

\subsection*{Treemacs Evil}
\label{sec:orgd0de917}
\begin{verbatim}
(use-package treemacs-evil
  :after treemacs evil
  :ensure t)
\end{verbatim}

\subsection*{Treemacs Projectile}
\label{sec:org1a9efa2}
\begin{verbatim}
(use-package treemacs-projectile
  :after treemacs projectile
  :ensure t)
\end{verbatim}

\subsection*{Treemacs Dired}
\label{sec:org35c25fc}
\begin{verbatim}
(use-package treemacs-icons-dired
  :after treemacs dired
  :ensure t
  :config (treemacs-icons-dired-mode))
\end{verbatim}

\subsection*{Treemacs Magit}
\label{sec:org128633e}
\begin{verbatim}
(use-package treemacs-magit
  :after treemacs magit
  :ensure t)
\end{verbatim}

\section*{Neotree}
\label{sec:orgeebc7e1}

\begin{verbatim}
(use-package neotree
  :bind
  ("<f7>" . neotree-toggle)
  :config
  (progn
    (setq neo-smart-open t)
    (setq neo-window-fixed-size nil)
    (evil-leader/set-key
      "tt" 'neotree-toggle
      "tp" 'neotree-projectile-action))

  ;; neotree 'icons' theme, which supports filetype icons
  (setq neo-theme (if (display-graphic-p) 'icons))
  (setq neo-theme 'icons)
  (setq neo-window-width 32)

  ;; Neotree bindings
  (add-hook 'neotree-mode-hook
            (lambda ()
              ; default Neotree bindings
              (define-key evil-normal-state-local-map (kbd "<tab>") 'neotree-enter)
              (define-key evil-normal-state-local-map (kbd "SPC") 'neotree-quick-look)
              (define-key evil-normal-state-local-map (kbd "q") 'neotree-hide)
              (define-key evil-normal-state-local-map (kbd "RET") 'neotree-enter)
              (define-key evil-normal-state-local-map (kbd "g") 'neotree-refresh)
              (define-key evil-normal-state-local-map (kbd "n") 'neotree-next-line)
              (define-key evil-normal-state-local-map (kbd "p") 'neotree-previous-line)
              (define-key evil-normal-state-local-map (kbd "A") 'neotree-stretch-toggle)
              (define-key evil-normal-state-local-map (kbd "H") 'neotree-hidden-file-toggle)
              (define-key evil-normal-state-local-map (kbd "|") 'neotree-enter-vertical-split)
              (define-key evil-normal-state-local-map (kbd "-") 'neotree-enter-horizontal-split)
              ; simulating NERDTree bindings in Neotree
              (define-key evil-normal-state-local-map (kbd "R") 'neotree-refresh)
              (define-key evil-normal-state-local-map (kbd "r") 'neotree-refresh)
              (define-key evil-normal-state-local-map (kbd "u") 'neotree-refresh)
              (define-key evil-normal-state-local-map (kbd "C") 'neotree-change-root)
              (define-key evil-normal-state-local-map (kbd "c") 'neotree-create-node))))
\end{verbatim}

\section*{dired-sidebar}
\label{sec:orgd67524f}

\begin{verbatim}
(require 'neotree)
\end{verbatim}

\subsection*{toggle neotree with F8}
\label{sec:org8b77c57}
\begin{verbatim}
(global-set-key [f6] 'dired-sidebar-toggle-sidebar)
\end{verbatim}

\subsection*{other settings for dired-sidebar}
\label{sec:orgd64659e}
\begin{verbatim}
(setq dired-sidebar-subtree-line-prefix "__")
(setq dired-sidebar-theme 'vscode)
(setq dired-sidebar-use-term-integration t)
(setq dired-sidebar-use-custom-font t)
\end{verbatim}

\section*{ranger}
\label{sec:orga1603e3}

\begin{verbatim}
(use-package ranger
  :ensure t
  :bind
  ("C-x C-j" . ranger)
  :config
  (setq ranger-show-hidden t) ;; show hidden files
)
\end{verbatim}

\section*{ace jump mode}
\label{sec:orge3a3215}
\begin{verbatim}
(use-package ace-jump-mode
  :ensure t
  :bind
  ("C-." . ace-jump-mode)
)
\end{verbatim}

\section*{PDF Tools}
\label{sec:orgd25fe53}

\subsection*{Install pdf-tools if its not already installed}
\label{sec:org77afd08}
\begin{verbatim}
;; (pdf-tools-install)
;; the docs say if i care about startup time, i should use pdf-loader-install instead of pdf-tools-install, but doenst say why
;; (pdf-loader-install)
\end{verbatim}

\subsection*{Make buffer refresh every 1 second to PDF-tools updates the changed pdf}
\label{sec:orga17866d}
\begin{verbatim}
(add-hook 'TeX-after-compilation-finished-functions #'TeX-revert-document-buffer)
;; (add-hook 'pdf-view-mode-hook 'auto-revert-mode)
;; (add-hook 'doc-view-mode-hook 'auto-revert-mode)
\end{verbatim}

\subsection*{PDF tools evil keybindings}
\label{sec:orgd98df24}
\begin{verbatim}
(evil-define-key 'normal pdf-view-mode-map
  "h" 'pdf-view-previous-page-command
  "j" (lambda () (interactive) (pdf-view-next-line-or-next-page 5))
  "k" (lambda () (interactive) (pdf-view-previous-line-or-previous-page 5))
  "l" 'pdf-view-next-page-command)
\end{verbatim}



\section*{Appearance}
\label{sec:orgdf0cb88}

\subsection*{cleaning the default UI}
\label{sec:orgdb747fe}

\begin{verbatim}
(setq inhibit-splash-screen t)

(blink-cursor-mode t)
(setq blink-cursor-blinks 0) ;; blink forever
(setq-default indicate-empty-lines t)
(setq-default line-spacing 3)
(setq frame-title-format '("Emacs"))
\end{verbatim}

\subsubsection*{Remove scroll bars from frames}
\label{sec:org62326bb}
\begin{verbatim}
(scroll-bar-mode -1)
\end{verbatim}

\subsubsection*{Remove menu bar and tool bar}
\label{sec:org0b6cc65}
\begin{verbatim}
(tool-bar-mode -1)
(menu-bar-mode -1)
\end{verbatim}

\subsection*{Applying my theme}
\label{sec:orge409b55}

\begin{verbatim}

(add-to-list 'custom-theme-load-path "~/dotfiles/emacs.d/themes/")
; theme options:
; atom-one-dark (doenst work well with emacsclient, ugly blue bg)
; dracula
; darktooth
; gruvbox-dark-hard
; gruvbox-dark-light
; gruvbox-dark-medium
; base16-default-dark-theme -- this one is good

(setq my-theme 'darkplus)

\end{verbatim}

Load the theme

\begin{verbatim}
(load-theme my-theme t)
\end{verbatim}


\begin{verbatim}

;; (defun load-my-theme (frame)
;;   "Function to load the theme in current FRAME.
;;   sed in conjunction
;;   with bellow snippet to load theme after the frame is loaded
;;   to avoid terminal breaking theme."
;;   (select-frame frame)
;;   (load-theme my-theme t))

;; ; make emacs load the theme after loading the frame
;; ; resolves issue with the theme not loading properly in terminal mode on emacsclient

;; ;; this if was breaking my emacs!!!!!
;;  (add-hook 'after-make-frame-functions #'load-my-theme)
\end{verbatim}


\subsection*{Highlight lines}
\label{sec:orgbc45f3d}

\begin{verbatim}
(use-package hl-line
  :defer nil
  :config
  (global-hl-line-mode)
)
\end{verbatim}

\subsection*{Highlight columns}
\label{sec:org48f43be}

\begin{verbatim}
(use-package col-highlight
  :disabled
  :defer nil
  :config
  (col-highlight-toggle-when-idle)
  (col-highlight-set-interval 2)
)
\end{verbatim}

\subsection*{Highlight columns}
\label{sec:org7d91b35}

Highlight crosshair (combination of hl-lines and hl-columns

\begin{verbatim}
(use-package crosshairs
  :disabled
  :defer nil
  :config
  (crosshairs-mode)
)
\end{verbatim}

\subsection*{Uniquify (unique files names in buffers)}
\label{sec:orgcf8c84e}

This package is included in emacs, so `:ensure nil` prevents use-package from trying to download it on Melpa

\begin{verbatim}
(use-package uniquify
  :defer 1
  :ensure nil
  :custom
  (uniquify-after-kill-buffer-p t)
  (uniquify-buffer-name-style 'post-forward)
  (uniquify-strip-common-suffix t)
)
\end{verbatim}

\subsection*{All The Icons - Icon package}
\label{sec:org7a0087f}

\begin{verbatim}
(use-package all-the-icons
  :defer 3
)
\end{verbatim}

\subsection*{doom-modeline}
\label{sec:org68ac74d}

Require and enable the doom-modeline
\begin{verbatim}
(require 'doom-modeline)
(doom-modeline-mode 1)
\end{verbatim}

Don’t compact font caches during GC (garbage collection).
\begin{verbatim}
;; (setq inhibit-compacting-font-caches t)
\end{verbatim}

Customize the doom-modeline (convert the comments to org later)

\begin{verbatim}
;; How tall the mode-line should be. It's only respected in GUI.
;; If the actual char height is larger, it respects the actual height.
(setq doom-modeline-height 23)

;; How wide the mode-line bar should be. It's only respected in GUI.
(setq doom-modeline-bar-width 3)

;; Determines the style used by `doom-modeline-buffer-file-name'.
;;
;; Given ~/Projects/FOSS/emacs/lisp/comint.el
;;   truncate-upto-project = ~/P/F/emacs/lisp/comint.el
;;   truncate-from-project = ~/Projects/FOSS/emacs/l/comint.el
;;   truncate-with-project = emacs/l/comint.el
;;   truncate-except-project = ~/P/F/emacs/l/comint.el
;;   truncate-upto-root = ~/P/F/e/lisp/comint.el
;;   truncate-all = ~/P/F/e/l/comint.el
;;   relative-from-project = emacs/lisp/comint.el
;;   relative-to-project = lisp/comint.el
;;   file-name = comint.el
;;   buffer-name = comint.el<2> (uniquify buffer name)
;;
;; If you are expereicing the laggy issue, especially while editing remote files
;; with tramp, please try `file-name' style.
;; Please refer to https://github.com/bbatsov/projectile/issues/657.
(setq doom-modeline-buffer-file-name-style 'truncate-upto-project)

;; Whether display icons in mode-line or not.
(setq doom-modeline-icon t)

;; Whether display the icon for major mode. It respects `doom-modeline-icon'.
(setq doom-modeline-major-mode-icon t)

;; Whether display color icons for `major-mode'. It respects
;; `doom-modeline-icon' and `all-the-icons-color-icons'.
(setq doom-modeline-major-mode-color-icon t)

;; Whether display icons for buffer states. It respects `doom-modeline-icon'.
(setq doom-modeline-buffer-state-icon t)

;; Whether display buffer modification icon. It respects `doom-modeline-icon'
;; and `doom-modeline-buffer-state-icon'.
(setq doom-modeline-buffer-modification-icon t)

;; Whether display minor modes in mode-line or not.
(setq doom-modeline-minor-modes nil)

;; If non-nil, a word count will be added to the selection-info modeline segment.
(setq doom-modeline-enable-word-count nil)

;; Whether display buffer encoding.
(setq doom-modeline-buffer-encoding t)

;; Whether display indentation information.
(setq doom-modeline-indent-info nil)

;; If non-nil, only display one number for checker information if applicable.
(setq doom-modeline-checker-simple-format t)

;; The maximum displayed length of the branch name of version control.
(setq doom-modeline-vcs-max-length 12)

;; Whether display perspective name or not. Non-nil to display in mode-line.
(setq doom-modeline-persp-name t)

;; Whether display icon for persp name. Nil to display a # sign. It respects `doom-modeline-icon'
(setq doom-modeline-persp-name-icon nil)

;; Whether display `lsp' state or not. Non-nil to display in mode-line.
(setq doom-modeline-lsp t)

;; Whether display github notifications or not. Requires `ghub` package.
(setq doom-modeline-github nil)

;; The interval of checking github.
(setq doom-modeline-github-interval (* 30 60))

;; Whether display environment version or not
(setq doom-modeline-env-version t)
;; Or for individual languages
;; (setq doom-modeline-env-enable-python t)
;; (setq doom-modeline-env-enable-ruby t)
;; (setq doom-modeline-env-enable-perl t)
;; (setq doom-modeline-env-enable-go t)
;; (setq doom-modeline-env-enable-elixir t)
;; (setq doom-modeline-env-enable-rust t)

;; Change the executables to use for the language version string
(setq doom-modeline-env-python-executable "python")
(setq doom-modeline-env-ruby-executable "ruby")
(setq doom-modeline-env-perl-executable "perl")
(setq doom-modeline-env-go-executable "go")
(setq doom-modeline-env-elixir-executable "iex")
(setq doom-modeline-env-rust-executable "rustc")

;; Whether display mu4e notifications or not. Requires `mu4e-alert' package.
(setq doom-modeline-mu4e t)

;; Whether display irc notifications or not. Requires `circe' package.
(setq doom-modeline-irc t)

;; Function to stylize the irc buffer names.
(setq doom-modeline-irc-stylize 'identity)
\end{verbatim}


this was commented with C-c ; so it doenst get exported in favor of doom-modeline
\subsection*{parrot-mode}
\label{sec:org43a924b}

Type of parrots available:

\begin{itemize}
\item default
\item confused
\item emacs
\item nyan
\item rotating
\item science
\item thumbsup
\end{itemize}

\begin{verbatim}
(use-package parrot
  :config
  ;; To see the party parrot in the modeline, turn on parrot mode:
  (parrot-mode)
  (parrot-set-parrot-type 'default)
  ;; Rotate the parrot when clicking on it (this can also be used to execute any function when clicking the parrot, like 'flyspell-buffer)
  (add-hook 'parrot-click-hook #'parrot-start-animation)
  ;; Rotate parrot when buffer is saved
  (add-hook 'after-save-hook #'parrot-start-animation)
  ;;/Rotation function keybindings for evil users
  (define-key evil-normal-state-map (kbd "[r") 'parrot-rotate-prev-word-at-point)
  (define-key evil-normal-state-map (kbd "]r") 'parrot-rotate-next-word-at-point)
  (add-hook 'mu4e-index-updated-hook #'parrot-start-animation)
)
\end{verbatim}

\subsection*{nyan-mode}
\label{sec:org67f02dd}

\begin{verbatim}
(use-package nyan-mode
   :if window-system
   :hook
   (after-init . nyan-mode)
   :config
   (setq nyan-cat-face-number 4)
   (setq nyan-animate-nyancat t)
   (setq nyan-wavy-trail t)
   (nyan-start-animation))
\end{verbatim}

\subsection*{solaire-mode}
\label{sec:org212fc60}

solaire-mode is an aesthetic plugin that helps visually distinguish
file-visiting windows from other types of windows (like popups or sidebars)
by giving them a slightly different -- often brighter -- background.

\begin{verbatim}
;; (use-package solaire-mode
;;   :config
;;   (solaire-mode)
;;   :hook
;;   (after-init . solaire-global-mode +1)
;;   ;; To enable solaire-mode unconditionally for certain modes:
;;   (ediff-prepare-buffer . solaire-mode)
;;   ;; if you use auto-revert-mode, this prevents solaire-mode from turning itself off every time Emacs reverts the file
;;   (after-revert- . turn-on-solaire-mode)
;;   ;; highlight the minibuffer when it is activated:
;;   (minibuffer-setup . solaire-mode-in-minibuffer)
;;   (after-change-major-mode . turn-on-solaire-mode)
;;   :config
;;   ;; if the bright and dark background colors are the wrong way around, use this
;;   ;; to switch the backgrounds of the `default` and `solaire-default-face` faces.
;;   ;; This should be used *after* you load the active theme!
;;   ;;  NOTE: This is necessary for themes in the doom-themes package!
;;   (solaire-mode-swap-bg))
\end{verbatim}

\subsection*{sublimity}
\label{sec:org39bafe7}

\begin{verbatim}
(use-package sublimity
  :ensure t
  :config
  (setq sublimity-scroll-weight 10
      sublimity-scroll-drift-length 5)
  ;; this is the only part of the config where i use `use-package' inside another package config.
  ;; the oficial docs appears to suggest this way
  (sublimity-mode 1)
)
\end{verbatim}

\subsubsection*{sublimity-scroll}
\label{sec:orgc2162f7}
\begin{verbatim}
(use-package sublimity-scroll
  :ensure nil
  :config
  (setq sublimity-scroll-weight 10)
  (setq sublimity-scroll-drift-length 5)
)
\end{verbatim}

\subsubsection*{sublimity-map (experimental)}
\label{sec:orgf9bff82}
\begin{verbatim}
(use-package sublimity-map
  :disabled t
  :ensure nil
  :config
  (setq sublimity-map-size 18)  ;; minimap width
  (setq sublimity-map-fraction 0.3)
  (setq sublimity-map-text-scale -7)
  (sublimity-map-set-delay 4) ;; minimap is displayed after 5 seconds of idle time

  ;; document this snippet better, not sure what it does, but it defines the font-family
  (add-hook 'sublimity-map-setup-hook
          (lambda ()
            (setq buffer-face-mode-face '(:family "Monospace"))
            (buffer-face-mode)))

)
\end{verbatim}

\subsubsection*{sublimity-attractive}
\label{sec:org3a623a2}
\begin{verbatim}
(use-package sublimity-attractive
  :disabled nil
  :ensure nil
  :config
  (setq sublimity-attractive-centering-width 110)

  ;; these are functions (NOT variables) to configure some UI parts
  ;; (sublimity-attractive-hide-bars)
  ;; (sublimity-attractive-hide-vertical-border)
  ;; (sublimity-attractive-hide-fringes)
  ;; (sublimity-attractive-hide-modelines)
)

\end{verbatim}

\subsection*{fancy-battery-mode}
\label{sec:org428734d}
\begin{verbatim}
(use-package fancy-battery
  :ensure t
  :config
  (add-hook 'after-init-hook #'fancy-battery-mode)
)
\end{verbatim}
\subsection*{beacon - flash light where cursor is}
\label{sec:org50ebe67}
\begin{verbatim}
(use-package beacon
  :ensure t
  :config
  (setq beacon-blink-when-window-scrolls nil)
  (setq beacon-dont-blink-major-modes '(t pdf-view-mode))
  (setq beacon-size 10)
  (beacon-mode 1)
)
\end{verbatim}




\section*{centaur-tabs}
\label{sec:org74f5fcd}

\begin{verbatim}
(use-package centaur-tabs
   ;; :load-path "~/.emacs.d/other/centaur-tabs"
   :hook
   (dashboard-mode . centaur-tabs-local-mode)
   (term-mode . centaur-tabs-local-mode)
   (calendar-mode . centaur-tabs-local-mode)
   (org-agenda-mode . centaur-tabs-local-mode)
   (helpful-mode . centaur-tabs-local-mode)
   :bind
   ("C-<prior>" . centaur-tabs-backward)
   ("C-<next>" . centaur-tabs-forward)
   ("C-c t s" . centaur-tabs-counsel-switch-group)
   ("C-c t p" . centaur-tabs-group-by-projectile-project)
   ("C-c t g" . centaur-tabs-group-buffer-groups)
   (:map evil-normal-state-map
   ("g t" . centaur-tabs-forward)
   ("g T" . centaur-tabs-backward))
   :config
   (centaur-tabs-mode t)
   (setq centaur-tabs-style "bar") ; types available: (alternative, bar, box, chamfer, rounded, slang, wave, zigzag)
   (setq centaur-tabs-height 32)
   (setq centaur-tabs-set-icons t)
   (setq centaur-tabs-set-modified-marker t) ;; display a marker indicating that a buffer has been modified (atom-style)
   (setq centaur-tabs-modified-marker "*")
   (setq centaur-tabs-set-bar 'left) ;; in previous config value was 'over
   (setq centaur-tabs-gray-out-icons 'buffer)
   (setq centaur-tabs-headline-match)
   ;; (centaur-tabs-enable-buffer-reordering)
   ;; (setq centaur-tabs-adjust-buffer-order t)
   (setq uniquify-separator "/")
   (setq uniquify-buffer-name-style 'forward)
)
\end{verbatim}

\section*{Minor modes}
\label{sec:orgee4e1bb}

\subsection*{which-key}
\label{sec:orgeee1625}

\begin{verbatim}
(use-package which-key
  :hook (after-init . which-key-mode))
  :config
  (setq which-key-idle-delay 0.2)
  (setq which-key-min-display-lines 3)
  (setq which-key-max-description-length 20)
  (setq which-key-max-display-columns 6)
\end{verbatim}

\subsection*{diff-hl (highlights uncommited diffs in bar aside from the line numbers)}
\label{sec:orgbd30b0c}

\begin{verbatim}
(use-package diff-hl
  :ensure t
  :hook
  (prog-mode . diff-hl-mode)
  (org-mode . diff-hl-mode)
  (dired-mode . diff-hl-mode)
  (magit-post-refresh . diff-hl-mode)
  :init
  ;; (add-hook 'prog-mode-hook #'diff-hl-mode)
  ;; (add-hook 'org-mode-hook #'diff-hl-mode)
  ;; (add-hook 'dired-mode-hook 'diff-hl-dired-mode)
  ;; (add-hook 'magit-post-refresh-hook 'diff-hl-magit-post-refresh)

  ;; Better looking colours for diff indicators
  (custom-set-faces
    '(diff-hl-change ((t (:background "#3a81c3"))))
    '(diff-hl-insert ((t (:background "#7ccd7c"))))
    '(diff-hl-delete ((t (:background "#ee6363"))))
  )

  :config
  (setq diff-hl-fringe-bmp-function 'diff-hl-fringe-bmp-from-type)
  (setq diff-hl-side 'left)
  (setq diff-hl-margin-side 'left)

  (diff-hl-margin-mode 1) ;; show the indicators in the margin
  (diff-hl-flydiff-mode 1) ;;  ;; On-the-fly diff updates
  (global-diff-hl-mode 1) ;; Enable diff-hl globally
)
\end{verbatim}

\subsection*{smartparens}
\label{sec:org792d452}
\begin{verbatim}
(use-package smartparens
  :ensure t
  :hook
  (after-init . smartparens-global-mode)
  :config
  (require 'smartparens-config)
  (sp-pair "=" "=" :actions '(wrap))
  (sp-pair "+" "+" :actions '(wrap))
  (sp-pair "<" ">" :actions '(wrap))
  (sp-pair "$" "$" :actions '(wrap)))
\end{verbatim}

\subsubsection*{evil-smartparens helps avoid conflicts between evil and smartparens}
\label{sec:org0a87c98}

\begin{verbatim}
(use-package evil-smartparens
  :ensure t
  :hook
  (smartparens-enabled . evil-smartparens-mode)
)
\end{verbatim}

\subsection*{Rainbow Delimiters}
\label{sec:orge8929e7}

This highlights matching parentheses acording to their depth
Helps editing lisp code

\begin{verbatim}
(use-package rainbow-delimiters
  :ensure t
  :hook
  (emacs-lisp-mode . rainbow-delimiters-mode)
  (prog-mode . rainbow-delimiters-mode)
)
\end{verbatim}

\subsection*{rainbow mode}
\label{sec:org2f64c5f}

\begin{verbatim}
Colorize hex, rgb and named color codes
\end{verbatim}


\begin{verbatim}
(use-package rainbow-mode
  :ensure t
  :hook
  (org-mode . rainbow-mode)
  (css-mode . rainbow-mode)
  (scss-mode . rainbow-mode)
  (php-mode . rainbow-mode)
  (html-mode . rainbow-mode)
  (web-mode . rainbow-mode)
  (js2-mode . rainbow-mode))
\end{verbatim}

\subsection*{Smartscan mode}
\label{sec:org68855c8}
\begin{verbatim}
Usage:
M-n and M-p move between symbols
M-' to replace all symbols in the buffer matching the one under point
C-u M-' to replace symbols in your current defun only (as used by narrow-to-defun.)
\end{verbatim}


\begin{verbatim}
(smartscan-mode 1)
\end{verbatim}

\subsection*{editorconfig}
\label{sec:orgf19fda1}

\begin{verbatim}
(use-package editorconfig
  :ensure t
  :config
  (editorconfig-mode 1))
\end{verbatim}

\subsection*{prettify symbols}
\label{sec:org646ad31}

\begin{verbatim}
this is built-in with emacs >= v24
\end{verbatim}

\begin{verbatim}
(global-prettify-symbols-mode 1)
(defun add-pretty-lambda ()
  "Make some word or string show as pretty Unicode symbols. See https://unicodelookup.com for more."
  (setq prettify-symbols-alist
        '(
          ("lambda" . 955)
          ("delta" . 120517)
          ("epsilon" . 120518)
          ("->" . 8594)
          ("<=" . 8804)
          (">=" . 8805)
          )))
(add-hook 'prog-mode-hook 'add-pretty-lambda)
(add-hook 'org-mode-hook 'add-pretty-lambda)
\end{verbatim}

\section*{quickrun (compile and execute code)}
\label{sec:org475240c}
\begin{verbatim}
(use-package quickrun
  :bind
  ("C-<f5>" . quickrun)
  ("M-<f5>" . quickrun-shell)
)
\end{verbatim}


\section*{Ruby}
\label{sec:orgee40c79}
\begin{verbatim}
(use-package ruby-mode
  :mode "\\.rb\\'"
  :interpreter "ruby"
  :ensure-system-package
  ((rubocop     . "gem install rubocop")
   (ruby-lint   . "gem install ruby-lint")
   (ripper-tags . "gem install ripper-tags")
   (pry         . "gem install pry"))
  :functions inf-ruby-keys
  :config
  (defun my-ruby-mode-hook ()
    (require 'inf-ruby)
    (inf-ruby-keys))
  (add-hook 'ruby-mode-hook 'my-ruby-mode-hook)
)
\end{verbatim}

\section*{Go}
\label{sec:org060550b}

\begin{verbatim}
(use-package go-mode
  :mode "\\.go\\'"
  :hook (before-save . gofmt-before-save)
)
\end{verbatim}


\section*{Emmet}
\label{sec:org204d6c3}

\begin{verbatim}
(use-package emmet-mode
  :ensure t
  :commands emmet-mode
  :init
    (setq emmet-indentation 2)
    (setq emmet-move-cursor-between-quotes t)
  :hook
    (sgml-mode . emmet-mode) ;; Auto-start on any markup modes
    (css-mode . emmet-mode) ;; enable Emmet's css abbreviation.
    (scss-mode . emmet-mode) ;; enable Emmet's css abbreviation.
    (html-mode . emmet-mode) ;; Auto-start on HTML files
    (web-mode . emmet-mode) ;; Auto-start on web-mode
  :config
  (unbind-key "<C-return>" emmet-mode-keymap)
  (unbind-key "C-M-<left>" emmet-mode-keymap)
  (unbind-key "C-M-<right>" emmet-mode-keymap)
  (setq emmet-expand-jsx-className? t)) ;; use emmet with JSX markup
\end{verbatim}

\section*{HTML}
\label{sec:org7090190}

\subsection*{Set HTML indentation to 4 spaces by default (only on html-mode)}
\label{sec:orge632ee1}
\begin{verbatim}
(add-hook 'html-mode-hook
  (lambda ()
    (set (make-local-variable 'sgml-basic-offset) 4)))
\end{verbatim}

\section*{YAML}
\label{sec:org6c32408}
\begin{verbatim}
(use-package yaml-mode
  :ensure t
)
\end{verbatim}

\section*{TOML}
\label{sec:orgf3997ae}
\begin{verbatim}
(use-package toml-mode
  :ensure t
)
\end{verbatim}

\section*{CSS and SCSS}
\label{sec:org00ef89c}

\subsection*{CSS}
\label{sec:org4d775e2}

\begin{verbatim}
(use-package css-mode
  :ensure t
  :mode "\\.css\\'"
  :init
  (setq css-indent-offset 2)
)
\end{verbatim}

\subsection*{SCSS}
\label{sec:org71f54ac}
\begin{verbatim}
(use-package scss-mode
  :ensure t
  ;; this mode doenst load using :mode from use-package, dunno why
  ;; :mode "\\.scss\\'"
  :init
  (setq scss-compile-at-save 'nil)
  :config
  (autoload 'scss-mode "scss-mode")
  (add-to-list 'auto-mode-alist '("\\.scss\\'" . scss-mode))
)
\end{verbatim}

\begin{verbatim}
(use-package helm-css-scss
  :ensure t
  :bind
  (:map isearch-mode-map
  ("s-i" . helm-css-scss-from-isearch)
  :map helm-css-scss-map
  ("s-i" . helm-css-scss-multi-from-helm-css-scss)
  :map css-mode-map
  ("s-i" . helm-css-scss)
  ("s-I" . helm-css-scss-back-to-last-point)
  :map scss-mode-map
  ("s-i" . helm-css-scss)
  ("s-I" . helm-css-scss-back-to-last-point)
  )
  :config
  (setq helm-css-scss-insert-close-comment-depth 2
        helm-css-scss-split-with-multiple-windows t
        helm-css-scss-split-direction 'split-window-vertically)
)
\end{verbatim}


\section*{Web-Mode}
\label{sec:orgb0c42bb}
\begin{verbatim}
(use-package web-mode
  :custom-face
  (css-selector ((t (:inherit default :foreground "#66CCFF"))))
  (font-lock-comment-face ((t (:foreground "#828282"))))
  :mode
  ("\\.phtml\\'" "\\.tpl\\.php\\'" "\\.[agj]sp\\'" "\\.as[cp]x\\'"
  "\\.erb\\'" "\\.mustache\\'" "\\.djhtml\\'" "\\.[t]?html?\\'")
  :config
  ;; web-mode indentation
  (setq web-mode-markup-indent-offset 4)
  (setq web-mode-css-indent-offset 2)
  (setq web-mode-code-indent-offset 2)
)
\end{verbatim}

\section*{JavaScript JS2-mode}
\label{sec:org23f2167}

\begin{verbatim}
;; js2-mode: enhanced JavaScript editing mode
;; https://github.com/mooz/js2-mode
(use-package js2-mode
  :mode
  ("\\.js$" . js2-mode)

  :hook
  (js2-mode . flycheck-mode)
  (js2-mode . company-mode)
  (js2-mode . tide-mode)
  (js2-mode . add-node-modules-path)
  :config
  ;; have 2 space indentation by default
  (setq js-indent-level 2)
  (setq js2-basic-offset 2)
  (setq js-chain-indent t)

  ;; use eslint_d insetad of eslint for faster linting
  ;; (setq flycheck-javascript-eslint-executable "eslint_d")

  ;; Try to highlight most ECMA built-ins
  (setq js2-highlight-level 3)

  ;; turn off all warnings in js2-mode
  (setq js2-mode-show-parse-errors t)
  (setq js2-mode-show-strict-warnings nil)
  (setq js2-strict-missing-semi-warning nil)
)
\end{verbatim}

\section*{PrettierJS}
\label{sec:orgb13ed77}

\begin{verbatim}
;; prettier-emacs: minor-mode to prettify javascript files on save
;; https://github.com/prettier/prettier-emacs
(use-package prettier-js
  :mode
  ("\\.js$" . prettier-js-mode)
  ("\\.scss$" . prettier-js-mode)
  :ensure-system-package
  (prettier . "npm install -g prettier")
  :hook
  (js2-mode . prettier-js-mode)
  (web-mode . prettier-js-mode)
  (rjsx-mode . prettier-js-mode)
  (css-mode . prettier-js-mode)
  (scss-mode . prettier-js-mode)
  (json-mode . prettier-js-mode)
  :config
  (setq prettier-js-args '("--bracket-spacing" "false"
                           "--print-width" "80"
                           "--tab-width" "2"
                           "--use-tabs" "false"
                           "--no-semi" "true"
                           "--single-quote" "true"
                           "--trailing-comma" "none"
                           "--no-bracket-spacing" "true"
                           "--jsx-bracket-same-line" "false"
                           "--arrow-parens" "avoid"))
)
\end{verbatim}


\section*{JSON Mode}
\label{sec:orgd8b6ed6}

\begin{verbatim}
;; json-mode: Major mode for editing JSON files with emacs
;; https://github.com/joshwnj/json-mode
(use-package json-mode
  :mode "\\.js\\(?:on\\|[hl]int\\(rc\\)?\\)\\'"
  :config
  (add-hook 'json-mode-hook #'prettier-js-mode)
  (setq json-reformat:indent-width 2)
  (setq json-reformat:pretty-string? t)
  (setq js-indent-level 2))
\end{verbatim}

\section*{ESLint}
\label{sec:orga0aa868}

eslintd-fix: Emacs minor-mode to automatically fix javascript with eslint\textsubscript{d}.
\url{https://github.com/aaronjensen/eslintd-fix/tree/master}

\begin{verbatim}
(use-package eslintd-fix
  :ensure t
  :ensure-system-package
  (eslint . "npm i -g eslint")
)
\end{verbatim}

\section*{RJSX Mode}
\label{sec:org4f019e6}

\url{https://github.com/felipeochoa/rjsx-mode}

\begin{verbatim}
(use-package rjsx-mode
    :after js2-mode
    :mode
    ("\\.jsx$" . rjsx-mode)
    ("components/.+\\.js$" . rjsx-mode)

    :config
    ;; for better jsx syntax-highlighting in web-mode
    ;; - courtesy of Patrick @halbtuerke
    (defadvice web-mode-highlight-part (around tweak-jsx activate)
      (if (equal web-mode-content-type "jsx")
        (let ((web-mode-enable-part-face nil))
          ad-do-it)
        ad-do-it))
)
\end{verbatim}


\section*{Typescript}
\label{sec:orgfe05516}

\subsection*{Typescript mode}
\label{sec:org0df4302}

\begin{verbatim}
(use-package typescript-mode
  :ensure t
  :mode (("\\.ts\\'" . typescript-mode)
         ("\\.tsx\\'" . typescript-mode))
)
\end{verbatim}

\begin{verbatim}
(defun setup-tide-mode ()
  (interactive)
  (tide-setup)
  (flycheck-mode +1)
  (setq flycheck-check-syntax-automatically '(save mode-enabled))
  (eldoc-mode +1)
  (tide-hl-identifier-mode +1)
  (company-mode +1)
)
\end{verbatim}

\subsection*{Tide}
\label{sec:org9c11231}
\begin{verbatim}
(use-package tide
  :ensure t
  :config
  (progn
    (add-hook 'before-save-hook 'tide-format-before-save)
    (add-hook 'typescript-mode-hook #'setup-tide-mode)
    (add-hook 'js2-mode-hook #'setup-tide-mode)
  )
)

\end{verbatim}


\section*{Angular}
\label{sec:orgf1cdad7}
\begin{verbatim}
(use-package ng2-mode
  :defer
  :hook
  (ng2-mode . prettier-js-mode)
)
\end{verbatim}
\section*{Devops tools setup and helpers (dockerfile, netlify)}
\label{sec:org1e5cbb4}

\begin{verbatim}
(require 'dockerfile-mode)
(add-to-list 'auto-mode-alist '("Dockerfile\\'" . dockerfile-mode))
\end{verbatim}


\section*{LSP}
\label{sec:org960345c}

\subsection*{LSP (language server protocol implementation for emacs)}
\label{sec:org5cfed66}

\begin{verbatim}
(use-package lsp-mode
  :ensure t
  :commands lsp
  :init
  (setq lsp-inhibit-message nil) ;; was `t`, changed to nil to see what it does
  (setq lsp-eldoc-render-all t)  ;; was `nil`, changed to nil to see what it does
  (setq lsp-highlight-symbol-at-point t)  ;; was `nil`, changed to nil to see what it does
  :hook
  ;; disabled lsp for javascript and typescript to use Tide-mode only
  ;; disabled lsp for typescript to use Tide-mode only
  ;;(typescript-mode . lsp)
  ;;(js2-mode . lsp)
  (js2-jsx-mode . lsp)
  (enh-ruby-mode . lsp)
)

\end{verbatim}

\begin{verbatim}
(use-package company-lsp
  :ensure t
  :custom
  ;; debug
  (lsp-print-io nil)
  (lsp-trace nil)
  (lsp-print-performance nil)
  ;; general
  (lsp-auto-guess-root t)
  (lsp-document-sync-method 'incremental) ;; none, full, incremental, or nil
  (lsp-response-timeout 10)
  (lsp-prefer-flymake t) ;; t(flymake), nil(lsp-ui), or :none
  :config
  (setq company-lsp-enable-snippet t)
  (setq company-lsp-async t)
  (setq company-lsp-cache-candidates t)
  (setq company-lsp-enable-recompletion t)
)
\end{verbatim}

\begin{verbatim}
(use-package lsp-ui
  :ensure t
  :hook
  (lsp-mode . lsp-ui-mode)
  :preface
  (defun ladicle/toggle-lsp-ui-doc ()
    (interactive)
    (if lsp-ui-doc-mode
      (progn
        (lsp-ui-doc-mode -1)
        (lsp-ui-doc--hide-frame)
      )
    (lsp-ui-doc-mode 1)
    )
  )
  :config
  ;; lsp-ui appearance
  (set-face-attribute 'lsp-ui-doc-background  nil :background "#f9f2d9")
  (add-hook 'lsp-ui-doc-frame-hook
    (lambda (frame _w)
      (set-face-attribute 'default frame :font "Overpass Mono 11")
    )
  )
  (set-face-attribute 'lsp-ui-sideline-global nil
                      :inherit 'shadow
                      :background "#f9f2d9")
  (setq ;; lsp-ui-doc
        lsp-ui-doc-enable t
        lsp-ui-doc-header t
        lsp-ui-doc-include-signature nil
        lsp-ui-doc-position 'at-point ;; top, bottom, or at-point
        lsp-ui-doc-max-width 100
        lsp-ui-doc-max-height 30
        lsp-ui-doc-use-childframe t
        lsp-ui-doc-use-webkit t
        ;; lsp-ui-flycheck
        lsp-ui-flycheck-enable t
        lsp-ui-flycheck-list-position 'right
        lsp-ui-flycheck-live-reporting t
        ;; lsp-ui-sideline
        lsp-ui-sideline-enable t
        lsp-ui-sideline-ignore-duplicate t
        lsp-ui-sideline-show-symbol t
        lsp-ui-sideline-show-hover t
        lsp-ui-sideline-show-diagnostics nil
        lsp-ui-sideline-show-code-actions t
        lsp-ui-sideline-code-actions-prefix ""
        lsp-ui-sideline-update-mode 'point
        ;; lsp-ui-imenu
        lsp-ui-imenu-enable t
        lsp-ui-imenu-kind-position 'top
        ;; lsp-ui-peek
        lsp-ui-peek-enable t
        lsp-ui-peek-peek-height 20
        lsp-ui-peek-list-width 40
        lsp-ui-peek-fontify 'on-demand) ;; never, on-demand, or always
  :bind
    (:map lsp-mode-map
      ("C-c C-r" . lsp-ui-peek-find-references)
      ("C-c C-j" . lsp-ui-peek-find-definitions)
      ("C-c g d" . lsp-goto-type-definition)
      ("C-c f d" . lsp-find-definition)
      ("C-c g i" . lsp-goto-implementation)
      ("C-c f i" . lsp-find-implementation)
      ("C-c i"   . lsp-ui-peek-find-implementation)
      ("C-c m"   . lsp-ui-imenu)
      ("C-c s"   . lsp-ui-sideline-mode)
      ("C-c d"   . ladicle/toggle-lsp-ui-doc)
    )
    ;; remap native find-definitions and references to use lsp-ui
    (:map lsp-ui-mode-map
      ([remap xref-find-definitions] . lsp-ui-peek-find-definitions)
      ([remap xref-find-references] . lsp-ui-peek-find-references)
      ("C-c u" . lsp-ui-imenu)
    )
)
\end{verbatim}

\subsection*{Disable <RET> for autocomplete and leave on TAB}
\label{sec:org96ed232}
\begin{verbatim}
;; (define-key ac-completing-map [return] nil)
;; (define-key ac-completing-map "\r" nil)
\end{verbatim}


\subsection*{enable autocompletion engine}
\label{sec:orgec7aff9}
\begin{verbatim}
(require 'auto-complete)
(global-auto-complete-mode t)
\end{verbatim}


\section*{Company}
\label{sec:org39621fd}

\begin{verbatim}
(use-package company
  :ensure t
  :defer t
  :init
  (global-company-mode)
  :bind
  (:map evil-insert-state-map
  ;; ("<tab>" . company-indent-or-complete-common)
  ("C-SPC" . company-indent-or-complete-common)
  )
  (:map company-active-map
  ("M-n" . nil)
  ("M-p" . nil)
  ("C-n" . company-select-next)
  ("C-p" . company-select-previous)
  ("<tab>" . company-complete-common-or-cycle)
  ("S-<tab>" . company-select-previous)
  ("<backtab>" . company-select-previous)
  ("C-d" . company-show-doc-buffer)
  )
  (:map company-search-map
   ("C-p" . company-select-previous)
   ("C-n" . company-select-next)
  )
  :config
  ;; Use Company for completion
  (progn
    (bind-key [remap completion-at-point] #'company-complete company-mode-map))
  (setq company-tooltip-limit 20)                      ; bigger popup window
  (setq company-minimum-prefix-length 1)               ; start completing after 1st char typed
  (setq company-idle-delay 0)                         ; decrease delay before autocompletion popup shows
  (setq company-echo-delay 0)                          ; remove annoying blinking
  (setq company-begin-commands '(self-insert-command)) ; start autocompletion only after typing
  ;; company-dabbrev
  (setq company-dabbrev-downcase nil)                  ;; Do not downcase completions by default.
  (setq company-dabbrev-ignore-case t)  ;; Even if I write something with the ‘wrong’ case, provide the ‘correct’ casing.
  (setq company-dabbrev-code-everywhere t)
  (setq company-dabbrev-other-buffers t)
  (setq company-dabbrev-code-other-buffers t)
  (setq company-selection-wrap-around t)               ; continue from top when reaching bottom
  (setq company-auto-complete 'company-explicit-action-p)
  (setq company-require-match nil)
  (setq company-tooltip-align-annotations t)
  (setq company-complete-number t)                     ;; Allow (lengthy) numbers to be eligible for completion.
  (setq company-show-numbers t)  ;; M-⟪num⟫ to select an option according to its number.
  (setq company-transformers '(company-sort-by-occurrence)) ; weight by frequency
  ;; (setq company-tooltip-flip-when-above t)
  ;; DELETE THIS PART IF USE PACKAGE :MAP WORKS
  ;; (define-key company-active-map (kbd "M-n") nil)
  ;; (define-key company-active-map (kbd "M-p") nil)
  ;; (define-key company-active-map (kbd "C-n") 'company-select-next)
  ;; (define-key company-active-map (kbd "C-p") 'company-select-previous)
  ;; (define-key company-active-map (kbd "TAB") 'company-complete-common-or-cycle)
  ;; (define-key company-active-map (kbd "<tab>") 'company-complete-common-or-cycle)
  ;; (define-key company-active-map (kbd "S-TAB") 'company-select-previous)
  ;; (define-key company-active-map (kbd "<backtab>") 'company-select-previous)
)
\end{verbatim}

\subsection*{Company emoji suport}
\label{sec:org4b5f736}

\begin{verbatim}
use `:` to use emojis
\end{verbatim}

\begin{verbatim}
(use-package company-emoji
  :ensure t
  :config
  (add-to-list 'company-backends 'company-emoji)
)
\end{verbatim}

\subsection*{Company-QuickHelp}
\label{sec:org8f34c88}
\begin{verbatim}
(use-package company-quickhelp          ; Documentation popups for Company
   :ensure t
   ;; :defer t
   :hook
   (global-company-mode . company-quickhelp-mode)
   :bind
   (:map company-active-map
   ("M-h" . company-quickhelp-manual-begin)
   )
   :config
   (setq company-quickhelp-delay 0.7)
   (company-quickhelp-mode)
)
\end{verbatim}

\subsection*{Company postframe}
\label{sec:orge30578b}

\begin{verbatim}
PS: this looks exactly the same as the usual company popup, except it doesn't disturb other overlays (like line numbers) in the buffer.
\end{verbatim}

\begin{verbatim}
(use-package company-posframe
  :ensure t
  :hook
  (company-mode . company-posframe-mode)
  (global-company-mode . company-posframe-mode)
)
\end{verbatim}

\subsection*{Company Box (icons in suggestions)}
\label{sec:org67d0b30}
\begin{verbatim}
;; (use-package company-box
;;   :ensure t
;;   :hook
;;   (company-mode . company-box-mode)
;;   (global-company-mode . company-box-mode)
;; )
\end{verbatim}


\subsection*{company-go}
\label{sec:org0e9ec81}
\begin{verbatim}
(use-package company-go
  :ensure t
  :defer t
  :init
  (with-eval-after-load 'company
    (add-to-list 'company-backends 'company-go))
)
\end{verbatim}

\subsection*{Company TabNine}
\label{sec:org52bb604}
\begin{verbatim}
this part is commented because TabNine is paid software
and not sure if i want to use it
\end{verbatim}

\begin{verbatim}
;; (use-package company-tabnine
;;   :demand
;;   :custom
;;   (company-tabnine-max-num-results 9)
;;   :bind
;;   (("M-q" . company-other-backend)
;;    ("C-z t" . company-tabnine))
;;   :config
;;   ;; Enable TabNine on default
;;   (add-to-list 'company-backends #'company-tabnine)

;;   ;; Integrate company-tabnine with lsp-mode
;;   (defun company//sort-by-tabnine (candidates)
;;     (if (or (functionp company-backend)
;;             (not (and (listp company-backend) (memq 'company-tabnine company-backend))))
;;         candidates
;;       (let ((candidates-table (make-hash-table :test #'equal))
;;             candidates-lsp
;;             candidates-tabnine)
;;         (dolist (candidate candidates)
;;           (if (eq (get-text-property 0 'company-backend candidate)
;;                   'company-tabnine)
;;               (unless (gethash candidate candidates-table)
;;                 (push candidate candidates-tabnine))
;;             (push candidate candidates-lsp)
;;             (puthash candidate t candidates-table)))
;;         (setq candidates-lsp (nreverse candidates-lsp))
;;         (setq candidates-tabnine (nreverse candidates-tabnine))
;;         (nconc (seq-take candidates-tabnine 3)
;;                (seq-take candidates-lsp 6)))))
;;   (add-hook 'lsp-after-open-hook
;;             (lambda ()
;;               (setq company-tabnine-max-num-results 3)
;;               (add-to-list 'company-transformers 'company//sort-by-tabnine t)
;;               (add-to-list 'company-backends '(company-lsp :with company-tabnine :separate)))))
\end{verbatim}

\section*{Yasnippets}
\label{sec:org9687edb}

\begin{verbatim}
(use-package yasnippet
  :ensure t
  :hook
  (prog-mode . yas-minor-mode)
  (text-mode . yas-minor-mode)
  :bind
  ;; ("<tab>" . yas-maybe-expand)
  ("C-<tab>" . yas-maybe-expand)
  (:map yas-minor-mode-map
  ;; yas-maybe-expand only expands if there are candidates.
  ;; if not, acts like binding is unbound and run whatever command is bound to that key normally
  ;; ("<tab>" . yas-maybe-expand)
  ;; Bind `C-c y' to `yas-expand' ONLY.
  ("C-c y" . yas-expand)
  ("C-SPC" . yas-expand)
  )
  :config
  ;; set snippets directory
  ;; (with-eval-after-load 'yasnippet
  ;;  (setq yas-snippet-dirs '(yasnippet-snippets-dir)))
  (setq yas-verbosity 1)                      ; No need to be so verbose
  (setq yas-wrap-around-region t)
  (yas-reload-all)
  ;; disabled global mode in favor or hooks in prog and text modes only
  ;; (yas-global-mode 1)
)
\end{verbatim}

\begin{verbatim}
(use-package yasnippet-snippets         ; Collection of snippets
  :ensure t
)
\end{verbatim}


\section*{Copy/Paste To/From System's Clipboard =D}
\label{sec:org762cebc}
this was supposed to be on the helper functions and macro section at the beggining of the file
but it has evil defined keybindings and had to be put after the evil section or emacs would complain it didnt know what evil is

\subsubsection*{Copy to system clipboard}
\label{sec:orgd91b9c7}

\begin{verbatim}
(defun copy-to-clipboard ()
"Make F8 and F9 Copy and Paste to/from OS Clipboard.  Super usefull."
(interactive)
(if (display-graphic-p)
    (progn
        (message "Yanked region to x-clipboard!")
        (call-interactively 'clipboard-kill-ring-save)
        )
    (if (region-active-p)
        (progn
        (shell-command-on-region (region-beginning) (region-end) "xsel -i -b")
        (message "Yanked region to clipboard!")
        (deactivate-mark))
    (message "No region active; can't yank to clipboard!")))
)
\end{verbatim}

\subsubsection*{Paste}
\label{sec:org5b56695}

\begin{verbatim}
(evil-define-command paste-from-clipboard()
(if (display-graphic-p)
    (progn
        (clipboard-yank)
        (message "graphics active")
        )
    (insert (shell-command-to-string "xsel -o -b")) ) )
\end{verbatim}

\begin{verbatim}
(global-set-key [f9] 'copy-to-clipboard)
(global-set-key [f10] 'paste-from-clipboard)
\end{verbatim}


\section*{End init.el file}
\label{sec:orgae4be57}
\begin{verbatim}
;; Local Variables:
;; coding: utf-8
;; no-byte-compile: t
;; End:


(provide 'init)
;;; .emacs ends here

\end{verbatim}
\end{document}
