% Created 2019-09-12 qui 19:03
% Intended LaTeX compiler: pdflatex
\documentclass[11pt]{article}
\usepackage[utf8]{inputenc}
\usepackage[T1]{fontenc}
\usepackage{graphicx}
\usepackage{grffile}
\usepackage{longtable}
\usepackage{wrapfig}
\usepackage{rotating}
\usepackage[normalem]{ulem}
\usepackage{amsmath}
\usepackage{textcomp}
\usepackage{amssymb}
\usepackage{capt-of}
\usepackage{hyperref}
\author{Gustavo P Borges}
\date{\today}
\title{gugutz emacs config\\\medskip
\large ORGfied configuration for Emacs}
\hypersetup{
 pdfauthor={Gustavo P Borges},
 pdftitle={gugutz emacs config},
 pdfkeywords={},
 pdfsubject={This file is compiled to init.el automatically on every save},
 pdfcreator={Emacs 26.2 (Org mode 9.2.4)}, 
 pdflang={English}}
\begin{document}

\maketitle
\setcounter{tocdepth}{0}
\tableofcontents


\section*{Recompile init.el everytime emacs.org is changed and saved}
\label{sec:orgf083100}

\begin{verbatim}
Moved this to beggining of the file to avoid it not being parsed when theres an error in the middle of the file
It was being recompiled without this function so i had to manually re-copy first-init.el to make it compile first time again and again
\end{verbatim}



\begin{verbatim}
(defun /util/tangle-init ()
  (interactive)
  "If the current buffer is init.org' the code-blocks are
tangled, and the tangled file is compiled."
  (when (equal (buffer-file-name)
               (expand-file-name (concat user-emacs-directory "emacs.org")))
    ;; Avoid running hooks when tangling.
    (let ((prog-mode-hook nil))
      (org-babel-tangle)
      (byte-compile-file (concat user-emacs-directory "init.el")))))
\end{verbatim}

\begin{verbatim}
(add-hook 'after-save-hook #'/util/tangle-init)
\end{verbatim}


\section*{General editor settings}
\label{sec:orgc55845a}

\subsection*{Allow access from emacsclient}
\label{sec:org811dc98}

\begin{verbatim}
(require 'server)
(unless (or (daemonp) (server-running-p))
  (server-start))
\end{verbatim}

\subsection*{Prevent emacs to create lockfiles (.\#files\#).}
\label{sec:orga0db61d}

PS: this also stops preventing editing colisions, so watch out

\begin{verbatim}
(setq create-lockfiles nil)
\end{verbatim}

\subsection*{Use the system clipboard}
\label{sec:org3457311}
\begin{verbatim}
(setq x-select-enable-clipboard t)
\end{verbatim}

\subsection*{Always follow symbolic links to edit the 'actual' file it points to}
\label{sec:org42c347a}

\begin{verbatim}
(setq vc-follow-symlinks t)
\end{verbatim}

\subsection*{Enable mouse support in terminal mode}
\label{sec:org4c04811}

\begin{verbatim}
(when (eq window-system nil)
  (xterm-mouse-mode 1))
\end{verbatim}

\subsection*{Save all tempfiles in \$TMPDIR/emacs\$UID/}
\label{sec:orgaa10a00}

\begin{verbatim}
(defconst emacs-tmp-dir (expand-file-name (format "emacs%d" (user-uid)) temporary-file-directory))
(setq backup-directory-alist
    `((".*" . ,emacs-tmp-dir)))
(setq auto-save-file-name-transforms
    `((".*" ,emacs-tmp-dir t)))
(setq auto-save-list-file-prefix
    emacs-tmp-dir)
\end{verbatim}

\subsection*{Disable the annoying Emacs bell ring (beep)}
\label{sec:orgbf1fef5}

\begin{verbatim}
(setq ring-bell-function 'ignore)
\end{verbatim}

\subsection*{Disable initial scratch message}
\label{sec:org888e9b4}
\begin{verbatim}
(setq initial-scratch-message nil)
\end{verbatim}
\subsection*{Create alias to yes-or-no anwsers (y-or-n-p}
\label{sec:org4abf6c6}
\begin{verbatim}
(defalias 'yes-or-no-p 'y-or-n-p)
(fset 'yes-or-no-p 'y-or-n-p) 
\end{verbatim}


\subsection*{show matching parenthesis}
\label{sec:orgec9465d}
\begin{verbatim}
; parentheses
(show-paren-mode t)
\end{verbatim}

\subsection*{default indentation}
\label{sec:orgfb60ba4}
\begin{verbatim}
(setq-default indent-tabs-mode nil)
(setq-default c-basic-offset 2)
\end{verbatim}

\subsection*{show line numbers}
\label{sec:orgdc90f6c}
\begin{verbatim}
(when (version<= "26.0.50" emacs-version )
  (global-display-line-numbers-mode))
\end{verbatim}

\subsection*{line number : pretty format}
\label{sec:orge562435}
\begin{verbatim}
(setq linum-format " %d ")
\end{verbatim}

\subsection*{superword-mode and subword-modes}
\label{sec:org2f8ef3e}
\begin{verbatim}
Alt+x subword-mode. It change all cursor movement/edit commands to stop in-between the “camelCase” words.
Alt+x superword-mode (emacs 24.4) is similar. It treats text like “x_y” as one word. Useful for “snake_case”.
subword-mode and superword-mode are mutally exclusive. Turning one on turns off the other.
\end{verbatim}


\begin{verbatim}
Enable global subword-mode (disabled - its enable per file according to language)
\end{verbatim}


\begin{verbatim}
;; (global-subword-mode 1)
\end{verbatim}

\subsection*{Turn on auto-revert mode (auto updates files changed on disk)}
\label{sec:org58ab644}
\begin{verbatim}
(global-auto-revert-mode 1)
(setq auto-revert-interval 0.5)
\end{verbatim}

\subsection*{Spellchecking}
\label{sec:org1d2991b}
\begin{verbatim}
(defconst *spell-check-support-enabled* t) ;; Enable with t if you prefer
\end{verbatim}

\subsection*{C-n insert newlines if the point is at the end of the buffer.}
\label{sec:org137ad7f}
\begin{verbatim}
Useful, as it means you won’t have to reach for the return key to add newlines!
\end{verbatim}

\begin{verbatim}
(setq next-line-add-newlines t)
\end{verbatim}



\subsection*{Increase, decrease and adjust font size}
\label{sec:org3fc381f}

\begin{verbatim}
(global-set-key (kbd "C-+") #'text-scale-increase)
(global-set-key (kbd "C-_") #'text-scale-decrease)
;; (global-set-key (kbd "C-)") #'text-scale-adjust)
\end{verbatim}

\subsection*{expand-region}
\label{sec:orgc98d361}
\begin{verbatim}
;; (require 'expand-region)
(global-set-key (kbd "C-TAB") 'er/expand-region)
\end{verbatim}

\subsection*{hippie-expand (native emacs expand function)}
\label{sec:orgfff8450}

\begin{verbatim}
(global-set-key "\M- " 'hippie-expand)
\end{verbatim}


\section*{Macros and helper functions}
\label{sec:orgf14d985}

\subsection*{Bindings}
\label{sec:orgb1e89d6}


\begin{verbatim}
These macros are to help me remap keys.
\end{verbatim}


\begin{verbatim}
(defmacro /bindings/define-prefix-keys (keymap prefix &rest body)
  (declare (indent defun))
  `(progn
     ,@(cl-loop for binding in body
                collect
                `(let ((seq ,(car binding))
                       (func ,(cadr binding))
                       (desc ,(caddr binding)))
                   (define-key ,keymap (kbd seq) func)
                   (when desc
                     (which-key-add-key-based-replacements
                       (if ,prefix
                           (concat ,prefix " " seq)
                         seq)
                       desc))))))

(defmacro /bindings/define-keys (keymap &rest body)
  (declare (indent defun))
  `(/bindings/define-prefix-keys ,keymap nil ,@body))

(defmacro /bindings/define-key (keymap sequence binding &optional description)
  (declare (indent defun))
  `(/bindings/define-prefix-keys ,keymap nil
     (,sequence ,binding ,description)))
\end{verbatim}


\subsection*{After}
\label{sec:org65acd41}

\begin{verbatim}
with-eval-after-load is a function that lets you defer execution of code until after a feature has been loaded.
It is very useful to only load some packages when they’re, and because of that it is extensively used in this setup. 
So of course there is a macro to make it simpler. It can also run code if a package has been installed by using “pkgname-autoloads” or only if multiple packages have been loaded.
This also avoids loading config for packages that haven’t been loaded yet, resulting in void variables of function definitions. 
This was take from milkypostman (along with some other things).
\end{verbatim}


\begin{verbatim}
;; examples
;; after [evil magit] (
  ;; execute after evil and magit have been loaded
;  )

;; macro definiton
(defmacro after (feature &rest body)
  "Executes BODY after FEATURE has been loaded.

FEATURE may be any one of:
    'evil            => (with-eval-after-load 'evil BODY)
    \"evil-autoloads\" => (with-eval-after-load \"evil-autolaods\" BODY)
    [evil cider]     => (with-eval-after-load 'evil
                          (with-eval-after-load 'cider
                            BODY))
"
  (declare (indent 1))
  (cond
   ((vectorp feature)
    (let ((prog (macroexp-progn body)))
      (cl-loop for f across feature
               do
               (progn
                 (setq prog (append `(',f) `(,prog)))
                 (setq prog (append '(with-eval-after-load) prog))))
      prog))
   (t
    `(with-eval-after-load ,feature ,@body))))
\end{verbatim}


\subsection*{Auto save function}
\label{sec:org7e1e865}

\begin{verbatim}
(defun my-save ()
  "Save file when leaving insert mode in Evil."
  (if (buffer-file-name)
      (evil-save)))
\end{verbatim}

\subsubsection*{This hook to the above function was breaking the evil-esc-delay 0}
\label{sec:org698b094}
\begin{verbatim}
;; (add-hook 'evil-insert-state-exit-hook 'my-save)
\end{verbatim}



\section*{GPG Encryption}
\label{sec:org5510d5d}

\begin{verbatim}
(require 'epa-file)
(epa-file-enable)
\end{verbatim}


\section*{Packages}
\label{sec:org94f182c}
\subsection*{package repositories}
\label{sec:org32d2d58}

\begin{verbatim}
(require 'package)
;; add melpa stable emacs package repository
(add-to-list 'package-archives '("melpa" . "https://melpa.org/packages/"))
(add-to-list 'package-archives '("gnu" . "https://elpa.gnu.org/packages/"))
(add-to-list 'package-archives '("org" . "http://orgmode.org/elpa/") t) ; Org-mode's repository
\end{verbatim}

\subsection*{initialize packages}
\label{sec:org7c52bae}
\begin{verbatim}
(package-initialize)
\end{verbatim}

moved this part to beggining of the file because if the
custom-safe-themes variable is not set before smart-mode-line (sml) activates
emacs asks 2 annoying confirmations on every startup before actually starting

\begin{verbatim}
(custom-set-variables
 ;; custom-set-variables was added by Custom.
 ;; If you edit it by hand, you could mess it up, so be careful.
 ;; Your init file should contain only one such instance.
 ;; If there is more than one, they won't work right.
 '(custom-safe-themes
   (quote
    ("84d2f9eeb3f82d619ca4bfffe5f157282f4779732f48a5ac1484d94d5ff5b279" "57f95012730e3a03ebddb7f2925861ade87f53d5bbb255398357731a7b1ac0e0" "3c83b3676d796422704082049fc38b6966bcad960f896669dfc21a7a37a748fa" default)))
 '(fci-rule-color "#3E4451")
 '(package-selected-packages
   (quote
   (pdf-tools ox-pandoc ox-reveal org-preview-html latex-preview-pane smart-mode-line-powerline-theme base16-theme gruvbox-theme darktooth-theme rainbow-mode smartscan restclient editorconfig prettier-js pandoc rjsx-mode js2-refactor web-mode evil-org multiple-cursors flycheck smart-mode-line ## evil-leader evil-commentary evil-surround htmlize magit neotree evil json-mode web-serverx org))))
(custom-set-faces
 ;; custom-set-faces was added by Custom.
 ;; If you edit it by hand, you could mess it up, so be careful.
 ;; Your init file should contain only one such instance.
 ;; If there is more than one, they won't work right.
 )
\end{verbatim}

\subsection*{Add the folder 'config' to emacs load-path so i can require stuff from there}
\label{sec:orgdc05d3c}

\begin{verbatim}
(add-to-list 'load-path (expand-file-name "config" user-emacs-directory))
;; (add-to-list 'load-path "~/dotfiles/emacs.d/config")

\end{verbatim}









\section*{Require my personal packages}
\label{sec:org3a1be9f}
\begin{verbatim}

(require 'evil.tau)
(require 'org.tau)
(require 'ruby.tau)
(require 'elixir.tau)

\end{verbatim}


\section*{Evil}
\label{sec:orge233c30}

\begin{verbatim}
All Evil settings are meant to be isolated in a separate file evil.tau.
\end{verbatim}

\subsection*{Require Evil related packages}
\label{sec:org4a3591a}

\begin{verbatim}
(require 'evil)
(evil-mode 1)
\end{verbatim}

\subsection*{Don't wait for any other keys after escape is pressed.}
\label{sec:org6a53348}
\begin{verbatim}
(setq evil-esc-delay 0)
\end{verbatim}

\subsection*{Make Evil look a bit more like (n) vim  (??)}
\label{sec:org8cc7190}
\begin{verbatim}
not sure what all these options do yet
\end{verbatim}


\begin{verbatim}
(setq evil-search-module 'isearch-regexp)
(setq evil-magic 'very-magic)
(setq evil-shift-width (symbol-value 'tab-width))
(setq evil-regexp-search t)
(setq evil-search-wrap t)
;; (setq evil-want-C-i-jump t)
(setq evil-want-C-u-scroll t)
(setq evil-want-fine-undo nil)
(setq evil-want-integration nil)
;; (setq evil-want-abbrev-on-insert-exit nil)
(setq evil-want-abbrev-expand-on-insert-exit nil)
;; move evil tag to beginning of modeline
(setq evil-mode-line-format '(before . mode-line-front-space))
\end{verbatim}

\subsection*{Window and buffer navigation with vim-like bindings}
\label{sec:orgbb97fd8}

\subsubsection*{vim-like navigation using C- HJKL (uppercase homerow keys)}
\label{sec:org9706fc9}
\begin{verbatim}
;; for some readon the bellow lines should be the default native way for navigation on emacs
;; but they dont work
;; using the above package instead til i find a solution
;
(windmove-default-keybindings 'control)
(global-set-key (kbd "C-H") 'windmove-left)
(global-set-key (kbd "C-L") 'windmove-right)
(global-set-key (kbd "C-K") 'windmove-up)
(global-set-key (kbd "C-J") 'windmove-down)
\end{verbatim}

\subsubsection*{vim-like navigation with C-w hjkl}
\label{sec:org3e71338}


\begin{verbatim}
Bellow i use the `define-keys` function to map window navigation to default Vim bindings <C-hjkl>
\end{verbatim}


\begin{verbatim}
First require the file with the function
\end{verbatim}


\begin{verbatim}
;; (require 'evil-tmux-navigator)
\end{verbatim}


\begin{verbatim}
Then create the keybindings 
\end{verbatim}

\begin{verbatim}
(define-prefix-command 'evil-window-map)
(define-key evil-window-map "h" 'evil-window-left)
(define-key evil-window-map "j" 'evil-window-down)
(define-key evil-window-map "k" 'evil-window-up)
(define-key evil-window-map "l" 'evil-window-right)
(define-key evil-window-map "b" 'evil-window-bottom-right)
(define-key evil-window-map "c" 'evil-window-delete)
(define-key evil-motion-state-map "\M-w" 'evil-window-map)
\end{verbatim}


\begin{verbatim}
;; (/bindings/define-keys evil-normal-state-map
  ;; ("C-w h" #'evil-window-left)
  ;; ("C-w j" #'evil-window-down)
  ;; ("C-w k" #'evil-window-up)
  ;; ("C-w l" #'evil-window-right))
\end{verbatim}


\begin{verbatim}
;; (/bindings/define-keys evil-normal-state-map
;;   ("C-w h" #'evil-window-left)
;;   ("C-w j" #'evil-window-down)
;;   ("C-w k" #'evil-window-up)
;;   ("C-w l" #'evil-window-right))
\end{verbatim}


\subsection*{make esc quit or cancel everything in Emacs}
\label{sec:orgcbfcd27}
\begin{verbatim}
(define-key evil-normal-state-map [escape] 'keyboard-quit)
(define-key evil-visual-state-map [escape] 'keyboard-quit)
(define-key minibuffer-local-map [escape] 'minibuffer-keyboard-quit)
(define-key minibuffer-local-ns-map [escape] 'minibuffer-keyboard-quit)
(define-key minibuffer-local-completion-map [escape] 'minibuffer-keyboard-quit)
(define-key minibuffer-local-must-match-map [escape] 'minibuffer-keyboard-quit)
(define-key minibuffer-local-isearch-map [escape] 'minibuffer-keyboard-quit)
\end{verbatim}

\subsection*{Cursor is alway black because of evil.}
\label{sec:orgb21f7a3}

\begin{verbatim}
Here is the workaround
(@see https://bitbucket.org/lyro/evil/issue/342/evil-default-cursor-setting-should-default)
\end{verbatim}

\begin{verbatim}
(setq evil-default-cursor t)
\end{verbatim}

\subsection*{recover native emacs commands that are overriden by evil}
\label{sec:orgad27a55}
\begin{verbatim}
this gives priority to native emacs behaviour rathen than Vim's
\end{verbatim}


\begin{verbatim}
(define-key evil-normal-state-map (kbd "SPC") 'ace-jump-mode)
(define-key evil-insert-state-map (kbd "C-e") 'move-end-of-line)
(define-key evil-insert-state-map (kbd "C-k") 'kill-line)
(define-key evil-normal-state-map (kbd "C-k") 'kill-line)
(define-key evil-insert-state-map (kbd "C-w") 'kill-region)
(define-key evil-normal-state-map (kbd "C-w") 'kill-region)
(define-key evil-visual-state-map (kbd "C-w") 'kill-region)
(define-key evil-visual-state-map (kbd "C-e") 'move-end-of-line)
(define-key evil-normal-state-map (kbd "C-e") 'move-end-of-line)
(define-key evil-normal-state-map (kbd "C-y") 'yank)
(define-key evil-insert-state-map (kbd "C-y") 'yank)
(define-key evil-visual-state-map (kbd "SPC") 'ace-jump-mode)
(define-key evil-normal-state-map "\C-e" 'evil-end-of-line)
(define-key evil-insert-state-map "\C-e" 'end-of-line)
(define-key evil-visual-state-map "\C-e" 'evil-end-of-line)
(define-key evil-motion-state-map "\C-e" 'evil-end-of-line)
(define-key evil-normal-state-map "\C-f" 'evil-forward-char)
(define-key evil-insert-state-map "\C-f" 'evil-forward-char)
(define-key evil-insert-state-map "\C-f" 'evil-forward-char)
(define-key evil-normal-state-map "\C-b" 'evil-backward-char)
(define-key evil-insert-state-map "\C-b" 'evil-backward-char)
(define-key evil-visual-state-map "\C-b" 'evil-backward-char)
(define-key evil-normal-state-map "\C-d" 'evil-delete-char)
(define-key evil-insert-state-map "\C-d" 'evil-delete-char)
(define-key evil-visual-state-map "\C-d" 'evil-delete-char)
(define-key evil-normal-state-map "\C-n" 'evil-next-line)
(define-key evil-insert-state-map "\C-n" 'evil-next-line)
(define-key evil-visual-state-map "\C-n" 'evil-next-line)
(define-key evil-normal-state-map "\C-p" 'evil-previous-line)
(define-key evil-insert-state-map "\C-p" 'evil-previous-line)
(define-key evil-visual-state-map "\C-p" 'evil-previous-line)
(define-key evil-normal-state-map "\C-w" 'evil-delete)
(define-key evil-insert-state-map "\C-w" 'evil-delete)
(define-key evil-visual-state-map "\C-w" 'evil-delete)
(define-key evil-normal-state-map "\C-y" 'yank)
(define-key evil-insert-state-map "\C-y" 'yank)
(define-key evil-visual-state-map "\C-y" 'yank)
(define-key evil-normal-state-map "\C-k" 'kill-line)
(define-key evil-insert-state-map "\C-k" 'kill-line)
(define-key evil-visual-state-map "\C-k" 'kill-line)
(define-key evil-normal-state-map "Q" 'call-last-kbd-macro)
(define-key evil-visual-state-map "Q" 'call-last-kbd-macro)
(define-key evil-insert-state-map "\C-e" 'end-of-line)
(define-key evil-insert-state-map "\C-r" 'search-backward)
\end{verbatim}


\begin{verbatim}
;; (define-key evil-window-map "\C-h" 'evil-window-left)
;; (define-key evil-window-map "\C-j" 'evil-window-down)
;; (define-key evil-window-map "\C-k" 'evil-window-up)
;; (define-key evil-window-map "\C-l" 'evil-window-right)
\end{verbatim}



\subsection*{change cursor color according to mode}
\label{sec:orgc248a18}

\begin{verbatim}
(setq evil-emacs-state-cursor '("#ff0000" box))
(setq evil-motion-state-cursor '("#FFFFFF" box))
(setq evil-normal-state-cursor '("#00ff00" box))
(setq evil-visual-state-cursor '("#abcdef" box))
(setq evil-insert-state-cursor '("#e2f00f" bar))
(setq evil-replace-state-cursor '("red" hbar))
(setq evil-operator-state-cursor '("red" hollow))
\end{verbatim}

\subsection*{multiple cursors}
\label{sec:org8c29ac4}

\begin{verbatim}
;; step 1, select thing in visual-mode (OPTIONAL)
;; step 2, `mc/mark-all-like-dwim' or `mc/mark-all-like-this-in-defun'
;; step 3, `ace-mc-add-multiple-cursors' to remove cursor, press RET to confirm
;; step 4, press s or S to start replace
;; step 5, press C-g to quit multiple-cursors
(define-key evil-visual-state-map (kbd "mn") 'mc/mark-next-like-this)
(define-key evil-visual-state-map (kbd "ma") 'mc/mark-all-like-this-dwim)
(define-key evil-visual-state-map (kbd "md") 'mc/mark-all-like-this-in-defun)
(define-key evil-visual-state-map (kbd "mm") 'ace-mc-add-multiple-cursors)
(define-key evil-visual-state-map (kbd "ms") 'ace-mc-add-single-cursor)
\end{verbatim}

\subsection*{evil-leader}
\label{sec:orge7ae41b}

\begin{verbatim}
(require 'evil-leader)
\end{verbatim}

\begin{verbatim}
(global-evil-leader-mode)
(evil-leader/set-leader ",")
(evil-leader/set-key
  "e" 'find-file
  "q" 'evil-quit
  "w" 'save-buffer
  "k" 'kill-buffer
  "b" 'switch-to-buffer
  "-" 'split-window-bellow
  "|" 'split-window-right)
\end{verbatim}

\subsection*{Evil Surround}
\label{sec:orgcdb88de}
\begin{verbatim}
@see https://github.com/timcharper/evil-surround for tutorial
\end{verbatim}


\begin{verbatim}
(require 'evil-surround)
(global-evil-surround-mode 1)
\end{verbatim}

\begin{verbatim}
(defun evil-surround-prog-mode-hook-setup ()
  "Documentation string, idk, put something here later."
  (push '(47 . ("/" . "/")) evil-surround-pairs-alist)
  (push '(40 . ("(" . ")")) evil-surround-pairs-alist)
  (push '(41 . ("(" . ")")) evil-surround-pairs-alist)
  (push '(91 . ("[" . "]")) evil-surround-pairs-alist)
  (push '(93 . ("[" . "]")) evil-surround-pairs-alist))
(add-hook 'prog-mode-hook 'evil-surround-prog-mode-hook-setup)
\end{verbatim}

\begin{verbatim}
(defun evil-surround-js-mode-hook-setup ()
  "ES6." ;  <-- this is a documentation string, a feature in Lisp
  ;; I believe this is for auto closing pairs
  (push '(?1 . ("{`" . "`}")) evil-surround-pairs-alist)
  (push '(?2 . ("${" . "}")) evil-surround-pairs-alist)
  (push '(?4 . ("(e) => " . "(e)")) evil-surround-pairs-alist)
  ;; ReactJS
  (push '(?3 . ("classNames(" . ")")) evil-surround-pairs-alist))
(add-hook 'js2-mode-hook 'evil-surround-js-mode-hook-setup)
\end{verbatim}

\begin{verbatim}
(defun evil-surround-emacs-lisp-mode-hook-setup ()
  (push '(?` . ("`" . "'")) evil-surround-pairs-alist))
(add-hook 'emacs-lisp-mode-hook 'evil-surround-emacs-lisp-mode-hook-setup)
(defun evil-surround-org-mode-hook-setup ()
  (push '(91 . ("[" . "]")) evil-surround-pairs-alist)
  (push '(93 . ("[" . "]")) evil-surround-pairs-alist)
  (push '(?= . ("=" . "=")) evil-surround-pairs-alist))
(add-hook 'org-mode-hook 'evil-surround-org-mode-hook-setup)
\end{verbatim}



\subsection*{Vim Commentary}
\label{sec:org0722754}
\begin{verbatim}
(require 'evil-commentary)
(evil-commentary-mode)
\end{verbatim}

\subsection*{Evil-Matchit}
\label{sec:orgf8c7b08}
\begin{verbatim}
(require 'evil-matchit)
(global-evil-matchit-mode 1)
\end{verbatim}


\section*{org-mode}
\label{sec:orgb1cc435}

\begin{verbatim}
The ORG part of the config compiles to a separate file, inside the config folder, called `org.el`
\end{verbatim}

\subsection*{Require ORG}
\label{sec:org291f647}

\begin{verbatim}
(require 'org)
\end{verbatim}

\subsection*{Resolve issue with Tab not working with ORG only in Normal VI Mode in terminal}
\label{sec:orgdc67caf}

(something with TAB on terminals being related to C-i\ldots{})

\begin{verbatim}
(add-hook 'org-mode-hook                                                                      
          (lambda ()                                                                          
        (define-key evil-normal-state-map (kbd "TAB") 'org-cycle))) 

;; (setq evil-want-C-i-jump nil)
\end{verbatim}



\subsection*{Show CLOSED tag line in closed TODO items}
\label{sec:orge3fbef0}

\begin{verbatim}
(setq org-log-done 'time)
\end{verbatim}

\subsection*{Prompt to leave a note when closing an item}
\label{sec:orge79888a}
\begin{verbatim}
(setq org-log-done 'note)
\end{verbatim}

\begin{NOTE}
Also achievable on a per file basis with: \#+STARTUP: logdone
\end{NOTE}

\subsection*{Function to activate export-on-save in org mode}
\label{sec:org455d727}

\begin{verbatim}
(defun toggle-org-html-export-on-save ()
  "Make Emacs auto-export to HTML when org file is saved.
Enable calling this function from the file with <M-x>."
  (interactive)
  (if (memq 'org-html-export-to-html after-save-hook)
      (progn
        (remove-hook 'after-save-hook 'org-html-export-to-html t)
        (message "Disabled org html export on save for current buffer..."))
    (add-hook 'after-save-hook 'org-html-export-to-html nil t)
    (message "Enabled org html export on save for current buffer...")))
\end{verbatim}

\subsection*{Add hook to auto-export automatically on saveing ORG files}
\label{sec:org22017a6}

\begin{verbatim}
(defun org-mode-export-hook ()
  "This exports to diffenent outputs everytime the file is saved.
This will be added to org-mode-hook, so it only activates on ORG files.
Generates outputs in these formats:
- PDF
- HTML
- RevealJS."
   (add-hook 'after-save-hook 'org-beamer-export-to-pdf t t)
   (add-hook 'after-save-hook 'org-reveal-export-to-html t t))

; Finally adds the above hook in org-mode-hook.
;; (add-hook 'org-mode-hook #'org-mode-export-hook)
\end{verbatim}


\subsection*{add suport for the ignore tag (ignores a headline without ignoring its content)}
\label{sec:orgd85509d}
\begin{verbatim}
(require 'ox-extra)
(ox-extras-activate '(ignore-headlines))
\end{verbatim}

\subsection*{Evil-ORG}
\label{sec:org8f41f40}

\begin{verbatim}
(after 'org
  (require 'evil-org)
  (require 'evil-org-agenda)
  (add-hook 'org-mode-hook #'evil-org-mode)
  (add-hook 'evil-org-mode-hook
            (lambda ()
              (evil-org-set-key-theme))))
\end{verbatim}

\begin{verbatim}
;; (add-hook 'org-mode-hook 'evil-org-mode)
;; (evil-org-set-key-theme '(navigation insert textobjects additional calendar))
;; (evil-org-agenda-set-keys)
\end{verbatim}

\subsection*{ox-pandoc}
\label{sec:org1761667}

As pandoc supports many number of formats, initial org-export-dispatch
shortcut menu does not show full of its supported formats. You can customize
org-pandoc-menu-entry variable (and probably restart Emacs) to change its
default menu entries.
If you want delayed loading of `ox-pandoc’ when org-pandoc-menu-entry
is customized, please consider the following settings in your init file``

\begin{verbatim}
(with-eval-after-load 'ox
  (require 'ox-pandoc))
\end{verbatim}

\begin{verbatim}
(require 'ox-pandoc)
\end{verbatim}

\begin{verbatim}
;; default options for all output formats
(setq org-pandoc-options '((standalone . t)))
;; cancel above settings only for 'docx' format
(setq org-pandoc-options-for-docx '((standalone . nil)))
;; special settings for beamer-pdf and latex-pdf exporters
(setq org-pandoc-options-for-beamer-pdf '((pdf-engine . "xelatex")))
(setq org-pandoc-options-for-latex-pdf '((pdf-engine . "luatex")))
;; special extensions for markdown_github output
(setq org-pandoc-format-extensions '(markdown_github+pipe_tables+raw_html))
\end{verbatim}

\subsection*{ox-twbs (exporter to twitter bootstrap html)}
\label{sec:orge287151}
\begin{verbatim}
(setq org-enable-bootstrap-support t)
\end{verbatim}

\subsection*{ReveaJS org-reveal:}
\label{sec:org500e21b}
\begin{verbatim}
This delay makes the options to export to RevealJS appear on the exporter menu (C-c C-e)
\end{verbatim}


\begin{verbatim}
(with-eval-after-load 'ox
  (require 'ox-reveal))
\end{verbatim}

\begin{verbatim}
(require 'ox-reveal)
\end{verbatim}


\subsection*{UTF8 pretty bullets in org mode}
\label{sec:org3bdd29f}
(require 'org-bullets)
(add-hook 'org-mode-hook (lambda () (org-bullets-mode 1)))


\section*{Helm}
\label{sec:orga8f7555}

\begin{verbatim}
(require 'helm)

(setq helm-bookmark-show-location t)
(setq helm-buffer-max-length 40)
(setq helm-split-window-inside-p t)
(setq helm-mode-fuzzy-match t)
(setq helm-ff-file-name-history-use-recentf t)
(setq helm-ff-skip-boring-files t)
(setq helm-follow-mode-persistent t)

(after 'helm-source
  (defun /helm/make-source (f &rest args)
    (let ((source-type (cadr args))
          (props (cddr args)))
      (unless (child-of-class-p source-type 'helm-source-async)
        (plist-put props :fuzzy-match t))
      (apply f args)))
  (advice-add 'helm-make-source :around '/helm/make-source))
\end{verbatim}


\subsection*{Other helm settings}
\label{sec:orgea8e9a8}

\begin{verbatim}
(after 'helm
  ;; take between 10-30% of screen space
  (setq helm-autoresize-min-height 10)
  (setq helm-autoresize-max-height 30)
  (helm-autoresize-mode t))
\end{verbatim}

\begin{verbatim}
Make helm replace the default Find-File and M-x
\end{verbatim}


\begin{verbatim}
(progn
(global-set-key [remap execute-extended-command] #'helm-M-x)
(global-set-key [remap find-file] #'helm-find-files)
(helm-mode t))
\end{verbatim}

\subsection*{Helm related bindings}
\label{sec:orge900aa9}

\begin{verbatim}
(after 'helm
  (require 'helm-config)
  (global-set-key (kbd "C-c h") #'helm-command-prefix)
  (global-unset-key (kbd "C-x c"))
  ;; (global-set-key (kbd "C-h a") #'helm-apropos)
  (global-set-key (kbd "C-x b") #'helm-buffers-list)
  (global-set-key (kbd "C-x C-b") #'helm-mini)
  (global-set-key (kbd "C-x C-f") #'helm-find-files)
  (global-set-key (kbd "C-x r b") #'helm-bookmarks)
  (global-set-key (kbd "M-x") #'helm-M-x)
  (global-set-key (kbd "M-y") #'helm-show-kill-ring)
  (global-set-key (kbd "M-:") #'helm-eval-expression-with-eldoc)
  (define-key helm-map (kbd "<tab>") #'helm-execute-persistent-action)
  (define-key helm-map (kbd "C-z") #'helm-select-action)
)
\end{verbatim}


\section*{Projectile}
\label{sec:org9b277b0}

\subsection*{Activate Projectile}
\label{sec:org4ffd080}
\begin{verbatim}
(projectile-mode +1)
(define-key projectile-mode-map (kbd "s-p") 'projectile-command-map)
(define-key projectile-mode-map (kbd "C-c p") 'projectile-command-map)
\end{verbatim}


\section*{Dired}
\label{sec:org50fe3b6}

\begin{verbatim}
(after 'dired
  (require 'dired-k)
  (setq dired-k-style 'git)
  (setq dired-k-human-readable t)
  (add-hook 'dired-initial-position-hook #'dired-k))
\end{verbatim}

\begin{verbatim}
(setq dired-dwin-target t)
\end{verbatim}


\section*{Magit}
\label{sec:org11b8a8e}

\subsection*{Load evil-magit with magit buffer}
\label{sec:orgb97d849}

\begin{verbatim}
(require 'evil-magit)
(evil-magit-init)
\end{verbatim}


\subsection*{define global keybing to magit-status}
\label{sec:org4eedf4e}

\begin{verbatim}
(global-set-key (kbd "C-x g") 'magit-status)
\end{verbatim}


\section*{treemacs (neotree like navigation)}
\label{sec:orgbb1c654}

\subsection*{Load treemacs and its relevant subpackages}
\label{sec:orgb30f0a4}
\begin{verbatim}
(require 'treemacs)

;; (after [treemacs evil] (
(require 'treemacs-evil)
 ;; ))

;; (after [treemacs magit] (
(require 'treemacs-magit)
 ;; ))

;; (after [treemacs projectile] (
(require 'treemacs-projectile)
 ;; ))
\end{verbatim}

\subsection*{toggle treemacs with F7}
\label{sec:orgc7658e6}
\begin{verbatim}
(global-set-key [f8] 'treemacs)
\end{verbatim}

\subsection*{treemacs-git-mode}
\label{sec:orgf613e03}
`treemacs-git-mode` is a global minor mode which enables treemacs to check for files’ and directories’ git status information and highlight them accordingly (see also the treemacs-git-\ldots{} faces). The mode is available in 3 variants: simple, extended and deferred:

\begin{verbatim}
(treemacs-git-mode 'deferred) 
\end{verbatim}


\subsection*{make treemacs open on emacs startup}
\label{sec:org6ff7d9a}
\begin{verbatim}
(add-hook 'after-init-hook 'treemacs)
\end{verbatim}


\section*{Neotree}
\label{sec:org494f5b3}

\begin{verbatim}
(require 'neotree)
\end{verbatim}

\subsection*{neotree 'icons' theme, which supports filetype icons}
\label{sec:org4fa32e6}
\begin{verbatim}
  ;; (after 'neotree
;; (setq neo-theme (if (display-graphic-p) 'icons))

  (setq neo-theme 'icons)
\end{verbatim}


\subsection*{set NeoTree default window width}
\label{sec:orgf0462c3}
\begin{verbatim}
(setq neo-window-width 32)
\end{verbatim}

\subsection*{toggle neotree with F8}
\label{sec:orga2d7a9b}
\begin{verbatim}
(global-set-key [f7] 'neotree-toggle)
\end{verbatim}


\subsection*{make nerdtree open on emacs startup}
\label{sec:org92ad762}
\begin{verbatim}
;; (add-hook 'after-init-hook #'neotree-toggle)
\end{verbatim}




\subsection*{make neotree window open and go the file currently opened}
\label{sec:orgdb77e4d}
\begin{verbatim}
(setq neo-smart-open t)
\end{verbatim}


\subsection*{solve keybinding conflicts between neotree with evil mode}
\label{sec:orge35949f}
\begin{verbatim}
(add-hook 'neotree-mode-hook
          (lambda ()
            ; default Neotree bindings
            (define-key evil-normal-state-local-map (kbd "TAB") 'neotree-enter)
            (define-key evil-normal-state-local-map (kbd "SPC") 'neotree-quick-look)
            (define-key evil-normal-state-local-map (kbd "q") 'neotree-hide)
            (define-key evil-normal-state-local-map (kbd "RET") 'neotree-enter)
            (define-key evil-normal-state-local-map (kbd "g") 'neotree-refresh)
            (define-key evil-normal-state-local-map (kbd "n") 'neotree-next-line)
            (define-key evil-normal-state-local-map (kbd "p") 'neotree-previous-line)
            (define-key evil-normal-state-local-map (kbd "A") 'neotree-stretch-toggle)
            (define-key evil-normal-state-local-map (kbd "H") 'neotree-hidden-file-toggle)
            (define-key evil-normal-state-local-map (kbd "|") 'neotree-enter-vertical-split)
            (define-key evil-normal-state-local-map (kbd "-") 'neotree-enter-horizontal-split)
            ; simulating NERDTree bindings in Neotree
            (define-key evil-normal-state-local-map (kbd "R") 'neotree-refresh)
            (define-key evil-normal-state-local-map (kbd "r") 'neotree-refresh)
            (define-key evil-normal-state-local-map (kbd "u") 'neotree-refresh)
            (define-key evil-normal-state-local-map (kbd "C") 'neotree-change-root)
            (define-key evil-normal-state-local-map (kbd "c") 'neotree-create-node)))
\end{verbatim}



\section*{eyebrowse (window navigation)}
\label{sec:orgc38dc53}

\begin{verbatim}
(require 'eyebrowse)
\end{verbatim}

\begin{verbatim}
;; (eyebrowse-mode t)
\end{verbatim}


\section*{Shell}
\label{sec:org6850658}

\subsection*{System Shell}
\label{sec:orgb9070ed}
\subsubsection*{Make system shell open in a split-window buffer at the bottom of the screen}
\label{sec:org87f24c9}

\begin{verbatim}
(defun /shell/new-window ()
    "Opens up a new shell in the directory associated with the current buffer's file." 
    (interactive)
    (let* ((parent (if (buffer-file-name)
                       (file-name-directory (buffer-file-name))
                     default-directory))
           (height (/ (window-total-height) 3))
           (name   (car (last (split-string parent "/" t)))))
      (split-window-vertically (- height))
      (other-window 1)
      (shell "new")
      (rename-buffer (concat "*shell: " name "*"))

      (insert (concat "ls"))
      ))

; Pull system shell in a new bottom window
(define-key evil-normal-state-map (kbd "\"") #'/shell/new-window)
(define-key evil-visual-state-map (kbd "\"") #'/shell/new-window)
(define-key evil-motion-state-map (kbd "\"") #'/shell/new-window)
\end{verbatim}


\subsection*{Eshell}
\label{sec:org6d20c27}

\subsubsection*{Make eshell open in a split-window buffer at the bottom of the screen}
\label{sec:org676b10d}

\begin{verbatim}
(defun /eshell/new-window ()
    "Opens up a new eshell in the directory associated with the current buffer's file.  The eshell is renamed to match that directory to make multiple eshell windows easier."
    (interactive)
    (let* ((parent (if (buffer-file-name)
                       (file-name-directory (buffer-file-name))
                     default-directory))
           (height (/ (window-total-height) 3))
           (name   (car (last (split-string parent "/" t)))))
      (split-window-vertically (- height))
      (other-window 1)
      (eshell "new")
      (rename-buffer (concat "*eshell: " name "*"))

      (insert (concat "ls"))
      (eshell-send-input)))

; Pull eshell in a new bottom window
(define-key evil-normal-state-map (kbd "!") #'/eshell/new-window)
(define-key evil-visual-state-map (kbd "!") #'/eshell/new-window)
(define-key evil-motion-state-map (kbd "!") #'/eshell/new-window)
\end{verbatim}




\section*{PDF Tools}
\label{sec:org9ffc18b}

\subsection*{Install pdf-tools if its not already installed}
\label{sec:org86a9212}
\begin{verbatim}
;; (pdf-tools-install)
;; the docs say if i care about startup time, i should use pdf-loader-install instead of pdf-tools-install, but doenst say why
;; (pdf-loader-install) 
\end{verbatim}

\subsection*{Make buffer refresh every 1 second to PDF-tools updates the changed pdf}
\label{sec:orgf95922d}
\begin{verbatim}
(add-hook 'TeX-after-compilation-finished-functions #'TeX-revert-document-buffer)
;; (add-hook 'pdf-view-mode-hook 'auto-revert-mode) 
;; (add-hook 'doc-view-mode-hook 'auto-revert-mode) 
\end{verbatim}

\subsection*{PDF tools evil keybindings}
\label{sec:orgcbd8127}
\begin{verbatim}
(evil-define-key 'normal pdf-view-mode-map
  "h" 'pdf-view-previous-page-command
  "j" (lambda () (interactive) (pdf-view-next-line-or-next-page 5))
  "k" (lambda () (interactive) (pdf-view-previous-line-or-previous-page 5))
  "l" 'pdf-view-next-page-command)
\end{verbatim}



\section*{Appearance}
\label{sec:org98afe30}

\subsection*{Applying my theme}
\label{sec:org2c88349}

\begin{verbatim}

(add-to-list 'custom-theme-load-path "~/dotfiles/emacs.d/themes/")
; theme options:
; atom-one-dark (doenst work well with emacsclient, ugly blue bg)
; dracula
; darktooth
; gruvbox-dark-hard
; gruvbox-dark-light
; gruvbox-dark-medium
; base16-default-dark-theme <-- this one is good

(setq my-theme 'doom-tomorrow-night)

\end{verbatim}

Load the theme

\begin{verbatim}
(load-theme my-theme t)
\end{verbatim}


\begin{verbatim}

;; (defun load-my-theme (frame)
;;   "Function to load the theme in current FRAME.
;;   sed in conjunction
;;   with bellow snippet to load theme after the frame is loaded
;;   to avoid terminal breaking theme."
;;   (select-frame frame)
;;   (load-theme my-theme t))

;; ; make emacs load the theme after loading the frame
;; ; resolves issue with the theme not loading properly in terminal mode on emacsclient

;; ;; this if was breaking my emacs!!!!!
;;  (add-hook 'after-make-frame-functions #'load-my-theme)
\end{verbatim}


\subsection*{doom-modeline}
\label{sec:orgd04c7cf}

Require and enable the doom-modeline
\begin{verbatim}
(require 'doom-modeline)
(doom-modeline-mode 1)
\end{verbatim}

Don’t compact font caches during GC (garbage collection). 
\begin{verbatim}
;; (setq inhibit-compacting-font-caches t)
\end{verbatim}

Customize the doom-modeline (convert the comments to org later

\begin{verbatim}
;; How tall the mode-line should be. It's only respected in GUI.
;; If the actual char height is larger, it respects the actual height.
(setq doom-modeline-height 23)

;; How wide the mode-line bar should be. It's only respected in GUI.
(setq doom-modeline-bar-width 3)

;; Determines the style used by `doom-modeline-buffer-file-name'.
;;
;; Given ~/Projects/FOSS/emacs/lisp/comint.el
;;   truncate-upto-project => ~/P/F/emacs/lisp/comint.el
;;   truncate-from-project => ~/Projects/FOSS/emacs/l/comint.el
;;   truncate-with-project => emacs/l/comint.el
;;   truncate-except-project => ~/P/F/emacs/l/comint.el
;;   truncate-upto-root => ~/P/F/e/lisp/comint.el
;;   truncate-all => ~/P/F/e/l/comint.el
;;   relative-from-project => emacs/lisp/comint.el
;;   relative-to-project => lisp/comint.el
;;   file-name => comint.el
;;   buffer-name => comint.el<2> (uniquify buffer name)
;;
;; If you are expereicing the laggy issue, especially while editing remote files
;; with tramp, please try `file-name' style.
;; Please refer to https://github.com/bbatsov/projectile/issues/657.
(setq doom-modeline-buffer-file-name-style 'truncate-upto-project)

;; Whether display icons in mode-line or not.
(setq doom-modeline-icon t)

;; Whether display the icon for major mode. It respects `doom-modeline-icon'.
(setq doom-modeline-major-mode-icon t)

;; Whether display color icons for `major-mode'. It respects
;; `doom-modeline-icon' and `all-the-icons-color-icons'.
(setq doom-modeline-major-mode-color-icon t)

;; Whether display icons for buffer states. It respects `doom-modeline-icon'.
(setq doom-modeline-buffer-state-icon t)

;; Whether display buffer modification icon. It respects `doom-modeline-icon'
;; and `doom-modeline-buffer-state-icon'.
(setq doom-modeline-buffer-modification-icon t)

;; Whether display minor modes in mode-line or not.
(setq doom-modeline-minor-modes nil)

;; If non-nil, a word count will be added to the selection-info modeline segment.
(setq doom-modeline-enable-word-count nil)

;; Whether display buffer encoding.
(setq doom-modeline-buffer-encoding t)

;; Whether display indentation information.
(setq doom-modeline-indent-info nil)

;; If non-nil, only display one number for checker information if applicable.
(setq doom-modeline-checker-simple-format t)

;; The maximum displayed length of the branch name of version control.
(setq doom-modeline-vcs-max-length 12)

;; Whether display perspective name or not. Non-nil to display in mode-line.
(setq doom-modeline-persp-name t)

;; Whether display icon for persp name. Nil to display a # sign. It respects `doom-modeline-icon'
(setq doom-modeline-persp-name-icon nil)

;; Whether display `lsp' state or not. Non-nil to display in mode-line.
(setq doom-modeline-lsp t)

;; Whether display github notifications or not. Requires `ghub` package.
(setq doom-modeline-github nil)

;; The interval of checking github.
(setq doom-modeline-github-interval (* 30 60))

;; Whether display environment version or not
(setq doom-modeline-env-version t)
;; Or for individual languages
;; (setq doom-modeline-env-enable-python t)
;; (setq doom-modeline-env-enable-ruby t)
;; (setq doom-modeline-env-enable-perl t)
;; (setq doom-modeline-env-enable-go t)
;; (setq doom-modeline-env-enable-elixir t)
;; (setq doom-modeline-env-enable-rust t)

;; Change the executables to use for the language version string
(setq doom-modeline-env-python-executable "python")
(setq doom-modeline-env-ruby-executable "ruby")
(setq doom-modeline-env-perl-executable "perl")
(setq doom-modeline-env-go-executable "go")
(setq doom-modeline-env-elixir-executable "iex")
(setq doom-modeline-env-rust-executable "rustc")

;; Whether display mu4e notifications or not. Requires `mu4e-alert' package.
(setq doom-modeline-mu4e t)

;; Whether display irc notifications or not. Requires `circe' package.
(setq doom-modeline-irc t)

;; Function to stylize the irc buffer names.
(setq doom-modeline-irc-stylize 'identity)
\end{verbatim}


this was commented with C-c ; so it doenst get exported in favor of doom-modeline 
\subsection*{parrot-mode}
\label{sec:orged3a72a}

Enable the party parrot
\begin{verbatim}
(require 'parrot)
;; To see the party parrot in the modeline, turn on parrot mode:
(parrot-mode)  
\end{verbatim}


Rotation function keybindings for vanilla emacs
\begin{verbatim}
(global-set-key (kbd "C-c p") 'parrot-rotate-prev-word-at-point)
(global-set-key (kbd "C-c n") 'parrot-rotate-next-word-at-point)
\end{verbatim}

Rotation function keybindings for evil users
\begin{verbatim}
(define-key evil-normal-state-map (kbd "[r") 'parrot-rotate-prev-word-at-point)
(define-key evil-normal-state-map (kbd "]r") 'parrot-rotate-next-word-at-point)
\end{verbatim}

Type of parrots available:

\begin{itemize}
\item default
\item confused
\item emacs
\item nyan
\item rotating
\item science
\item thumbsup
\end{itemize}

\begin{verbatim}
(parrot-set-parrot-type 'default)
\end{verbatim}

Seconds between animation frames (can be a decimal number)
\begin{verbatim}
; parrot-animation-frame-interval  
\end{verbatim}

Minimum width of the window, below which party parrot mode will be disabled.
\begin{verbatim}
; parrot-minimum-window-width 
\end{verbatim}


To enable parrot animation, nil for a static image.
\begin{verbatim}
; (parrot-animate-parrot t) 
\end{verbatim}

Number of spaces of padding before and after the parrot.
\begin{verbatim}
; parrot-spaces-before 
; parrot-spaces-after
\end{verbatim}

\begin{verbatim}
; - number of times the parrot will cycle through its gif.
parrot-num-rotations 
\end{verbatim}


Add hook to mu4e so the parrot rotates when new email arrives
\begin{verbatim}
(add-hook 'mu4e-index-updated-hook #'parrot-start-animation)
\end{verbatim}

Rotate the parrot when clicking on it (this can also be used to execute any function when clicking the parrot, like 'flyspell-buffer)
\begin{verbatim}
(add-hook 'parrot-click-hook #'parrot-start-animation)
\end{verbatim}

Rotate parrot when buffer is saved
\begin{verbatim}
(add-hook 'after-save-hook #'parrot-start-animation)
\end{verbatim}

\subsection*{nyan-mode}
\label{sec:orgd229a16}

\begin{verbatim}
(require 'nyan-mode)
(nyan-mode)
\end{verbatim}

\begin{verbatim}
;; (setq nyan-cat-face-number 1)
\end{verbatim}

I had to add this hook because setting nyan-animate-nyancat to t alone wasnt animating the cat
\begin{verbatim}
(add-hook 'after-init-hook #'nyan-start-animation)
(setq nyan-animate-nyancat t)
\end{verbatim}

\begin{verbatim}
(setq nyan-wavy-trail t)
\end{verbatim}

\subsection*{solaire-mode}
\label{sec:org2cc4e4e}

solaire-mode is an aesthetic plugin that helps visually distinguish file-visiting windows from other types of windows (like popups or sidebars) by giving them a slightly different -- often brighter -- background.

\begin{verbatim}
(require 'solaire-mode)
\end{verbatim}

Enable solaire-mode anywhere it can be enabled
\begin{verbatim}
(solaire-global-mode +1)
\end{verbatim}

To enable solaire-mode unconditionally for certain modes:
\begin{verbatim}
(add-hook 'ediff-prepare-buffer-hook #'solaire-mode)
\end{verbatim}


\ldots{}if you use auto-revert-mode, this prevents solaire-mode from turning itself off every time Emacs reverts the file
\begin{verbatim}
(add-hook 'after-revert-hook #'turn-on-solaire-mode)
\end{verbatim}

highlight the minibuffer when it is activated:
\begin{verbatim}
(add-hook 'minibuffer-setup-hook #'solaire-mode-in-minibuffer)
\end{verbatim}

if the bright and dark background colors are the wrong way around, use this
to switch the backgrounds of the `default` and `solaire-default-face` faces.
This should be used \textbf{after} you load the active theme!

NOTE: This is necessary for themes in the doom-themes package!
\begin{verbatim}
(solaire-mode-swap-bg)
\end{verbatim}

\subsection*{cleaning the default UI}
\label{sec:org55301aa}

\begin{verbatim}
(setq inhibit-splash-screen t)

(blink-cursor-mode t)
(setq blink-cursor-blinks 0) ;; blink forever
(setq-default indicate-empty-lines t)
(setq-default line-spacing 3)
(setq frame-title-format '("Emacs"))
\end{verbatim}

\subsubsection*{Remove scroll bars from frames}
\label{sec:org90664c9}
\begin{verbatim}
(scroll-bar-mode -1)
\end{verbatim}

\subsubsection*{Remove menu bar and tool bar}
\label{sec:org46f5f53}
\begin{verbatim}
(tool-bar-mode -1)
(menu-bar-mode -1)
\end{verbatim}


\section*{centaur-tabs}
\label{sec:org50a7e1d}

\begin{verbatim}
(require 'centaur-tabs)
(centaur-tabs-mode t)
\end{verbatim}

\begin{verbatim}
(global-set-key (kbd "C-<prior>")  'centaur-tabs-backward)
(global-set-key (kbd "C-<next>") 'centaur-tabs-forward)
\end{verbatim}

\subsection*{Tab Style for centaur-tabs}
\label{sec:org0aa69d5}
Types available: 
\begin{itemize}
\item alternate
\item bar
\item box
\item chamfer
\item rounded
\item slang
\item wave
\item zigzag
\end{itemize}

\begin{verbatim}
(setq centaur-tabs-style "bar")
\end{verbatim}

\subsection*{Set the tabs height}
\label{sec:org53676d8}
\begin{verbatim}
(setq centaur-tabs-height 32)
\end{verbatim}

\subsection*{Use custom icons on tabs from all-the-icons}
\label{sec:orgec76008}
\begin{verbatim}
(setq centaur-tabs-set-icons t)
\end{verbatim}


\begin{verbatim}
(setq centaur-tabs-set-bar 'over)
\end{verbatim}

\subsection*{Gray out unselected tabs}
\label{sec:org55eff4a}
\begin{verbatim}
(setq centaur-tabs-gray-out-icons 'buffer)
\end{verbatim}

\subsection*{Customize the modified marker}
\label{sec:org8b77790}
To display a marker indicating that a buffer has been modified (atom-style)
\begin{verbatim}
(setq centaur-tabs-set-modified-marker t)
\end{verbatim}

\subsection*{To change the displayed string for the modified-marker}
\label{sec:org9e4fb30}
\begin{verbatim}
(setq centaur-tabs-modified-marker "*")
\end{verbatim}


\subsection*{Enable vim-like tab motions}
\label{sec:orgfd8d957}
\begin{verbatim}
(define-key evil-normal-state-map (kbd "g t") 'centaur-tabs-forward)
(define-key evil-normal-state-map (kbd "g T") 'centaur-tabs-backward)
\end{verbatim}



\section*{Minor modes}
\label{sec:org9de4594}

\subsection*{Highlighting stuff in code (numbers, operators and escape sequences)}
\label{sec:org1d28261}
\begin{verbatim}
(add-hook 'prog-mode-hook 'highlight-numbers-mode)
(add-hook 'prog-mode-hook 'highlight-operators-mode)
(add-hook 'prog-mode-hook 'hes-mode)    ;; highlight escape sequences
\end{verbatim}

\subsection*{which-key}
\label{sec:org2ec2d8a}

\begin{verbatim}
(require 'which-key)
(setq which-key-idle-delay 0.2)
(setq which-key-min-display-lines 3)
(setq which-key-max-description-length 20)
(setq which-key-max-display-columns 6)
(which-key-mode)
\end{verbatim}

\subsection*{diff-hl (highlights uncommited diffs in bar aside from the line numbers)}
\label{sec:org0f80897}
\begin{verbatim}
(global-diff-hl-mode)
\end{verbatim}

\subsection*{smartparens}
\label{sec:org14cce7e}

\begin{verbatim}
(require 'smartparens-config)
(add-hook 'prog-mode-hook #'smartparens-mode)
\end{verbatim}

\subsubsection*{evil-smartparens helps avoid conflicts between evil and smartparens}
\label{sec:orgd887e25}

\begin{verbatim}
(add-hook 'smartparens-enabled-hook #'evil-smartparens-mode)
\end{verbatim}

\subsection*{rainbow mode}
\label{sec:org6ebc8a4}
\subsubsection*{enable rainbow-mode on relevant filetypes}
\label{sec:org929656c}

\begin{verbatim}
Colorize hex, rgb and named color codes
\end{verbatim}


\begin{verbatim}
(add-hook 'org-mode-hook 'rainbow-mode)
(add-hook 'css-mode-hook 'rainbow-mode)
(add-hook 'php-mode-hook 'rainbow-mode)
(add-hook 'html-mode-hook 'rainbow-mode)
(add-hook 'web-mode-hook 'rainbow-mode)
(add-hook 'js2-mode-hook 'rainbow-mode)
\end{verbatim}

\subsection*{Emmet}
\label{sec:orge13a207}

\subsubsection*{Add hook to any markup file to load emmet-mode}
\label{sec:org8bcf0e6}
\begin{verbatim}
(add-hook 'sgml-mode-hook 'emmet-mode) ;; Auto-start on any markup modes
(add-hook 'css-mode-hook  'emmet-mode) ;; enable Emmet's css abbreviation. 
\end{verbatim}

\subsubsection*{Use emmet with JSX markup}
\label{sec:orgbc50cab}
\begin{verbatim}
(setq emmet-expand-jsx-className? t) ;; default nil
\end{verbatim}

\subsection*{Smartscan mode}
\label{sec:orgb17194b}
\begin{verbatim}
Usage:
M-n and M-p move between symbols
M-' to replace all symbols in the buffer matching the one under point
C-u M-' to replace symbols in your current defun only (as used by narrow-to-defun.)
\end{verbatim}


\begin{verbatim}
(smartscan-mode 1)
\end{verbatim}

\section*{FlyCheck linter}
\label{sec:orgc078652}

\begin{verbatim}
(add-hook 'after-init-hook #'global-flycheck-mode)
\end{verbatim}

\subsection*{Turn flycheck inline extension after flycheck starts}
\label{sec:org05bd1d7}

\begin{verbatim}
(with-eval-after-load 'flycheck
  (global-flycheck-inline-mode))
;; (with-eval-after-load 'flycheck
;;   (add-hook 'flycheck-mode-hook #'turn-on-flycheck-inline))
\end{verbatim}



\section*{Languages Setup}
\label{sec:orge00b4c0}

\subsection*{web mode}
\label{sec:org040cbce}

\subsubsection*{Require Web-Mode}
\label{sec:orgadfe77f}
\begin{verbatim}
(require 'web-mode)
\end{verbatim}

\subsubsection*{web-mode script/code offset indentation (for JavaScript, Java, PHP, Ruby, Go, VBScript, Python, etc.)}
\label{sec:org2fa8f84}
\begin{verbatim}
(setq web-mode-code-indent-offset 2)
\end{verbatim}

\subsection*{JavaScript}
\label{sec:org5378f6e}

\subsubsection*{Associate Javascript files with js2-mode}
\label{sec:org6f29420}

\begin{verbatim}
Use js2-mode for JS files  
\end{verbatim}


\begin{verbatim}
(add-to-list 'auto-mode-alist '("\\.js?\\'" . js2-mode))
\end{verbatim}

\begin{verbatim}
Use js2-mode for Typescript files  
\end{verbatim}


\begin{verbatim}
(add-to-list 'auto-mode-alist '("\\.ts?\\'" . js2-mode))
\end{verbatim}

\begin{verbatim}
Use js2-jsx-mode for JSX files  
\end{verbatim}

\begin{verbatim}
(add-to-list 'auto-mode-alist '("\\.jsx?\\'" . js2-jsx-mode))
\end{verbatim}

\subsubsection*{PrettierJS}
\label{sec:org2d6060b}

\begin{itemize}
\item Require first so i can actually use it
\label{sec:orgf35e3dc}

\begin{verbatim}
(require 'prettier-js)
\end{verbatim}

\item add prettier to js2 and rjsx minor modes
\label{sec:org813bf63}

\begin{verbatim}
(add-hook 'js2-mode-hook 'prettier-js-mode)
(add-hook 'web-mode-hook 'prettier-js-mode)
(add-hook 'rjsx-mode-hook 'prettier-js-mode)
\end{verbatim}

\item Add prettier to Web-Mode
\label{sec:org1a46a6e}
\begin{verbatim}
(defun enable-minor-mode (my-pair)
  "Enable minor mode if filename match the regexp.  MY-PAIR is a cons cell (regexp . minor-mode)."
  (if (buffer-file-name)
      (if (string-match (car my-pair) buffer-file-name)
      (funcall (cdr my-pair)))))
\end{verbatim}

\begin{itemize}
\item And then hook to web-mode like this:
\label{sec:orgca8b13f}

\begin{verbatim}
(add-hook 'web-mode-hook #'(lambda ()
                            (enable-minor-mode
                             '("\\.js?\\'" . prettier-js-mode)
                             '("\\.jsx?\\'" . prettier-js-mode)
                             '("\\.css?\\'" . prettier-js-mode))))
\end{verbatim}
\end{itemize}
\end{itemize}

\subsubsection*{js2-refactor}
\label{sec:org16e240a}
\begin{verbatim}
(add-hook 'js2-mode-hook #'js2-refactor-mode)
\end{verbatim}


\begin{itemize}
\item choose js2-refactor keybinding scheme (this can be changed easily)
\label{sec:orgf7a49f5}

\begin{verbatim}
(js2r-add-keybindings-with-prefix "C-c C-m")
\end{verbatim}
\end{itemize}

\subsection*{Elixir}
\label{sec:orgfc30dfa}

\begin{verbatim}
This elixir part of the config compiles to a separate files, inside the config folder, named `elixir.tau.el`
\end{verbatim}

\subsubsection*{Install elixir-mode if not already installed}
\label{sec:org9daae4d}

\begin{verbatim}
(unless (package-installed-p 'elixir-mode)
(package-install 'elixir-mode))  
\end{verbatim}

\subsubsection*{Require elixir-mode}
\label{sec:org695eb38}

\begin{verbatim}
(require 'elixir-mode)
\end{verbatim}

\subsubsection*{Elixir file types associations}
\label{sec:orgb6a5f4b}
\begin{verbatim}
(add-to-list 'auto-mode-alist '("\\.ex\\'" . elixir-mode))
(add-to-list 'auto-mode-alist '("\\.exs\\'" . elixir-mode))
;; Use web-mode for elixir template files (eex)
(add-to-list 'auto-mode-alist '("\\.eex\\'" . web-mode))
\end{verbatim}

\subsubsection*{Create a buffer-local hook to run elixir-format on save, only when we enable elixir-mode.}
\label{sec:orgcce3a7b}

\begin{verbatim}
(add-hook 'elixir-mode-hook
          (lambda () (add-hook 'before-save-hook 'elixir-format nil t)))
\end{verbatim}


\subsubsection*{Alchemist}
\label{sec:orgf313a4e}

\begin{itemize}
\item Install Alchemist if already not installed
\label{sec:org9af99af}
\begin{verbatim}
(unless (package-installed-p 'alchemist)
  (package-install 'alchemist)) 
\end{verbatim}
\end{itemize}

\subsection*{Ruby}
\label{sec:org95a1383}

\begin{verbatim}
This Ruby part of the config compiles to a separate file, inside the config folder, called `ruby.tau.el`
\end{verbatim}

\subsubsection*{Associate common Ruby files with  enh-ruby-mode}
\label{sec:orgbb02eea}
\begin{verbatim}
(add-to-list 'auto-mode-alist
             '("\\(?:\\.rb\\|ru\\|rake\\|thor\\|jbuilder\\|gemspec\\|podspec\\|/\\(?:Gem\\|Rake\\|Cap\\|Thor\\|Vagrant\\|Guard\\|Pod\\)file\\)\\'" . enh-ruby-mode))
\end{verbatim}
\subsubsection*{Require enhanced-ruby major mode replacement for native ruby-mode}
\label{sec:orgbc415b7}
\begin{verbatim}
(require 'enh-ruby-mode)
\end{verbatim}

\subsubsection*{this bellow is optional, i only use it because it complains about not finding ruby in /usr/local/bin}
\label{sec:org62103bc}
``This is also easily solvable by creating a symbolic link to the ruby shim to /usr/local/bin/ruby
\begin{verbatim}
(setq enh-ruby-program "~/.rbenv/shims/ruby") ; so that still works if ruby points to ruby1.8
\end{verbatim}

\begin{verbatim}
(setq-default
  ruby-use-encoding-map nil
  ruby-insert-encoding-magic-comment nil)
\end{verbatim}

\subsubsection*{Enable superword-mode for enh-ruby-mode}
\label{sec:org086a047}
It change all cursor movement/edit commands to stop in-between the “camelCase” words.

\begin{verbatim}
(add-hook 'enh-ruby-mode-hook 'superword-mode)
\end{verbatim}

\subsubsection*{Enhanced Ruby Mode defines its own specific faces with the hook erm-define-faces. If your theme is already defining those faces, to not overwrite them, just remove the hook with:}
\label{sec:org89eefcd}
\begin{verbatim}
(remove-hook 'enh-ruby-mode-hook 'erm-define-faces)
\end{verbatim}


\begin{verbatim}
(after 'page-break-lines
            (push 'ruby-mode page-break-lines-modes))
\end{verbatim}

\begin{verbatim}
(require 'rspec-mode)
\end{verbatim}

\subsubsection*{Inferior ruby}
\label{sec:orgb7e83a7}
\begin{verbatim}
(require 'inf-ruby)
(add-hook 'enh-ruby-mode-hook 'inf-ruby-minor-mode)
\end{verbatim}

\subsubsection*{Ruby compilation}
\label{sec:orgab7d8cb}
\begin{verbatim}
(require 'ruby-compilation)

(after 'enh-ruby-mode
            (let ((m ruby-mode-map))
              (define-key m [S-f7] 'ruby-compilation-this-buffer)
              (define-key m [f7] 'ruby-compilation-this-test)))

(after 'ruby-compilation
            (defalias 'rake 'ruby-compilation-rake))
\end{verbatim}

\subsubsection*{ri support}
\label{sec:org87abdee}
\begin{verbatim}
;; (require 'yari)
;; (defalias 'ri 'yari)
\end{verbatim}



\begin{verbatim}
;; (require 'goto-gem)
\end{verbatim}


\begin{verbatim}
(require 'bundler)
\end{verbatim}


\begin{verbatim}
;; (when (maybe-require 'yard-mode)
;;   (add-hook 'ruby-mode-hook 'yard-mode)
;;   (add-hook 'enh-ruby-mode-hook 'yard-mode)
;;   (after 'yard-mode
;;               (diminish 'yard-mode)))
\end{verbatim}

\subsubsection*{Rubocop}
\label{sec:org3766029}
\begin{verbatim}
(require 'rubocop)
\end{verbatim}

\subsection*{Go}
\label{sec:org052f118}

\subsubsection*{Enable go-mode}
\label{sec:orgba97b4e}
\begin{verbatim}
(autoload 'go-mode "go-mode" "Major mode for editing Go code." t)
\end{verbatim}

\subsubsection*{Associate Go files with go-mode}
\label{sec:org20a3058}
\begin{verbatim}
(add-to-list 'auto-mode-alist '("\\.go\\'" . go-mode))
\end{verbatim}

\subsection*{Elm}
\label{sec:org7143fc9}
\subsubsection*{Company backend for elm}
\label{sec:org1af8320}
\begin{verbatim}

(with-eval-after-load 'company
  (add-to-list 'company-backends 'company-elm))

\end{verbatim}

\subsection*{Haskell}
\label{sec:org0027f85}
\begin{verbatim}
(add-hook 'haskell-mode-hook #'flycheck-haskell-setup)
; ; (require 'haskell-interactive-mode)
; (add-hook 'haskell-mode-hook 'turn-on-haskell-indent)
; (eval-after-load 'flycheck
;                  '(add-hook 'flycheck-mode-hook #'flycheck-haskell-setup))
; (add-hook 'haskell-mode-hook (lambda ()
;                                (electric-indent-mode -1)))
; (add-hook 'haskell-mode-hook 'interactive-haskell-mode)
; (add-hook 'haskell-mode-hook (lambda () (global-set-key (kbd "<f5>") 'haskell-process-cabal-build)))
\end{verbatim}

\subsection*{PHP}
\label{sec:org91eabe1}

\begin{verbatim}
Use web-mode for PHP files  
\end{verbatim}


\begin{verbatim}
(add-to-list 'auto-mode-alist '("\\.php?\\'" . web-mode))
(add-to-list 'auto-mode-alist '("\\.inc?\\'" . web-mode))
\end{verbatim}

\subsection*{HTML}
\label{sec:org7abdef3}

\subsubsection*{Associate HTML with web-mode}
\label{sec:org8de5ce5}

\begin{verbatim}
(add-to-list 'auto-mode-alist '("\\.html?\\'" . web-mode))
(add-to-list 'auto-mode-alist '("\\.phtml\\'" . web-mode))
\end{verbatim}

\subsubsection*{HTML element offset indentation}
\label{sec:org11ee253}
\begin{verbatim}
(setq web-mode-markup-indent-offset 4)
\end{verbatim}

\subsection*{CSS}
\label{sec:org89aa09f}

\subsubsection*{Open CSS files with web-mode}
\label{sec:orge8c93b5}
\begin{verbatim}
(add-to-list 'auto-mode-alist '("\\.css?\\'" . web-mode))
\end{verbatim}

\subsubsection*{CSS offset indentation}
\label{sec:org9862038}
\begin{verbatim}
(setq web-mode-css-indent-offset 2)
\end{verbatim}

\subsection*{YAML}
\label{sec:org00badeb}

\begin{verbatim}
(require 'yaml-mode)
(add-to-list 'auto-mode-alist '("\\.yml\\'" . yaml-mode))
\end{verbatim}


\subsubsection*{Unlike python-mode, this mode follows the Emacs convention of not binding the ENTER key to `newline-and-indent'.  To get this behavior, add the key definition to `yaml-mode-hook':}
\label{sec:org0ca6140}

\begin{verbatim}
(add-hook 'yaml-mode-hook
  '(lambda ()
    (define-key yaml-mode-map "\C-m" 'newline-and-indent))) 
\end{verbatim}

\subsection*{\LaTeX{}}
\label{sec:orgc6f286b}

\begin{verbatim}
(require 'ox-latex)
\end{verbatim}

\subsubsection*{AucTex settings}
\label{sec:org892a796}

\begin{verbatim}
(require 'tex)
\end{verbatim}

Three steps are required (as according to ORG official docs) to setup AucTex with Emacs:

\begin{itemize}
\item 1) Tell emacs where the \LaTeX{} related bins are located in the system
\label{sec:org4748a72}

\begin{verbatim}
(setq exec-path (append exec-path '("/usr/bin/tex")))
\end{verbatim}

\item 2) Load AucTex
\label{sec:org87df796}

\begin{verbatim}
;; (load "auctex.el" nil t t)
;; (load "preview-latex.el" nil t t)
\end{verbatim}

\item 3) Add Latex to list of org-babel loaded languages
\label{sec:org48748ed}

\#+END\textsubscript{SRC}
\begin{verbatim}
(org-babel-do-load-languages
 'org-babel-load-languages
 '((latex . t)))
\end{verbatim}

\begin{verbatim}
(setq TeX-auto-save t)
(setq TeX-parse-self t)
(setq-default TeX-master nil)
\end{verbatim}


\begin{verbatim}
(add-hook 'LaTeX-mode-hook 'visual-line-mode)
(add-hook 'LaTeX-mode-hook 'flyspell-mode)
(add-hook 'LaTeX-mode-hook 'LaTeX-math-mode)   
\end{verbatim}
\end{itemize}




\subsubsection*{Latex Classes}
\label{sec:org746007d}
\begin{itemize}
\item Add the beamer presentation class template to org
\label{sec:org38a417f}
\begin{verbatim}
(add-to-list 'org-latex-classes
             '("beamer"
               "\\documentclass\[presentation\]\{beamer\}"
               ("\\section\{%s\}" . "\\section*\{%s\}")
               ("\\subsection\{%s\}" . "\\subsection*\{%s\}")
               ("\\subsubsection\{%s\}" . "\\subsubsection*\{%s\}")))
\end{verbatim}


\item Add the memoir class template to org
\label{sec:orge93dcd5}

The Sections and Heading Levels gets configured as follows: 

\begin{center}
\begin{tabular}{lrl}
Division & <c>Level & <c>org-equivalent\\
\book & -2 & *\\
\part & -1 & **\\
\chapter & 0 & \textbf{*}\\
\section & 1 & \textbf{**}\\
\subsection & 2 & \textbf{\textbf{*}}\\
\subsubsection & 3 & \textbf{\textbf{**}}\\
\paragraph & 4 & \textbf{\textbf{\textbf{*}}}\\
\subparagraph & 5 & \textbf{\textbf{\textbf{**}}}\\
\end{tabular}
\end{center}


\begin{verbatim}
(add-to-list 'org-latex-classes
             '("memoir"
               "\\documentclass\[a4paper\]\{memoir\}"
               ("\\book\{%s\}" . "\\book*\{%s\}")
               ("\\part\{%s\}" . "\\part*\{%s\}")
               ("\\chapter\{%s\}" . "\\chapter*\{%s\}")
               ("\\section\{%s\}" . "\\section*\{%s\}")
               ("\\subsection\{%s\}" . "\\subsection*\{%s\}")
               ("\\subsubsection\{%s\}" . "\\subsubsection*\{%s\}")))
\end{verbatim}

\item Add abntex2 class to org list of latex classes
\label{sec:org16a178e}
This class is based on the Memoir class
The Sections and Heading Levels gets configured as follows: 

\begin{center}
\begin{tabular}{lrl}
Division & <c>Level & <c>org-equivalent\\
\part & -1 & *\\
\chapter & 0 & **\\
\section & 1 & \textbf{*}\\
\subsection & 2 & \textbf{**}\\
\subsubsection & 3 & \textbf{\textbf{*}}\\
\paragraph & 4 & \textbf{\textbf{**}}\\
\subparagraph & 5 & \textbf{\textbf{\textbf{*}}}\\
\end{tabular}
\end{center}
\begin{verbatim}
(add-to-list 'org-latex-classes
            '("abntex2"
              "\\documentclass\[a4paper,oneside,12pt\]\{abntex2\}"
              ("\\chapter\{%s\}" . "\\chapter*\{%s\}")
              ("\\section\{%s\}" . "\\section*\{%s\}")
              ("\\subsection\{%s\}" . "\\subsection*\{%s\}")
              ("\\subsubsection\{%s\}" . "\\subsubsection*\{%s\}")
              ("\\subsubsection\{%s\}" . "\\subsubsection*\{%s\}")
              ("\\paragraph\{%s\}" . "\\paragraph*\{%s\}")))
\end{verbatim}
\end{itemize}


\subsubsection*{Enable latex-preview-pane}
\label{sec:org7e98401}
\begin{verbatim}
(latex-preview-pane-enable)
\end{verbatim}

\subsubsection*{To compile documents to PDF by default add the following to your \textasciitilde{}/.emacs.}
\label{sec:org7122be2}

\begin{verbatim}
(setq TeX-PDF-mode t)
\end{verbatim}

\begin{itemize}
\item If it doesn’t work, try this :
\label{sec:org3f0730c}

\begin{verbatim}
(TeX-global-PDF-mode t)
\end{verbatim}
\end{itemize}


\subsubsection*{To highlight (or font-lock) the “\section{title}” lines:}
\label{sec:orgc3d8b07}

\begin{verbatim}
(font-lock-add-keywords
   'latex-mode
   `((,(concat "^\\s-*\\\\\\("
               "\\(documentclass\\|\\(sub\\)?section[*]?\\)"
               "\\(\\[[^]% \t\n]*\\]\\)?{[-[:alnum:]_ ]+"
               "\\|"
               "\\(begin\\|end\\){document"
               "\\)}.*\n?")
      (0 'your-face append))))
\end{verbatim}


\subsubsection*{Convert quotes to \LaTeX{} Smartquotes}
\label{sec:org1a48e1a}

\begin{verbatim}
(setq org-export-with-smart-quotes t)
\end{verbatim}

\subsubsection*{Keep latex logfiles}
\label{sec:org63fad50}
\begin{verbatim}
(setq org-latex-remove-logfiles nil)
\end{verbatim}


\subsection*{Markdown}
\label{sec:orgdd06628}

\subsubsection*{Enable markdown-mode}
\label{sec:org8124812}
\begin{verbatim}
(autoload 'mardown-mode "markdown-mode")
(add-to-list 'auto-mode-alist '("\\.md\\'" . markdown-mode))
\end{verbatim}


\section*{Devops tools setup and helpers (dockerfile, netlify)}
\label{sec:orgbf48e26}

\begin{verbatim}
(require 'dockerfile-mode)
(add-to-list 'auto-mode-alist '("Dockerfile\\'" . dockerfile-mode)) 
\end{verbatim}


\section*{Autocompletion and Snippets}
\label{sec:org334d1f8}

\subsection*{LSP (language server protocol implementation for emacs)}
\label{sec:org0f6958f}

\subsubsection*{require and create lsp-mode hook for every programming language}
\label{sec:orgdcfafd0}
\begin{verbatim}
(require 'lsp-mode)
(add-hook 'prog-mode-hook #'lsp) 
(add-hook 'enh-ruby-mode-hook #'lsp) 
(add-hook 'js2-mode-hook #'lsp) 
(add-hook 'js2-jsx-mode-hook #'lsp) 
\end{verbatim}

\subsubsection*{lsp-ui}
\label{sec:orga717767}

\begin{verbatim}
This contains all the higher level UI modules of lsp-mode, like flycheck support and code lenses.
\end{verbatim}


\begin{verbatim}
(require 'lsp-ui)
(add-hook 'lsp-mode-hook 'lsp-ui-mode)
\end{verbatim}

\subsubsection*{company-lsp}
\label{sec:org8f92314}

\begin{verbatim}
Company completion backend for lsp-mode. 
\end{verbatim}


\begin{verbatim}
(require 'company-lsp)
(push 'company-lsp company-backends)
\end{verbatim}

\subsection*{Disable <RET> for autocomplete and leave on TAB}
\label{sec:org3a50079}
\begin{verbatim}
;; (define-key ac-completing-map [return] nil)
;; (define-key ac-completing-map "\r" nil)
\end{verbatim}


\subsection*{enable autocompletion engine}
\label{sec:org770c1b3}
\begin{verbatim}
(require 'auto-complete)
(global-auto-complete-mode t)
\end{verbatim}


\subsection*{Company mode (Complete Anything)}
\label{sec:orge955e54}

\subsubsection*{Basic settings for company-mode}
\label{sec:org24857d8}
\begin{verbatim}
(require 'company)
(global-company-mode t)
(setq company-tooltip-limit 20)                      ; bigger popup window
(setq company-minimum-prefix-length 1)               ; start completing after 1st char typed
(setq company-idle-delay .1)                         ; decrease delay before autocompletion popup shows
(setq company-echo-delay 0)                          ; remove annoying blinking
(setq company-begin-commands '(self-insert-command)) ; start autocompletion only after typing
(setq company-dabbrev-downcase nil)                  ; Do not convert to lowercase
(setq company-dabbrev-ignore-case t)
(setq company-dabbrev-code-everywhere t)
(setq company-selection-wrap-around t)               ; continue from top when reaching bottom
(setq company-auto-complete 'company-explicit-action-p)
\end{verbatim}

\subsubsection*{Enable company-mode in all buffers}
\label{sec:orgdb27fce}
\begin{verbatim}
(add-hook 'after-init-hook 'global-company-mode)
\end{verbatim}

\subsubsection*{Bind <TAB> to company-indent-or-complete}
\label{sec:org77bdebe}
\begin{verbatim}
(add-hook 'after-init-hook 'global-company-mode)

(after "company-autoloads"
   (define-key evil-insert-state-map (kbd "TAB")
     #'company-indent-or-complete-common))
\end{verbatim}


\subsection*{Yasnippets}
\label{sec:orgd3dc01e}

\begin{verbatim}
(require 'yasnippet)
(yas-global-mode 1)
\end{verbatim}

\begin{verbatim}
(setq yas-snippet-dirs
      '("~/.emacs.d/snippets"                 ;; personal snippets
        ))
\end{verbatim}


\section*{Copy/Paste To/From System's Clipboard =D}
\label{sec:orgccfceb7}
this was supposed to be on the helper functions and macro section at the beggining of the file
but it has evil defined keybindings and had to be put after the evil section or emacs would complain it didnt know what evil is

\subsubsection*{Copy}
\label{sec:org7428cf8}

\begin{verbatim}
(defun copy-to-clipboard ()
  "Make F8 and F9 Copy and Paste to/from OS Clipboard.  Super usefull."
  (interactive)
  (if (display-graphic-p)
      (progn
        (message "Yanked region to x-clipboard!")
        (call-interactively 'clipboard-kill-ring-save)
        )
    (if (region-active-p)
        (progn
          (shell-command-on-region (region-beginning) (region-end) "xsel -i -b")
          (message "Yanked region to clipboard!")
          (deactivate-mark))
      (message "No region active; can't yank to clipboard!")))
  )
\end{verbatim}


\subsubsection*{Paste}
\label{sec:org567ad4e}

\begin{verbatim}
(evil-define-command paste-from-clipboard()
  (if (display-graphic-p)
      (progn
        (clipboard-yank)
        (message "graphics active")
        )
    (insert (shell-command-to-string "xsel -o -b")) ) )
\end{verbatim}

\begin{verbatim}
(global-set-key [f9] 'copy-to-clipboard)
(global-set-key [f10] 'paste-from-clipboard)
\end{verbatim}


\section*{Provide my personal packages in separate files}
\label{sec:orgac9fc2a}

\subsection*{Provide the evil.tau.el file (this need to be the last thing on the file)}
\label{sec:orgcd68534}

\begin{verbatim}
(provide 'evil.tau)
;;; evil.tau.el ends here...
\end{verbatim}

\subsection*{Provide the org.tau.el file (this needs to be the last part of the org config file)}
\label{sec:orgd843a7b}

\begin{verbatim}
(provide 'org.tau)
;;; elixir.tau.el ends here...
\end{verbatim}

\subsection*{Provide the ruby.tau.el file (this needs to be the last part of the org config file)}
\label{sec:org9c8aea8}

\begin{verbatim}
(provide 'ruby.tau)
;;; elixir.tau.el ends here...
\end{verbatim}

\subsection*{Provide the elixir.tau.el file (this needs to be the last part of the org config file)}
\label{sec:org7add388}

\begin{verbatim}
(provide 'elixir.tau)
;;; elixir.tau.el ends here...
\end{verbatim}


\section*{End init.el file}
\label{sec:orgff50ad9}
\begin{verbatim}
;; Local Variables:
;; coding: utf-8
;; no-byte-compile: t
;; End:


(provide 'init)
;;; .emacs ends here

\end{verbatim}
\end{document}